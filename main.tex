\documentclass[12pt,arial,letterpaper]{book}
\linespread{1.5}
\usepackage{eumat}
\usepackage{graphicx}
\graphicspath{{images/}}
\usepackage{titlesec}
\titleformat{\chapter}[display]
 {\huge\bfseries\centering}
 {\chaptertitlename\ \thechapter}{20pt}{\huge}
\renewcommand{\chaptername}{BAB}
\pagestyle{plain}
\renewcommand{\contentsname}{Daftar Isi}

\begin{document}
\clearpage
\thispagestyle{empty}
\frontmatter
\begin{center}
    \huge{\textbf{TUGAS PROYEK \\ APLIKASI KOMPUTER}}
\end{center}
\vspace{3cm}
\begin{minipage}{17cm}
    \begin{center}
        \includegraphics[width=6cm]{images/UNY.png}
    \end{center}
\end{minipage}

\vspace{3cm}

\begin{center}
    \large{Disusun oleh: \\ Amalia Intan Arvitasari \\ 22305144026}
\end{center}

\vspace{4cm}
\begin{center}
    \large{\textbf{PROGRAM STUDI MATEMATIKA}} \\
    \large{\textbf{JURUSAN PENDIDIKAN MATEMATIKA}} \\
    \large{\textbf{FAKULTAS MATEMATIKA DAN ILMU PENGETAHUAN ALAM}} \\
    \large{\textbf{UNIVERSITAS NEGERI YOGYAKARTA}} \\
    \large{\textbf{2023}}
\end{center}

\clearpage
\thispagestyle{empty}
\tableofcontents
\mainmatter

\chapter{EMT Untuk Aljabar}
\begin{eulernotebook}
\eulerheading{EMT untuk Perhitungan Aljabar}
\begin{eulercomment}
Pada notebook ini Anda belajar menggunakan EMT untuk melakukan
berbagai perhitungan terkait dengan materi atau topik dalam Aljabar.
Kegiatan yang harus Anda lakukan adalah sebagai berikut:

- Membaca secara cermat dan teliti notebook ini;\\
- Menerjemahkan teks bahasa Inggris ke bahasa Indonesia;\\
- Mencoba contoh-contoh perhitungan (perintah EMT) dengan cara
meng-ENTER setiap perintah EMT yang ada (pindahkan kursor ke baris
perintah)\\
- Jika perlu Anda dapat memodifikasi perintah yang ada dan memberikan
keterangan/penjelasan tambahan terkait hasilnya.\\
- Menyisipkan baris-baris perintah baru untuk mengerjakan soal-soal
Aljabar dari file PDF yang saya berikan;\\
- Memberi catatan hasilnya.\\
- Jika perlu tuliskan soalnya pada teks notebook (menggunakan format
LaTeX).\\
- Gunakan tampilan hasil semua perhitungan yang eksak atau simbolik
dengan format LaTeX. (Seperti contoh-contoh pada notebook ini.)

\end{eulercomment}
\eulersubheading{Contoh pertama}
\begin{eulercomment}
Menyederhanakan bentuk aljabar:

\end{eulercomment}
\begin{eulerformula}
\[
6x^{-3}y^5\times -7x^2y^{-9}
\]
\end{eulerformula}
\begin{eulercomment}
\end{eulercomment}
\begin{eulerprompt}
>$&6*x^(-3)*y^5*-7*x^2*y^(-9)
\end{eulerprompt}
\begin{eulerformula}
\[
-\frac{42}{x\,y^4}
\]
\end{eulerformula}
\begin{eulercomment}
Menjabarkan:

\end{eulercomment}
\begin{eulerformula}
\[
(6x^{-3}+y^5)(-7x^2-y^{-9})
\]
\end{eulerformula}
\begin{eulerprompt}
>$&showev('expand((6*x^(-3)+y^5)*(-7*x^2-y^(-9))))
\end{eulerprompt}
\begin{eulerformula}
\[
{\it expand}\left(\left(-\frac{1}{y^9}-7\,x^2\right)\,\left(y^5+
 \frac{6}{x^3}\right)\right)=-7\,x^2\,y^5-\frac{1}{y^4}-\frac{6}{x^3
 \,y^9}-\frac{42}{x}
\]
\end{eulerformula}
\begin{eulercomment}
Contoh Soal dan Penyelesaian latihan R2\\
1. Lakukan penyederhanaan operasi berikut\\
\end{eulercomment}
\begin{eulerformula}
\[
6xy^3\times9x^4y^2
\]
\end{eulerformula}
\begin{eulerttcomment}
   Penyelesaian:
\end{eulerttcomment}
\begin{eulerprompt}
>$&6*x*y^3*9*x^4*y^2
\end{eulerprompt}
\begin{eulerformula}
\[
54\,x^5\,y^5
\]
\end{eulerformula}
\begin{eulercomment}
2. Lakukan penyerhanaan operasi berikut ini\\
\end{eulercomment}
\begin{eulerformula}
\[
\left(\frac{24a^{10}b^{-8}c^7}{12a^6b^{-3}c^5}\right)^{-5}
\]
\end{eulerformula}
\begin{eulerttcomment}
   Penyelesaian:
\end{eulerttcomment}
\begin{eulerprompt}
>$&((24*a^10*b^(-8)*c^7)/(12*a^6*b^(-3)*c^5))^(-5)
\end{eulerprompt}
\begin{eulerformula}
\[
\frac{b^{25}}{32\,a^{20}\,c^{10}}
\]
\end{eulerformula}
\begin{eulercomment}
3. Lakukan penyederhanaan operasi berikut ini\\
\end{eulercomment}
\begin{eulerformula}
\[
\left(\frac{125p^{12}q^{-14}r^{22}}{25p^8q^6r^{-15}}\right)^{-4}
\]
\end{eulerformula}
\begin{eulerttcomment}
   Penyelesaian:
\end{eulerttcomment}
\begin{eulerprompt}
>$&((125*p^12*q^(-14)*r^22)/(25*p^8*q^6*r^(-15)))^(-4)
\end{eulerprompt}
\begin{eulerformula}
\[
\frac{q^{80}}{625\,p^{16}\,r^{148}}
\]
\end{eulerformula}
\begin{eulercomment}
4. Lakukan penyederhanaan operasi berikut ini\\
\end{eulercomment}
\begin{eulerformula}
\[
(m^{x-b}n^{x+b})^x(m^bn^{-b})^x
\]
\end{eulerformula}
\begin{eulerttcomment}
   Penyelesaian:
\end{eulerttcomment}
\begin{eulerprompt}
>$&((m^(x-b)*n(x+b))^x*(m^b*n^(-b))^x)
\end{eulerprompt}
\begin{eulerformula}
\[
\left(\frac{m^{b}}{n^{b}}\right)^{x}\,\left(m^{x-b}\,n\left(x+b
 \right)\right)^{x}
\]
\end{eulerformula}
\begin{eulercomment}
\end{eulercomment}
\eulersubheading{Baris Perintah}
\begin{eulercomment}
Baris perintah Euler terdiri dari satu atau beberapa perintah Euler
diikuti dengan titik koma ";" atau koma ",". Titik koma mencegah
pencetakan hasilnya. Koma setelah perintah terakhir dapat dihilangkan.\\
Baris perintah berikut hanya akan mencetak hasil ekspresi, bukan tugas
atau perintah format.
\end{eulercomment}
\begin{eulerprompt}
>r:=2; h:=4; pi*r^2*h/3
\end{eulerprompt}
\begin{euleroutput}
  16.7551608191
\end{euleroutput}
\begin{eulercomment}
Perintah harus dipisahkan dengan yang kosong. Baris perintah berikut
mencetak dua hasilnya.
\end{eulercomment}
\begin{eulerprompt}
>pi*2*r*h, %+2*pi*r*h // Ingat tanda % menyatakan hasil perhitungan terakhir sebelumnya
\end{eulerprompt}
\begin{euleroutput}
  50.2654824574
  100.530964915
\end{euleroutput}
\begin{eulercomment}
Baris perintah dijalankan sesuai urutan yang ditekan pengguna kembali.
Jadi, Anda mendapatkan nilai baru setiap kali Anda menjalankan baris
kedua.
\end{eulercomment}
\begin{eulerprompt}
>x := 1;
>x := cos(x) // nilai cosinus (x dalam radian)
\end{eulerprompt}
\begin{euleroutput}
  0.540302305868
\end{euleroutput}
\begin{eulerprompt}
>x := cos(x)
\end{eulerprompt}
\begin{euleroutput}
  0.857553215846
\end{euleroutput}
\begin{eulercomment}
Jika dua jalur dihubungkan dengan "..." kedua jalur akan selalu
dijalankan secara bersamaan.
\end{eulercomment}
\begin{eulerprompt}
>x := 1.5; ...
>x := (x+2/x)/2, x := (x+2/x)/2, x := (x+2/x)/2, 
\end{eulerprompt}
\begin{euleroutput}
  1.41666666667
  1.41421568627
  1.41421356237
\end{euleroutput}
\begin{eulercomment}
Ini juga merupakan cara yang baik untuk menyebarkan perintah panjang
ke dua baris atau lebih. Anda dapat menekan Ctrl+Return untuk membagi
baris menjadi dua pada posisi kursor saat ini, atau Ctlr+Back untuk
menggabungkan baris.

Untuk melipat semua multi-garis tekan Ctrl+L. Maka garis-garis
berikutnya hanya akan terlihat, jika salah satunya mendapat
fokus.Untuk melipat satu multi-baris, mulailah baris pertama dengan
"\%+"
\end{eulercomment}
\begin{eulerprompt}
>%+ x=4+5; ...
\end{eulerprompt}
\begin{eulercomment}
Garis yang dimulai dengan \%\% tidak akan terlihat sama sekali.
\end{eulercomment}
\begin{euleroutput}
  81
\end{euleroutput}
\begin{eulercomment}
Euler mendukung loop di baris perintah, asalkan cocok ke dalam satu
baris atau multi-baris. Tentu saja, pembatasan ini tidak berlaku dalam\\
program. Untuk informasi lebih lanjut lihat pendahuluan berikut.
\end{eulercomment}
\begin{eulerprompt}
>x=1; for i=1 to 5; x := (x+2/x)/2, end; // menghitung akar 2
\end{eulerprompt}
\begin{euleroutput}
  1.5
  1.41666666667
  1.41421568627
  1.41421356237
  1.41421356237
\end{euleroutput}
\begin{eulercomment}
Tidak masalah menggunakan multi-baris. Pastikan baris diakhiri dengan
"...".\\
\end{eulercomment}
\begin{eulerttcomment}
 
\end{eulerttcomment}
\begin{eulerprompt}
>x := 1.5; // comments go here before the ...
>repeat xnew:=(x+2/x)/2; until xnew~=x; ...
>   x := xnew; ...
>end; ...
>x,
\end{eulerprompt}
\begin{euleroutput}
  1.41421356237
\end{euleroutput}
\begin{eulercomment}
Struktur bersyarat juga berfungsi.
\end{eulercomment}
\begin{eulerprompt}
>if E^pi>pi^E; then "Thought so!", endif;
\end{eulerprompt}
\begin{euleroutput}
  Thought so!
\end{euleroutput}
\begin{eulercomment}
Saat Anda menjalankan perintah, kursor dapat berada di posisi mana pun
di baris perintah. Anda dapat kembali ke perintah sebelumnya \\
atau melompat ke perintah berikutnya dengan tombol panah. Atau Anda
dapat mengklik bagian komentar di atas perintah untuk membuka \\
perintah.

Saat Anda menggerakkan kursor di sepanjang garis, pasangan tanda
kurung atau tanda kurung buka dan tutup akan disorot. Juga, \\
perhatikan baris status. Setelah tanda kurung buka dari fungsi sqrt(),
baris status akan menampilkan teks bantuan untuk fungsi tersebut. \\
Jalankan perintah dengan kunci kembali.
\end{eulercomment}
\begin{eulerprompt}
>sqrt(sin(10°)/cos(20°))
\end{eulerprompt}
\begin{euleroutput}
  0.429875017772
\end{euleroutput}
\begin{eulercomment}
Untuk melihat bantuan untuk perintah terbaru, buka jendela bantuan
dengan F1. Di sana, Anda dapat memasukkan teks untuk\\
dicari. Pada baris kosong, bantuan untuk jendela bantuan akan
ditampilkan. Anda dapat menekan escape untuk menghapus garis,\\
atau untuk menutup jendela bantuan.

Anda dapat mengklik dua kali pada perintah apa pun untuk membuka
bantuan untuk perintah ini. Coba klik dua kali pada exp command di
bawah pada baris perintah.
\end{eulercomment}
\begin{eulerprompt}
>exp(log(2.5))
\end{eulerprompt}
\begin{euleroutput}
  2.5
\end{euleroutput}
\begin{eulercomment}
Anda juga dapat menyalin dan menempel di Euler. Gunakan Ctrl-C dan
Ctrl-V untuk ini. Untuk menandai teks, seret mouse atau gunakan shift \\
bersamaan dengan tombol kursor apa pun. Selain itu, Anda dapat
menyalin tanda kurung yang disorot.
\end{eulercomment}
\begin{eulercomment}

\end{eulercomment}
\eulersubheading{Sintaks Dasar}
\begin{eulercomment}
Euler mengetahui fungsi matematika biasa. Seperti yang Anda lihat di
atas, fungsi trigonometri bekerja dalam radian atau derajat. Untuk\\
mengonversi ke derajat, tambahkan simbol derajat (dengan tombol F7) ke
nilainya, atau gunakan fungsi rad(x). Fungsi akar kuadrat disebut\\
sqrt di Euler. Tentu saja, x\textasciicircum{}(1/2) juga dimungkinkan.

Untuk mensetting variabel, gunakan "=" atau ":=". Demi kejelasan,
pendahuluan ini menggunakan bentuk yang terakhir. Spasi tidak penting.\\
Tapi jarak antar perintah diharapkan.

Beberapa perintah dalam satu baris dipisahkan dengan "," atau ";".
Titik koma menekan keluaran perintah. Di akhir baris perintah, ","\\
diasumsikan, jika ";" hilang.
\end{eulercomment}
\begin{eulerprompt}
>g:=9.81; t:=2.5; 1/2*g*t^2
\end{eulerprompt}
\begin{euleroutput}
  30.65625
\end{euleroutput}
\begin{eulercomment}
EMT menggunakan sintaks pemrograman untuk ekspresi. Memasuki

\end{eulercomment}
\begin{eulerformula}
\[
e^2 \cdot \left( \frac{1}{3+4 \log(0.6)}+\frac{1}{7} \right)
\]
\end{eulerformula}
\begin{eulercomment}
Anda harus mengatur tanda kurung yang benar dan menggunakan / untuk
pecahan. Perhatikan tanda kurung yang disorot untuk mendapatkan
bantuan. Perhatikan bahwa konstanta Euler e diberi nama E dalam EMT.
\end{eulercomment}
\begin{eulerprompt}
>E^2*(1/(3+4*log(0.6))+1/7)
\end{eulerprompt}
\begin{euleroutput}
  8.77908249441
\end{euleroutput}
\begin{eulercomment}
Untuk menghitung ekspresi rumit seperti

\end{eulercomment}
\begin{eulerformula}
\[
\left(\frac{\frac17 + \frac18 + 2}{\frac13 + \frac12}\right)^2 \pi
\]
\end{eulerformula}
\begin{eulercomment}
Anda harus memasukkannya dalam formulir baris.
\end{eulercomment}
\begin{eulerprompt}
>((1/7 + 1/8 + 2) / (1/3 + 1/2))^2 * pi
\end{eulerprompt}
\begin{euleroutput}
  23.2671801626
\end{euleroutput}
\begin{eulercomment}
Letakkan tanda kurung dengan hati-hati di sekitar sub-ekspresi yang
perlu dihitung terlebih dahulu. EMT membantu Anda dengan\\
menyorot ekspresi yang mengakhiri tanda kurung tutup. Anda juga harus
memasukkan nama "pi" untuk huruf Yunani pi.

Hasil perhitungan ini berupa bilangan floating point. Ini secara
default dicetak dengan akurasi sekitar 12 digit. Di baris perintah
berikut, kita juga mempelajari bagaimana kita bisa merujuk ke hasil
sebelumnya dalam baris yang sama.
\end{eulercomment}
\begin{eulerprompt}
>1/3+1/7, fraction %
\end{eulerprompt}
\begin{euleroutput}
  0.47619047619
  10/21
\end{euleroutput}
\begin{eulercomment}
Perintah Euler dapat berupa ekspresi atau perintah primitif. Ekspresi
terbuat dari operator dan fungsi. Jika perlu, harus berisi tanda
kurung untuk memaksakan urutan eksekusi yang benar. Jika ragu,
memasang braket adalah ide yang bagus. Perhatikan bahwa EMT
menampilkan tanda kurung buka dan tutup saat mengedit baris perintah.
\end{eulercomment}
\begin{eulerprompt}
>(cos(pi/4)+1)^3*(sin(pi/4)+1)^2
\end{eulerprompt}
\begin{euleroutput}
  14.4978445072
\end{euleroutput}
\begin{eulercomment}
Operator numerik Euler meliputi

\end{eulercomment}
\begin{eulerttcomment}
 + unary atau operator plus
 - unary atau operator minus
 *, /
 .hasil kali matriks
 a^b pangkat untuk a positif atau bilangan bulat b(a**b juga
 berfungsi)
 n! operator faktorial
\end{eulerttcomment}
\begin{eulercomment}

dan masih banyak lagi

Berikut beberapa fungsi yang mungkin Anda perlukan. Berikut beberapa
fungsi yang mungkin Anda perlukan. Masih banyak lagi.

sin,cos,tan,atan,asin,acos,rad,deg log,exp,log10,sqrt,logbase\\
bin,logbin,logfac,mod,floor,ceil,round,abs,sign\\
conj,re,im,arg,conj , nyata, beta kompleks, betai, gamma, gamma
kompleks, ellrf, ellf,\\
ellrd, elle bitand, bitor, bitxor, bitnot.

Beberapa perintah mempunyai alias, misalnya ln untuk log.
\end{eulercomment}
\begin{eulerprompt}
>ln(E^2), arctan(tan(0.5))
\end{eulerprompt}
\begin{euleroutput}
  2
  0.5
\end{euleroutput}
\begin{eulerprompt}
>sin(30°)
\end{eulerprompt}
\begin{euleroutput}
  0.5
\end{euleroutput}
\begin{eulercomment}
Pastikan untuk menggunakan tanda kurung (tanda kurung bulat), setiap
kali ada keraguan tentang urutan eksekusi! Berikut ini tidak sama
dengan\\
(2\textasciicircum{}3)\textasciicircum{}4, yang merupakan default untuk 2\textasciicircum{}3\textasciicircum{}4 di EMT (beberapa sistem
numerik melakukannya dengan cara lain).

\end{eulercomment}
\begin{eulerprompt}
>2^3^4, (2^3)^4, 2^(3^4)
\end{eulerprompt}
\begin{euleroutput}
  2.41785163923e+24
  4096
  2.41785163923e+24
\end{euleroutput}
\begin{eulercomment}
Contoh dan Penyelesaian\\
1. Tentukan hasil dari operasi bilangan dibawah ini\\
\end{eulercomment}
\begin{eulerformula}
\[
\left(\frac{4(8-6)^2-4*3+2*8}{3^1+19^0}\right)
\]
\end{eulerformula}
\begin{eulerttcomment}
   Penyelesaian:
\end{eulerttcomment}
\begin{eulerprompt}
>$&(4*(8-6)^2-4*3+2*8)/(3^1+19^0)
\end{eulerprompt}
\begin{eulerformula}
\[
5
\]
\end{eulerformula}
\begin{eulerttcomment}
   Jadi hasil dari operasi bilangan diatas adalah 5
\end{eulerttcomment}
\begin{eulercomment}
2. Tentukan hasil dari operasi bilangan berikut\\
\end{eulercomment}
\begin{eulerformula}
\[
\left(\frac{[4(8-6)^2+4](3-2*8)}{2^2(2^3+5)}\right)
\]
\end{eulerformula}
\begin{eulerttcomment}
   Penyelesaian:
\end{eulerttcomment}
\begin{eulerprompt}
>$&((4*(8-6)^2+4)*(3-2*8))/(2^2*(2^3+5))
\end{eulerprompt}
\begin{eulerformula}
\[
-5
\]
\end{eulerformula}
\begin{eulerttcomment}
   Jadi hasil dari operasi diatas adalah -5
\end{eulerttcomment}
\begin{eulercomment}
3. Tentukan nilai dari log 0,01 + log 1000\\
\end{eulercomment}
\begin{eulerttcomment}
   Penyelesaian:
\end{eulerttcomment}
\begin{eulerprompt}
>log(0.01)+log(1000)
\end{eulerprompt}
\begin{euleroutput}
  2.30258509299
\end{euleroutput}
\begin{eulercomment}
Jadi nilai penjumlahan logaritma diatas adalah 2.30

4.Tentukan nilai dari sin\textasciicircum{}(-1\}(sin(7pi/6))\\
\end{eulercomment}
\begin{eulerttcomment}
  Penyelesaian:
\end{eulerttcomment}
\begin{eulerprompt}
>arcsin(sin(7*pi/6))
\end{eulerprompt}
\begin{euleroutput}
  -0.523598775598
\end{euleroutput}
\eulersubheading{Bilangan Real}
\begin{eulercomment}
Tipe data primer pada Euler adalah bilangan real. Real
direpresentasikan dalam format IEEE dengan akurasi sekitar 16 digit
desimal.
\end{eulercomment}
\begin{eulerprompt}
>longest 1/3
\end{eulerprompt}
\begin{euleroutput}
       0.3333333333333333 
\end{euleroutput}
\begin{eulercomment}
Representasi ganda internal membutuhkan 8 byte.
\end{eulercomment}
\begin{eulerprompt}
>printdual(1/3)
\end{eulerprompt}
\begin{euleroutput}
  1.0101010101010101010101010101010101010101010101010101*2^-2
\end{euleroutput}
\begin{eulerprompt}
>printhex(1/3)
\end{eulerprompt}
\begin{euleroutput}
  5.5555555555554*16^-1
\end{euleroutput}
\eulersubheading{String}
\begin{eulercomment}
Sebuah string di Euler didefinisikan dengan "...".
\end{eulercomment}
\begin{eulerprompt}
>"A string can contain anything."
\end{eulerprompt}
\begin{euleroutput}
  A string can contain anything.
\end{euleroutput}
\begin{eulercomment}
String dapat digabungkan dengan \textbar{} atau dengan +. Ini juga berfungsi
dengan angka, yang dalam hal ini diubah menjadi string.
\end{eulercomment}
\begin{eulerprompt}
>"The area of the circle with radius " + 2 + " cm is " + pi*4 + " cm^2."
\end{eulerprompt}
\begin{euleroutput}
  The area of the circle with radius 2 cm is 12.5663706144 cm^2.
\end{euleroutput}
\begin{eulercomment}
Fungsi print juga mengubah angka menjadi string. Ini dapat memerlukan
sejumlah digit dan sejumlah tempat (0 untuk keluaran padat), dan\\
optimalnya satuan.\\
\end{eulercomment}
\begin{eulerttcomment}
 
\end{eulerttcomment}
\begin{eulerprompt}
>"Golden Ratio : " + print((1+sqrt(5))/2,5,0)
\end{eulerprompt}
\begin{euleroutput}
  Golden Ratio : 1.61803
\end{euleroutput}
\begin{eulercomment}
Ada string khusus none yang tidak dicetak. Itu dikembalikan oleh
beberapa fungsi, ketika hasilnya tidak menjadi masalah. (Ini
dikembalikan secara otomatis, jika fungsi tidak memiliki pernyataan
return.)
\end{eulercomment}
\begin{eulerprompt}
>none
\end{eulerprompt}
\begin{eulercomment}
Untuk mengonversi string menjadi angka, cukup evaluasi saja. Ini juga
berfungsi untuk ekspresi (lihat di bawah).
\end{eulercomment}
\begin{eulerprompt}
>"1234.5"()
\end{eulerprompt}
\begin{euleroutput}
  1234.5
\end{euleroutput}
\begin{eulercomment}
Untuk mendefinisikan vektor string, gunakan notasi vektor [...].
\end{eulercomment}
\begin{eulerprompt}
>v:=["affe","charlie","bravo"]
\end{eulerprompt}
\begin{euleroutput}
  affe
  charlie
  bravo
\end{euleroutput}
\begin{eulercomment}
Vektor string kosong dilambangkan dengan [none]. Vektor string dapat
digabungkan.
\end{eulercomment}
\begin{eulerprompt}
>w:=[none]; w|v|v
\end{eulerprompt}
\begin{euleroutput}
  affe
  charlie
  bravo
  affe
  charlie
  bravo
\end{euleroutput}
\begin{eulercomment}
String dapat berisi karakter Unicode. Secara internal, string ini
berisi kode UTF-8. Untuk menghasilkan string seperti itu, gunakan
u"..." dan salah satu entitas HTML

String Unicode dapat digabungkan seperti string lainnya.
\end{eulercomment}
\begin{eulerprompt}
>u"&alpha; = " + 45 + u"&deg;" // pdfLaTeX mungkin gagal menampilkan secara benar
\end{eulerprompt}
\begin{euleroutput}
  α = 45°
\end{euleroutput}
\begin{eulercomment}
I
\end{eulercomment}
\begin{eulercomment}
Di komentar, entitas yang sama seperti α, β, dll, dapat
digunakan. Ini mungkin merupakan alternatif cepat untuk
Lateks(Keterangan lebih lanjut pada komentar di bawah)
\end{eulercomment}
\begin{eulercomment}
Ada beberapa fungsi untuk membuat atau menganalisis string unicode.
Fungsi strtochar() akan mengenali string Unicode dan menerjemahkannya
dengan benar.
\end{eulercomment}
\begin{eulerprompt}
>v=strtochar(u"&Auml; is a German letter")
\end{eulerprompt}
\begin{euleroutput}
  [196,  32,  105,  115,  32,  97,  32,  71,  101,  114,  109,  97,  110,
  32,  108,  101,  116,  116,  101,  114]
\end{euleroutput}
\begin{eulercomment}
Hasilnya adalah vektor angka Unicode. Fungsi kebalikannya adalah
chartoutf().
\end{eulercomment}
\begin{eulerprompt}
>v[1]=strtochar(u"&Uuml;")[1]; chartoutf(v)
\end{eulerprompt}
\begin{euleroutput}
  Ü is a German letter
\end{euleroutput}
\begin{eulercomment}
Fungsi utf() dapat menerjemahkan string dengan entitas dalam variabel
menjadi string Unicode.
\end{eulercomment}
\begin{eulerprompt}
>s="We have &alpha;=&beta;."; utf(s) // pdfLaTeX mungkin gagal menampilkan secara benar
\end{eulerprompt}
\begin{euleroutput}
  We have α=β.
\end{euleroutput}
\begin{eulercomment}
Dimungkinkan juga untuk menggunakan entitas numerik.
\end{eulercomment}
\begin{eulerprompt}
>u"&#196;hnliches"
\end{eulerprompt}
\begin{euleroutput}
  Ähnliches
\end{euleroutput}
\eulersubheading{Nilai Boolean}
\begin{eulercomment}
Nilai Boolean diwakili dengan 1=true atau 0=false di Euler. String
dapat dibandingkan, seperti halnya angka.
\end{eulercomment}
\begin{eulerprompt}
>2<1, "apel"<"banana"
\end{eulerprompt}
\begin{euleroutput}
  0
  1
\end{euleroutput}
\begin{eulercomment}
"and" adalah operator "\&\&" dan "or" adalah operator "\textbar{}\textbar{}", seperti
dalam bahasa C. (Kata "and" dan "or" hanya dapat digunakan dalam \\
kondisi "if".)
\end{eulercomment}
\begin{eulerprompt}
>2<E && E<3
\end{eulerprompt}
\begin{euleroutput}
  1
\end{euleroutput}
\begin{eulercomment}
Operator Boolean mematuhi aturan bahasa matriks.
\end{eulercomment}
\begin{eulerprompt}
>(1:10)>5, nonzeros(%)
\end{eulerprompt}
\begin{euleroutput}
  [0,  0,  0,  0,  0,  1,  1,  1,  1,  1]
  [6,  7,  8,  9,  10]
\end{euleroutput}
\begin{eulercomment}
Anda dapat menggunakan fungsi nonzeros() untuk mengekstrak elemen
tertentu dari vektor. Dalam contoh ini, kita menggunakan kondisi
isprime(n).
\end{eulercomment}
\begin{eulerprompt}
>N=2|3:2:99 // N berisi elemen 2 dan bilangan2 ganjil dari 3 s.d. 99
\end{eulerprompt}
\begin{euleroutput}
  [2,  3,  5,  7,  9,  11,  13,  15,  17,  19,  21,  23,  25,  27,  29,
  31,  33,  35,  37,  39,  41,  43,  45,  47,  49,  51,  53,  55,  57,
  59,  61,  63,  65,  67,  69,  71,  73,  75,  77,  79,  81,  83,  85,
  87,  89,  91,  93,  95,  97,  99]
\end{euleroutput}
\begin{eulerprompt}
>N[nonzeros(isprime(N))] //pilih anggota2 N yang prima
\end{eulerprompt}
\begin{euleroutput}
  [2,  3,  5,  7,  11,  13,  17,  19,  23,  29,  31,  37,  41,  43,  47,
  53,  59,  61,  67,  71,  73,  79,  83,  89,  97]
\end{euleroutput}
\eulersubheading{Format Keluaran}
\begin{eulercomment}
Format keluaran default EMT mencetak 12 digit. Untuk memastikan bahwa
kami melihat defaultnya, kami mengatur ulang formatnya.
\end{eulercomment}
\begin{eulerprompt}
>defformat; pi
\end{eulerprompt}
\begin{euleroutput}
  3.14159265359
\end{euleroutput}
\begin{eulercomment}
Secara internal, EMT menggunakan standar IEEE untuk bilangan ganda
dengan sekitar 16 digit desimal. Untuk melihat jumlah digit secara
lengkap gunakan perintah "longestformat", atau kita gunakan operator
"longest" untuk menampilkan hasilnya dalam format terpanjang.
\end{eulercomment}
\begin{eulerprompt}
>longest pi
\end{eulerprompt}
\begin{euleroutput}
        3.141592653589793 
\end{euleroutput}
\begin{eulercomment}
Berikut adalah representasi heksadesimal internal dari bilangan ganda.
\end{eulercomment}
\begin{eulerprompt}
>printhex(pi)
\end{eulerprompt}
\begin{euleroutput}
  3.243F6A8885A30*16^0
\end{euleroutput}
\begin{eulercomment}
Format keluaran dapat diubah secara permanen dengan perintah format.
\end{eulercomment}
\begin{eulerprompt}
>format(12,5); 1/3, pi, sin(1)
\end{eulerprompt}
\begin{euleroutput}
      0.33333 
      3.14159 
      0.84147 
\end{euleroutput}
\begin{eulercomment}
Standarnya adalah format(12).
\end{eulercomment}
\begin{eulerprompt}
>format(12); 1/3
\end{eulerprompt}
\begin{euleroutput}
  0.333333333333
\end{euleroutput}
\begin{eulercomment}
Fungsi seperti "shortestformat", "shortformat", "longformat" berfungsi
untuk vektor dengan cara berikut.
\end{eulercomment}
\begin{eulerprompt}
>shortestformat; random(3,8)
\end{eulerprompt}
\begin{euleroutput}
    0.66    0.2   0.89   0.28   0.53   0.31   0.44    0.3 
    0.28   0.88   0.27    0.7   0.22   0.45   0.31   0.91 
    0.19   0.46  0.095    0.6   0.43   0.73   0.47   0.32 
\end{euleroutput}
\begin{eulercomment}
Format default untuk skalar adalah format(12). Tapi ini bisa diubah.
\end{eulercomment}
\begin{eulerprompt}
>setscalarformat(5); pi
\end{eulerprompt}
\begin{euleroutput}
  3.1416
\end{euleroutput}
\begin{eulercomment}
Fungsi"longestformat" juga mengatur format skala.
\end{eulercomment}
\begin{eulerprompt}
>longestformat; pi
\end{eulerprompt}
\begin{euleroutput}
  3.141592653589793
\end{euleroutput}
\begin{eulercomment}
Sebagai referensi, berikut adalah daftar format keluaran terpenting.

\end{eulercomment}
\begin{eulerttcomment}
 shortestformat shortformat longformat, longestformat
 format(length,digits) goodformat(length)
 fracformat(length)
 defformat
\end{eulerttcomment}
\begin{eulercomment}

Akurasi internal EMT adalah sekitar 16 tempat desimal, yang merupakan
standar IEEE. Nomor disimpan dalam format internal ini.

Namun format keluaran EMT dapat diatur dengan cara yang fleksibel.
\end{eulercomment}
\begin{eulerprompt}
>longestformat; pi,
\end{eulerprompt}
\begin{euleroutput}
  3.141592653589793
\end{euleroutput}
\begin{eulerprompt}
>format(10,5); pi
\end{eulerprompt}
\begin{euleroutput}
    3.14159 
\end{euleroutput}
\begin{eulercomment}
Defaultnya adalah defformat().
\end{eulercomment}
\begin{eulerprompt}
>defformat; // default
\end{eulerprompt}
\begin{eulercomment}
Ada operator pendek yang hanya mencetak satu nilai. Operator
"longest(terpanjang)" akan mencetak semua digit nomor yang valid.
\end{eulercomment}
\begin{eulerprompt}
>longest pi^2/2
\end{eulerprompt}
\begin{euleroutput}
        4.934802200544679 
\end{euleroutput}
\begin{eulercomment}
Ada juga operator pendek untuk mencetak hasil dalam format pecahan.
Kami sudah menggunakannya di atas.
\end{eulercomment}
\begin{eulerprompt}
>fraction 1+1/2+1/3+1/4
\end{eulerprompt}
\begin{euleroutput}
  25/12
\end{euleroutput}
\begin{eulercomment}
Karena format internal menggunakan cara biner untuk menyimpan angka,
nilai 0,1 tidak akan direpresentasikan secara tepat. Kesalahannya
bertambah sedikit, seperti yang Anda lihat pada perhitungan berikut.\\
Akurasi internal EMT adalah sekitar 16 tempat desimal, yang merupakan
standar IEEE. Nomor disimpan dalam format internal ini.
\end{eulercomment}
\begin{eulerprompt}
>longest 0.1+0.1+0.1+0.1+0.1+0.1+0.1+0.1+0.1+0.1-1
\end{eulerprompt}
\begin{euleroutput}
   -1.110223024625157e-16 
\end{euleroutput}
\begin{eulercomment}
Tetapi dengan "longformat" default Anda tidak akan menyadarinya. Untuk
kenyamanan, keluaran angka yang sangat kecil adalah 0.
\end{eulercomment}
\begin{eulerprompt}
>0.1+0.1+0.1+0.1+0.1+0.1+0.1+0.1+0.1+0.1-1
\end{eulerprompt}
\begin{euleroutput}
  0
\end{euleroutput}
\eulerheading{Ekspresi}
\begin{eulercomment}
String atau nama dapat digunakan untuk menyimpan ekspresi matematika,
yang dapat dievaluasi dengan EMT. Untuk ini, gunakan tanda kurung\\
setelah ekspresi. Jika Anda ingin menggunakan string sebagai ekspresi,
gunakan konvensi untuk menamainya "fx" atau "fxy" dll. Ekspresi lebih\\
diutamakan daripada fungsi.

Variabel global dapat digunakan dalam evaluasi
\end{eulercomment}
\begin{eulerprompt}
>r:=2; fx:="pi*r^2"; longest fx()
\end{eulerprompt}
\begin{euleroutput}
        12.56637061435917 
\end{euleroutput}
\begin{eulercomment}
Parameter ditetapkan ke x, y, dan z dalam urutan itu. Parameter
tambahan dapat ditambahkan menggunakan parameter yang ditetapkan.
\end{eulercomment}
\begin{eulerprompt}
>fx:="a*sin(x)^2"; fx(5,a=-1)
\end{eulerprompt}
\begin{euleroutput}
  -0.919535764538
\end{euleroutput}
\begin{eulercomment}
Perhatikan bahwa ekspresi akan selalu menggunakan variabel global,
meskipun ada variabel dalam fungsi dengan nama yang sama. (Jika\\
tidak, evaluasi ekspresi dalam fungsi dapat memberikan hasil yang
sangat membingungkan bagi pengguna yang memanggil fungsi tersebut).
\end{eulercomment}
\begin{eulerprompt}
>at:=4; function f(expr,x,at) := expr(x); ...
>f("at*x^2",3,5) // computes 4*3^2 not 5*3^2
\end{eulerprompt}
\begin{euleroutput}
  36
\end{euleroutput}
\begin{eulercomment}
Jika Anda ingin menggunakan nilai lain untuk "at" selain nilai global,
Anda perlu menambahkan "at=value".
\end{eulercomment}
\begin{eulerprompt}
>at:=4; function f(expr,x,a) := expr(x,at=a); ...
>f("at*x^2",3,5)
\end{eulerprompt}
\begin{euleroutput}
  45
\end{euleroutput}
\begin{eulercomment}
Sebagai referensi, kami mencatat bahwa koleksi panggilan (dibahas di
tempat lain) dapat berisi ekspresi. Jadi kita bisa membuat contoh \\
di atas sebagai berikut.
\end{eulercomment}
\begin{eulerprompt}
>at:=4; function f(expr,x) := expr(x); ...
>f(\{\{"at*x^2",at=5\}\},3)
\end{eulerprompt}
\begin{euleroutput}
  45
\end{euleroutput}
\begin{eulercomment}
Ekspresi dalam x sering digunakan seperti halnya fungsi.\\
Perhatikan bahwa mendefinisikan fungsi dengan nama yang sama seperti
ekspresi simbolik global akan menghapus variabel ini untuk\\
menghindari kebingungan antara ekspresi simbolik dan fungsi.
\end{eulercomment}
\begin{eulerprompt}
>f &= 5*x;
>function f(x) := 6*x;
>f(2)
\end{eulerprompt}
\begin{euleroutput}
  12
\end{euleroutput}
\begin{eulercomment}
Berdasarkan konvensi, ekspresi simbolik atau numerik harus diberi nama
fx, fxy dll. Skema penamaan ini tidak boleh digunakan untuk fungsi.
\end{eulercomment}
\begin{eulerprompt}
>fx &= diff(x^x,x); $&fx
\end{eulerprompt}
\begin{eulerformula}
\[
x^{x}\,\left(\log x+1\right)
\]
\end{eulerformula}
\begin{eulercomment}
Bentuk ekspresi khusus memungkinkan variabel apa pun sebagai parameter
yang tidak disebutkan namanya untuk mengevaluasi ekspresi, bukan hanya
"x", "y", dll. Untuk ini, mulailah ekspresi dengan "@(variabel) ...".
\end{eulercomment}
\begin{eulerprompt}
>"@(a,b) a^2+b^2", %(4,5)
\end{eulerprompt}
\begin{euleroutput}
  @(a,b) a^2+b^2
  41
\end{euleroutput}
\begin{eulercomment}
Hal ini memungkinkan untuk memanipulasi ekspresi dalam variabel lain
untuk fungsi EMT yang memerlukan ekspresi dalam "x".

Cara paling dasar untuk mendefinisikan suatu fungsi sederhana adalah
dengan menyimpan rumusnya dalam ekspresi simbolik atau numerik. Jika
variabel utamanya adalah x, ekspresi dapat dievaluasi seperti halnya
fungsi.

Seperti yang Anda lihat pada contoh berikut, variabel global terlihat
selama evaluasi.
\end{eulercomment}
\begin{eulerprompt}
>fx &= x^3-a*x;  ...
>a=1.2; fx(0.5)
\end{eulerprompt}
\begin{euleroutput}
  -0.475
\end{euleroutput}
\begin{eulercomment}
Semua variabel lain dalam ekspresi dapat ditentukan dalam evaluasi
menggunakan parameter yang ditetapkan.
\end{eulercomment}
\begin{eulerprompt}
>fx(0.5,a=1.1)
\end{eulerprompt}
\begin{euleroutput}
  -0.425
\end{euleroutput}
\begin{eulercomment}
Sebuah ekspresi tidak perlu bersifat simbolis. Hal ini diperlukan,
jika ekspresi berisi fungsi, yang hanya diketahui di kernel numerik,\\
bukan di Maxima

\begin{eulercomment}
\eulerheading{Matematika Simbolik}
\begin{eulercomment}
EMT melakukan matematika simbolis dengan bantuan Maxima. Untuk
detailnya, mulailah dengan tutorial berikut, atau telusuri referensi\\
untuk Maxima. Para ahli di Maxima harus memperhatikan bahwa ada
perbedaan sintaksis antara sintaksis asli Maxima dan sintaksis\\
default ekspresi simbolik di EMT.

Matematika simbolik diintegrasikan ke dalam Euler dengan \&. Ekspresi
apa pun yang dimulai dengan \& adalah ekspresi simbolis. Itu\\
dievaluasi dan dicetak oleh Maxima.

Pertama-tama, Maxima memiliki aritmatika "tak terbatas" yang dapat
menangani bilangan yang sangat besar
\end{eulercomment}
\begin{eulerprompt}
>$&44!
\end{eulerprompt}
\begin{eulerformula}
\[
2658271574788448768043625811014615890319638528000000000
\]
\end{eulerformula}
\begin{eulercomment}
Dengan cara ini, Anda dapat menghitung hasil yang besar dengan tepat.
Mari kita menghitung.\\
\end{eulercomment}
\begin{eulerformula}
\[
C(44,10) = \frac{44!}{34! \cdot 10!}
\]
\end{eulerformula}
\begin{eulerprompt}
>$& 44!/(34!*10!) // nilai C(44,10)
\end{eulerprompt}
\begin{eulerformula}
\[
2481256778
\]
\end{eulerformula}
\begin{eulercomment}
Tentu saja, Maxima memiliki fungsi yang lebih efisien untuk
ini(seperti halnya bagian numerik EMT).
\end{eulercomment}
\begin{eulerprompt}
>$binomial(44,10) //menghitung C(44,10) menggunakan fungsi binomial()
\end{eulerprompt}
\begin{eulerformula}
\[
2481256778
\]
\end{eulerformula}
\begin{eulercomment}
Untuk mempelajari lebih lanjut tentang fungsi tertentu, klik dua kali.
Misalnya, mencoba klik dua kali pada "\&binomial" di baris \\
perintah sebelumnya. Ini membuka dokumentasi Maxima yang disediakan
oleh penulis program tersebut.

Anda akan mengetahui bahwa cara berikut juga bisa dilakukan.

\end{eulercomment}
\begin{eulerformula}
\[
C(x,3)=\frac{x!}{(x-3)!3!}=\frac{(x-2)(x-1)x}{6}
\]
\end{eulerformula}
\begin{eulerprompt}
>$binomial(x,3) // C(x,3)
\end{eulerprompt}
\begin{eulerformula}
\[
\frac{\left(x-2\right)\,\left(x-1\right)\,x}{6}
\]
\end{eulerformula}
\begin{eulercomment}
Jika Anda ingin mengganti x dengan nilai tertentu, gunakan "with".
\end{eulercomment}
\begin{eulerprompt}
>$&binomial(x,3) with x=10 // substitusi x=10 ke C(x,3)
\end{eulerprompt}
\begin{eulerformula}
\[
120
\]
\end{eulerformula}
\begin{eulercomment}
Dengan begitu Anda bisa menggunakan solusi suatu persamaan di
persamaan lain.

Ekspresi simbolik dicetak oleh Maxima dalam bentuk 2D. Alasannya
adalah adanya tanda simbolis khusus pada string tersebut.\\
Seperti yang telah Anda lihat pada contoh sebelumnya dan contoh
berikut, jika Anda telah menginstal LaTeX, Anda dapat mencetak
ekspresi\\
simbolik dengan Latex. Jika tidak, perintah berikut akan mengeluarkan
pesan kesalahan.\\
Untuk mencetak ekspresi simbolik dengan LaTeX, gunakan \textdollar{} di depan \&
(atau Anda dapat menghilangkan \&) sebelum perintah. Jangan jalankan
perintah Maxima dengan \textdollar{}, jika Anda belum menginstal LaTeX.
\end{eulercomment}
\begin{eulerprompt}
>$(3+x)/(x^2+1)
\end{eulerprompt}
\begin{eulerformula}
\[
\frac{x+3}{x^2+1}
\]
\end{eulerformula}
\begin{eulercomment}
Ekspresi simbolik diurai oleh Euler. Jika Anda memerlukan sintaksis
kompleks dalam satu ekspresi, Anda dapat mengapit ekspresi\\
tersebut di "...". Menggunakan lebih dari sekadar ekspresi sederhana
bisa saja dilakukan, namun sangat tidak disarankan.
\end{eulercomment}
\begin{eulerprompt}
>&"v := 5; v^2"
\end{eulerprompt}
\begin{euleroutput}
  
                                    25
  
\end{euleroutput}
\begin{eulercomment}
Untuk kelengkapannya, kami mencatat bahwa ekspresi simbolik dapat
digunakan dalam program, namun perlu diapit dalam tanda \\
kutip. Selain itu, akan jauh lebih efektif untuk memanggil Maxima pada
waktu kompilasi jika memungkinkan.
\end{eulercomment}
\begin{eulerprompt}
>$&expand((1+x)^4), $&factor(diff(%,x)) // diff: turunan, factor: faktor
\end{eulerprompt}
\begin{eulerformula}
\[
x^4+4\,x^3+6\,x^2+4\,x+1
\]
\end{eulerformula}
\begin{eulerformula}
\[
4\,\left(x+1\right)^3
\]
\end{eulerformula}
\begin{eulercomment}
Sekali lagi, \% mengacu pada hasil sebelumnya.

Untuk mempermudah kami menyimpan solusi ke variabel simbolik. Variabel
simbolik didefinisikan dengan "\&=".
\end{eulercomment}
\begin{eulerprompt}
>fx &= (x+1)/(x^4+1); $&fx
\end{eulerprompt}
\begin{eulerformula}
\[
\frac{x+1}{x^4+1}
\]
\end{eulerformula}
\begin{eulercomment}
Ekspresi simbolik dapat digunakan dalam ekspresi simbolik lainnya.
\end{eulercomment}
\begin{eulerprompt}
>$&factor(diff(fx,x))
\end{eulerprompt}
\begin{eulerformula}
\[
\frac{-3\,x^4-4\,x^3+1}{\left(x^4+1\right)^2}
\]
\end{eulerformula}
\begin{eulercomment}
Input langsung dari perintah Maxima juga tersedia. Mulai baris
perintah dengan "::". Sintaks Maxima disesuaikan dengan sintaks
EMT(disebut "mode kompatibilitas").
\end{eulercomment}
\begin{eulerprompt}
>&factor(20!)
\end{eulerprompt}
\begin{euleroutput}
  
                           2432902008176640000
  
\end{euleroutput}
\begin{eulerprompt}
>::: factor(10!)
\end{eulerprompt}
\begin{euleroutput}
  
                                 8  4  2
                                2  3  5  7
  
\end{euleroutput}
\begin{eulerprompt}
>:: factor(20!)
\end{eulerprompt}
\begin{euleroutput}
  
                          18  8  4  2
                         2   3  5  7  11 13 17 19
  
\end{euleroutput}
\begin{eulercomment}
Jika Anda ahli dalam Maxima, Anda mungkin ingin menggunakan sintaks
asli Maxima. Anda dapat melakukan ini dengan ":::".
\end{eulercomment}
\begin{eulerprompt}
>::: av:g$ av^2;
\end{eulerprompt}
\begin{euleroutput}
  
                                     2
                                    g
  
\end{euleroutput}
\begin{eulerprompt}
>fx &= x^3*exp(x), $fx
\end{eulerprompt}
\begin{euleroutput}
  
                                   3  x
                                  x  E
  
\end{euleroutput}
\begin{eulerformula}
\[
x^3\,e^{x}
\]
\end{eulerformula}
\begin{eulercomment}
Variabel tersebut dapat digunakan dalam ekspresi simbolik lainnya.
Perhatikan, bahwa dalam perintah berikut sisi kanan \&= dievaluasi\\
sebelum ditugaskan ke Fx.
\end{eulercomment}
\begin{eulerprompt}
>&(fx with x=5), $%, &float(%)
\end{eulerprompt}
\begin{euleroutput}
  
                                       5
                                  125 E
  
\end{euleroutput}
\begin{eulerformula}
\[
125\,e^5
\]
\end{eulerformula}
\begin{euleroutput}
  
                            18551.64488782208
  
\end{euleroutput}
\begin{eulerprompt}
>fx(5)
\end{eulerprompt}
\begin{euleroutput}
  18551.6448878
\end{euleroutput}
\begin{eulercomment}
Untuk mengevaluasi ekspresi dengan nilai variabel tertentu, Anda dapat
menggunakan operator "with".

Baris perintah berikut juga menunjukkan bahwa Maxima dapat
mengevaluasi ekspresi secara numerik dengan float().
\end{eulercomment}
\begin{eulerprompt}
>&(fx with x=10)-(fx with x=5), &float(%)
\end{eulerprompt}
\begin{euleroutput}
  
                                  10        5
                            1000 E   - 125 E
  
  
                           2.20079141499189e+7
  
\end{euleroutput}
\begin{eulerprompt}
>$factor(diff(fx,x,2))
\end{eulerprompt}
\begin{eulerformula}
\[
x\,\left(x^2+6\,x+6\right)\,e^{x}
\]
\end{eulerformula}
\begin{eulercomment}
Untuk mendapatkan kode Lateks untuk sebuah ekspresi, Anda dapat
menggunakan perintah tex.
\end{eulercomment}
\begin{eulerprompt}
>tex(fx)
\end{eulerprompt}
\begin{euleroutput}
  x^3\(\backslash\),e^\{x\}
\end{euleroutput}
\begin{eulercomment}
Ekspresi simbolik dapat dievaluasi seperti halnya ekspresi numerik.
\end{eulercomment}
\begin{eulerprompt}
>fx(0.5)
\end{eulerprompt}
\begin{euleroutput}
  0.206090158838
\end{euleroutput}
\begin{eulercomment}
Dalam ekspresi simbolis, ini tidak berhasil, karena Maxima tidak
mendukungnya. Sebagai gantinya, gunakan sintaks "with" (bentuk \\
perintah at(...) yang lebih bagus dari Maxima).
\end{eulercomment}
\begin{eulerprompt}
>$&fx with x=1/2
\end{eulerprompt}
\begin{eulerformula}
\[
\frac{\sqrt{e}}{8}
\]
\end{eulerformula}
\begin{eulercomment}
Penugasannya juga bisa bersifat simbolis
\end{eulercomment}
\begin{eulerprompt}
>$&fx with x=1+t
\end{eulerprompt}
\begin{eulerformula}
\[
\left(t+1\right)^3\,e^{t+1}
\]
\end{eulerformula}
\begin{eulercomment}
Perintah solve menyelesaikan ekspresi simbolik untuk variabel di
Maxima. Hasilnya adalah vektor solusi.
\end{eulercomment}
\begin{eulerprompt}
>$&solve(x^2+x=4,x)
\end{eulerprompt}
\begin{eulerformula}
\[
\left[ x=\frac{-\sqrt{17}-1}{2} , x=\frac{\sqrt{17}-1}{2} \right] 
\]
\end{eulerformula}
\begin{eulercomment}
Bandingkan dengan perintah numerik "solve" di Euler, yang memerlukan
nilai awal, dan opsional nilai target.
\end{eulercomment}
\begin{eulerprompt}
>solve("x^2+x",1,y=4)
\end{eulerprompt}
\begin{euleroutput}
  1.56155281281
\end{euleroutput}
\begin{eulercomment}
Nilai numerik dari solusi simbolik dapat dihitung dengan evaluasi
hasil simbolik. Euler akan membacakan tugas x= dst. Jika Anda tidak \\
memerlukan hasil numerik untuk perhitungan lebih lanjut, Anda juga
dapat membiarkan Maxima menemukan nilai numeriknya.
\end{eulercomment}
\begin{eulerprompt}
>sol &= solve(x^2+2*x=4,x); $&sol, sol(), $&float(sol)
\end{eulerprompt}
\begin{eulerformula}
\[
\left[ x=-\sqrt{5}-1 , x=\sqrt{5}-1 \right] 
\]
\end{eulerformula}
\begin{euleroutput}
  [-3.23607,  1.23607]
\end{euleroutput}
\begin{eulerformula}
\[
\left[ x=-3.23606797749979 , x=1.23606797749979 \right] 
\]
\end{eulerformula}
\begin{eulercomment}
Untuk mendapatkan solusi simbolik tertentu, seseorang dapat
menggunakan "with" dan index.
\end{eulercomment}
\begin{eulerprompt}
>$&solve(x^2+x=1,x), x2 &= x with %[2]; $&x2
\end{eulerprompt}
\begin{eulerformula}
\[
\left[ x=\frac{-\sqrt{5}-1}{2} , x=\frac{\sqrt{5}-1}{2} \right] 
\]
\end{eulerformula}
\begin{eulerformula}
\[
\frac{\sqrt{5}-1}{2}
\]
\end{eulerformula}
\begin{eulercomment}
Untuk menyelesaikan sistem persamaan, gunakan vektor persamaan.
Hasilnya adalah vektor solusi.
\end{eulercomment}
\begin{eulerprompt}
>sol &= solve([x+y=3,x^2+y^2=5],[x,y]); $&sol, $&x*y with sol[1]
\end{eulerprompt}
\begin{eulerformula}
\[
\left[ \left[ x=2 , y=1 \right]  , \left[ x=1 , y=2 \right] 
  \right] 
\]
\end{eulerformula}
\begin{eulerformula}
\[
2
\]
\end{eulerformula}
\begin{eulercomment}
Ekspresi simbolis dapat memiliki bendera, yang menunjukkan perlakuan
khusus di Maxima. Beberapa flag dapat digunakan sebagai perintah juga,
yang lainnya tidak. Bendera ditambahkan dengan "\textbar{}" (bentuk yang lebih
bagus dari "ev(...,flags)"
\end{eulercomment}
\begin{eulerprompt}
>$& diff((x^3-1)/(x+1),x) //turunan bentuk pecahan
\end{eulerprompt}
\begin{eulerformula}
\[
\frac{3\,x^2}{x+1}-\frac{x^3-1}{\left(x+1\right)^2}
\]
\end{eulerformula}
\begin{eulerprompt}
>$& diff((x^3-1)/(x+1),x) | ratsimp //menyederhanakan pecahan
\end{eulerprompt}
\begin{eulerformula}
\[
\frac{2\,x^3+3\,x^2+1}{x^2+2\,x+1}
\]
\end{eulerformula}
\begin{eulerprompt}
>$&factor(%)
\end{eulerprompt}
\begin{eulerformula}
\[
\frac{2\,x^3+3\,x^2+1}{\left(x+1\right)^2}
\]
\end{eulerformula}
\eulerheading{Fungsi}
\begin{eulercomment}
Dalam EMT, fungsi adalah program yang didefinisikan dengan perintah
"fungsi". Ini bisa berupa fungsi satu baris atau fungsi multibaris.

ungsi satu baris dapat berupa numerik atau simbolik. Fungsi satu baris
numerik didefinisikan oleh ":=".
\end{eulercomment}
\begin{eulerprompt}
>function f(x) := x*sqrt(x^2+1)
\end{eulerprompt}
\begin{eulercomment}
Untuk gambaran umum, kami menunjukkan semua kemungkinan definisi untuk
fungsi satu baris. Suatu fungsi dapat dievaluasi sama \\
seperti fungsi Euler bawaan lainnya.
\end{eulercomment}
\begin{eulerprompt}
>f(2)
\end{eulerprompt}
\begin{euleroutput}
  4.472135955
\end{euleroutput}
\begin{eulercomment}
Fungsi ini juga dapat digunakan untuk vektor, mengikuti bahasa matriks
Euler, karena ekspresi yang digunakan dalam fungsi tersebut
divektorkan.
\end{eulercomment}
\begin{eulerprompt}
>f(0:0.1:1)
\end{eulerprompt}
\begin{euleroutput}
  [0,  0.100499,  0.203961,  0.313209,  0.430813,  0.559017,  0.699714,
  0.854459,  1.0245,  1.21083,  1.41421]
\end{euleroutput}
\begin{eulercomment}
Fungsi dapat diplot. Daripada ekspresi, kita hanya perlu memberikan
nama fungsinya.

erbeda dengan ekspresi simbolik atau numerik, nama fungsi harus
diberikan dalam string.
\end{eulercomment}
\begin{eulerprompt}
>solve("f",1,y=1)
\end{eulerprompt}
\begin{euleroutput}
  0.786151377757
\end{euleroutput}
\begin{eulercomment}
Secara default, jika Anda perlu menimpa fungsi bawaan, Anda harus
menambahkan kata kunci "timpa". Menimpa fungsi bawaan berbahaya dan \\
dapat menyebabkan masalah pada fungsi lain yang bergantung pada fungsi
tersebut.

Anda masih dapat memanggil fungsi bawaan sebagai "\_...", jika fungsi
tersebut ada di inti Euler.
\end{eulercomment}
\begin{eulerprompt}
>function overwrite sin (x) := _sin(x°) // redine sine in degrees
>sin(45)
\end{eulerprompt}
\begin{euleroutput}
  0.707106781187
\end{euleroutput}
\begin{eulercomment}
Sebaiknya kita menghilangkan definisi ulang sin
\end{eulercomment}
\begin{eulerprompt}
>forget sin; sin(pi/4)
\end{eulerprompt}
\begin{euleroutput}
  0.707106781187
\end{euleroutput}
\eulersubheading{Parameter Default}
\begin{eulercomment}
Fungsi numerik dapat memiliki parameter default
\end{eulercomment}
\begin{eulerprompt}
>function f(x,a=1) := a*x^2
\end{eulerprompt}
\begin{eulercomment}
Menghilangkan parameter ini akan menggunakan nilai default.
\end{eulercomment}
\begin{eulerprompt}
>f(4)
\end{eulerprompt}
\begin{euleroutput}
  16
\end{euleroutput}
\begin{eulercomment}
Mengaturnya akan menimpa nilai default
\end{eulercomment}
\begin{eulerprompt}
>f(4,5)
\end{eulerprompt}
\begin{euleroutput}
  80
\end{euleroutput}
\begin{eulercomment}
Parameter yang ditetapkan akan menimpanya juga. Ini digunakan oleh
banyak fungsi Euler seperti plot2d, plot3d.
\end{eulercomment}
\begin{eulerprompt}
>f(4,a=1)
\end{eulerprompt}
\begin{euleroutput}
  16
\end{euleroutput}
\begin{eulercomment}
Jika suatu variabel bukan parameter, maka harus bersifat global.
Fungsi satu baris dapat melihat variabel global.
\end{eulercomment}
\begin{eulerprompt}
>function f(x) := a*x^2
>a=6; f(2)
\end{eulerprompt}
\begin{euleroutput}
  24
\end{euleroutput}
\begin{eulercomment}
Namun parameter yang ditetapkan mengesampingkan nilai global.

Jika argumen tidak ada dalam daftar parameter yang telah ditentukan
sebelumnya, argumen tersebut harus dideklarasikan dengan ":="!
\end{eulercomment}
\begin{eulerprompt}
>f(2,a:=5)
\end{eulerprompt}
\begin{euleroutput}
  20
\end{euleroutput}
\begin{eulercomment}
Fungsi simbolik didefinisikan dengan "\&=". Mereka didefinisikan di
Euler dan Maxima, dan bekerja di kedua dunia.\\
Ekspresi yang menentukan dijalankan melalui Maxima sebelum definisi.
\end{eulercomment}
\begin{eulerprompt}
>function g(x) &= x^3-x*exp(-x); $&g(x)
\end{eulerprompt}
\begin{eulerformula}
\[
x^3-x\,e^ {- x }
\]
\end{eulerformula}
\begin{eulercomment}
Fungsi simbolik dapat digunakan dalam ekspresi simbolik.
\end{eulercomment}
\begin{eulerprompt}
>$&diff(g(x),x), $&% with x=4/3
\end{eulerprompt}
\begin{eulerformula}
\[
x\,e^ {- x }-e^ {- x }+3\,x^2
\]
\end{eulerformula}
\begin{eulerformula}
\[
\frac{e^ {- \frac{4}{3} }}{3}+\frac{16}{3}
\]
\end{eulerformula}
\begin{eulercomment}
Mereka juga dapat digunakan dalam ekspresi numerik. Tentu saja, ini
hanya akan berfungsi jika EMT dapat menafsirkan semua yang ada \\
di dalam fungsi tersebut.
\end{eulercomment}
\begin{eulerprompt}
>g(5+g(1))
\end{eulerprompt}
\begin{euleroutput}
  178.635099908
\end{euleroutput}
\begin{eulercomment}
Mereka dapat digunakan untuk mendefinisikan fungsi atau ekspresi
simbolik lainnya.
\end{eulercomment}
\begin{eulerprompt}
>function G(x) &= factor(integrate(g(x),x)); $&G(c) // integrate: mengintegralkan
\end{eulerprompt}
\begin{eulerformula}
\[
\frac{e^ {- c }\,\left(c^4\,e^{c}+4\,c+4\right)}{4}
\]
\end{eulerformula}
\begin{eulerprompt}
>solve(&g(x),0.5)
\end{eulerprompt}
\begin{euleroutput}
  0.703467422498
\end{euleroutput}
\begin{eulercomment}
Berikut ini juga berfungsi, karena Euler menggunakan ekspresi simbolik
dalam fungsi g, jika tidak menemukan variabel simbolik g, dan jika
terdapat fungsi simbolik g.
\end{eulercomment}
\begin{eulerprompt}
>solve(&g,0.5)
\end{eulerprompt}
\begin{euleroutput}
  0.703467422498
\end{euleroutput}
\begin{eulerprompt}
>function P(x,n) &= (2*x-1)^n; $&P(x,n)
\end{eulerprompt}
\begin{eulerformula}
\[
\left(2\,x-1\right)^{n}
\]
\end{eulerformula}
\begin{eulerprompt}
>function Q(x,n) &= (x+2)^n; $&Q(x,n)
\end{eulerprompt}
\begin{eulerformula}
\[
\left(x+2\right)^{n}
\]
\end{eulerformula}
\begin{eulerprompt}
>$&P(x,4), $&expand(%)
\end{eulerprompt}
\begin{eulerformula}
\[
\left(2\,x-1\right)^4
\]
\end{eulerformula}
\begin{eulerformula}
\[
16\,x^4-32\,x^3+24\,x^2-8\,x+1
\]
\end{eulerformula}
\begin{eulerprompt}
>P(3,4)
\end{eulerprompt}
\begin{euleroutput}
  625
\end{euleroutput}
\begin{eulerprompt}
>$&P(x,4)+ Q(x,3), $&expand(%)
\end{eulerprompt}
\begin{eulerformula}
\[
\left(2\,x-1\right)^4+\left(x+2\right)^3
\]
\end{eulerformula}
\begin{eulerformula}
\[
16\,x^4-31\,x^3+30\,x^2+4\,x+9
\]
\end{eulerformula}
\begin{eulerprompt}
>$&P(x,4)-Q(x,3), $&expand(%), $&factor(%)
\end{eulerprompt}
\begin{eulerformula}
\[
\left(2\,x-1\right)^4-\left(x+2\right)^3
\]
\end{eulerformula}
\begin{eulerformula}
\[
16\,x^4-33\,x^3+18\,x^2-20\,x-7
\]
\end{eulerformula}
\begin{eulerformula}
\[
16\,x^4-33\,x^3+18\,x^2-20\,x-7
\]
\end{eulerformula}
\begin{eulerprompt}
>$&P(x,4)*Q(x,3), $&expand(%), $&factor(%)
\end{eulerprompt}
\begin{eulerformula}
\[
\left(x+2\right)^3\,\left(2\,x-1\right)^4
\]
\end{eulerformula}
\begin{eulerformula}
\[
16\,x^7+64\,x^6+24\,x^5-120\,x^4-15\,x^3+102\,x^2-52\,x+8
\]
\end{eulerformula}
\begin{eulerformula}
\[
\left(x+2\right)^3\,\left(2\,x-1\right)^4
\]
\end{eulerformula}
\begin{eulerprompt}
>$&P(x,4)/Q(x,1), $&expand(%), $&factor(%)
\end{eulerprompt}
\begin{eulerformula}
\[
\frac{\left(2\,x-1\right)^4}{x+2}
\]
\end{eulerformula}
\begin{eulerformula}
\[
\frac{16\,x^4}{x+2}-\frac{32\,x^3}{x+2}+\frac{24\,x^2}{x+2}-\frac{8
 \,x}{x+2}+\frac{1}{x+2}
\]
\end{eulerformula}
\begin{eulerformula}
\[
\frac{\left(2\,x-1\right)^4}{x+2}
\]
\end{eulerformula}
\begin{eulerprompt}
>function f(x) &= x^3-x; $&f(x)
\end{eulerprompt}
\begin{eulerformula}
\[
x^3-x
\]
\end{eulerformula}
\begin{eulercomment}
Dengan \&= fungsinya bersifat simbolis, dan dapat digunakan dalam
ekspresi simbolik lainnya.
\end{eulercomment}
\begin{eulerprompt}
>$&integrate(f(x),x)
\end{eulerprompt}
\begin{eulerformula}
\[
\frac{x^4}{4}-\frac{x^2}{2}
\]
\end{eulerformula}
\begin{eulercomment}
Dengan := fungsinya numerik. Contoh yang baik adalah integral tentu\\
\end{eulercomment}
\begin{eulerformula}
\[
f(x) = \int_1^x t^t \, dt,
\]
\end{eulerformula}
\begin{eulercomment}
yang tidak dapat dievaluasi secara simbolis.\\
Jika kita mendefinisikan ulang fungsi tersebut dengan kata kunci “map”
maka dapat digunakan untuk vektor x. Secara internal,fungsi ini
dipanggil untuk semua nilai x satu kali, dan hasilnya disimpan dalam
vektor.
\end{eulercomment}
\begin{eulerprompt}
>function map f(x) := integrate("x^x",1,x)
>f(0:0.5:2)
\end{eulerprompt}
\begin{euleroutput}
  [-0.783431,  -0.410816,  0,  0.676863,  2.05045]
\end{euleroutput}
\begin{eulercomment}
Fungsi dapat memiliki nilai default untuk parameter.
\end{eulercomment}
\begin{eulerprompt}
>function mylog (x,base=10) := ln(x)/ln(base);
\end{eulerprompt}
\begin{eulercomment}
Sekarang fungsinya bisa dipanggil dengan atau tanpa parameter "base".
\end{eulercomment}
\begin{eulerprompt}
>mylog(100), mylog(2^6.7,2)
\end{eulerprompt}
\begin{euleroutput}
  2
  6.7
\end{euleroutput}
\begin{eulercomment}
Selain itu, dimungkinkan untuk menggunakan parameter yang ditetapkan.
\end{eulercomment}
\begin{eulerprompt}
>mylog(E^2,base=E)
\end{eulerprompt}
\begin{euleroutput}
  2
\end{euleroutput}
\begin{eulercomment}
Seringkali, kita ingin menggunakan fungsi untuk vektor di satu tempat,
dan untuk elemen individual di tempat lain. Hal ini dimungkinkan \\
dengan parameter vektor.
\end{eulercomment}
\begin{eulerprompt}
>function f([a,b]) &= a^2+b^2-a*b+b; $&f(a,b), $&f(x,y)
\end{eulerprompt}
\begin{eulerformula}
\[
b^2-a\,b+b+a^2
\]
\end{eulerformula}
\begin{eulerformula}
\[
y^2-x\,y+y+x^2
\]
\end{eulerformula}
\begin{eulercomment}
Fungsi simbolik seperti ini dapat digunakan untuk variabel simbolik.\\
Namun fungsinya juga dapat digunakan untuk vektor numerik.
\end{eulercomment}
\begin{eulerprompt}
>v=[3,4]; f(v)
\end{eulerprompt}
\begin{euleroutput}
  17
\end{euleroutput}
\begin{eulercomment}
Ada juga fungsi yang murni simbolik, yang tidak dapat digunakan secara
numerik.
\end{eulercomment}
\begin{eulerprompt}
>function lapl(expr,x,y) &&= diff(expr,x,2)+diff(expr,y,2)//turunan parsial kedua
\end{eulerprompt}
\begin{euleroutput}
  
                   diff(expr, y, 2) + diff(expr, x, 2)
  
\end{euleroutput}
\begin{eulerprompt}
>$&realpart((x+I*y)^4), $&lapl(%,x,y)
\end{eulerprompt}
\begin{eulerformula}
\[
y^4-6\,x^2\,y^2+x^4
\]
\end{eulerformula}
\begin{eulerformula}
\[
0
\]
\end{eulerformula}
\begin{eulercomment}
Namun tentu saja, mereka dapat digunakan dalam ekspresi simbolik atau
dalam definisi fungsi simbolik.
\end{eulercomment}
\begin{eulerprompt}
>function f(x,y) &= factor(lapl((x+y^2)^5,x,y)); $&f(x,y)
\end{eulerprompt}
\begin{eulerformula}
\[
10\,\left(y^2+x\right)^3\,\left(9\,y^2+x+2\right)
\]
\end{eulerformula}
\begin{eulercomment}
Ringkasan

- \&= Mendefinisikan fungsi simbolik,\\
- := Mendefinisikan fungsi numerik,\\
- \&\&= Mendefinisikan fungsi simbolik murni.

\begin{eulercomment}
\eulerheading{Pemecahan Ekspresi }
\begin{eulercomment}
Ekspresi dapat diselesaikan secara numerik dan simbolis.\\
Untuk menyelesaikan ekspresi sederhana dari satu variabel, kita dapat
menggunakan fungsi solve(). Dibutuhkan nilai awal untuk memulai\\
pencarian. Secara internal, solve() menggunakan metode secant.
\end{eulercomment}
\begin{eulerprompt}
>solve("x^2-2",1)
\end{eulerprompt}
\begin{euleroutput}
  1.41421356237
\end{euleroutput}
\begin{eulercomment}
Ini juga berfungsi untuk ekspresi simbolik. Ambil fungsi berikut.
\end{eulercomment}
\begin{eulerprompt}
>$&solve(x^2=2,x)
\end{eulerprompt}
\begin{eulerformula}
\[
\left[ x=-\sqrt{2} , x=\sqrt{2} \right] 
\]
\end{eulerformula}
\begin{eulerprompt}
>$&solve(x^2-2,x)
\end{eulerprompt}
\begin{eulerformula}
\[
\left[ x=-\sqrt{2} , x=\sqrt{2} \right] 
\]
\end{eulerformula}
\begin{eulerprompt}
>$&solve(a*x^2+b*x+c=0,x)
\end{eulerprompt}
\begin{eulerformula}
\[
\left[ x=\frac{-\sqrt{b^2-4\,a\,c}-b}{2\,a} , x=\frac{\sqrt{b^2-4\,
 a\,c}-b}{2\,a} \right] 
\]
\end{eulerformula}
\begin{eulerprompt}
>$&solve([a*x+b*y=c,d*x+e*y=f],[x,y])
\end{eulerprompt}
\begin{eulerformula}
\[
\left[ \left[ x=-\frac{c\,e}{b\,\left(d-5\right)-a\,e} , y=\frac{c
 \,\left(d-5\right)}{b\,\left(d-5\right)-a\,e} \right]  \right] 
\]
\end{eulerformula}
\begin{eulerprompt}
>px &= 4*x^8+x^7-x^4-x; $&px
\end{eulerprompt}
\begin{eulerformula}
\[
4\,x^8+x^7-x^4-x
\]
\end{eulerformula}
\begin{eulercomment}
Sekarang kita mencari titik yang polinomialnya adalah 2. Dalam
solve(), nilai target default y=0 dapat diubah dengan variabel yang
ditetapkan.\\
Kami menggunakan y=2 dan memeriksa dengan mengevaluasi polinomial pada
hasil sebelumnya.
\end{eulercomment}
\begin{eulerprompt}
>solve(px,1,y=2), px(%)
\end{eulerprompt}
\begin{euleroutput}
  0.966715594851
  2
\end{euleroutput}
\begin{eulercomment}
Memecahkan ekspresi simbolik dalam bentuk simbolik mengembalikan
daftar solusi. Kami menggunakan pemecah simbolis solve()y ang
disediakan oleh Maxima.
\end{eulercomment}
\begin{eulerprompt}
>sol &= solve(x^2-x-1,x); $&sol
\end{eulerprompt}
\begin{eulerformula}
\[
\left[ x=\frac{1-\sqrt{5}}{2} , x=\frac{\sqrt{5}+1}{2} \right] 
\]
\end{eulerformula}
\begin{eulercomment}
Cara termudah untuk mendapatkan nilai numerik adalah dengan
mengevaluasi solusi secara numerik seperti halnya ekspresi.
\end{eulercomment}
\begin{eulerprompt}
>longest sol()
\end{eulerprompt}
\begin{euleroutput}
      -0.6180339887498949       1.618033988749895 
\end{euleroutput}
\begin{eulercomment}
Untuk menggunakan solusi secara simbolis dalam ekspresi lain, cara
termudah adalah "with".
\end{eulercomment}
\begin{eulerprompt}
>$&x^2 with sol[1], $&expand(x^2-x-1 with sol[2])
\end{eulerprompt}
\begin{eulerformula}
\[
\frac{\left(\sqrt{5}-1\right)^2}{4}
\]
\end{eulerformula}
\begin{eulerformula}
\[
0
\]
\end{eulerformula}
\begin{eulercomment}
Penyelesaian sistem persamaan secara simbolis dapat dilakukan dengan
vektor persamaan dan solver simbolis solve(). Jawabannya adalah daftar\\
daftar persamaan.
\end{eulercomment}
\begin{eulerprompt}
>$&solve([x+y=2,x^3+2*y+x=4],[x,y])
\end{eulerprompt}
\begin{eulerformula}
\[
\left[ \left[ x=-1 , y=3 \right]  , \left[ x=1 , y=1 \right]  , 
 \left[ x=0 , y=2 \right]  \right] 
\]
\end{eulerformula}
\begin{eulercomment}
Fungsi f() dapat melihat variabel global. Namun seringkali kita ingin
menggunakan parameter lokal.

\end{eulercomment}
\begin{eulerformula}
\[
a^x-x^a = 0.1
\]
\end{eulerformula}
\begin{eulercomment}
dengan a=3.
\end{eulercomment}
\begin{eulerprompt}
>function f(x,a) := x^a-a^x;
\end{eulerprompt}
\begin{eulercomment}
Salah satu cara untuk meneruskan parameter tambahan ke f() adalah
dengan menggunakan daftar dengan nama fungsi dan parameternya (cara
lainnya adalah parameter titik koma).
\end{eulercomment}
\begin{eulerprompt}
>solve(\{\{"f",3\}\},2,y=0.1)
\end{eulerprompt}
\begin{euleroutput}
  2.54116291558
\end{euleroutput}
\begin{eulercomment}
Ini juga berfungsi dengan ekspresi. Namun kemudian, elemen daftar
bernama harus digunakan. (Lebih lanjut tentang daftar di tutorial
tentang sintaks EMT).
\end{eulercomment}
\begin{eulerprompt}
>solve(\{\{"x^a-a^x",a=3\}\},2,y=0.1)
\end{eulerprompt}
\begin{euleroutput}
  2.54116291558
\end{euleroutput}
\eulerheading{Menyelesaikan Pertidaksamaan}
\begin{eulercomment}
Untuk menyelesaikan pertidaksamaan, EMT tidak akan dapat melakukannya,
melainkan dengan bantuan Maxima, artinya secara eksak (simbolik).
Perintah Maxima yang digunakan adalah fourier\_elim(), yang harus
dipanggil dengan perintah "load(fourier\_elim)" terlebih dahulu.
\end{eulercomment}
\begin{eulerprompt}
>&load(fourier_elim)
\end{eulerprompt}
\begin{euleroutput}
  
          C:/Program Files/Euler x64/maxima/share/maxima/5.35.1/share/f\(\backslash\)
  ourier_elim/fourier_elim.lisp
  
\end{euleroutput}
\begin{eulerprompt}
>$&fourier_elim([x^2 - 1>0],[x]) // x^2-1 > 0
\end{eulerprompt}
\begin{eulerformula}
\[
\left[ 1<x \right] \lor \left[ x<-1 \right] 
\]
\end{eulerformula}
\begin{eulerprompt}
>$&fourier_elim([x^2 - 1<0],[x]) // x^2-1 < 0
\end{eulerprompt}
\begin{eulerformula}
\[
\left[ -1<x , x<1 \right] 
\]
\end{eulerformula}
\begin{eulerprompt}
>$&fourier_elim([x^2 - 1 # 0],[x]) // x^-1 <> 0
\end{eulerprompt}
\begin{eulerformula}
\[
\left[ -1<x , x<1 \right] \lor \left[ 1<x \right] \lor \left[ x<-1
  \right] 
\]
\end{eulerformula}
\begin{eulerprompt}
>$&fourier_elim([x # 6],[x])
\end{eulerprompt}
\begin{eulerformula}
\[
\left[ x<6 \right] \lor \left[ 6<x \right] 
\]
\end{eulerformula}
\begin{eulerprompt}
>$&fourier_elim([x < 1, x > 1],[x]) // tidak memiliki penyelesaian
\end{eulerprompt}
\begin{eulerformula}
\[
{\it emptyset}
\]
\end{eulerformula}
\begin{eulerprompt}
>$&fourier_elim([minf < x, x < inf],[x]) // solusinya R
\end{eulerprompt}
\begin{eulerformula}
\[
{\it universalset}
\]
\end{eulerformula}
\begin{eulerprompt}
>$&fourier_elim([x^3 - 1 > 0],[x])
\end{eulerprompt}
\begin{eulerformula}
\[
\left[ 1<x , x^2+x+1>0 \right] \lor \left[ x<1 , -x^2-x-1>0
  \right] 
\]
\end{eulerformula}
\begin{eulerprompt}
>$&fourier_elim([cos(x) < 1/2],[x]) // ??? gagal
\end{eulerprompt}
\begin{eulerformula}
\[
\left[ 1-2\,\cos x>0 \right] 
\]
\end{eulerformula}
\begin{eulerprompt}
>$&fourier_elim([y-x < 5, x - y < 7, 10 < y],[x,y]) // sistem pertidaksamaan
\end{eulerprompt}
\begin{eulerformula}
\[
\left[ y-5<x , x<y+7 , 10<y \right] 
\]
\end{eulerformula}
\begin{eulerprompt}
>$&fourier_elim([y-x < 5, x - y < 7, 10 < y],[y,x])
\end{eulerprompt}
\begin{eulerformula}
\[
\left[ {\it max}\left(10 , x-7\right)<y , y<x+5 , 5<x \right] 
\]
\end{eulerformula}
\begin{eulerprompt}
>$&fourier_elim((x + y < 5) and (x - y >8),[x,y])
\end{eulerprompt}
\begin{eulerformula}
\[
\left[ y+8<x , x<5-y , y<-\frac{3}{2} \right] 
\]
\end{eulerformula}
\begin{eulerprompt}
>$&fourier_elim(((x + y < 5) and x < 1) or  (x - y >8),[x,y])
\end{eulerprompt}
\begin{eulerformula}
\[
\left[ y+8<x \right] \lor \left[ x<{\it min}\left(1 , 5-y\right)
  \right] 
\]
\end{eulerformula}
\begin{eulerprompt}
>&fourier_elim([max(x,y) > 6, x # 8, abs(y-1) > 12],[x,y])
\end{eulerprompt}
\begin{euleroutput}
  
          [6 < x, x < 8, y < - 11] or [8 < x, y < - 11]
   or [x < 8, 13 < y] or [x = y, 13 < y] or [8 < x, x < y, 13 < y]
   or [y < x, 13 < y]
  
\end{euleroutput}
\begin{eulerprompt}
>$&fourier_elim([(x+6)/(x-9) <= 6],[x])
\end{eulerprompt}
\begin{eulerformula}
\[
\left[ x=12 \right] \lor \left[ 12<x \right] \lor \left[ x<9
  \right] 
\]
\end{eulerformula}
\eulerheading{Bahasa Matriks}
\begin{eulercomment}
Dokumentasi inti EMT berisi pembahasan rinci tentang bahasa matriks
Euler.\\
Vektor dan matriks dimasukkan dengan tanda kurung siku, elemen
dipisahkan dengan koma, baris dipisahkan dengan titik koma.\\
rated by semicolons.
\end{eulercomment}
\begin{eulerprompt}
>A=[1,2;3,4]
\end{eulerprompt}
\begin{euleroutput}
              1             2 
              3             4 
\end{euleroutput}
\begin{eulercomment}
Hasil kali matriks dilambangkan dengan titik.
\end{eulercomment}
\begin{eulerprompt}
>b=[3;4]
\end{eulerprompt}
\begin{euleroutput}
              3 
              4 
\end{euleroutput}
\begin{eulerprompt}
>b' // transpose b
\end{eulerprompt}
\begin{euleroutput}
  [3,  4]
\end{euleroutput}
\begin{eulerprompt}
>inv(A) //inverse A
\end{eulerprompt}
\begin{euleroutput}
             -2             1 
            1.5          -0.5 
\end{euleroutput}
\begin{eulerprompt}
>A.b //perkalian matriks
\end{eulerprompt}
\begin{euleroutput}
             11 
             25 
\end{euleroutput}
\begin{eulerprompt}
>A.inv(A)
\end{eulerprompt}
\begin{euleroutput}
              1             0 
              0             1 
\end{euleroutput}
\begin{eulercomment}
Poin utama dari bahasa matriks adalah semua fungsi dan operator
bekerja elemen demi elemen.
\end{eulercomment}
\begin{eulerprompt}
>A.A
\end{eulerprompt}
\begin{euleroutput}
              7            10 
             15            22 
\end{euleroutput}
\begin{eulerprompt}
>A^2 //perpangkatan elemen2 A
\end{eulerprompt}
\begin{euleroutput}
              1             4 
              9            16 
\end{euleroutput}
\begin{eulerprompt}
>A.A.A
\end{eulerprompt}
\begin{euleroutput}
             37            54 
             81           118 
\end{euleroutput}
\begin{eulerprompt}
>power(A,3) //perpangkatan matriks
\end{eulerprompt}
\begin{euleroutput}
             37            54 
             81           118 
\end{euleroutput}
\begin{eulerprompt}
>A/A //pembagian elemen-elemen matriks yang seletak
\end{eulerprompt}
\begin{euleroutput}
              1             1 
              1             1 
\end{euleroutput}
\begin{eulerprompt}
>A/b //pembagian elemen2 A oleh elemen2 b kolom demi kolom (karena b vektor kolom)
\end{eulerprompt}
\begin{euleroutput}
       0.333333      0.666667 
           0.75             1 
\end{euleroutput}
\begin{eulerprompt}
>A\(\backslash\)b // hasilkali invers A dan b, A^(-1)b 
\end{eulerprompt}
\begin{euleroutput}
             -2 
            2.5 
\end{euleroutput}
\begin{eulerprompt}
>inv(A).b
\end{eulerprompt}
\begin{euleroutput}
             -2 
            2.5 
\end{euleroutput}
\begin{eulerprompt}
>A\(\backslash\)A   //A^(-1)A
\end{eulerprompt}
\begin{euleroutput}
              1             0 
              0             1 
\end{euleroutput}
\begin{eulerprompt}
>inv(A).A
\end{eulerprompt}
\begin{euleroutput}
              1             0 
              0             1 
\end{euleroutput}
\begin{eulerprompt}
>A*A //perkalin elemen-elemen matriks seletak
\end{eulerprompt}
\begin{euleroutput}
              1             4 
              9            16 
\end{euleroutput}
\begin{eulercomment}
Ini bukan hasil kali matriks, melainkan perkalian elemen demi elemen.
Hal yang sama juga berlaku untuk vektor.
\end{eulercomment}
\begin{eulerprompt}
>b^2 // perpangkatan elemen-elemen matriks/vektor
\end{eulerprompt}
\begin{euleroutput}
              9 
             16 
\end{euleroutput}
\begin{eulercomment}
Jika salah satu operan adalah vektor atau skalar, maka operan tersebut
diperluas secara alami.
\end{eulercomment}
\begin{eulerprompt}
>2*A
\end{eulerprompt}
\begin{euleroutput}
              2             4 
              6             8 
\end{euleroutput}
\begin{eulercomment}
Misalnya, jika operan adalah vektor kolom, elemennya diterapkan ke
semua baris A.
\end{eulercomment}
\begin{eulerprompt}
>[1,2]*A
\end{eulerprompt}
\begin{euleroutput}
              1             4 
              3             8 
\end{euleroutput}
\begin{eulercomment}
Jika ini adalah vektor baris, maka diterapkan ke semua kolom A.
\end{eulercomment}
\begin{eulerprompt}
>A*[2,3]
\end{eulerprompt}
\begin{euleroutput}
              2             6 
              6            12 
\end{euleroutput}
\begin{eulercomment}
Kita dapat membayangkan perkalian ini seolah-olah vektor baris v telah
diduplikasi untuk membentuk matriks yang sama berukuran A.
\end{eulercomment}
\begin{eulerprompt}
>dup([1,2],2) // dup: menduplikasi/menggandakan vektor [1,2] sebanyak 2 kali (baris)
\end{eulerprompt}
\begin{euleroutput}
              1             2 
              1             2 
\end{euleroutput}
\begin{eulerprompt}
>A*dup([1,2],2) 
\end{eulerprompt}
\begin{euleroutput}
              1             4 
              3             8 
\end{euleroutput}
\begin{eulercomment}
Hal ini juga berlaku untuk dua vektor dimana yang satu adalah vektor
baris dan yang lainnya adalah vektor kolom. Kami menghitung i*j untuk
i,j dari 1 sampai 5. Caranya adalah dengan mengalikan 1:5 dengan
transposenya. Bahasa matriks Euler secara otomatis menghasilkan tabel
nilai.
\end{eulercomment}
\begin{eulerprompt}
>(1:5)*(1:5)' // hasilkali elemen-elemen vektor baris dan vektor kolom
\end{eulerprompt}
\begin{euleroutput}
              1             2             3             4             5 
              2             4             6             8            10 
              3             6             9            12            15 
              4             8            12            16            20 
              5            10            15            20            25 
\end{euleroutput}
\begin{eulercomment}
Sekali lagi, ingatlah bahwa ini bukan produk matriks!
\end{eulercomment}
\begin{eulerprompt}
>(1:5).(1:5)' // hasilkali vektor baris dan vektor kolom
\end{eulerprompt}
\begin{euleroutput}
  55
\end{euleroutput}
\begin{eulerprompt}
>sum((1:5)*(1:5)) // sama hasilnya
\end{eulerprompt}
\begin{euleroutput}
  55
\end{euleroutput}
\begin{eulercomment}
Bahkan operator seperti \textless{} atau == bekerja dengan cara yang sama.
\end{eulercomment}
\begin{eulerprompt}
>(1:10)<6 // menguji elemen-elemen yang kurang dari 6
\end{eulerprompt}
\begin{euleroutput}
  [1,  1,  1,  1,  1,  0,  0,  0,  0,  0]
\end{euleroutput}
\begin{eulercomment}
Misalnya, kita dapat menghitung jumlah elemen yang memenuhi kondisi
tertentu dengan fungsi sum().
\end{eulercomment}
\begin{eulerprompt}
>sum((1:10)<6) // banyak elemen yang kurang dari 6
\end{eulerprompt}
\begin{euleroutput}
  5
\end{euleroutput}
\begin{eulercomment}
Euler memiliki operator perbandingan, seperti "==", yang memeriksa
kesetaraan.\\
Kita mendapatkan vektor 0 dan 1, dimana 1 berarti benar.
\end{eulercomment}
\begin{eulerprompt}
>t=(1:10)^2; t==25 //menguji elemen2 t yang sama dengan 25 (hanya ada 1)
\end{eulerprompt}
\begin{euleroutput}
  [0,  0,  0,  0,  1,  0,  0,  0,  0,  0]
\end{euleroutput}
\begin{eulercomment}
Dari vektor tersebut, "bukan nol" memilih elemen bukan nol.\\
Dalam hal ini, kita mendapatkan indeks semua elemen lebih besar dari
50.
\end{eulercomment}
\begin{eulerprompt}
>nonzeros(t>50) //indeks elemen2 t yang lebih besar daripada 50
\end{eulerprompt}
\begin{euleroutput}
  [8,  9,  10]
\end{euleroutput}
\begin{eulercomment}
Tentu saja, kita dapat menggunakan vektor indeks ini untuk mendapatkan
nilai yang sesuai dalam t.
\end{eulercomment}
\begin{eulerprompt}
>t[nonzeros(t>50)] //elemen2 t yang lebih besar daripada 50
\end{eulerprompt}
\begin{euleroutput}
  [64,  81,  100]
\end{euleroutput}
\begin{eulercomment}
Sebagai contoh, mari kita cari semua kuadrat bilangan 1 sampai 1000,
yaitu 5 modulo 11 dan 3 modulo 13.
\end{eulercomment}
\begin{eulerprompt}
>t=1:1000; nonzeros(mod(t^2,11)==5 && mod(t^2,13)==3)
\end{eulerprompt}
\begin{euleroutput}
  [4,  48,  95,  139,  147,  191,  238,  282,  290,  334,  381,  425,
  433,  477,  524,  568,  576,  620,  667,  711,  719,  763,  810,  854,
  862,  906,  953,  997]
\end{euleroutput}
\begin{eulercomment}
EMT tidak sepenuhnya efektif untuk perhitungan bilangan bulat. Ia
menggunakan floating point presisi ganda secara internal.
Namun,seringkali hal ini sangat berguna.

Kita dapat memeriksa primalitasnya. Mari kita cari tahu, berapa banyak
persegi ditambah 1 yang merupakan bilangan prima.
\end{eulercomment}
\begin{eulerprompt}
>t=1:1000; length(nonzeros(isprime(t^2+1)))
\end{eulerprompt}
\begin{euleroutput}
  112
\end{euleroutput}
\begin{eulercomment}
Fungsi nonzeros() hanya berfungsi untuk vektor. Untuk matriks, ada
mnonzeros().
\end{eulercomment}
\begin{eulerprompt}
>seed(2); A=random(3,4)
\end{eulerprompt}
\begin{euleroutput}
       0.765761      0.401188      0.406347      0.267829 
        0.13673      0.390567      0.495975      0.952814 
       0.548138      0.006085      0.444255      0.539246 
\end{euleroutput}
\begin{eulercomment}
Ini mengembalikan indeks elemen, yang bukan nol.
\end{eulercomment}
\begin{eulerprompt}
>k=mnonzeros(A<0.4) //indeks elemen2 A yang kurang dari 0,4
\end{eulerprompt}
\begin{euleroutput}
              1             4 
              2             1 
              2             2 
              3             2 
\end{euleroutput}
\begin{eulercomment}
Indeks ini dapat digunakan untuk mengatur elemen ke nilai tertentu.
\end{eulercomment}
\begin{eulerprompt}
>mset(A,k,0) //mengganti elemen2 suatu matriks pada indeks tertentu
\end{eulerprompt}
\begin{euleroutput}
       0.765761      0.401188      0.406347             0 
              0             0      0.495975      0.952814 
       0.548138             0      0.444255      0.539246 
\end{euleroutput}
\begin{eulercomment}
Fungsi mset() juga dapat mengatur elemen pada indeks ke entri beberapa
matriks lainnya.
\end{eulercomment}
\begin{eulerprompt}
>mset(A,k,-random(size(A)))
\end{eulerprompt}
\begin{euleroutput}
       0.765761      0.401188      0.406347     -0.126917 
      -0.122404     -0.691673      0.495975      0.952814 
       0.548138     -0.483902      0.444255      0.539246 
\end{euleroutput}
\begin{eulercomment}
Dan dimungkinkan untuk mendapatkan elemen dalam vektor
\end{eulercomment}
\begin{eulerprompt}
>mget(A,k)
\end{eulerprompt}
\begin{euleroutput}
  [0.267829,  0.13673,  0.390567,  0.006085]
\end{euleroutput}
\begin{eulercomment}
Fungsi lain yang berguna adalah ekstrem, yang mengembalikan nilai
minimal dan maksimal di setiap baris matriks\\
dan posisi mereka.
\end{eulercomment}
\begin{eulerprompt}
>ex=extrema(A)
\end{eulerprompt}
\begin{euleroutput}
       0.267829             4      0.765761             1 
        0.13673             1      0.952814             4 
       0.006085             2      0.548138             1 
\end{euleroutput}
\begin{eulercomment}
Kita dapat menggunakan ini untuk mengekstrak nilai maksimal di setiap
baris.
\end{eulercomment}
\begin{eulerprompt}
>ex[,3]'
\end{eulerprompt}
\begin{euleroutput}
  [0.765761,  0.952814,  0.548138]
\end{euleroutput}
\begin{eulercomment}
Ini tentu saja sama dengan fungsi max().
\end{eulercomment}
\begin{eulerprompt}
>max(A)'
\end{eulerprompt}
\begin{euleroutput}
  [0.765761,  0.952814,  0.548138]
\end{euleroutput}
\begin{eulercomment}
Namun dengan mget(), kita dapat mengekstrak indeks dan menggunakan
informasi ini untuk mengekstrak elemen pada posisi yang sama dari
matriks lain.
\end{eulercomment}
\begin{eulerprompt}
>j=(1:rows(A))'|ex[,4], mget(-A,j)
\end{eulerprompt}
\begin{euleroutput}
              1             1 
              2             4 
              3             1 
  [-0.765761,  -0.952814,  -0.548138]
\end{euleroutput}
\begin{eulercomment}
\begin{eulercomment}
\eulerheading{Fungsi Matriks Lainnya (Matriks Pembangun) }
\begin{eulercomment}
Untuk membangun sebuah matriks, kita dapat menyusun satu matriks di
atas matriks lainnya. Jika keduanya tidak mempunyai jumlah kolom yang
sama,yang lebih pendek akan diisi dengan 0.
\end{eulercomment}
\begin{eulerprompt}
>v=1:3; v_v
\end{eulerprompt}
\begin{euleroutput}
              1             2             3 
              1             2             3 
\end{euleroutput}
\begin{eulercomment}
Demikian pula, kita dapat melekatkan matriks ke matriks lain secara
berdampingan, jika keduanya mempunyai jumlah baris yang sama.
\end{eulercomment}
\begin{eulerprompt}
>A=random(3,4); A|v'
\end{eulerprompt}
\begin{euleroutput}
       0.032444     0.0534171      0.595713      0.564454             1 
        0.83916      0.175552      0.396988       0.83514             2 
      0.0257573      0.658585      0.629832      0.770895             3 
\end{euleroutput}
\begin{eulercomment}
Jika jumlah barisnya tidak sama, matriks yang lebih pendek diisi
dengan 0.\\
Ada pengecualian untuk aturan ini. Bilangan real yang melekat pada
suatu matriks akan digunakan sebagai kolom yang diisi dengan bilangan
tersebut bilangan real.
\end{eulercomment}
\begin{eulerprompt}
>A|1
\end{eulerprompt}
\begin{euleroutput}
       0.032444     0.0534171      0.595713      0.564454             1 
        0.83916      0.175552      0.396988       0.83514             1 
      0.0257573      0.658585      0.629832      0.770895             1 
\end{euleroutput}
\begin{eulercomment}
Dimungkinkan untuk membuat matriks vektor baris dan kolom.
\end{eulercomment}
\begin{eulerprompt}
>[v;v]
\end{eulerprompt}
\begin{euleroutput}
              1             2             3 
              1             2             3 
\end{euleroutput}
\begin{eulerprompt}
>[v',v']
\end{eulerprompt}
\begin{euleroutput}
              1             1 
              2             2 
              3             3 
\end{euleroutput}
\begin{eulercomment}
Tujuan utamanya adalah untuk menafsirkan ekspresi vektor untuk vektor
kolom.
\end{eulercomment}
\begin{eulerprompt}
>"[x,x^2]"(v')
\end{eulerprompt}
\begin{euleroutput}
              1             1 
              2             4 
              3             9 
\end{euleroutput}
\begin{eulercomment}
Untuk mendapatkan ukuran A, kita bisa menggunakan fungsi berikut.
\end{eulercomment}
\begin{eulerprompt}
>C=zeros(2,4); rows(C), cols(C), size(C), length(C)
\end{eulerprompt}
\begin{euleroutput}
  2
  4
  [2,  4]
  4
\end{euleroutput}
\begin{eulercomment}
Untuk vektor, ada panjang().
\end{eulercomment}
\begin{eulerprompt}
>length(2:10)
\end{eulerprompt}
\begin{euleroutput}
  9
\end{euleroutput}
\begin{eulercomment}
Masih banyak fungsi lain yang menghasilkan matriks.
\end{eulercomment}
\begin{eulerprompt}
>ones(2,2)
\end{eulerprompt}
\begin{euleroutput}
              1             1 
              1             1 
\end{euleroutput}
\begin{eulercomment}
Ini juga dapat digunakan dengan satu parameter. Untuk mendapatkan
vektor dengan bilangan selain 1, gunakan yang berikut ini.
\end{eulercomment}
\begin{eulerprompt}
>ones(5)*6
\end{eulerprompt}
\begin{euleroutput}
  [6,  6,  6,  6,  6]
\end{euleroutput}
\begin{eulercomment}
Juga matriks bilangan acak dapat dihasilkan dengan acak (uniform
distribution) atau normal (Gauß distribution).
\end{eulercomment}
\begin{eulerprompt}
>random(2,2)
\end{eulerprompt}
\begin{euleroutput}
        0.66566      0.831835 
          0.977      0.544258 
\end{euleroutput}
\begin{eulercomment}
Berikut adalah fungsi lain yang berguna, yang merestrukturisasi elemen
matriks menjadi matriks lain.
\end{eulercomment}
\begin{eulerprompt}
>redim(1:9,3,3) // menyusun elemen2 1, 2, 3, ..., 9 ke bentuk matriks 3x3
\end{eulerprompt}
\begin{euleroutput}
              1             2             3 
              4             5             6 
              7             8             9 
\end{euleroutput}
\begin{eulercomment}
Dengan fungsi berikut, kita dapat menggunakan fungsi ini dan fungsi
dup untuk menulis fungsi rep(), yang mengulangi a vektor n kali.
\end{eulercomment}
\begin{eulerprompt}
>function rep(v,n) := redim(dup(v,n),1,n*cols(v))
\end{eulerprompt}
\begin{eulercomment}
Mari kita uji.
\end{eulercomment}
\begin{eulerprompt}
>rep(1:3,5)
\end{eulerprompt}
\begin{euleroutput}
  [1,  2,  3,  1,  2,  3,  1,  2,  3,  1,  2,  3,  1,  2,  3]
\end{euleroutput}
\begin{eulercomment}
Fungsi multdup() menduplikasi elemen vektor.
\end{eulercomment}
\begin{eulerprompt}
>multdup(1:3,5), multdup(1:3,[2,3,2])
\end{eulerprompt}
\begin{euleroutput}
  [1,  1,  1,  1,  1,  2,  2,  2,  2,  2,  3,  3,  3,  3,  3]
  [1,  1,  2,  2,  2,  3,  3]
\end{euleroutput}
\begin{eulercomment}
Fungsi flipx() dan flipy() mengembalikan urutan baris atau kolom
matriks. Fungsi flipx() membalik secara horizontal.
\end{eulercomment}
\begin{eulerprompt}
>flipx(1:5) //membalik elemen2 vektor baris
\end{eulerprompt}
\begin{euleroutput}
  [5,  4,  3,  2,  1]
\end{euleroutput}
\begin{eulercomment}
Untuk rotasi, Euler memiliki rotleft() dan rotright().
\end{eulercomment}
\begin{eulerprompt}
>rotleft(1:5) // memutar elemen2 vektor baris
\end{eulerprompt}
\begin{euleroutput}
  [2,  3,  4,  5,  1]
\end{euleroutput}
\begin{eulercomment}
Fungsi khusus adalah drop(v,i), yang menghilangkan elemen dengan
indeks di i dari vektor v
\end{eulercomment}
\begin{eulerprompt}
>drop(10:20,3)
\end{eulerprompt}
\begin{euleroutput}
  [10,  11,  13,  14,  15,  16,  17,  18,  19,  20]
\end{euleroutput}
\begin{eulercomment}
Perhatikan bahwa vektor i di drop(v,i) mengacu pada indeks elemen di
v, bukan nilai elemen. Jika Anda ingin menghapus elemen, Anda perlu
mencari elemennya terlebih dahulu. Fungsi indexof(v,x) dapat digunakan
untuk mencari elemen x dalam vektor yang diurutkan v.
\end{eulercomment}
\begin{eulerprompt}
>v=primes(50), i=indexof(v,10:20), drop(v,i)
\end{eulerprompt}
\begin{euleroutput}
  [2,  3,  5,  7,  11,  13,  17,  19,  23,  29,  31,  37,  41,  43,  47]
  [0,  5,  0,  6,  0,  0,  0,  7,  0,  8,  0]
  [2,  3,  5,  7,  23,  29,  31,  37,  41,  43,  47]
\end{euleroutput}
\begin{eulercomment}
Seperti yang Anda lihat, tidak ada salahnya memasukkan indeks di luar
rentang (seperti 0), indeks ganda, atau indeks yang tidak diurutkan.
\end{eulercomment}
\begin{eulerprompt}
>drop(1:10,shuffle([0,0,5,5,7,12,12]))
\end{eulerprompt}
\begin{euleroutput}
  [1,  2,  3,  4,  6,  8,  9,  10]
\end{euleroutput}
\begin{eulercomment}
Ada beberapa fungsi khusus untuk mengatur diagonal atau menghasilkan
matriks diagonal.\\
Kita mulai dengan matriks identitas.
\end{eulercomment}
\begin{eulerprompt}
>A=id(5) // matriks identitas 5x5
\end{eulerprompt}
\begin{euleroutput}
              1             0             0             0             0 
              0             1             0             0             0 
              0             0             1             0             0 
              0             0             0             1             0 
              0             0             0             0             1 
\end{euleroutput}
\begin{eulercomment}
Kemudian kita atur diagonal bawah (-1) menjadi 1:4.
\end{eulercomment}
\begin{eulerprompt}
>setdiag(A,-1,1:4) //mengganti diagonal di bawah diagonal utama
\end{eulerprompt}
\begin{euleroutput}
              1             0             0             0             0 
              1             1             0             0             0 
              0             2             1             0             0 
              0             0             3             1             0 
              0             0             0             4             1 
\end{euleroutput}
\begin{eulercomment}
Perhatikan bahwa kami tidak mengubah matriks A. Kami mendapatkan
matriks baru sebagai hasil dari setdiag().

Berikut adalah fungsi yang mengembalikan matriks tri-diagonal.
\end{eulercomment}
\begin{eulerprompt}
>function tridiag (n,a,b,c) := setdiag(setdiag(b*id(n),1,c),-1,a); ...
>tridiag(5,1,2,3)
\end{eulerprompt}
\begin{euleroutput}
              2             3             0             0             0 
              1             2             3             0             0 
              0             1             2             3             0 
              0             0             1             2             3 
              0             0             0             1             2 
\end{euleroutput}
\begin{eulercomment}
Diagonal suatu matriks juga dapat diekstraksi dari matriks tersebut.
Untuk mendemonstrasikannya, kami merestrukturisasi vektor 1:9 menjadi
matriks 3x3.
\end{eulercomment}
\begin{eulerprompt}
>A=redim(1:9,3,3)
\end{eulerprompt}
\begin{euleroutput}
              1             2             3 
              4             5             6 
              7             8             9 
\end{euleroutput}
\begin{eulercomment}
Sekarang kita dapat mengekstrak diagonalnya.
\end{eulercomment}
\begin{eulerprompt}
>d=getdiag(A,0)
\end{eulerprompt}
\begin{euleroutput}
  [1,  5,  9]
\end{euleroutput}
\begin{eulercomment}
Misalnya kita membagi matriks dengan diagonalnya. Bahasa matriks
memperhatikan bahwa vektor kolom d diterapkan pada matriks baris demi
baris.
\end{eulercomment}
\begin{eulerprompt}
>fraction A/d'
\end{eulerprompt}
\begin{euleroutput}
          1         2         3 
        4/5         1       6/5 
        7/9       8/9         1 
\end{euleroutput}
\eulerheading{Vektorisasi}
\begin{eulercomment}
Hampir semua fungsi di Euler juga berfungsi untuk input matriks dan
vektor, jika hal ini masuk akal.\\
Misalnya, fungsi sqrt() menghitung akar kuadrat dari semua elemen
vektor atau matriks.
\end{eulercomment}
\begin{eulerprompt}
>sqrt(1:3)
\end{eulerprompt}
\begin{euleroutput}
  [1,  1.41421,  1.73205]
\end{euleroutput}
\begin{eulercomment}
Jadi Anda dapat dengan mudah membuat tabel nilai. Ini adalah salah
satu cara untuk memplot suatu fungsi (alternatifnya menggunakan
ekspresi).
\end{eulercomment}
\begin{eulerprompt}
>x=1:0.01:5; y=log(x)/x^2; // terlalu panjang untuk ditampikan
\end{eulerprompt}
\begin{eulercomment}
Dengan ini dan operator titik dua a:delta:b, vektor nilai fungsi dapat
dihasilkan dengan mudah.

alam contoh berikut, kita menghasilkan vektor nilai t[i] dengan jarak
0,1 dari -1 hingga 1. Kemudian kita menghasilkan a vektor nilai fungsi

\end{eulercomment}
\begin{eulerformula}
\[
s = t^3-t
\]
\end{eulerformula}
\begin{eulerprompt}
>t=-1:0.1:1; s=t^3-t
\end{eulerprompt}
\begin{euleroutput}
  [0,  0.171,  0.288,  0.357,  0.384,  0.375,  0.336,  0.273,  0.192,
  0.099,  0,  -0.099,  -0.192,  -0.273,  -0.336,  -0.375,  -0.384,
  -0.357,  -0.288,  -0.171,  0]
\end{euleroutput}
\begin{eulercomment}
EMT memperluas operator untuk skalar, vektor, dan matriks dengan cara
yang jelas.

Misalnya vektor kolom dikali vektor baris diperluas ke matriks, jika
operator diterapkan. Berikut ini, v' adalah vektor yang dialihkan\\
(vektor kolom).
\end{eulercomment}
\begin{eulerprompt}
>shortest (1:5)*(1:5)'
\end{eulerprompt}
\begin{euleroutput}
       1      2      3      4      5 
       2      4      6      8     10 
       3      6      9     12     15 
       4      8     12     16     20 
       5     10     15     20     25 
\end{euleroutput}
\begin{eulercomment}
Perhatikan, ini sangat berbeda dengan perkalian matriks. Hasil kali
matriks dilambangkan dengan titik "." di EMT.
\end{eulercomment}
\begin{eulerprompt}
>(1:5).(1:5)'
\end{eulerprompt}
\begin{euleroutput}
  55
\end{euleroutput}
\begin{eulercomment}
Secara default, vektor baris dicetak dalam format ringkas.
\end{eulercomment}
\begin{eulerprompt}
>[1,2,3,4]
\end{eulerprompt}
\begin{euleroutput}
  [1,  2,  3,  4]
\end{euleroutput}
\begin{eulercomment}
Untuk matriks operator khusus . menunjukkan perkalian matriks, dan A'
menunjukkan transposisi. Matriks 1x1 dapat digunakan seperti bilangan
real.
\end{eulercomment}
\begin{eulerprompt}
>v:=[1,2]; v.v', %^2
\end{eulerprompt}
\begin{euleroutput}
  5
  25
\end{euleroutput}
\begin{eulercomment}
Untuk mengubah urutan matriks kita menggunakan apostrophe.
\end{eulercomment}
\begin{eulerprompt}
>v=1:4; v'
\end{eulerprompt}
\begin{euleroutput}
              1 
              2 
              3 
              4 
\end{euleroutput}
\begin{eulercomment}
Jadi kita dapat menghitung matriks A dikalikan vektor b.
\end{eulercomment}
\begin{eulerprompt}
>A=[1,2,3,4;5,6,7,8]; A.v'
\end{eulerprompt}
\begin{euleroutput}
             30 
             70 
\end{euleroutput}
\begin{eulercomment}
Perhatikan bahwa v masih merupakan vektor baris. Jadi v'.v berbeda
dengan v.v'.
\end{eulercomment}
\begin{eulerprompt}
>v'.v
\end{eulerprompt}
\begin{euleroutput}
              1             2             3             4 
              2             4             6             8 
              3             6             9            12 
              4             8            12            16 
\end{euleroutput}
\begin{eulercomment}
v.v' menghitung norm v kuadrat untuk vektor baris v. Hasilnya adalah
vektor 1x1, yang bekerja seperti vektor nyata nomor.
\end{eulercomment}
\begin{eulerprompt}
>v.v'
\end{eulerprompt}
\begin{euleroutput}
  30
\end{euleroutput}
\begin{eulercomment}
Ada juga fungsi norm (bersama dengan banyak fungsi Aljabar Linier
lainnya).
\end{eulercomment}
\begin{eulerprompt}
>norm(v)^2
\end{eulerprompt}
\begin{euleroutput}
  30
\end{euleroutput}
\begin{eulercomment}
Operator dan fungsi mematuhi bahasa matriks Euler.

Berikut ringkasan peraturannya.\\
-Suatu fungsi yang diterapkan pada vektor atau matriks diterapkan pada
setiap elemen.\\
-Operator yang mengoperasikan dua matriks dengan ukuran yang sama
diterapkan secara berpasangan pada elemen-elemen matriks.\\
- Jika kedua matriks mempunyai dimensi yang berbeda, keduanya
diekspansi secara wajar sehingga mempunyai ukuran yang sama.

Misalnya, nilai skalar dikalikan vektor dengan mengalikan nilai dengan
setiap elemen vektor. Atau matriks dikalikan vektor (dengan *,\\
bukan .) memperluas vektor ke ukuran matriks dengan menduplikasinya.

Berikut ini adalah kasus sederhana dengan operator \textasciicircum{}.
\end{eulercomment}
\begin{eulerprompt}
>[1,2,3]^2
\end{eulerprompt}
\begin{euleroutput}
  [1,  4,  9]
\end{euleroutput}
\begin{eulercomment}
Ini kasus yang lebih rumit. Vektor baris dikalikan vektor kolom
memperluas keduanya dengan cara menduplikasi.
\end{eulercomment}
\begin{eulerprompt}
>v:=[1,2,3]; v*v'
\end{eulerprompt}
\begin{euleroutput}
              1             2             3 
              2             4             6 
              3             6             9 
\end{euleroutput}
\begin{eulercomment}
Perhatikan bahwa perkalian skalar menggunakan perkalian matriks, bukan
*!
\end{eulercomment}
\begin{eulerprompt}
>v.v'
\end{eulerprompt}
\begin{euleroutput}
  14
\end{euleroutput}
\begin{eulercomment}
Ada banyak fungsi matriks. Kami memberikan daftar singkat. Anda harus
membaca dokumentasi untuk informasi lebih lanjut tentang perintah ini.

\end{eulercomment}
\begin{eulerttcomment}
  sum,prod menghitung jumlah dan hasil kali baris
  cumsum,cumprod melakukan hal yang sama secara kumulatif
  menghitung nilai ekstrem setiap baris
  extrema mengembalikan vektor dengan informasi ekstrem
\end{eulerttcomment}
\begin{eulercomment}

\end{eulercomment}
\begin{eulerttcomment}
  diag(A,i) mengembalikan himpunan diagonal ke-i
  setdiag(A,i,v)mengatur diagonal ke-i
  id(n) matriks identitasnya
  det(A) determinannya
  charpoly(A) polinomial karakteristik
  eigenvalues(A) nilai eigennya
\end{eulerttcomment}
\begin{eulerprompt}
>v*v, sum(v*v), cumsum(v*v)
\end{eulerprompt}
\begin{euleroutput}
  [1,  4,  9]
  14
  [1,  5,  14]
\end{euleroutput}
\begin{eulercomment}
Operator : menghasilkan vektor baris dengan spasi yang sama, opsional
dengan ukuran langkah.
\end{eulercomment}
\begin{eulerprompt}
>1:4, 1:2:10
\end{eulerprompt}
\begin{euleroutput}
  [1,  2,  3,  4]
  [1,  3,  5,  7,  9]
\end{euleroutput}
\begin{eulercomment}
Untuk menggabungkan matriks dan vektor terdapat operator "\textbar{}" Dan "\_".
\end{eulercomment}
\begin{eulerprompt}
>[1,2,3]|[4,5], [1,2,3]_1
\end{eulerprompt}
\begin{euleroutput}
  [1,  2,  3,  4,  5]
              1             2             3 
              1             1             1 
\end{euleroutput}
\begin{eulercomment}
Elemen-elemen matriks disebut dengan "A[i,j]".
\end{eulercomment}
\begin{eulerprompt}
>A:=[1,2,3;4,5,6;7,8,9]; A[2,3]
\end{eulerprompt}
\begin{euleroutput}
  6
\end{euleroutput}
\begin{eulercomment}
Untuk vektor baris atau kolom, v[i] adalah elemen ke-i dari vektor
tersebut. Untuk matriks, ini mengembalikan baris ke-i yang lengkap\\
dari matriks tersebut.
\end{eulercomment}
\begin{eulerprompt}
>v:=[2,4,6,8]; v[3], A[3]
\end{eulerprompt}
\begin{euleroutput}
  6
  [7,  8,  9]
\end{euleroutput}
\begin{eulercomment}
Indeks juga dapat berupa vektor baris dari indeks. : menunjukkan semua
indeks.
\end{eulercomment}
\begin{eulerprompt}
>v[1:2], A[:,2]
\end{eulerprompt}
\begin{euleroutput}
  [2,  4]
              2 
              5 
              8 
\end{euleroutput}
\begin{eulercomment}
Bentuk singkat untuk menghilangkan indeks sepenuhnya.
\end{eulercomment}
\begin{eulerprompt}
>A[,2:3]
\end{eulerprompt}
\begin{euleroutput}
              2             3 
              5             6 
              8             9 
\end{euleroutput}
\begin{eulercomment}
Untuk tujuan vektorisasi, elemen matriks dapat diakses seolah-olah
elemen tersebut adalah vektor.
\end{eulercomment}
\begin{eulerprompt}
>A\{4\}
\end{eulerprompt}
\begin{euleroutput}
  4
\end{euleroutput}
\begin{eulercomment}
Matriks juga dapat diratakan menggunakan fungsi redim(). Ini
diimplementasikan dalam fungsi flatten().
\end{eulercomment}
\begin{eulerprompt}
>redim(A,1,prod(size(A))), flatten(A)
\end{eulerprompt}
\begin{euleroutput}
  [1,  2,  3,  4,  5,  6,  7,  8,  9]
  [1,  2,  3,  4,  5,  6,  7,  8,  9]
\end{euleroutput}
\begin{eulercomment}
Untuk menggunakan matriks pada tabel, mari kita atur ulang ke format
default, dan hitung tabel nilai sinus dan kosinus.\\
Perhatikan bahwa sudut dinyatakan dalam radian secara default.
\end{eulercomment}
\begin{eulerprompt}
>defformat; w=0°:45°:360°; w=w'; deg(w)
\end{eulerprompt}
\begin{euleroutput}
              0 
             45 
             90 
            135 
            180 
            225 
            270 
            315 
            360 
\end{euleroutput}
\begin{eulercomment}
Sekarang kita menambahkan kolom ke matriks.
\end{eulercomment}
\begin{eulerprompt}
>M = deg(w)|w|cos(w)|sin(w)
\end{eulerprompt}
\begin{euleroutput}
              0             0             1             0 
             45      0.785398      0.707107      0.707107 
             90        1.5708             0             1 
            135       2.35619     -0.707107      0.707107 
            180       3.14159            -1             0 
            225       3.92699     -0.707107     -0.707107 
            270       4.71239             0            -1 
            315       5.49779      0.707107     -0.707107 
            360       6.28319             1             0 
\end{euleroutput}
\begin{eulercomment}
Dengan menggunakan bahasa matriks, kita dapat menghasilkan beberapa
tabel dari beberapa fungsi sekaligus.\\
Dalam contoh berikut, kita menghitung t[j]\textasciicircum{}i untuk i dari 1 hingga n.
Kita mendapatkan sebuah matriks, dimana setiap barisnya adalah tabel
dari t\textasciicircum{}i untuk satu i. Artinya matriks memiliki elemen \\
\end{eulercomment}
\begin{eulerformula}
\[
a_{i,j} = t_j^i, \quad 1 \le j \le 101, \quad 1 \le i \le n
\]
\end{eulerformula}
\begin{eulercomment}
Fungsi yang tidak berfungsi untuk masukan vektor harus "divektorkan".\\
Hal ini dapat dicapai dengan "map" kata kunci dalam definisi fungsi.
Kemudian fungsi tersebut akan dievaluasi untuk setiap elemen parameter
vektor. Integrasi numerik integrate() hanya berfungsi untuk batas
interval skalar. Jadi kita perlu membuat vektorisasinya
\end{eulercomment}
\begin{eulerprompt}
>function map f(x) := integrate("x^x",1,x)
\end{eulerprompt}
\begin{eulercomment}
Kata kunci "map" membuat vektorisasi fungsi tersebut. Fungsinya
sekarang akan berfungsi untuk vektor bilangan
\end{eulercomment}
\begin{eulerprompt}
>f([1:5])
\end{eulerprompt}
\begin{euleroutput}
  [0,  2.05045,  13.7251,  113.336,  1241.03]
\end{euleroutput}
\eulerheading{Sub Matriks dan Elemen Matriks}
\begin{eulercomment}
Untuk mengakses elemen matriks, gunakan notasi braket.
\end{eulercomment}
\begin{eulerprompt}
>A=[1,2,3;4,5,6;7,8,9], A[2,2]
\end{eulerprompt}
\begin{euleroutput}
              1             2             3 
              4             5             6 
              7             8             9 
  5
\end{euleroutput}
\begin{eulercomment}
Kita dapat mengakses baris matriks secara lengkap.
\end{eulercomment}
\begin{eulerprompt}
>A[2]
\end{eulerprompt}
\begin{euleroutput}
  [4,  5,  6]
\end{euleroutput}
\begin{eulercomment}
Dalam kasus vektor baris atau kolom, ini mengembalikan elemen vektor.
\end{eulercomment}
\begin{eulerprompt}
>v=1:3; v[2]
\end{eulerprompt}
\begin{euleroutput}
  2
\end{euleroutput}
\begin{eulercomment}
Untuk memastikan, Anda mendapatkan baris pertama untuk matriks 1xn dan
mxn, tentukan semua kolom menggunakan kolom kedua yang kosong indeks.
\end{eulercomment}
\begin{eulerprompt}
>A[2,]
\end{eulerprompt}
\begin{euleroutput}
  [4,  5,  6]
\end{euleroutput}
\begin{eulercomment}
Jika indeks adalah vektor dari indeks, Euler akan mengembalikan baris
matriks yang sesuai.

Di sini kita menginginkan baris pertama dan kedua A.
\end{eulercomment}
\begin{eulerprompt}
>A[[1,2]]
\end{eulerprompt}
\begin{euleroutput}
              1             2             3 
              4             5             6 
\end{euleroutput}
\begin{eulercomment}
Kita bahkan dapat menyusun ulang A menggunakan vektor indeks.
Tepatnya, kita tidak mengubah A di sini, tetapi menghitung a versi A
yang dipesan ulang.
\end{eulercomment}
\begin{eulerprompt}
>A[[3,2,1]]
\end{eulerprompt}
\begin{euleroutput}
              7             8             9 
              4             5             6 
              1             2             3 
\end{euleroutput}
\begin{eulercomment}
Trick indeks juga bekerja dengan kolom.

Contoh ini memilih semua baris A dan kolom kedua dan ketiga.
\end{eulercomment}
\begin{eulerprompt}
>A[1:3,2:3]
\end{eulerprompt}
\begin{euleroutput}
              2             3 
              5             6 
              8             9 
\end{euleroutput}
\begin{eulercomment}
Untuk singkatan ":" menunjukkan semua indeks baris atau kolom.
\end{eulercomment}
\begin{eulerprompt}
>A[:,3]
\end{eulerprompt}
\begin{euleroutput}
              3 
              6 
              9 
\end{euleroutput}
\begin{eulercomment}
Alternatifnya, biarkan indeks pertama kosong.
\end{eulercomment}
\begin{eulerprompt}
>A[,2:3]
\end{eulerprompt}
\begin{euleroutput}
              2             3 
              5             6 
              8             9 
\end{euleroutput}
\begin{eulercomment}
Kita juga bisa mendapatkan baris terakhir A.
\end{eulercomment}
\begin{eulerprompt}
>A[-1]
\end{eulerprompt}
\begin{euleroutput}
  [7,  8,  9]
\end{euleroutput}
\begin{eulercomment}
Sekarang mari kita ubah elemen A dengan menetapkan submatriks A ke
beberapa nilai. Hal ini sebenarnya mengubah matriks tersimpan A.
\end{eulercomment}
\begin{eulerprompt}
>A[1,1]=4
\end{eulerprompt}
\begin{euleroutput}
              4             2             3 
              4             5             6 
              7             8             9 
\end{euleroutput}
\begin{eulercomment}
Kita juga dapat memberikan nilai pada baris A.
\end{eulercomment}
\begin{eulerprompt}
>A[1]=[-1,-1,-1]
\end{eulerprompt}
\begin{euleroutput}
             -1            -1            -1 
              4             5             6 
              7             8             9 
\end{euleroutput}
\begin{eulercomment}
Kita bahkan dapat menetapkan sub-matriks jika ukurannya sesuai.
\end{eulercomment}
\begin{eulerprompt}
>A[1:2,1:2]=[5,6;7,8]
\end{eulerprompt}
\begin{euleroutput}
              5             6            -1 
              7             8             6 
              7             8             9 
\end{euleroutput}
\begin{eulercomment}
Selain itu, beberapa jalan pintas diperbolehkan.
\end{eulercomment}
\begin{eulerprompt}
>A[1:2,1:2]=0
\end{eulerprompt}
\begin{euleroutput}
              0             0            -1 
              0             0             6 
              7             8             9 
\end{euleroutput}
\begin{eulercomment}
Peringatan: Indeks di luar batas mengembalikan matriks kosong, atau
pesan kesalahan, bergantung pada pengaturan sistem. Standarnya adalah
pesan kesalahan. Namun perlu diingat bahwa indeks negatif dapat
digunakan untuk mengakses elemen matriks yang dihitung dari akhir.
\end{eulercomment}
\begin{eulerprompt}
>A[4]
\end{eulerprompt}
\begin{euleroutput}
  Row index 4 out of bounds!
  Error in:
  A[4] ...
      ^
\end{euleroutput}
\eulerheading{Menyortir dan mengacak}
\begin{eulercomment}
Fungsi sort() mengurutkan vektor baris.
\end{eulercomment}
\begin{eulerprompt}
>sort([5,6,4,8,1,9])
\end{eulerprompt}
\begin{euleroutput}
  [1,  4,  5,  6,  8,  9]
\end{euleroutput}
\begin{eulercomment}
Seringkali perlu mengetahui indeks vektor yang diurutkan dalam vektor
aslinya. Ini dapat digunakan untuk menyusun ulang vektor lain\\
dengan cara yang sama.

Mari kita mengacak sebuah vektor.
\end{eulercomment}
\begin{eulerprompt}
>v=shuffle(1:10)
\end{eulerprompt}
\begin{euleroutput}
  [4,  5,  10,  6,  8,  9,  1,  7,  2,  3]
\end{euleroutput}
\begin{eulercomment}
Indeks berisi urutan v.
\end{eulercomment}
\begin{eulerprompt}
>\{vs,ind\}=sort(v); v[ind]
\end{eulerprompt}
\begin{euleroutput}
  [1,  2,  3,  4,  5,  6,  7,  8,  9,  10]
\end{euleroutput}
\begin{eulercomment}
Ini juga berfungsi untuk vektor string.
\end{eulercomment}
\begin{eulerprompt}
>s=["a","d","e","a","aa","e"]
\end{eulerprompt}
\begin{euleroutput}
  a
  d
  e
  a
  aa
  e
\end{euleroutput}
\begin{eulerprompt}
>\{ss,ind\}=sort(s); ss
\end{eulerprompt}
\begin{euleroutput}
  a
  a
  aa
  d
  e
  e
\end{euleroutput}
\begin{eulercomment}
Seperti yang Anda lihat, posisi entri ganda agak acak.
\end{eulercomment}
\begin{eulerprompt}
>ind
\end{eulerprompt}
\begin{euleroutput}
  [4,  1,  5,  2,  6,  3]
\end{euleroutput}
\begin{eulercomment}
Fungsi unik mengembalikan daftar elemen unik vektor yang diurutkan.
\end{eulercomment}
\begin{eulerprompt}
>intrandom(1,10,10), unique(%)
\end{eulerprompt}
\begin{euleroutput}
  [4,  4,  9,  2,  6,  5,  10,  6,  5,  1]
  [1,  2,  4,  5,  6,  9,  10]
\end{euleroutput}
\begin{eulercomment}
Ini juga berfungsi untuk vektor string.
\end{eulercomment}
\begin{eulerprompt}
>unique(s)
\end{eulerprompt}
\begin{euleroutput}
  a
  aa
  d
  e
\end{euleroutput}
\eulerheading{Aljabar Linear}
\begin{eulercomment}
EMT memiliki banyak sekali fungsi untuk menyelesaikan masalah sistem
linier, sistem sparse, atau regresi.

Untuk sistem linier Ax=b, Anda dapat menggunakan algoritma Gauss,
matriks invers, atau linear fit. Operator A\textbackslash{}b menggunakan versi\\
algoritma Gauss.
\end{eulercomment}
\begin{eulerprompt}
>A=[1,2;3,4]; b=[5;6]; A\(\backslash\)b
\end{eulerprompt}
\begin{euleroutput}
             -4 
            4.5 
\end{euleroutput}
\begin{eulercomment}
Contoh lain, kita membuat matriks berukuran 200x200 dan jumlah
baris-barisnya. Kemudian kita selesaikan Ax=b menggunakan matriks
invers. Kami mengukur kesalahan sebagai deviasi maksimal semua elemen
dari 1, yang tentu saja merupakan solusi yang tepat.
\end{eulercomment}
\begin{eulerprompt}
>A=normal(200,200); b=sum(A); longest totalmax(abs(inv(A).b-1))
\end{eulerprompt}
\begin{euleroutput}
    8.790745908981989e-13 
\end{euleroutput}
\begin{eulercomment}
Jika sistem tidak mempunyai solusi, kecocokan linier meminimalkan
norma kesalahan Ax-b
\end{eulercomment}
\begin{eulerprompt}
>A=[1,2,3;4,5,6;7,8,9]
\end{eulerprompt}
\begin{euleroutput}
              1             2             3 
              4             5             6 
              7             8             9 
\end{euleroutput}
\begin{eulercomment}
Determinan matriks ini adalah 0
\end{eulercomment}
\begin{eulerprompt}
>det(A)
\end{eulerprompt}
\begin{euleroutput}
  0
\end{euleroutput}
\eulerheading{Matriks Simbolik}
\begin{eulercomment}
Maxima memiliki matriks simbolik. Tentu saja Maxima dapat digunakan
untuk permasalahan aljabar linier sederhana seperti itu. Kita \\
dapat mendefinisikan matriks untuk Euler dan Maxima dengan \&:=, lalu
menggunakannya dalam ekspresi simbolik. Bentuk [...] yang \\
biasa untuk mendefinisikan matriks dapat digunakan di Euler untuk
mendefinisikan matriks simbolik.
\end{eulercomment}
\begin{eulerprompt}
>A &= [a,1,1;1,a,1;1,1,a]; $A
\end{eulerprompt}
\begin{eulerformula}
\[
\begin{pmatrix}a & 1 & 1 \\ 1 & a & 1 \\ 1 & 1 & a \\ \end{pmatrix}
\]
\end{eulerformula}
\begin{eulerprompt}
>$&det(A), $&factor(%)
\end{eulerprompt}
\begin{eulerformula}
\[
a\,\left(a^2-1\right)-2\,a+2
\]
\end{eulerformula}
\begin{eulerformula}
\[
\left(a-1\right)^2\,\left(a+2\right)
\]
\end{eulerformula}
\begin{eulerprompt}
>$&invert(A) with a=0
\end{eulerprompt}
\begin{eulerformula}
\[
\begin{pmatrix}-\frac{1}{2} & \frac{1}{2} & \frac{1}{2} \\ \frac{1
 }{2} & -\frac{1}{2} & \frac{1}{2} \\ \frac{1}{2} & \frac{1}{2} & -
 \frac{1}{2} \\ \end{pmatrix}
\]
\end{eulerformula}
\begin{eulerprompt}
>A &= [1,a;b,2]; $A
\end{eulerprompt}
\begin{eulerformula}
\[
\begin{pmatrix}1 & a \\ b & 2 \\ \end{pmatrix}
\]
\end{eulerformula}
\begin{eulercomment}
Seperti semua variabel simbolik, matriks ini dapat digunakan dalam
ekspresi simbolik lainnya.
\end{eulercomment}
\begin{eulerprompt}
>$&det(A-x*ident(2)), $&solve(%,x)
\end{eulerprompt}
\begin{eulerformula}
\[
\left(1-x\right)\,\left(2-x\right)-a\,b
\]
\end{eulerformula}
\begin{eulerformula}
\[
\left[ x=\frac{3-\sqrt{4\,a\,b+1}}{2} , x=\frac{\sqrt{4\,a\,b+1}+3
 }{2} \right] 
\]
\end{eulerformula}
\begin{eulercomment}
Nilai eigen juga dapat dihitung secara otomatis. Hasilnya adalah
sebuah vektor dengan dua vektor nilai eigen dan multiplisitas.
\end{eulercomment}
\begin{eulerprompt}
>$&eigenvalues([a,1;1,a])
\end{eulerprompt}
\begin{eulerformula}
\[
\left[ \left[ a-1 , a+1 \right]  , \left[ 1 , 1 \right]  \right] 
\]
\end{eulerformula}
\begin{eulercomment}
Untuk mengekstrak vektor eigen tertentu memerlukan pengindeksan yang
cermat.
\end{eulercomment}
\begin{eulerprompt}
>$&eigenvectors([a,1;1,a]), &%[2][1][1]
\end{eulerprompt}
\begin{eulerformula}
\[
\left[ \left[ \left[ a-1 , a+1 \right]  , \left[ 1 , 1 \right] 
  \right]  , \left[ \left[ \left[ 1 , -1 \right]  \right]  , \left[ 
 \left[ 1 , 1 \right]  \right]  \right]  \right] 
\]
\end{eulerformula}
\begin{euleroutput}
  
                                 [1, - 1]
  
\end{euleroutput}
\begin{eulercomment}
Matriks simbolik dapat dievaluasi dalam Euler secara numerik sama
seperti ekspresi simbolik lainnya
\end{eulercomment}
\begin{eulerprompt}
>A(a=4,b=5)
\end{eulerprompt}
\begin{euleroutput}
              1             4 
              5             2 
\end{euleroutput}
\begin{eulercomment}
Dalam ekspresi simbolik, gunakan with.
\end{eulercomment}
\begin{eulerprompt}
>$&A with [a=4,b=5]
\end{eulerprompt}
\begin{eulerformula}
\[
\begin{pmatrix}1 & 4 \\ 5 & 2 \\ \end{pmatrix}
\]
\end{eulerformula}
\begin{eulercomment}
Akses ke deretan matriks simbolik berfungsi sama seperti matriks
numerik.
\end{eulercomment}
\begin{eulerprompt}
>$&A[1]
\end{eulerprompt}
\begin{eulerformula}
\[
\left[ 1 , a \right] 
\]
\end{eulerformula}
\begin{eulercomment}
Ekspresi simbolis dapat berisi tugas. Dan itu mengubah matriks A.
\end{eulercomment}
\begin{eulerprompt}
>&A[1,1]:=t+1; $&A
\end{eulerprompt}
\begin{eulerformula}
\[
\begin{pmatrix}t+1 & a \\ b & 2 \\ \end{pmatrix}
\]
\end{eulerformula}
\begin{eulercomment}
Ada fungsi simbolik di Maxima untuk membuat vektor dan matriks. Untuk
ini, lihat dokumentasi Maxima atau tutorial tentang Maxima di EMT.
\end{eulercomment}
\begin{eulerprompt}
>v &= makelist(1/(i+j),i,1,3); $v
\end{eulerprompt}
\begin{eulerformula}
\[
\left[ \frac{1}{j+1} , \frac{1}{j+2} , \frac{1}{j+3} \right] 
\]
\end{eulerformula}
\begin{eulerttcomment}
 
\end{eulerttcomment}
\begin{eulerprompt}
>B &:= [1,2;3,4]; $B, $&invert(B)
\end{eulerprompt}
\begin{eulerformula}
\[
\begin{pmatrix}1 & 2 \\ 3 & 4 \\ \end{pmatrix}
\]
\end{eulerformula}
\begin{eulerformula}
\[
\begin{pmatrix}-2 & 1 \\ \frac{3}{2} & -\frac{1}{2} \\ 
 \end{pmatrix}
\]
\end{eulerformula}
\begin{eulercomment}
Hasilnya dapat dievaluasi secara numerik dalam Euler. Untuk informasi
lebih lanjut tentang Maxima, lihat pengenalan Maxima.
\end{eulercomment}
\begin{eulerprompt}
>$&invert(B)()
\end{eulerprompt}
\begin{euleroutput}
             -2             1 
            1.5          -0.5 
\end{euleroutput}
\begin{eulercomment}
Euler juga memiliki fungsi kuat xinv(), yang melakukan upaya lebih
besar dan mendapatkan hasil yang lebih tepat.

Perhatikan, bahwa dengan \&:= matriks B telah didefinisikan sebagai
simbolik dalam ekspresi simbolik dan numerik dalam\\
ekspresi numerik. Jadi kita bisa menggunakannya di sini.
\end{eulercomment}
\begin{eulerprompt}
>longest B.xinv(B)
\end{eulerprompt}
\begin{euleroutput}
                        1                       0 
                        0                       1 
\end{euleroutput}
\begin{eulercomment}
Misalnya nilai eigen dari A dapat dihitung secara numerik.
\end{eulercomment}
\begin{eulerprompt}
>A=[1,2,3;4,5,6;7,8,9]; real(eigenvalues(A))
\end{eulerprompt}
\begin{euleroutput}
  [16.1168,  -1.11684,  0]
\end{euleroutput}
\begin{eulercomment}
Atau secara simbolis. Lihat tutorial tentang Maxima untuk detailnya.
\end{eulercomment}
\begin{eulerprompt}
>$&eigenvalues(@A)
\end{eulerprompt}
\begin{eulerformula}
\[
\left[ \left[ \frac{15-3\,\sqrt{33}}{2} , \frac{3\,\sqrt{33}+15}{2}
  , 0 \right]  , \left[ 1 , 1 , 1 \right]  \right] 
\]
\end{eulerformula}
\eulerheading{Nilai Numerik dalam Ekspresi simbolis}
\begin{eulercomment}
Ekspresi simbolis hanyalah string yang berisi ekspresi. Jika kita
ingin mendefinisikan nilai untuk ekspresi simbolik dan ekspresi \\
numerik, kita harus menggunakan "\&:=".
\end{eulercomment}
\begin{eulerprompt}
>A &:= [1,pi;4,5]
\end{eulerprompt}
\begin{euleroutput}
              1       3.14159 
              4             5 
\end{euleroutput}
\begin{eulercomment}
Masih terdapat perbedaan antara bentuk numerik dan simbolik. Saat
mentransfer matriks ke bentuk simbolik, pendekatan pecahan\\
untuk real akan digunakan.
\end{eulercomment}
\begin{eulerprompt}
>$&A
\end{eulerprompt}
\begin{eulerformula}
\[
\begin{pmatrix}1 & \frac{1146408}{364913} \\ 4 & 5 \\ \end{pmatrix}
\]
\end{eulerformula}
\begin{eulercomment}
Untuk menghindari hal ini, ada fungsi "mxmset(variabel)".
\end{eulercomment}
\begin{eulerprompt}
>mxmset(A); $&A
\end{eulerprompt}
\begin{eulerformula}
\[
\begin{pmatrix}1 & 3.141592653589793 \\ 4 & 5 \\ \end{pmatrix}
\]
\end{eulerformula}
\begin{eulercomment}
Maxima juga dapat menghitung dengan bilangan floating point, bahkan
dengan bilangan mengambang besar dengan 32 digit.Namun evaluasinya
jauh lebih lambat.
\end{eulercomment}
\begin{eulerprompt}
>$&bfloat(sqrt(2)), $&float(sqrt(2))
\end{eulerprompt}
\begin{eulerformula}
\[
1.4142135623730950488016887242097_B \times 10^{0}
\]
\end{eulerformula}
\begin{eulerformula}
\[
1.414213562373095
\]
\end{eulerformula}
\begin{eulercomment}
Ketepatan angka floating point besar dapat diubah.
\end{eulercomment}
\begin{eulerprompt}
>&fpprec:=100; &bfloat(pi)
\end{eulerprompt}
\begin{euleroutput}
  
          3.14159265358979323846264338327950288419716939937510582097494\(\backslash\)
  4592307816406286208998628034825342117068b0
  
\end{euleroutput}
\begin{eulercomment}
Variabel numerik dapat digunakan dalam ekspresi simbolik apa pun
menggunakan "@var".

erhatikan bahwa ini hanya diperlukan, jika variabel telah
didefinisikan dengan ":=" atau "=" sebagai variabel numerik.
\end{eulercomment}
\begin{eulerprompt}
>B:=[1,pi;3,4]; $&det(@B)
\end{eulerprompt}
\begin{eulerformula}
\[
-5.424777960769379
\]
\end{eulerformula}
\begin{eulercomment}
\begin{eulercomment}
\eulerheading{Demo - Suku Bunga}
\begin{eulercomment}
Di bawah ini, kami menggunakan Euler Math Toolbox (EMT) untuk
menghitung suku bunga. Kami melakukannya secara numerik dan simbolis
untuk menunjukkan kepada Anda bagaimana Euler dapat digunakan untuk
memecahkan masalah kehidupan nyata.

Asumsikan Anda memiliki modal awal sebesar 5.000 (katakanlah dalam
dolar).
\end{eulercomment}
\begin{eulerprompt}
>K=5000
\end{eulerprompt}
\begin{euleroutput}
  5000
\end{euleroutput}
\begin{eulercomment}
Sekarang kami mengasumsikan tingkat bunga 3\% per tahun. Mari kita
tambahkan satu tarif sederhana dan hitung hasilnya.
\end{eulercomment}
\begin{eulerprompt}
>K*1.03
\end{eulerprompt}
\begin{euleroutput}
  5150
\end{euleroutput}
\begin{eulercomment}
Euler juga akan memahami sintaks berikut.
\end{eulercomment}
\begin{eulerprompt}
>K+K*3%
\end{eulerprompt}
\begin{euleroutput}
  5150
\end{euleroutput}
\begin{eulercomment}
Namun lebih mudah menggunakan faktor tersebut
\end{eulercomment}
\begin{eulerprompt}
>q=1+3%, K*q
\end{eulerprompt}
\begin{euleroutput}
  1.03
  5150
\end{euleroutput}
\begin{eulercomment}
Selama 10 tahun, kita cukup mengalikan faktor-faktornya dan
mendapatkan nilai akhir dengan tingkat bunga majemuk.
\end{eulercomment}
\begin{eulerprompt}
>K*q^10
\end{eulerprompt}
\begin{euleroutput}
  6719.58189672
\end{euleroutput}
\begin{eulercomment}
Untuk keperluan kita, kita dapat mengatur formatnya menjadi 2 digit
setelah titik desimal.
\end{eulercomment}
\begin{eulerprompt}
>format(12,2); K*q^10
\end{eulerprompt}
\begin{euleroutput}
      6719.58 
\end{euleroutput}
\begin{eulercomment}
Mari kita cetak yang dibulatkan menjadi 2 digit dalam satu kalimat
lengkap.
\end{eulercomment}
\begin{eulerprompt}
>"Starting from " + K + "$ you get " + round(K*q^10,2) + "$."
\end{eulerprompt}
\begin{euleroutput}
  Starting from 5000$ you get 6719.58$.
\end{euleroutput}
\begin{eulercomment}
Bagaimana jika kita ingin mengetahui hasil antara dari tahun 1 sampai
tahun ke 9? Untuk ini, bahasa matriks Euler sangat membantu. Anda
tidak perlu menulis satu perulangan, tetapi cukup masuk
\end{eulercomment}
\begin{eulerprompt}
>K*q^(0:10)
\end{eulerprompt}
\begin{euleroutput}
  Real 1 x 11 matrix
  
      5000.00     5150.00     5304.50     5463.64     ...
\end{euleroutput}
\begin{eulercomment}
Bagaimana keajaiban ini terjadi? Pertama, ekspresi 0:10 mengembalikan
vektor bilangan bulat.
\end{eulercomment}
\begin{eulerprompt}
>short 0:10
\end{eulerprompt}
\begin{euleroutput}
  [0,  1,  2,  3,  4,  5,  6,  7,  8,  9,  10]
\end{euleroutput}
\begin{eulercomment}
Kemudian semua operator dan fungsi di Euler dapat diterapkan pada
vektor elemen demi elemen. Jadi
\end{eulercomment}
\begin{eulerprompt}
>short q^(0:10)
\end{eulerprompt}
\begin{euleroutput}
  [1,  1.03,  1.0609,  1.0927,  1.1255,  1.1593,  1.1941,  1.2299,
  1.2668,  1.3048,  1.3439]
\end{euleroutput}
\begin{eulercomment}
Adalah vektor faktor q\textasciicircum{}0 sampai q\textasciicircum{}10. Ini dikalikan dengan K, dan kita
mendapatkan vektor nilainya.
\end{eulercomment}
\begin{eulerprompt}
>VK=K*q^(0:10);
\end{eulerprompt}
\begin{eulercomment}
Tentu saja, cara realistis untuk menghitung suku bunga ini adalah
dengan membulatkannya ke sen terdekat tahun. Mari kita tambahkan
fungsi untuk ini.
\end{eulercomment}
\begin{eulerprompt}
>function oneyear (K) := round(K*q,2)
\end{eulerprompt}
\begin{eulercomment}
Mari kita bandingkan kedua hasil tersebut, dengan dan tanpa
pembulatan.
\end{eulercomment}
\begin{eulerprompt}
>longest oneyear(1234.57), longest 1234.57*q
\end{eulerprompt}
\begin{euleroutput}
                  1271.61 
                1271.6071 
\end{euleroutput}
\begin{eulercomment}
Sekarang tidak ada rumus sederhana untuk tahun ke-n, dan kita harus
mengulanginya selama bertahun-tahun. Euler menyediakan banyak hal
solusi untuk ini.\\
Cara termudah adalah fungsi iterate, yang mengulangi fungsi tertentu
beberapa kali.
\end{eulercomment}
\begin{eulerprompt}
>VKr=iterate("oneyear",5000,10)
\end{eulerprompt}
\begin{euleroutput}
  Real 1 x 11 matrix
  
      5000.00     5150.00     5304.50     5463.64     ...
\end{euleroutput}
\begin{eulercomment}
Kami dapat mencetaknya dengan cara yang ramah, menggunakan format kami
dengan tempat desimal tetap.
\end{eulercomment}
\begin{eulerprompt}
>VKr'
\end{eulerprompt}
\begin{euleroutput}
      5000.00 
      5150.00 
      5304.50 
      5463.64 
      5627.55 
      5796.38 
      5970.27 
      6149.38 
      6333.86 
      6523.88 
      6719.60 
\end{euleroutput}
\begin{eulercomment}
Untuk mendapatkan elemen vektor tertentu, kami menggunakan indeks
dalam tanda kurung siku.
\end{eulercomment}
\begin{eulerprompt}
>VKr[2], VKr[1:3]
\end{eulerprompt}
\begin{euleroutput}
      5150.00 
      5000.00     5150.00     5304.50 
\end{euleroutput}
\begin{eulercomment}
Anehnya, kita juga bisa menggunakan vektor indeks. Ingatlah bahwa 1:3
menghasilkan vektor [1,2,3].\\
Mari kita bandingkan elemen terakhir dari nilai yang dibulatkan dengan
nilai penuh.
\end{eulercomment}
\begin{eulerprompt}
>VKr[-1], VK[-1]
\end{eulerprompt}
\begin{euleroutput}
      6719.60 
      6719.58 
\end{euleroutput}
\begin{eulercomment}
Perbedaannya sangat kecil.

\begin{eulercomment}
\eulerheading{Menyelesaikan Persamaan}
\begin{eulercomment}
Sekarang kita mengambil fungsi yang lebih maju, yang menambahkan
tingkat uang tertentu setiap tahunnya.
\end{eulercomment}
\begin{eulerprompt}
>function onepay (K) := K*q+R
\end{eulerprompt}
\begin{eulercomment}
Kita tidak perlu menentukan q atau R untuk definisi fungsi. Hanya jika
kita menjalankan perintah, kita harus melakukannya\\
mendefinisikan nilai-nilai ini. Kami memilih R=200.
\end{eulercomment}
\begin{eulerprompt}
>R=200; iterate("onepay",5000,10)
\end{eulerprompt}
\begin{euleroutput}
  Real 1 x 11 matrix
  
      5000.00     5350.00     5710.50     6081.82     ...
\end{euleroutput}
\begin{eulercomment}
Bagaimana jika kita menghapus jumlah yang sama setiap tahun?
\end{eulercomment}
\begin{eulerprompt}
>R=-200; iterate("onepay",5000,10)
\end{eulerprompt}
\begin{euleroutput}
  Real 1 x 11 matrix
  
      5000.00     4950.00     4898.50     4845.45     ...
\end{euleroutput}
\begin{eulercomment}
Kami melihat uangnya berkurang. Jelasnya, jika kita hanya mendapat
bunga sebesar 150 pada tahun pertama, namun menghapus 200, kita
kehilangan uang setiap tahunnya.

Bagaimana kita dapat menentukan berapa tahun uang tersebut akan
bertahan? Kita harus menulis satu lingkaran untuk ini. Cara termudah\\
adalah dengan melakukan iterasi cukup lama.
\end{eulercomment}
\begin{eulerprompt}
>VKR=iterate("onepay",5000,50)
\end{eulerprompt}
\begin{euleroutput}
  Real 1 x 51 matrix
  
      5000.00     4950.00     4898.50     4845.45     ...
\end{euleroutput}
\begin{eulercomment}
Dengan menggunakan bahasa matriks, kita dapat menentukan nilai negatif
pertama dengan cara berikut.
\end{eulercomment}
\begin{eulerprompt}
>min(nonzeros(VKR<0))
\end{eulerprompt}
\begin{euleroutput}
        48.00 
\end{euleroutput}
\begin{eulercomment}
Alasannya adalah bukan nol (VKR\textless{}0) mengembalikan vektor indeks i,
dengan VKR[i]\textless{}0, dan min menghitung indeks minimal.

karena vektor selalu dimulai dengan indeks 1, maka jawabannya adalah
47 tahun.

Fungsi iterate() memiliki satu trik lagi. Ini dapat mengambil kondisi
akhir sebagai argumen. Kemudian akan mengembalikan nilai dan\\
jumlah iterasi.
\end{eulercomment}
\begin{eulerprompt}
>\{x,n\}=iterate("onepay",5000,till="x<0"); x, n,
\end{eulerprompt}
\begin{euleroutput}
       -19.83 
        47.00 
\end{euleroutput}
\begin{eulercomment}
Mari kita coba menjawab pertanyaan yang lebih ambigu. Asumsikan kita
mengetahui bahwa nilainya adalah 0 setelah 50 tahun. Berapa\\
tingkat bunganya?

Ini adalah pertanyaan yang hanya bisa dijawab secara numerik. Di bawah
ini, kita akan mendapatkan rumus yang diperlukan.\\
Kemudian Anda akan melihat bahwa tidak ada rumus yang mudah untuk
menentukan tingkat suku bunga. Namun untuk saat ini, kami menargetkan\\
solusi numerik.

Langkah pertama adalah mendefinisikan fungsi yang melakukan iterasi
sebanyak n kali. Kami menambahkan semua parameter ke fungsi ini.
\end{eulercomment}
\begin{eulerprompt}
>function f(K,R,P,n) := iterate("x*(1+P/100)+R",K,n;P,R)[-1]
\end{eulerprompt}
\begin{eulercomment}
Iterasinya sama seperti di atas

\end{eulercomment}
\begin{eulerformula}
\[
x_{n+1} = x_n \cdot \left(1+ \frac{P}{100}\right) + R
\]
\end{eulerformula}
\begin{eulercomment}
Namun kami tidak lagi menggunakan nilai global R dalam ekspresi kami.
Fungsi seperti iterate() memiliki trik khusus di Euler. Anda dapat\\
meneruskan nilai variabel dalam ekspresi sebagai parameter titik koma.
Dalam hal ini P dan R.\\
Apalagi kami hanya tertarik pada nilai terakhir. Jadi kita ambil
indeks [-1].

Mari kita coba tes.
\end{eulercomment}
\begin{eulerprompt}
>f(5000,-200,3,47)
\end{eulerprompt}
\begin{euleroutput}
       -19.83 
\end{euleroutput}
\begin{eulercomment}
Sekarang kita bisa menyelesaikan masalah kita.
\end{eulercomment}
\begin{eulerprompt}
>solve("f(5000,-200,x,50)",3)
\end{eulerprompt}
\begin{euleroutput}
         3.15 
\end{euleroutput}
\begin{eulercomment}
Rutinitas penyelesaian menyelesaikan ekspresi=0 untuk variabel x.
Jawabannya adalah 3,15\% per tahun. Kami mengambil nilai awal 3\%\\
untuk algoritma. Fungsi solve() selalu membutuhkan nilai awal.

Kita dapat menggunakan fungsi yang sama untuk menyelesaikan pertanyaan
berikut: Berapa banyak yang dapat kita keluarkan per tahun\\
sehingga modal awal habis setelah 20 tahun dengan asumsi tingkat bunga
3\% per tahun.
\end{eulercomment}
\begin{eulerprompt}
>solve("f(5000,x,3,20)",-200)
\end{eulerprompt}
\begin{euleroutput}
      -336.08 
\end{euleroutput}
\begin{eulercomment}
Perhatikan bahwa Anda tidak dapat menyelesaikan jumlah tahun, karena
fungsi kami mengasumsikan n sebagai nilai bilangan bulat.

\begin{eulercomment}
\eulerheading{Solusi Simbolis Masalah Suku Bunga}
\begin{eulercomment}
Kita dapat menggunakan bagian simbolis dari Euler untuk mempelajari
masalahnya. Pertama kita mendefinisikan fungsi onepay() kita secara
simbolis.
\end{eulercomment}
\begin{eulerprompt}
>function op(K) &= K*q+R; $&op(K)
\end{eulerprompt}
\begin{eulerformula}
\[
R+q\,K
\]
\end{eulerformula}
\begin{eulercomment}
Sekarang kita dapat mengulanginya.
\end{eulercomment}
\begin{eulerprompt}
>$&op(op(op(op(K)))), $&expand(%)
\end{eulerprompt}
\begin{eulerformula}
\[
q\,\left(q\,\left(q\,\left(R+q\,K\right)+R\right)+R\right)+R
\]
\end{eulerformula}
\begin{eulerformula}
\[
q^3\,R+q^2\,R+q\,R+R+q^4\,K
\]
\end{eulerformula}
\begin{eulercomment}
Kami melihat sebuah pola. Setelah n periode yang kita miliki

\end{eulercomment}
\begin{eulerformula}
\[
K_n = q^n K + R (1+q+\ldots+q^{n-1}) = q^n K + \frac{q^n-1}{q-1} R
\]
\end{eulerformula}
\begin{eulercomment}
Rumusnya adalah rumus jumlah geometri yang diketahui Maxima.
\end{eulercomment}
\begin{eulerprompt}
>&sum(q^k,k,0,n-1); $& % = ev(%,simpsum)
\end{eulerprompt}
\begin{eulerformula}
\[
\sum_{k=0}^{n-1}{q^{k}}=\frac{q^{n}-1}{q-1}
\]
\end{eulerformula}
\begin{eulercomment}
Ini agak rumit. Jumlahnya dievaluasi dengan tanda "simpsum" untuk
menguranginya menjadi hasil bagi.\\
Mari kita membuat fungsi untuk ini.
\end{eulercomment}
\begin{eulerprompt}
>function fs(K,R,P,n) &= (1+P/100)^n*K + ((1+P/100)^n-1)/(P/100)*R; $&fs(K,R,P,n)
\end{eulerprompt}
\begin{eulerformula}
\[
\frac{100\,\left(\left(\frac{P}{100}+1\right)^{n}-1\right)\,R}{P}+K
 \,\left(\frac{P}{100}+1\right)^{n}
\]
\end{eulerformula}
\begin{eulercomment}
Fungsinya sama dengan fungsi f kita sebelumnya. Tapi ini lebih
efektif.
\end{eulercomment}
\begin{eulerprompt}
>longest f(5000,-200,3,47), longest fs(5000,-200,3,47)
\end{eulerprompt}
\begin{euleroutput}
       -19.82504734650985 
       -19.82504734652684 
\end{euleroutput}
\begin{eulercomment}
Sekarang kita dapat menggunakannya untuk menanyakan waktu n. Kapan
modal kita habis? Perkiraan awal kami adalah 30 tahun.
\end{eulercomment}
\begin{eulerprompt}
>solve("fs(5000,-330,3,x)",30)
\end{eulerprompt}
\begin{euleroutput}
        20.51 
\end{euleroutput}
\begin{eulercomment}
Jawaban ini mengatakan akan menjadi negatif setelah 21 tahun.\\
Kita juga dapat menggunakan sisi simbolis Euler untuk menghitung rumus
pembayaran.\\
Asumsikan kita mendapatkan pinjaman sebesar K, dan membayar n
pembayaran sebesar R (dimulai setelah tahun pertama) meninggalkan sisa
hutang sebesar Kn (pada saat pembayaran terakhir. Rumusnya jelas
\end{eulercomment}
\begin{eulerprompt}
>equ &= fs(K,R,P,n)=Kn; $&equ
\end{eulerprompt}
\begin{eulerformula}
\[
\frac{100\,\left(\left(\frac{P}{100}+1\right)^{n}-1\right)\,R}{P}+K
 \,\left(\frac{P}{100}+1\right)^{n}={\it Kn}
\]
\end{eulerformula}
\begin{eulercomment}
Biasanya rumus ini diberikan dalam bentuk\\
\end{eulercomment}
\begin{eulerformula}
\[
i = \frac{P}{100}
\]
\end{eulerformula}
\begin{eulerprompt}
>equ &= (equ with P=100*i); $&equ
\end{eulerprompt}
\begin{eulerformula}
\[
\frac{\left(\left(i+1\right)^{n}-1\right)\,R}{i}+\left(i+1\right)^{
 n}\,K={\it Kn}
\]
\end{eulerformula}
\begin{eulercomment}
Kita dapat menyelesaikan nilai R secara simbolis.\\
\end{eulercomment}
\begin{eulerformula}
\[
i = \frac{P}{100}
\]
\end{eulerformula}
\begin{eulerprompt}
>$&solve(equ,R)
\end{eulerprompt}
\begin{eulerformula}
\[
\left[ R=\frac{i\,{\it Kn}-i\,\left(i+1\right)^{n}\,K}{\left(i+1
 \right)^{n}-1} \right] 
\]
\end{eulerformula}
\begin{eulercomment}
Seperti yang Anda lihat dari rumusnya, fungsi ini mengembalikan
kesalahan floating point untuk i=0. Euler tetap merencanakannya.

Tentu saja, kami memiliki batasan berikut.
\end{eulercomment}
\begin{eulerprompt}
>$&limit(R(5000,0,x,10),x,0)
\end{eulerprompt}
\begin{eulerformula}
\[
\lim_{x\rightarrow 0}{R\left(5000 , 0 , x , 10\right)}
\]
\end{eulerformula}
\begin{eulercomment}
Yang jelas tanpa bunga kita harus membayar kembali 10 tarif 500.

Persamaan tersebut juga dapat diselesaikan untuk n. Akan terlihat
lebih bagus jika kita menerapkan beberapa penyederhanaan padanya.
\end{eulercomment}
\begin{eulerprompt}
>fn &= solve(equ,n) | ratsimp; $&fn
\end{eulerprompt}
\begin{eulerformula}
\[
\left[ n=\frac{\log \left(\frac{R+i\,{\it Kn}}{R+i\,K}\right)}{
 \log \left(i+1\right)} \right] 
\]
\end{eulerformula}
\eulersubheading{Latihan R2}
\begin{eulercomment}
1. Lakukan penyederhanaan pada operasi berikut\\
\end{eulercomment}
\begin{eulerformula}
\[
\left(\frac{24a^{10}b^{-8}c^7}{12a^6b^{-3}c^5}\right)^{-5}
\]
\end{eulerformula}
\begin{eulerttcomment}
   Penyelesaian:
\end{eulerttcomment}
\begin{eulerprompt}
>$&((24*a^10*b^(-8)*c^7)/(12*a^6*b^(-3)*c^5))^(-5)
\end{eulerprompt}
\begin{eulerformula}
\[
\frac{b^{25}}{32\,a^{20}\,c^{10}}
\]
\end{eulerformula}
\begin{eulercomment}
2. Lakukan penyederhanaan operasi berikut ini\\
\end{eulercomment}
\begin{eulerformula}
\[
\left(\frac{125p^{12}q^{-14}r^{22}}{25p^8q^6r^{-15}}\right)^{-4}
\]
\end{eulerformula}
\begin{eulerttcomment}
   Penyelesaian:
\end{eulerttcomment}
\begin{eulerprompt}
>$&((125*p^12*q^(-14)*r^22)/(25*p^8*q^6*r^(-15)))^(-4)
\end{eulerprompt}
\begin{eulerformula}
\[
\frac{q^{80}}{625\,p^{16}\,r^{148}}
\]
\end{eulerformula}
\begin{eulercomment}
3. Lakukan penyederhanaan operasi berikut ini\\
\end{eulercomment}
\begin{eulerformula}
\[
(m^{x-b}n^{x+b})^x(m^bn^{-b})^x
\]
\end{eulerformula}
\begin{eulerttcomment}
    Penyelesaian:
\end{eulerttcomment}
\begin{eulerprompt}
>$&((m^(x-b)*n(x+b))^x*(m^b*n^(-b))^x)
\end{eulerprompt}
\begin{eulerformula}
\[
\left(\frac{m^{b}}{n^{b}}\right)^{x}\,\left(m^{x-b}\,n\left(x+b
 \right)\right)^{x}
\]
\end{eulerformula}
\begin{eulercomment}
4. Lakukan penyederhanaan operasi berikut ini\\
\end{eulercomment}
\begin{eulerformula}
\[
\left(\frac{3x^ay^b}{-3x^{a}y^{b}}\right)^{2}
\]
\end{eulerformula}
\begin{eulerttcomment}
   Penyelesaian:
\end{eulerttcomment}
\begin{eulerprompt}
>$&((3*x^a*y^b)/(-3*x^a*y^b))^2
\end{eulerprompt}
\begin{eulerformula}
\[
1
\]
\end{eulerformula}
\begin{eulercomment}
5. Tentukan hasil dari operasi bilangan dibawah ini\\
\end{eulercomment}
\begin{eulerformula}
\[
\left(\frac{4(8-6)^2-4*3+2*8}{3^1+19^0}\right)
\]
\end{eulerformula}
\begin{eulerttcomment}
   Penyelesaian:
\end{eulerttcomment}
\begin{eulerprompt}
>$&(4*(8-6)^2-4*3+2*8)/(3^1+19^0)
\end{eulerprompt}
\begin{eulerformula}
\[
5
\]
\end{eulerformula}
\begin{eulercomment}
6. Tentukan hasil dari operasi bilangan berikut\\
\end{eulercomment}
\begin{eulerformula}
\[
\left(\frac{[4(8-6)^2+4](3-2*8)}{2^2(2^3+5)}\right)
\]
\end{eulerformula}
\begin{eulerttcomment}
   Penyelesaian:
\end{eulerttcomment}
\begin{eulerprompt}
>$&((4*(8-6)^2+4)*(3-2*8))/(2^2*(2^3+5))
\end{eulerprompt}
\begin{eulerformula}
\[
-5
\]
\end{eulerformula}
\begin{eulercomment}
Ke-enam soal-soal diatas merupakan penyederhanaan suatu perhitungan
simbolik atau operasi-operasi dalam aljabar.
\end{eulercomment}
\eulersubheading{Latihan R3}
\begin{eulercomment}
Perform the indicated operations\\
1.\\
\end{eulercomment}
\begin{eulerformula}
\[
(3x-2)^2
\]
\end{eulerformula}
\begin{eulercomment}
Penyelesaian:
\end{eulercomment}
\begin{eulerprompt}
>function P(x,n) &= (3*x-2)^n; $&P(x,n)
\end{eulerprompt}
\begin{eulerformula}
\[
\left(3\,x-2\right)^{n}
\]
\end{eulerformula}
\begin{eulerprompt}
>$&P(x,2), $&expand(%)
\end{eulerprompt}
\begin{eulerformula}
\[
\left(3\,x-2\right)^2
\]
\end{eulerformula}
\begin{eulerformula}
\[
9\,x^2-12\,x+4
\]
\end{eulerformula}
\begin{eulercomment}
2.\\
\end{eulercomment}
\begin{eulerformula}
\[
(3b+1)(b-2)
\]
\end{eulerformula}
\begin{eulerttcomment}
  Penyelesaian:
\end{eulerttcomment}
\begin{eulerprompt}
>function P(b) &= (3*b+1); $&P(b)
\end{eulerprompt}
\begin{eulerformula}
\[
3\,b+1
\]
\end{eulerformula}
\begin{eulerprompt}
>function q(b) &= (b-2); $&q(b)
\end{eulerprompt}
\begin{eulerformula}
\[
b-2
\]
\end{eulerformula}
\begin{eulerprompt}
>$&P(b)*q(b), $&expand(%), $&factor(%)
\end{eulerprompt}
\begin{eulerformula}
\[
\left(b-2\right)\,\left(3\,b+1\right)
\]
\end{eulerformula}
\begin{eulerformula}
\[
3\,b^2-5\,b-2
\]
\end{eulerformula}
\begin{eulerformula}
\[
\left(b-2\right)\,\left(3\,b+1\right)
\]
\end{eulerformula}
\begin{eulercomment}
3.\\
\end{eulercomment}
\begin{eulerformula}
\[
(3x^2-2x-x^3+2)-(5x^2-8x-x^3+4)
\]
\end{eulerformula}
\begin{eulercomment}
Penyelesaian:
\end{eulercomment}
\begin{eulerprompt}
>$&(3*x^2-2*x-x^3+2)-(5*x^2-8*x-x^3+4)
\end{eulerprompt}
\begin{eulerformula}
\[
-2\,x^2+6\,x-2
\]
\end{eulerformula}
\begin{eulercomment}
4.\\
\end{eulercomment}
\begin{eulerformula}
\[
(2x^4-3x^2+7x)-(5x^3+2x^2-3x+5)
\]
\end{eulerformula}
\begin{eulercomment}
Penyelesaian:
\end{eulercomment}
\begin{eulerprompt}
>$&(2*x^4-3*x^2+7*x)-(5*x^3+2*x^2-3*x+5)
\end{eulerprompt}
\begin{eulerformula}
\[
2\,x^4-5\,x^3-5\,x^2+10\,x-5
\]
\end{eulerformula}
\begin{eulercomment}
5.\\
\end{eulercomment}
\begin{eulerformula}
\[
(2x+3y+z-7)+(4x-2y-z+8)+(-3x+y-2z-4)
\]
\end{eulerformula}
\begin{eulerttcomment}
  Penyelesaian:
\end{eulerttcomment}
\begin{eulerprompt}
>$&(2*x+3*y+z-7)+(4*x-2*y-z+8)+(-3*x+y-2*z-4)
\end{eulerprompt}
\begin{eulerformula}
\[
-2\,z+2\,y+3\,x-3
\]
\end{eulerformula}
\begin{eulercomment}
6.\\
\end{eulercomment}
\begin{eulerformula}
\[
(t^a+4)(t^a-7)
\]
\end{eulerformula}
\begin{eulerttcomment}
  Penyelesaian:
\end{eulerttcomment}
\begin{eulerprompt}
>function P(t,n) &= (t^a+4)^n; $&P(t,n)
\end{eulerprompt}
\begin{eulerformula}
\[
\left(t^{a}+4\right)^{n}
\]
\end{eulerformula}
\begin{eulerprompt}
>function q(t,n) &= (t^a-7)^n; $&q(t,n)
\end{eulerprompt}
\begin{eulerformula}
\[
\left(t^{a}-7\right)^{n}
\]
\end{eulerformula}
\begin{eulerprompt}
>$&P(t,1)*q(t,1), $&expand(%), $&factor(%)
\end{eulerprompt}
\begin{eulerformula}
\[
\left(t^{a}-7\right)\,\left(t^{a}+4\right)
\]
\end{eulerformula}
\begin{eulerformula}
\[
t^{2\,a}-3\,t^{a}-28
\]
\end{eulerformula}
\begin{eulerformula}
\[
\left(t^{a}-7\right)\,\left(t^{a}+4\right)
\]
\end{eulerformula}
\eulersubheading{Latihan R4}
\begin{eulercomment}
Tentukan faktor dari persamaan berikut\\
1.\\
\end{eulercomment}
\begin{eulerformula}
\[
t^2+8t+15
\]
\end{eulerformula}
\begin{eulerttcomment}
  Penyelesaian:
\end{eulerttcomment}
\begin{eulerprompt}
>gt &= (t^2+8*t+15); $&gt
\end{eulerprompt}
\begin{eulerformula}
\[
t^2+8\,t+15
\]
\end{eulerformula}
\begin{eulerprompt}
>$&factor((gt))
\end{eulerprompt}
\begin{eulerformula}
\[
\left(t+3\right)\,\left(t+5\right)
\]
\end{eulerformula}
\begin{eulercomment}
2.\\
\end{eulercomment}
\begin{eulerformula}
\[
 a^3 + 24a^2 + 144a
\]
\end{eulerformula}
\begin{eulerttcomment}
  Penyelesaian:
\end{eulerttcomment}
\begin{eulerprompt}
>ga &= (a^3+24*a^2+144*a); $&ga
\end{eulerprompt}
\begin{eulerformula}
\[
a^3+24\,a^2+144\,a
\]
\end{eulerformula}
\begin{eulerprompt}
>$&factor((ga))
\end{eulerprompt}
\begin{eulerformula}
\[
a\,\left(a+12\right)^2
\]
\end{eulerformula}
\begin{eulercomment}
3.\\
\end{eulercomment}
\begin{eulerformula}
\[
x^4+11x^2-80
\]
\end{eulerformula}
\begin{eulercomment}
Penyelesaian:
\end{eulercomment}
\begin{eulerprompt}
>fx &= (x^4+11*x^2-80); $&fx
\end{eulerprompt}
\begin{eulerformula}
\[
x^4+11\,x^2-80
\]
\end{eulerformula}
\begin{eulerprompt}
>$&factor((fx))
\end{eulerprompt}
\begin{eulerformula}
\[
\left(x^2-5\right)\,\left(x^2+16\right)
\]
\end{eulerformula}
\begin{eulercomment}
4.\\
\end{eulercomment}
\begin{eulerformula}
\[
m^6+8m^3-20
\]
\end{eulerformula}
\begin{eulercomment}
Penyelesaian:
\end{eulercomment}
\begin{eulerprompt}
>fm &= (m^6+8*m^3-20); $&fm
\end{eulerprompt}
\begin{eulerformula}
\[
m^6+8\,m^3-20
\]
\end{eulerformula}
\begin{eulerprompt}
>$&factor((fm))
\end{eulerprompt}
\begin{eulerformula}
\[
\left(m^3-2\right)\,\left(m^3+10\right)
\]
\end{eulerformula}
\begin{eulercomment}
5.\\
\end{eulercomment}
\begin{eulerformula}
\[
x^6-2x^5+x^4-x^2+2x-1
\]
\end{eulerformula}
\begin{eulerttcomment}
 Penyelesaian:
\end{eulerttcomment}
\begin{eulerprompt}
>fx &= (x^6-2*x^5+x^4-x^2+2*x-1); $&fx
\end{eulerprompt}
\begin{eulerformula}
\[
x^6-2\,x^5+x^4-x^2+2\,x-1
\]
\end{eulerformula}
\begin{eulerprompt}
>$&factor((fx))
\end{eulerprompt}
\begin{eulerformula}
\[
\left(x-1\right)^3\,\left(x+1\right)\,\left(x^2+1\right)
\]
\end{eulerformula}
\eulersubheading{Latihan R5}
\begin{eulercomment}
Tentukan penyelesaian atau solusi dari persamaan dibawah\\
1.\\
\end{eulercomment}
\begin{eulerformula}
\[
x^2+3x-28
\]
\end{eulerformula}
\begin{eulerttcomment}
 Penyelesaian:
\end{eulerttcomment}
\begin{eulerprompt}
>$&solve(x^2+3*x-28,x)
\end{eulerprompt}
\begin{eulerformula}
\[
\left[ x=4 , x=-7 \right] 
\]
\end{eulerformula}
\begin{eulercomment}
Hasil di atas diperoleh dengan memfaktorkan ekspresi atau persamaan
sehingga didapat solusi(nilai x) yaitu 4 dan -7

2. \\
\end{eulercomment}
\begin{eulerformula}
\[
x^2+100=20x
\]
\end{eulerformula}
\begin{eulerttcomment}
 Penyelesaian:
\end{eulerttcomment}
\begin{eulerprompt}
>$&solve(x^2-20*x+100,x)
\end{eulerprompt}
\begin{eulerformula}
\[
\left[ x=10 \right] 
\]
\end{eulerformula}
\begin{eulercomment}
3.\\
\end{eulercomment}
\begin{eulerformula}
\[
5x^2-75=0
\]
\end{eulerformula}
\begin{eulerttcomment}
 Penyelesaian:
\end{eulerttcomment}
\begin{eulerprompt}
>$&solve(5*x^2-75=0,x)
\end{eulerprompt}
\begin{eulerformula}
\[
\left[ x=-\sqrt{15} , x=\sqrt{15} \right] 
\]
\end{eulerformula}
\begin{eulercomment}
4. \\
\end{eulercomment}
\begin{eulerformula}
\[
21n^2-10=n
\]
\end{eulerformula}
\begin{eulerttcomment}
 Penyelesaian:
\end{eulerttcomment}
\begin{eulerprompt}
>sol &= solve(21*n^2-n-10,n); $&sol
\end{eulerprompt}
\begin{eulerformula}
\[
\left[ n=\frac{5}{7} , n=-\frac{2}{3} \right] 
\]
\end{eulerformula}
\begin{eulercomment}
5.\\
\end{eulercomment}
\begin{eulerformula}
\[
12a^2-28=5a
\]
\end{eulerformula}
\begin{eulerttcomment}
 Penyelesaian: 
\end{eulerttcomment}
\begin{eulerprompt}
>sol &= solve(12*a^2-5*a-28,a); $&sol
\end{eulerprompt}
\begin{eulerformula}
\[
\left[ a=-\frac{4}{3} , a=\frac{7}{4} \right] 
\]
\end{eulerformula}
\begin{eulercomment}
6.\\
\end{eulercomment}
\begin{eulerformula}
\[
6z^2-18=0
\]
\end{eulerformula}
\begin{eulerttcomment}
 Penyelesaian:
\end{eulerttcomment}
\begin{eulerprompt}
>$&solve(6*z^2-18,z)
\end{eulerprompt}
\begin{eulerformula}
\[
\left[ z=-\sqrt{3} , z=\sqrt{3} \right] 
\]
\end{eulerformula}
\eulersubheading{Latihan R6}
\begin{eulercomment}
Sederhanakan persamaan dibawah ini\\
1).\\
\end{eulercomment}
\begin{eulerformula}
\[
\left(\frac{6-x}{x^2-36}\right)
\]
\end{eulerformula}
\begin{eulercomment}
2).\\
\end{eulercomment}
\begin{eulerformula}
\[
\left(\frac{x^2-4}{x^2-4x+4}\right)
\]
\end{eulerformula}
\begin{eulercomment}
3).\\
\end{eulercomment}
\begin{eulerformula}
\[
\left(\frac{x^2+2x-3}{x^2-9}\right)
\]
\end{eulerformula}
\begin{eulercomment}
4).\\
\end{eulercomment}
\begin{eulerformula}
\[
\left(\frac{4-x}{x^2+4x-32}\right)
\]
\end{eulerformula}
\begin{eulercomment}
5).\\
\end{eulercomment}
\begin{eulerformula}
\[
\left(\frac{x^3-6x^2+9x}{x^3-3x^2}\right)
\]
\end{eulerformula}
\begin{eulercomment}
Penyelesaian:\\
Nomor 1
\end{eulercomment}
\begin{eulerprompt}
>$&((6-x)/x^2-36); $&factor(%)
\end{eulerprompt}
\begin{eulerformula}
\[
\frac{-36\,x^2-x+6}{x^2}
\]
\end{eulerformula}
\begin{eulercomment}
Nomor 2
\end{eulercomment}
\begin{eulerprompt}
>$&((x^2-4)/(x^2-4*x+4)); $&factor(%)
\end{eulerprompt}
\begin{eulerformula}
\[
\frac{x+2}{x-2}
\]
\end{eulerformula}
\begin{eulercomment}
Nomor 3
\end{eulercomment}
\begin{eulerprompt}
>$&((x^2+2*x-3)/(x^2-9)); $&factor(%)
\end{eulerprompt}
\begin{eulerformula}
\[
\frac{x-1}{x-3}
\]
\end{eulerformula}
\begin{eulercomment}
Nomor 4
\end{eulercomment}
\begin{eulerprompt}
>$&((4-x)/(x^2+4*x-32)); $&factor(%)
\end{eulerprompt}
\begin{eulerformula}
\[
-\frac{1}{x+8}
\]
\end{eulerformula}
\begin{eulercomment}
Nomor 5
\end{eulercomment}
\begin{eulerprompt}
>$&((x^3-6*x^2+9*x)/(x^3-3*x^2)); $&factor(%)
\end{eulerprompt}
\begin{eulerformula}
\[
\frac{x-3}{x}
\]
\end{eulerformula}
\eulersubheading{Latihan 2.3}
\begin{eulercomment}
Tentukan masing-masing nilai berikut ini\\
1).\\
\end{eulercomment}
\begin{eulerformula}
\[
(f\circ g)(-1)
\]
\end{eulerformula}
\begin{eulercomment}
2)\\
\end{eulercomment}
\begin{eulerformula}
\[
(h\circ f)(1)
\]
\end{eulerformula}
\begin{eulercomment}
3)\\
\end{eulercomment}
\begin{eulerformula}
\[
(f\circ h)(-3)
\]
\end{eulerformula}
\begin{eulercomment}
4)\\
\end{eulercomment}
\begin{eulerformula}
\[
(f\circ f)(-4)
\]
\end{eulerformula}
\begin{eulercomment}
5)\\
\end{eulercomment}
\begin{eulerformula}
\[
(h\circ g)(3)
\]
\end{eulerformula}
\begin{eulercomment}
6)\\
\end{eulercomment}
\begin{eulerformula}
\[
(g\circ f)(5)
\]
\end{eulerformula}
\begin{eulercomment}
Penyelesaian:
\end{eulercomment}
\begin{eulerprompt}
>function f(x):= 3*x+1;
>function g(x):= x^2-2*x-6;
>function h(x):= x^3
\end{eulerprompt}
\begin{eulercomment}
Nomor 1
\end{eulercomment}
\begin{eulerprompt}
>f(g(-1))
\end{eulerprompt}
\begin{euleroutput}
        -8.00 
\end{euleroutput}
\begin{eulercomment}
Nomor 2
\end{eulercomment}
\begin{eulerprompt}
>h(f(1))
\end{eulerprompt}
\begin{euleroutput}
        64.00 
\end{euleroutput}
\begin{eulercomment}
Nomor 3
\end{eulercomment}
\begin{eulerprompt}
>f(h(-3))
\end{eulerprompt}
\begin{euleroutput}
       -80.00 
\end{euleroutput}
\begin{eulercomment}
Nomor 4
\end{eulercomment}
\begin{eulerprompt}
>f(f(-4))
\end{eulerprompt}
\begin{euleroutput}
       -32.00 
\end{euleroutput}
\begin{eulercomment}
Nomor 5
\end{eulercomment}
\begin{eulerprompt}
>h(g(3))
\end{eulerprompt}
\begin{euleroutput}
       -27.00 
\end{euleroutput}
\begin{eulercomment}
Nomor 6
\end{eulercomment}
\begin{eulerprompt}
>g(f(5))
\end{eulerprompt}
\begin{euleroutput}
       218.00 
\end{euleroutput}
\eulersubheading{Latihan 3.1}
\begin{eulercomment}
Tentukan penyederhanaan dalam bentuk a+bi dimana a dan b adalah
bilangan real\\
1).\\
\end{eulercomment}
\begin{eulerformula}
\[
(-5+3i)+(7+8i)
\]
\end{eulerformula}
\begin{eulercomment}
2).\\
\end{eulercomment}
\begin{eulerformula}
\[
(12+3i)+(-8+5i)
\]
\end{eulerformula}
\begin{eulercomment}
3)\\
\end{eulercomment}
\begin{eulerformula}
\[
 7i(2-5i)
\]
\end{eulerformula}
\begin{eulercomment}
4)\\
\end{eulercomment}
\begin{eulerformula}
\[
-2i(-8+3i)
\]
\end{eulerformula}
\begin{eulercomment}
5)\\
\end{eulercomment}
\begin{eulerformula}
\[
(13+9i)-(8+2i)
\]
\end{eulerformula}
\begin{eulercomment}
Penyelesaian:\\
Nomor 1
\end{eulercomment}
\begin{eulerprompt}
>$((-5+3*i)+(7+8*i))
\end{eulerprompt}
\begin{eulerformula}
\[
11\,i+2
\]
\end{eulerformula}
\begin{eulercomment}
Nomor 2
\end{eulercomment}
\begin{eulerprompt}
>$((12+3*i)+(-8+5*i))
\end{eulerprompt}
\begin{eulerformula}
\[
8\,i+4
\]
\end{eulerformula}
\begin{eulercomment}
Nomor 3
\end{eulercomment}
\begin{eulerprompt}
>$&expand((7*i)*(2-5*i))
\end{eulerprompt}
\begin{eulerformula}
\[
14\,i-35\,i^2
\]
\end{eulerformula}
\begin{eulercomment}
Nomor 4
\end{eulercomment}
\begin{eulerprompt}
>$&expand((-2*i)*(-8+3*i))
\end{eulerprompt}
\begin{eulerformula}
\[
16\,i-6\,i^2
\]
\end{eulerformula}
\begin{eulercomment}
Nomor 5
\end{eulercomment}
\begin{eulerprompt}
>$((13+9*i)-(8+2*i))
\end{eulerprompt}
\begin{eulerformula}
\[
7\,i+5
\]
\end{eulerformula}
\eulersubheading{Latihan 3.5}
\begin{eulercomment}
Tentukan penyelesaian dan notasi interval untuk himpunan solusi
berikut\\
Nomor 1
\end{eulercomment}
\begin{eulerprompt}
>&load(fourier_elim)
\end{eulerprompt}
\begin{euleroutput}
  
          C:/Program Files/Euler x64/maxima/share/maxima/5.35.1/share/f\(\backslash\)
  ourier_elim/fourier_elim.lisp
  
\end{euleroutput}
\begin{eulerprompt}
>$&fourier_elim(abs(4*x)>20,[x])
\end{eulerprompt}
\begin{eulerformula}
\[
\left[ 5<x \right] \lor \left[ x<-5 \right] 
\]
\end{eulerformula}
\begin{eulercomment}
Nomor 2
\end{eulercomment}
\begin{eulerprompt}
>$&fourier_elim(abs(x+8)<9,[x])
\end{eulerprompt}
\begin{eulerformula}
\[
\left[ -17<x , x<1 \right] 
\]
\end{eulerformula}
\begin{eulercomment}
Nomor 3
\end{eulercomment}
\begin{eulerprompt}
>$&fourier_elim(abs(x-5)>0.1,[x])
\end{eulerprompt}
\begin{eulerformula}
\[
\left[ x<4.9 \right] \lor \left[ 5.1<x \right] 
\]
\end{eulerformula}
\begin{eulercomment}
Nomor 4
\end{eulercomment}
\begin{eulerprompt}
>$&fourier_elim(abs(3*x+5)<0,[x])
\end{eulerprompt}
\begin{eulerformula}
\[
{\it emptyset}
\]
\end{eulerformula}
\begin{eulercomment}
Nomor 5
\end{eulercomment}
\begin{eulerprompt}
>$&fourier_elim(abs(2*x-4)<-5,[x])
\end{eulerprompt}
\begin{eulerformula}
\[
{\it emptyset}
\]
\end{eulerformula}
\end{eulernotebook}

[[[[[[[[\chapter{EMT plot 2D}
\begin{eulernotebook}
\eulerheading{Menggambar Grafik 2D dengan EMT}
\begin{eulercomment}
Notebook ini menjelaskan tentang cara menggambar berbagaikurva dan
grafik 2D dengan software EMT. EMT menyediakan fungsi plot2d() untuk
menggambar berbagai kurva dan grafik dua dimensi (2D).\\
\end{eulercomment}
\eulersubheading{Plot Dasar}
\begin{eulercomment}
Ada fungsi plot yang sangat mendasar. Terdapat koordinat layar yang
selalu berkisar antara 0 hingga 1024 di setiap sumbu, tidak peduli
apakah layarnya berbentuk persegi atau tidak. Semut terdapat koordinat
plot, yang dapat diatur dengan setplot(). Pemetaan antar koordinat
bergantung pada jendela plot saat ini. Misalnya, shrinkwindow()
default menyisakan ruang untuk label sumbu dan judul plot.

Dalam contoh ini, kita hanya menggambar beberapa garis acak dengan
berbagai warna. Untuk rincian tentang fungsi-fungsi ini,\\
pelajari fungsi inti EMT.
\end{eulercomment}
\begin{eulerprompt}
>clg; // clear screen
>window(0,0,1024,1024); // use all of the window
>setplot(0,1,0,1); // set plot coordinates
>hold on; // start overwrite mode
>n=100; X=random(n,2); Y=random(n,2);  // get random points
>colors=rgb(random(n),random(n),random(n)); // get random colors
>loop 1 to n; color(colors[#]); plot(X[#],Y[#]); end; // plot
>hold off; // end overwrite mode
>insimg; // insert to notebook
\end{eulerprompt}
\eulerimg{17}{images/EMT4Plot2D-001.png}
\begin{eulerprompt}
>reset;
\end{eulerprompt}
\begin{eulercomment}
Grafik perlu ditahan, karena perintah plot() akan menghapus jendela
plot.

Untuk menghapus semua yang kami lakukan, kami menggunakan reset().

Untuk menampilkan gambar hasil plot di layar notebook, perintah
plot2d() dapat diakhiri dengan titik dua (:). Cara lain adalah
perintah plot2d() diakhiri dengan titik koma (;), kemudian menggunakan
perintah insimg() untuk menampilkan gambar hasil plot.

Contoh lain, kita menggambar plot sebagai sisipan di plot lain. Hal
ini dilakukan dengan mendefinisikan jendela plot yang lebih
kecil.Perhatikan bahwa jendela ini tidak memberikan ruang untuk label
sumbu di luar jendela plot. Kita harus menambahkan beberapa margin
untuk ini sesuai kebutuhan. Perhatikan bahwa kita menyimpan dan
memulihkan jendela penuh, dan menahan plot saat ini sementara kita
memplot inset.
\end{eulercomment}
\begin{eulerprompt}
>plot2d("x^3-x");
>xw=200; yw=100; ww=300; hw=300;
>ow=window();
>window(xw,yw,xw+ww,yw+hw);
>hold on;
>barclear(xw-50,yw-10,ww+60,ww+60);
>plot2d("x^4-x",grid=6):
\end{eulerprompt}
\eulerimg{27}{images/EMT4Plot2D-002.png}
\begin{eulerprompt}
>hold off;
>window(ow);
\end{eulerprompt}
\begin{eulercomment}
Plot dengan banyak gambar dicapai dengan cara yang sama. Ada fungsi
utilitas figure() untuk ini.

Contoh soal:\\
Tentukan plot dari fungsi berikut:\\
\end{eulercomment}
\begin{eulerformula}
\[
x^2+2x+1
\]
\end{eulerformula}
\begin{eulercomment}
Penyelesaian:
\end{eulercomment}
\begin{eulerprompt}
> plot2d("x^2+2x+1",grid=2):
\end{eulerprompt}
\eulerimg{27}{images/EMT4Plot2D-004.png}
\begin{eulercomment}
\end{eulercomment}
\eulersubheading{Aspek Plot}
\begin{eulercomment}
Plot default menggunakan jendela plot persegi. Anda dapat mengubahnya
dengan fungsi aspek(). Jangan lupa untuk mengatur ulang aspeknya
nanti. Anda juga dapat mengubah default ini di menu dengan "Set
Aspect" ke rasio aspek tertentu atau ke ukuran jendela grafik saat
ini.

Tapi Anda juga bisa mengubahnya untuk satu plot. Untuk ini, ukuran
area plot saat ini diubah, dan jendela diatur sehingga label memiliki
cukup ruang.
\end{eulercomment}
\begin{eulerprompt}
>aspect(2); // rasio panjang dan lebar 2:1
>plot2d(["sin(x)","cos(x)"],0,2pi):
\end{eulerprompt}
\eulerimg{13}{images/EMT4Plot2D-005.png}
\begin{eulerprompt}
>aspect();
>reset;
\end{eulerprompt}
\begin{eulercomment}
Fungsi reset() mengembalikan default plot termasuk rasio aspek.

Contoh soal:\\
Tentukan plot dari fungsi berikut:\\
\end{eulercomment}
\begin{eulerformula}
\[
sin 2x
\]
\end{eulerformula}
\begin{eulerformula}
\[
cos(x)^3
\]
\end{eulerformula}
\begin{eulercomment}
Penyelesaian:
\end{eulercomment}
\begin{eulerprompt}
>aspect(1); // rasio panjang dan lebar 1:1
>plot2d(["sin(2x)","cos(x)^3"],0,2pi):
\end{eulerprompt}
\eulerimg{27}{images/EMT4Plot2D-008.png}
\begin{eulercomment}
\begin{eulercomment}
\eulerheading{Plot 2D di Euler}
\begin{eulercomment}
EMT Math Toolbox memiliki plot dalam 2D, baik untuk data maupun
fungsi. EMT menggunakan fungsi plot2d. Fungsi ini dapat memplot fungsi
dan data.

Dimungkinkan untuk membuat plot di Maxima menggunakan Gnuplot atau
dengan Python menggunakan Math Plot Lib.

Euler dapat membuat plot 2D

- ekspresi\\
- fungsi, variabel, atau kurva berparameter,\\
- vektor nilai x-y,\\
- awan titik di pesawat,\\
- kurva implisit dengan level atau wilayah level.\\
- Fungsi kompleks

Gaya plot mencakup berbagai gaya untuk garis dan titik, plot batang,
dan plot berbayang.\\
\begin{eulercomment}
\eulerheading{Plot Ekspresi atau Variabel}
\begin{eulercomment}
Ekspresi tunggal dalam "x" (misalnya "4*x\textasciicircum{}2") atau nama suatu fungsi\\
(misalnya "f") menghasilkan grafik fungsi tersebut.\\
Berikut adalah contoh paling dasar, yang menggunakan rentang default
dan menetapkan rentang y yang tepat agar sesuai dengan plot fungsinya.\\
Catatan: Jika Anda mengakhiri baris perintah dengan titik dua ":",
plot akan dimasukkan ke dalam jendela teks. Jika tidak, tekan TAB
untuk melihat plot jika jendela plot tertutup.
\end{eulercomment}
\begin{eulerprompt}
>plot2d("x^2"):
\end{eulerprompt}
\eulerimg{27}{images/EMT4Plot2D-009.png}
\begin{eulerprompt}
>aspect(1.5); plot2d("x^3-x"):
\end{eulerprompt}
\eulerimg{17}{images/EMT4Plot2D-010.png}
\begin{eulerprompt}
>a:=5.6; plot2d("exp(-a*x^2)/a"); insimg(30); // menampilkan gambar hasil plot setinggi 25 baris
\end{eulerprompt}
\eulerimg{17}{images/EMT4Plot2D-011.png}
\begin{eulercomment}
Dari beberapa contoh sebelumnya Anda dapat melihat bahwa aslinya
gambar plot menggunakan sumbu X dengan rentang nilai dari -2 sampai
dengan 2. Untuk mengubah rentang nilai X dan Y, Anda dapat menambahkan
nilai-nilai batas X (dan Y) di belakang ekspresi yang digambar.

The plot range is set with the following assigned parameters

- a,b: rentang x (default -2,2)\\
- c,d: rentang y (default: skala dengan nilai)\\
- r: alternatifnya radius di sekitar pusat plot\\
- cx,cy: koordinat pusat plot(default 0,0)
\end{eulercomment}
\begin{eulerprompt}
>plot2d("x^3-x",-1,2):
\end{eulerprompt}
\eulerimg{17}{images/EMT4Plot2D-012.png}
\begin{eulerprompt}
>plot2d("sin(x)",-2*pi,2*pi): // plot sin(x) pada interval [-2pi, 2pi]
\end{eulerprompt}
\eulerimg{17}{images/EMT4Plot2D-013.png}
\begin{eulerprompt}
>plot2d("cos(x)","sin(3*x)",xmin=0,xmax=2pi):
\end{eulerprompt}
\eulerimg{17}{images/EMT4Plot2D-014.png}
\begin{eulercomment}
Alternatif untuk titik dua adalah perintah insimg(baris), yang
menyisipkan plot yang menempati sejumlah baris teks tertentu.

Dalam opsi, plot dapat diatur agar muncul

- di jendela terpisah yang dapat diubah ukurannya,\\
- di jendela buku catatan.

Lebih banyak gaya dapat dicapai dengan perintah plot tertentu.

Bagaimanapun, tekan tombol tabulator untuk melihat plotnya, jika
tersembunyi.\\
Untuk membagi jendela menjadi beberapa plot, gunakan perintah
figure(). Dalam contoh, kita memplot x\textasciicircum{}1 hingga x\textasciicircum{}4 menjadi 4 bagian
jendela. gambar(0) mengatur ulang jendela default.
\end{eulercomment}
\begin{eulerprompt}
>reset;
>figure(2,2); ...
>for n=1 to 4; figure(n); plot2d("x^"+n); end; ...
>figure(0):
\end{eulerprompt}
\eulerimg{27}{images/EMT4Plot2D-015.png}
\begin{eulercomment}
Di plot2d(), ada gaya alternatif yang tersedia dengan grid=x. Untuk
gambaran umum, kami menampilkan berbagai gaya kisi dalam satu gambar
(lihat di bawah untuk perintah figure()). Gaya grid=0 tidak
disertakan. Ini tidak menunjukkan kisi dan bingkai.
\end{eulercomment}
\begin{eulerprompt}
>figure(3,3); ...
>for k=1:9; figure(k); plot2d("x^3-x",-2,1,grid=k); end; ...
>figure(0):
\end{eulerprompt}
\eulerimg{27}{images/EMT4Plot2D-016.png}
\begin{eulercomment}
Jika argumen pada plot2d() adalah ekspresi yang diikuti oleh empat
angka, angka-angka tersebut adalah rentang x dan y untuk plot
tersebut.

Alternatifnya, a, b, c, d dapat ditentukan sebagai parameter yang
ditetapkan sebagai a=... dll.

Pada contoh berikut, kita mengubah gaya kisi, menambahkan label, dan
menggunakan label vertikal untuk sumbu y.
\end{eulercomment}
\begin{eulerprompt}
>aspect(1.5); plot2d("sin(x)",0,2pi,-1.2,1.2,grid=3,xl="x",yl="sin(x)"):
\end{eulerprompt}
\eulerimg{17}{images/EMT4Plot2D-017.png}
\begin{eulerprompt}
>plot2d("sin(x)+cos(2*x)",0,4pi):
\end{eulerprompt}
\eulerimg{17}{images/EMT4Plot2D-018.png}
\begin{eulercomment}
Gambar yang dihasilkan dengan memasukkan plot ke dalam jendela teks
disimpan di direktori yang sama dengan buku catatan, secara default di
subdirektori bernama "gambar". Mereka juga digunakan oleh ekspor HTML.

Anda cukup menandai gambar apa saja dan menyalinnya ke clipboard
dengan Ctrl-C. Tentu saja, Anda juga dapat mengekspor grafik saat ini
dengan fungsi di menu File.

Fungsi atau ekspresi di plot2d dievaluasi secara adaptif. Agar lebih
cepat, nonaktifkan plot adaptif dengan \textless{}adaptive dan tentukan jumlah
subinterval dengan n=... Hal ini hanya diperlukan dalam kasus yang
jarang terjadi.
\end{eulercomment}
\begin{eulerprompt}
>plot2d("sign(x)*exp(-x^2)",-1,1,<adaptive,n=10000):
\end{eulerprompt}
\eulerimg{17}{images/EMT4Plot2D-019.png}
\begin{eulerprompt}
>plot2d("x^x",r=1.2,cx=1,cy=1):
\end{eulerprompt}
\eulerimg{17}{images/EMT4Plot2D-020.png}
\begin{eulercomment}
Perhatikan bahwa x\textasciicircum{}x tidak ditentukan untuk x\textless{}=0. Fungsi plot2d
menangkap kesalahan ini, dan mulai membuat plot segera setelah
fungsinya ditentukan. Ini berfungsi untuk semua fungsi yang
mengembalikan NAN di luar jangkauan definisinya.
\end{eulercomment}
\begin{eulerprompt}
>plot2d("log(x)",-0.1,2):
\end{eulerprompt}
\eulerimg{17}{images/EMT4Plot2D-021.png}
\begin{eulercomment}
Parameter square=true (atau \textgreater{}square) memilih rentang y secara otomatis
sehingga hasilnya adalah jendela plot persegi. Perhatikan bahwa secara
default, Euler menggunakan spasi persegi di dalam jendela plot.
\end{eulercomment}
\begin{eulerprompt}
>plot2d("x^3-x",>square):
\end{eulerprompt}
\eulerimg{17}{images/EMT4Plot2D-022.png}
\begin{eulerprompt}
>plot2d(''integrate("sin(x)*exp(-x^2)",0,x)'',0,2): // plot integral
\end{eulerprompt}
\eulerimg{17}{images/EMT4Plot2D-023.png}
\begin{eulercomment}
Jika Anda memerlukan lebih banyak ruang untuk label y, panggil
shrinkwindow() dengan parameter lebih kecil, atau tetapkan nilai
positif untuk "lebih kecil" di plot2d().
\end{eulercomment}
\begin{eulerprompt}
>plot2d("gamma(x)",1,10,yl="y-values",smaller=6,<vertical):
\end{eulerprompt}
\eulerimg{17}{images/EMT4Plot2D-024.png}
\begin{eulercomment}
Ekspresi simbolik juga dapat digunakan karena disimpan sebagai
ekspresi string sederhana.
\end{eulercomment}
\begin{eulerprompt}
>x=linspace(0,2pi,1000); plot2d(sin(5x),cos(7x)):
\end{eulerprompt}
\eulerimg{17}{images/EMT4Plot2D-025.png}
\begin{eulerprompt}
>a:=5.6; expr &= exp(-a*x^2)/a; // define expression
>plot2d(expr,-2,2): // plot from -2 to 2
\end{eulerprompt}
\eulerimg{17}{images/EMT4Plot2D-026.png}
\begin{eulerprompt}
>plot2d(expr,r=1,thickness=2): // plot in a square around (0,0)
\end{eulerprompt}
\eulerimg{17}{images/EMT4Plot2D-027.png}
\begin{eulerprompt}
>plot2d(&diff(expr,x),>add,style="--",color=red): // add another plot
\end{eulerprompt}
\eulerimg{17}{images/EMT4Plot2D-028.png}
\begin{eulerprompt}
>plot2d(&diff(expr,x,2),a=-2,b=2,c=-2,d=1): // plot in rectangle
\end{eulerprompt}
\eulerimg{17}{images/EMT4Plot2D-029.png}
\begin{eulerprompt}
>plot2d(&diff(expr,x),a=-2,b=2,>square): // keep plot square
\end{eulerprompt}
\eulerimg{17}{images/EMT4Plot2D-030.png}
\begin{eulerprompt}
>plot2d("x^2",0,1,steps=1,color=red,n=10):
\end{eulerprompt}
\eulerimg{17}{images/EMT4Plot2D-031.png}
\begin{eulerprompt}
>plot2d("x^2",>add,steps=2,color=blue,n=10):
\end{eulerprompt}
\eulerimg{17}{images/EMT4Plot2D-032.png}
\begin{eulercomment}
Contoh soal:\\
1. Tentukan plot dari fungsi berikut:\\
\end{eulercomment}
\begin{eulerformula}
\[
x^2+2x+1, -5<x<5
\]
\end{eulerformula}
\begin{eulerprompt}
>plot2d("x^2+2x+1",-5,5): // plot x^2+2x+1 pada interval [-5, 5]
\end{eulerprompt}
\eulerimg{17}{images/EMT4Plot2D-034.png}
\begin{eulercomment}
2. Tentukan plot dari integral fungsi berikut:\\
\end{eulercomment}
\begin{eulerformula}
\[
(cos 2x)(-x^4)
\]
\end{eulerformula}
\begin{eulerprompt}
>plot2d(''integrate("cos(2x)*exp(-x^4)",0,x)'',0,2): // plot integral
\end{eulerprompt}
\eulerimg{17}{images/EMT4Plot2D-036.png}
\begin{eulercomment}
3. Tentukan plot dari fungsi berikut:\\
\end{eulercomment}
\begin{eulerformula}
\[
4x^3-x^2, -1<x<1, x\in Z
\]
\end{eulerformula}
\begin{eulercomment}
Penyelesaian: 
\end{eulercomment}
\begin{eulerprompt}
>figure(3,2); ...
>for k=1:6; figure(k); plot2d("4x^3-x^2",-1,1,grid=k); end; ...
>figure(0):
\end{eulerprompt}
\eulerimg{17}{images/EMT4Plot2D-038.png}
\begin{eulercomment}
4. Tentukan plot fungsi berikut sebagai fungsi tangga\\
\end{eulercomment}
\begin{eulerformula}
\[
2(x)^2
\]
\end{eulerformula}
\begin{eulercomment}
Penyelesaian: 
\end{eulercomment}
\begin{eulerprompt}
> plot2d("2*x^2",-1,1,steps=1,color=red,n=6):
\end{eulerprompt}
\eulerimg{17}{images/EMT4Plot2D-040.png}
\eulerheading{Fungsi dalam satu Parameter}
\begin{eulercomment}
Fungsi plot yang paling penting untuk plot planar adalah plot2d().
Fungsi ini diimplementasikan dalam bahasa Euler di file "plot.e", yang
dimuat di awal program.

Berikut beberapa contoh penggunaan suatu fungsi. Seperti biasa di EMT,
fungsi yang berfungsi untuk fungsi atau ekspresi lain, Anda bisa
meneruskan parameter tambahan (selain x) yang bukan variabel global ke
fungsi dengan parameter titik koma atau dengan kumpulan panggilan.
\end{eulercomment}
\begin{eulerprompt}
>function f(x,a) := x^2/a+a*x^2-x; // define a function
>a=0.3; plot2d("f",0,1;a): // plot with a=0.3
\end{eulerprompt}
\eulerimg{17}{images/EMT4Plot2D-041.png}
\begin{eulerprompt}
>plot2d("f",0,1;0.4): // plot with a=0.4
\end{eulerprompt}
\eulerimg{17}{images/EMT4Plot2D-042.png}
\begin{eulerprompt}
>plot2d(\{\{"f",0.2\}\},0,1): // plot with a=0.2
\end{eulerprompt}
\eulerimg{17}{images/EMT4Plot2D-043.png}
\begin{eulerprompt}
>plot2d(\{\{"f(x,b)",b=0.1\}\},0,1): // plot with 0.1
\end{eulerprompt}
\eulerimg{17}{images/EMT4Plot2D-044.png}
\begin{eulerprompt}
>function f(x) := x^3-x; ...
>plot2d("f",r=1):
\end{eulerprompt}
\eulerimg{17}{images/EMT4Plot2D-045.png}
\begin{eulercomment}
Berikut ini ringkasan fungsi yang diterima

- ekspresi atau ekspresi simbolik di x\\
- fungsi atau fungsi simbolik dengan nama "f"\\
- fungsi simbolik hanya dengan nama f

Fungsi plot2d() juga menerima fungsi simbolik. Untuk fungsi simbolik,
namanya saja yang berfungsi.
\end{eulercomment}
\begin{eulerprompt}
>function f(x) &= diff(x^x,x)
\end{eulerprompt}
\begin{euleroutput}
  
                              x
                             x  (log(x) + 1)
  
\end{euleroutput}
\begin{eulerprompt}
>plot2d(f,0,2):
\end{eulerprompt}
\eulerimg{17}{images/EMT4Plot2D-046.png}
\begin{eulercomment}
Tentu saja, untuk ekspresi atau ekspresi simbolik, nama variabel sudah
cukup untuk memplotnya.
\end{eulercomment}
\begin{eulerprompt}
>expr &= sin(x)*exp(-x)
\end{eulerprompt}
\begin{euleroutput}
  
                                - x
                               E    sin(x)
  
\end{euleroutput}
\begin{eulerprompt}
>plot2d(expr,0,3pi):
\end{eulerprompt}
\eulerimg{17}{images/EMT4Plot2D-047.png}
\begin{eulerprompt}
>function f(x) &= x^x;
>plot2d(f,r=1,cx=1,cy=1,color=blue,thickness=2);
>plot2d(&diff(f(x),x),>add,color=red,style="-.-"):
\end{eulerprompt}
\eulerimg{17}{images/EMT4Plot2D-048.png}
\begin{eulercomment}
Untuk gaya garis ada berbagai pilihan.

- gaya="...". Pilih dari "-", "--", "-.", ".", ".-.", "-.-".\\
- color: Lihat di bawah untuk warna.\\
- ketebalan: Defaultnya adalah 1.

Warna dapat dipilih sebagai salah satu warna default, atau sebagai
warna RGB.

- 0..15: indeks warna default.\\
- konstanta warna: putih, hitam, merah, hijau, biru, cyan, zaitun,
abu-abu muda, abu-abu, abu-abu tua, oranye, hijau muda, pirus, biru
muda, oranye muda, kuning\\
- rgb(merah,hijau,biru): parameternya real di [0,1].
\end{eulercomment}
\begin{eulerprompt}
>plot2d("exp(-x^2)",r=2,color=red,thickness=3,style="--"):
\end{eulerprompt}
\eulerimg{17}{images/EMT4Plot2D-049.png}
\begin{eulercomment}
Berikut adalah tampilan warna EMT yang telah ditentukan sebelumnya.
\end{eulercomment}
\begin{eulerprompt}
>aspect(2); columnsplot(ones(1,16),lab=0:15,grid=0,color=0:15):
\end{eulerprompt}
\eulerimg{13}{images/EMT4Plot2D-050.png}
\begin{eulercomment}
Tapi Anda bisa menggunakan warna apa saja.
\end{eulercomment}
\begin{eulerprompt}
>columnsplot(ones(1,16),grid=0,color=rgb(0,0,linspace(0,1,15))):
\end{eulerprompt}
\eulerimg{13}{images/EMT4Plot2D-051.png}
\begin{eulercomment}
Contoh soal:\\
1. Tentukan plot dari fungsi berikut:\\
\end{eulercomment}
\begin{eulerformula}
\[
f(x)=x^3-2
\]
\end{eulerformula}
\begin{eulercomment}
Penyelesaian:
\end{eulercomment}
\begin{eulerprompt}
>function f(x)&=x^3-2;
>plot2d(f,0,5):
\end{eulerprompt}
\eulerimg{13}{images/EMT4Plot2D-053.png}
\begin{eulercomment}
2. Tentukan plot dari fungsi berikut:\\
\end{eulercomment}
\begin{eulerformula}
\[
f(x)=x^3
\]
\end{eulerformula}
\begin{eulercomment}
Penyelesaian: 
\end{eulercomment}
\begin{eulerprompt}
>function f(x) &= x^3;
>plot2d(f,r=1,cx=1,cy=1,color=green,thickness=3);
>plot2d(&diff(f(x),x),>add,color=red,style="-.-"):
\end{eulerprompt}
\eulerimg{13}{images/EMT4Plot2D-055.png}
\eulersubheading{Menggambar Beberapa Kurva pada bidang koordinat yang sama}
\begin{eulercomment}
Plot lebih dari satu fungsi (multiple function) ke dalam satu jendela
dapat dilakukan dengan berbagai cara. Salah satu metodenya adalah
menggunakan \textgreater{}add untuk beberapa panggilan ke plot2d secara
keseluruhan, kecuali panggilan pertama. Kami telah menggunakan fitur
ini pada contoh di atas.
\end{eulercomment}
\begin{eulerprompt}
>aspect(); plot2d("cos(x)",r=2,grid=6); plot2d("x",style=".",>add):
\end{eulerprompt}
\eulerimg{27}{images/EMT4Plot2D-056.png}
\begin{eulerprompt}
>aspect(1.5); plot2d("sin(x)",0,2pi); plot2d("cos(x)",color=blue,style="--",>add):
\end{eulerprompt}
\eulerimg{17}{images/EMT4Plot2D-057.png}
\begin{eulercomment}
Salah satu kegunaan \textgreater{}add adalah untuk menambahkan titik pada kurva.
\end{eulercomment}
\begin{eulerprompt}
>plot2d("sin(x)",0,pi); plot2d(2,sin(2),>points,>add):
\end{eulerprompt}
\eulerimg{17}{images/EMT4Plot2D-058.png}
\begin{eulercomment}
Kita tambahkan titik perpotongan dengan label (pada posisi "cl" untuk
kiri tengah), dan masukkan hasilnya ke dalam buku catatan. Kami juga
menambahkan judul pada plot.
\end{eulercomment}
\begin{eulerprompt}
>plot2d(["cos(x)","x"],r=1.1,cx=0.5,cy=0.5, ...
>  color=[black,blue],style=["-","."], ...
>  grid=1);
>x0=solve("cos(x)-x",1);  ...
>  plot2d(x0,x0,>points,>add,title="Intersection Demo");  ...
>  label("cos(x) = x",x0,x0,pos="cl",offset=20):
\end{eulerprompt}
\eulerimg{17}{images/EMT4Plot2D-059.png}
\begin{eulercomment}
Dalam demo berikut, kita memplot fungsi sin(x)=sin(x)/x dan ekspansi
Taylor ke-8 dan ke-16. Kami menghitung perluasan ini menggunakan
Maxima melalui ekspresi simbolik.

Plot ini dilakukan dalam perintah multi-baris berikut dengan tiga
panggilan ke plot2d(). Yang kedua dan ketiga memiliki kumpulan tanda
\textgreater{}add, yang membuat plot menggunakan rentang sebelumnya.

Kami menambahkan kotak label yang menjelaskan fungsinya.
\end{eulercomment}
\begin{eulerprompt}
>$taylor(sin(x)/x,x,0,4)
\end{eulerprompt}
\begin{eulerformula}
\[
\frac{x^4}{120}-\frac{x^2}{6}+1
\]
\end{eulerformula}
\begin{eulerprompt}
>plot2d("sinc(x)",0,4pi,color=green,thickness=2); ...
>  plot2d(&taylor(sin(x)/x,x,0,8),>add,color=blue,style="--"); ...
>  plot2d(&taylor(sin(x)/x,x,0,16),>add,color=red,style="-.-"); ...
>  labelbox(["sinc","T8","T16"],styles=["-","--","-.-"], ...
>    colors=[black,blue,red]):
\end{eulerprompt}
\eulerimg{17}{images/EMT4Plot2D-061.png}
\begin{eulercomment}
Dalam contoh berikut, kami menghasilkan Polinomial Bernstein.

\end{eulercomment}
\begin{eulerformula}
\[
B_i(x) = \binom{n}{i} x^i (1-x)^{n-i}
\]
\end{eulerformula}
\begin{eulerprompt}
>plot2d("(1-x)^10",0,1); // plot first function
>for i=1 to 10; plot2d("bin(10,i)*x^i*(1-x)^(10-i)",>add); end;
>insimg;
\end{eulerprompt}
\eulerimg{17}{images/EMT4Plot2D-063.png}
\begin{eulercomment}
Cara kedua adalah dengan menggunakan pasangan matriks bernilai x dan
matriks bernilai y yang berukuran sama.

Kami menghasilkan matriks nilai dengan satu Polinomial Bernstein di
setiap baris. Untuk ini, kita cukup menggunakan vektor kolom i. Lihat
pendahuluan tentang bahasa matriks untuk mempelajari lebih detail.
\end{eulercomment}
\begin{eulerprompt}
>x=linspace(0,1,500);
>n=10; k=(0:n)'; // n is row vector, k is column vector
>y=bin(n,k)*x^k*(1-x)^(n-k); // y is a matrix then
>plot2d(x,y):
\end{eulerprompt}
\eulerimg{17}{images/EMT4Plot2D-064.png}
\begin{eulercomment}
Perhatikan bahwa parameter warna dapat berupa vektor. Kemudian setiap
warna digunakan untuk setiap baris matriks.
\end{eulercomment}
\begin{eulerprompt}
>x=linspace(0,1,200); y=x^(1:10)'; plot2d(x,y,color=1:10):
\end{eulerprompt}
\eulerimg{17}{images/EMT4Plot2D-065.png}
\begin{eulercomment}
Metode lain adalah menggunakan vektor ekspresi (string). Anda kemudian
dapat menggunakan susunan warna, susunan gaya, dan susunan ketebalan
dengan panjang yang sama.
\end{eulercomment}
\begin{eulerprompt}
>plot2d(["sin(x)","cos(x)"],0,2pi,color=4:5): 
\end{eulerprompt}
\eulerimg{17}{images/EMT4Plot2D-066.png}
\begin{eulerprompt}
>plot2d(["sin(x)","cos(x)"],0,2pi): // plot vector of expressions
\end{eulerprompt}
\eulerimg{17}{images/EMT4Plot2D-067.png}
\begin{eulercomment}
Kita bisa mendapatkan vektor seperti itu dari Maxima menggunakan
makelist() dan mxm2str().
\end{eulercomment}
\begin{eulerprompt}
>v &= makelist(binomial(10,i)*x^i*(1-x)^(10-i),i,0,10) // make list
\end{eulerprompt}
\begin{euleroutput}
  
                  10            9              8  2             7  3
          [(1 - x)  , 10 (1 - x)  x, 45 (1 - x)  x , 120 (1 - x)  x , 
             6  4             5  5             4  6             3  7
  210 (1 - x)  x , 252 (1 - x)  x , 210 (1 - x)  x , 120 (1 - x)  x , 
            2  8              9   10
  45 (1 - x)  x , 10 (1 - x) x , x  ]
  
\end{euleroutput}
\begin{eulerprompt}
>mxm2str(v) // get a vector of strings from the symbolic vector
\end{eulerprompt}
\begin{euleroutput}
  (1-x)^10
  10*(1-x)^9*x
  45*(1-x)^8*x^2
  120*(1-x)^7*x^3
  210*(1-x)^6*x^4
  252*(1-x)^5*x^5
  210*(1-x)^4*x^6
  120*(1-x)^3*x^7
  45*(1-x)^2*x^8
  10*(1-x)*x^9
  x^10
\end{euleroutput}
\begin{eulerprompt}
>plot2d(mxm2str(v),0,1): // plot functions
\end{eulerprompt}
\eulerimg{17}{images/EMT4Plot2D-068.png}
\begin{eulercomment}
Alternatif lain adalah dengan menggunakan bahasa matriks Euler.

Jika suatu ekspresi menghasilkan matriks fungsi, dengan satu fungsi di
setiap baris, semua fungsi tersebut akan diplot ke dalam satu plot.

Untuk ini, gunakan vektor parameter dalam bentuk vektor kolom. Jika
array warna ditambahkan maka akan digunakan untuk setiap baris plot.
\end{eulercomment}
\begin{eulerprompt}
>n=(1:10)'; plot2d("x^n",0,1,color=1:10):
\end{eulerprompt}
\eulerimg{17}{images/EMT4Plot2D-069.png}
\begin{eulercomment}
Ekspresi dan fungsi satu baris dapat melihat variabel global.

Jika Anda tidak dapat menggunakan variabel global, Anda perlu
menggunakan fungsi dengan parameter tambahan, dan meneruskan parameter
ini sebagai parameter titik koma.

Berhati-hatilah, untuk meletakkan semua parameter yang ditetapkan di
akhir perintah plot2d. Dalam contoh ini kita meneruskan a=5 ke fungsi
f, yang kita plot dari -10 hingga 10.
\end{eulercomment}
\begin{eulerprompt}
>function f(x,a) := 1/a*exp(-x^2/a); ...
>plot2d("f",-10,10;5,thickness=2,title="a=5"):
\end{eulerprompt}
\eulerimg{17}{images/EMT4Plot2D-070.png}
\begin{eulercomment}
Alternatifnya, gunakan koleksi dengan nama fungsi dan semua parameter
tambahan. Daftar khusus ini disebut kumpulan panggilan, dan ini adalah
cara yang lebih disukai untuk meneruskan argumen ke suatu fungsi yang
kemudian diteruskan sebagai argumen ke fungsi lain.

Pada contoh berikut, kita menggunakan loop untuk memplot beberapa
fungsi (lihat tutorial tentang pemrograman loop).
\end{eulercomment}
\begin{eulerprompt}
>plot2d(\{\{"f",1\}\},-10,10); ...
>for a=2:10; plot2d(\{\{"f",a\}\},>add); end:
\end{eulerprompt}
\eulerimg{17}{images/EMT4Plot2D-071.png}
\begin{eulercomment}
Kita dapat mencapai hasil yang sama dengan cara berikut menggunakan
bahasa matriks EMT. Setiap baris matriks f(x,a) merupakan satu fungsi.
Selain itu, kita dapat mengatur warna untuk setiap baris matriks. Klik
dua kali pada fungsi getspectral() untuk penjelasannya.
\end{eulercomment}
\begin{eulerprompt}
>x=-10:0.01:10; a=(1:10)'; plot2d(x,f(x,a),color=getspectral(a/10)):
\end{eulerprompt}
\eulerimg{17}{images/EMT4Plot2D-072.png}
\begin{eulercomment}
Contoh Soal:\\
1. Tentukan plot dari fungsi berikut:\\
\end{eulercomment}
\begin{eulerformula}
\[
2x^n
\]
\end{eulerformula}
\begin{eulercomment}
Penyelesaian:
\end{eulercomment}
\begin{eulerprompt}
>n=(1:5)'; plot2d("2x^n",0,3,color=1:5):
\end{eulerprompt}
\eulerimg{17}{images/EMT4Plot2D-074.png}
\eulersubheading{Label Teks}
\begin{eulercomment}
Dekorasi sederhana bisa berupa

- judul dengan judul="..."\\
- label x dan y dengan xl="...", yl="..."\\
- label teks lain dengan label("...",x,y)

Perintah label akan memplot ke plot saat ini pada koordinat plot
(x,y). Hal ini memerlukan argumen posisional.
\end{eulercomment}
\begin{eulerprompt}
>plot2d("x^3-x",-1,2,title="y=x^3-x",yl="y",xl="x"):
\end{eulerprompt}
\eulerimg{17}{images/EMT4Plot2D-075.png}
\begin{eulerprompt}
>expr := "log(x)/x"; ...
>  plot2d(expr,0.5,5,title="y="+expr,xl="x",yl="y"); ...
>  label("(1,0)",1,0); label("Max",E,expr(E),pos="lc"):
\end{eulerprompt}
\eulerimg{17}{images/EMT4Plot2D-076.png}
\begin{eulercomment}
Ada juga fungsi labelbox(), yang dapat menampilkan fungsi dan teks.
Dibutuhkan vektor string dan warna, satu item untuk setiap fungsi.
\end{eulercomment}
\begin{eulerprompt}
>function f(x) &= x^2*exp(-x^2);  ...
>plot2d(&f(x),a=-3,b=3,c=-1,d=1);  ...
>plot2d(&diff(f(x),x),>add,color=blue,style="--"); ...
>labelbox(["function","derivative"],styles=["-","--"], ...
>   colors=[black,blue],w=0.4):
\end{eulerprompt}
\eulerimg{17}{images/EMT4Plot2D-077.png}
\begin{eulercomment}
Kotak ini berlabuh di kanan atas secara default, tetapi \textgreater{}kiri berlabuh
di kiri atas. Anda dapat memindahkannya ke tempat mana pun yang Anda
suka. Posisi jangkar berada di pojok kanan atas kotak, dan angkanya
merupakan pecahan dari ukuran jendela grafis. Lebarnya otomatis.

Untuk plot titik, kotak label juga berfungsi. Tambahkan parameter
\textgreater{}points, atau vektor bendera, satu untuk setiap label.

Pada contoh berikut, hanya ada satu fungsi. Jadi kita bisa menggunakan
string sebagai pengganti vektor string. Kami mengatur warna teks
menjadi hitam untuk contoh ini.
\end{eulercomment}
\begin{eulerprompt}
>n=10; plot2d(0:n,bin(n,0:n),>addpoints); ...
>labelbox("Binomials",styles="[]",>points,x=0.1,y=0.1, ...
>tcolor=black,>left):
\end{eulerprompt}
\eulerimg{17}{images/EMT4Plot2D-078.png}
\begin{eulercomment}
Gaya plot ini juga tersedia di statplot(). Seperti di plot2d() warna
dapat diatur untuk setiap baris plot. Masih banyak lagi plot khusus
untuk keperluan statistik (lihat tutorial tentang statistik).
\end{eulercomment}
\begin{eulerprompt}
>statplot(1:10,random(2,10),color=[red,blue]):
\end{eulerprompt}
\eulerimg{17}{images/EMT4Plot2D-079.png}
\begin{eulercomment}
Fitur serupa adalah fungsi textbox().

Lebarnya secara default adalah lebar maksimal baris teks. Tapi itu
bisa diatur oleh pengguna juga.
\end{eulercomment}
\begin{eulerprompt}
>function f(x) &= exp(-x)*sin(2*pi*x); ...
>plot2d("f(x)",0,2pi); ...
>textbox(latex("\(\backslash\)text\{Example of a damped oscillation\}\(\backslash\) f(x)=e^\{-x\}sin(2\(\backslash\)pi x)"),w=0.85):
\end{eulerprompt}
\eulerimg{17}{images/EMT4Plot2D-080.png}
\begin{eulercomment}
Label teks, judul, kotak label, dan teks lainnya dapat berisi string
Unicode (lihat sintaks EMT untuk mengetahui lebih lanjut tentang
string Unicode).
\end{eulercomment}
\begin{eulerprompt}
>plot2d("x^3-x",title=u"x &rarr; x&sup3; - x"):
\end{eulerprompt}
\eulerimg{17}{images/EMT4Plot2D-081.png}
\begin{eulercomment}
Label pada sumbu x dan y bisa vertikal, begitu juga dengan sumbunya.
\end{eulercomment}
\begin{eulerprompt}
>plot2d("sinc(x)",0,2pi,xl="x",yl=u"x &rarr; sinc(x)",>vertical):
\end{eulerprompt}
\eulerimg{17}{images/EMT4Plot2D-082.png}
\begin{eulercomment}
Contoh Soal:\\
1. Tentukan plot dari fungsi berikut\\
\end{eulercomment}
\begin{eulerformula}
\[
y=x^2-2x
\]
\end{eulerformula}
\begin{eulercomment}
Penyelesaian:
\end{eulercomment}
\begin{eulerprompt}
> plot2d("x^2-2x",-1,2,title="y=x^2-2x",yl="y",xl="x"):
\end{eulerprompt}
\eulerimg{17}{images/EMT4Plot2D-084.png}
\begin{eulercomment}
2. Tentukan plot dari fungsi berikut dan turunannya!\\
\end{eulercomment}
\begin{eulerformula}
\[
f(x)= x^3-2x
\]
\end{eulerformula}
\begin{eulercomment}
Penyelesaian:
\end{eulercomment}
\begin{eulerprompt}
>function f(x) &= x^3-2*x; ...
>plot2d(&f(x),a=-3,b=3,c=-1,d=1); ...
>plot2d(&diff(f(x),x),>add,color=red,style="--"); ...
>labelbox(["fungsi","turunan"],styles=["-","--"], ...
>colors=[black,red],w=0.4):
\end{eulerprompt}
\eulerimg{17}{images/EMT4Plot2D-086.png}
\eulersubheading{LaTeX}
\begin{eulercomment}
Anda juga dapat memplot rumus LaTeX jika Anda telah menginstal sistem
LaTeX. Saya merekomendasikan MiKTeX. Jalur ke biner "lateks" dan
"dvipng" harus berada di jalur sistem, atau Anda harus mengatur LaTeX
di menu opsi.

Perhatikan, penguraian LaTeX lambat. Jika Anda ingin menggunakan LaTeX
dalam plot animasi, Anda harus memanggil latex() sebelum loop satu
kali dan menggunakan hasilnya (gambar dalam matriks RGB).

Pada plot berikut, kami menggunakan LaTeX untuk label x dan y, label,
kotak label, dan judul plot.
\end{eulercomment}
\begin{eulerprompt}
>plot2d("exp(-x)*sin(x)/x",a=0,b=2pi,c=0,d=1,grid=6,color=blue, ...
>  title=latex("\(\backslash\)text\{Function $\(\backslash\)Phi$\}"), ...
>  xl=latex("\(\backslash\)phi"),yl=latex("\(\backslash\)Phi(\(\backslash\)phi)")); ...
>textbox( ...
>  latex("\(\backslash\)Phi(\(\backslash\)phi) = e^\{-\(\backslash\)phi\} \(\backslash\)frac\{\(\backslash\)sin(\(\backslash\)phi)\}\{\(\backslash\)phi\}"),x=0.8,y=0.5); ...
>label(latex("\(\backslash\)Phi",color=blue),1,0.4):
\end{eulerprompt}
\eulerimg{17}{images/EMT4Plot2D-087.png}
\begin{eulercomment}
Seringkali, kita menginginkan spasi dan label teks yang tidak
konformal pada sumbu x. Kita bisa menggunakan xaxis() dan yaxis()
seperti yang akan kita tunjukkan nanti.

Cara termudah adalah membuat plot kosong dengan bingkai menggunakan
grid=4, lalu menambahkan grid dengan ygrid() dan xgrid(). Pada contoh
berikut, kami menggunakan tiga string LaTeX untuk label pada sumbu x
dengan xtick().
\end{eulercomment}
\begin{eulerprompt}
>plot2d("sinc(x)",0,2pi,grid=4,<ticks); ...
>ygrid(-2:0.5:2,grid=6); ...
>xgrid([0:2]*pi,<ticks,grid=6);  ...
>xtick([0,pi,2pi],["0","\(\backslash\)pi","2\(\backslash\)pi"],>latex):
\end{eulerprompt}
\eulerimg{17}{images/EMT4Plot2D-088.png}
\begin{eulercomment}
Tentu saja fungsinya juga bisa digunakan.
\end{eulercomment}
\begin{eulerprompt}
>function map f(x) ...
\end{eulerprompt}
\begin{eulerudf}
  if x>0 then return x^4
  else return x^2
  endif
  endfunction
\end{eulerudf}
\begin{eulercomment}
Parameter "map" membantu menggunakan fungsi untuk vektor. Untuk\\
plot, itu tidak perlu. Tapi untuk menunjukkan vektorisasi itu\\
berguna, kita menambahkan beberapa poin penting ke plot di x=-1, x=0
dan x=1.

Pada plot berikut, kami juga memasukkan beberapa kode LaTeX. Kami
menggunakannya untuk dua label dan kotak teks. Tentu saja, Anda hanya
bisa menggunakannya LaTeX jika Anda telah menginstal LaTeX dengan
benar.
\end{eulercomment}
\begin{eulerprompt}
>plot2d("f",-1,1,xl="x",yl="f(x)",grid=6);  ...
>plot2d([-1,0,1],f([-1,0,1]),>points,>add); ...
>label(latex("x^3"),0.72,f(0.72)); ...
>label(latex("x^2"),-0.52,f(-0.52),pos="ll"); ...
>textbox( ...
>  latex("f(x)=\(\backslash\)begin\{cases\} x^3 & x>0 \(\backslash\)\(\backslash\) x^2 & x \(\backslash\)le 0\(\backslash\)end\{cases\}"), ...
>  x=0.7,y=0.2):
\end{eulerprompt}
\eulerimg{17}{images/EMT4Plot2D-089.png}
\eulersubheading{Interaksi pengguna}
\begin{eulercomment}
Saat memplot suatu fungsi atau ekspresi, parameter \textgreater{}pengguna
memungkinkan pengguna untuk memperbesar dan menggeser plot dengan
tombol kursor atau mouse. Pengguna bisa

- perbesar dengan + atau -\\
- pindahkan plot dengan tombol kursor\\
- pilih jendela plot dengan mouse\\
- atur ulang tampilan dengan spasi\\
- keluar dengan kembali

Tombol spasi akan mengatur ulang plot ke jendela plot aslinya.

Saat memplot data, flag \textgreater{}user hanya akan menunggu penekanan tombol.
\end{eulercomment}
\begin{eulerprompt}
>plot2d(\{\{"x^3-a*x",a=1\}\},>user,title="Press any key!"):
\end{eulerprompt}
\eulerimg{17}{images/EMT4Plot2D-090.png}
\begin{eulerprompt}
>plot2d("exp(x)*sin(x)",user=true, ...
>  title="+/- or cursor keys (return to exit)"):
\end{eulerprompt}
\eulerimg{17}{images/EMT4Plot2D-091.png}
\begin{eulercomment}
Berikut ini menunjukkan cara interaksi pengguna tingkat lanjut (lihat
tutorial tentang pemrograman untuk detailnya).

Fungsi bawaan mousedrag() menunggu aktivitas mouse atau keyboard. Ini
melaporkan mouse ke bawah, gerakan mouse atau mouse ke atas, dan
penekanan tombol. Fungsi dragpoints() memanfaatkan ini, dan
memungkinkan pengguna menyeret titik mana pun dalam plot.

Kita membutuhkan fungsi plot terlebih dahulu. Misalnya, kita melakukan
interpolasi pada 5 titik dengan polinomial. Fungsi tersebut harus
diplot ke dalam area plot yang tetap.
\end{eulercomment}
\begin{eulerprompt}
>function plotf(xp,yp,select) ...
\end{eulerprompt}
\begin{eulerudf}
    d=interp(xp,yp);
    plot2d("interpval(xp,d,x)";d,xp,r=2);
    plot2d(xp,yp,>points,>add);
    if select>0 then
      plot2d(xp[select],yp[select],color=red,>points,>add);
    endif;
    title("Drag one point, or press space or return!");
  endfunction
\end{eulerudf}
\begin{eulercomment}
Perhatikan parameter titik koma di plot2d (d dan xp), yang diteruskan
ke evaluasi fungsi interp(). Tanpa ini, kita harus menulis fungsi
plotinterp() terlebih dahulu, mengakses nilainya secara global.

Sekarang kita menghasilkan beberapa nilai acak, dan membiarkan
pengguna menyeret titiknya.
\end{eulercomment}
\begin{eulerprompt}
>t=-1:0.5:1; dragpoints("plotf",t,random(size(t))-0.5):
\end{eulerprompt}
\eulerimg{17}{images/EMT4Plot2D-092.png}
\begin{eulercomment}
Ada juga fungsi yang memplot fungsi lain bergantung pada vektor
parameter, dan memungkinkan pengguna menyesuaikan parameter ini.

Pertama kita membutuhkan fungsi plot.
\end{eulercomment}
\begin{eulerprompt}
>function plotf([a,b]) := plot2d("exp(a*x)*cos(2pi*b*x)",0,2pi;a,b);
\end{eulerprompt}
\begin{eulercomment}
Kemudian kita memerlukan nama untuk parameter, nilai awal dan matriks
rentang nx2, opsional garis judul.\\
Ada penggeser interaktif, yang dapat menetapkan nilai oleh pengguna.
Fungsi dragvalues() menyediakan ini.
\end{eulercomment}
\begin{eulerprompt}
>dragvalues("plotf",["a","b"],[-1,2],[[-2,2];[1,10]], ...
>  heading="Drag these values:",hcolor=black):
\end{eulerprompt}
\eulerimg{17}{images/EMT4Plot2D-093.png}
\begin{eulercomment}
Dimungkinkan untuk membatasi nilai yang diseret menjadi bilangan
bulat. Sebagai contoh, kita menulis fungsi plot, yang memplot
polinomial Taylor berderajat n ke fungsi cosinus.
\end{eulercomment}
\begin{eulerprompt}
>function plotf(n) ...
\end{eulerprompt}
\begin{eulerudf}
  plot2d("cos(x)",0,2pi,>square,grid=6);
  plot2d(&"taylor(cos(x),x,0,@n)",color=blue,>add);
  textbox("Taylor polynomial of degree "+n,0.1,0.02,style="t",>left);
  endfunction
\end{eulerudf}
\begin{eulercomment}
Sekarang kita memperbolehkan derajat n bervariasi dari 0 hingga 20
dalam 20 perhentian. Hasil dragvalues() digunakan untuk memplot sketsa
dengan n ini, dan untuk memasukkan plot ke dalam buku catatan.
\end{eulercomment}
\begin{eulerprompt}
>nd=dragvalues("plotf","degree",2,[0,20],20,y=0.8, ...
>   heading="Drag the value:"); ...
>plotf(nd):
\end{eulerprompt}
\eulerimg{17}{images/EMT4Plot2D-094.png}
\begin{eulercomment}
Berikut ini adalah demonstrasi sederhana dari fungsinya. Pengguna
dapat menggambar jendela plot, meninggalkan jejak titik.
\end{eulercomment}
\begin{eulerprompt}
>function dragtest ...
\end{eulerprompt}
\begin{eulerudf}
    plot2d(none,r=1,title="Drag with the mouse, or press any key!");
    start=0;
    repeat
      \{flag,m,time\}=mousedrag();
      if flag==0 then return; endif;
      if flag==2 then
        hold on; mark(m[1],m[2]); hold off;
      endif;
    end
  endfunction
\end{eulerudf}
\begin{eulerprompt}
>dragtest // lihat hasilnya dan cobalah lakukan!
\end{eulerprompt}
\eulersubheading{Style Plot 2D}
\begin{eulercomment}
Secara default, EMT menghitung tick sumbu otomatis dan menambahkan
label ke setiap tick. Ini dapat diubah dengan parameter grid. Gaya
default sumbu dan label dapat diubah. Selain itu, label dan judul
dapat ditambahkan secara manual. Untuk menyetel ulang ke gaya default,
gunakan reset().
\end{eulercomment}
\begin{eulerprompt}
>aspect();
>figure(3,4); ...
> figure(1); plot2d("x^3-x",grid=0); ... // no grid, frame or axis
> figure(2); plot2d("x^3-x",grid=1); ... // x-y-axis
> figure(3); plot2d("x^3-x",grid=2); ... // default ticks
> figure(4); plot2d("x^3-x",grid=3); ... // x-y- axis with labels inside
> figure(5); plot2d("x^3-x",grid=4); ... // no ticks, only labels
> figure(6); plot2d("x^3-x",grid=5); ... // default, but no margin
> figure(7); plot2d("x^3-x",grid=6); ... // axes only
> figure(8); plot2d("x^3-x",grid=7); ... // axes only, ticks at axis
> figure(9); plot2d("x^3-x",grid=8); ... // axes only, finer ticks at axis
> figure(10); plot2d("x^3-x",grid=9); ... // default, small ticks inside
> figure(11); plot2d("x^3-x",grid=10); ...// no ticks, axes only
> figure(0):
\end{eulerprompt}
\eulerimg{27}{images/EMT4Plot2D-095.png}
\begin{eulercomment}
Parameter \textless{}frame mematikan frame, dan framecolor=blue mengatur frame
menjadi warna biru.

Jika Anda menginginkan tanda centang Anda sendiri, Anda dapat
menggunakan style=0, dan menambahkan semuanya nanti.
\end{eulercomment}
\begin{eulerprompt}
>aspect(1.5); 
>plot2d("x^3-x",grid=0); // plot
>frame; xgrid([-1,0,1]); ygrid(0): // add frame and grid
\end{eulerprompt}
\eulerimg{17}{images/EMT4Plot2D-096.png}
\begin{eulercomment}
Untuk judul plot dan label sumbu, lihat contoh berikut.
\end{eulercomment}
\begin{eulerprompt}
>plot2d("exp(x)",-1,1);
>textcolor(black); // set the text color to black
>title(latex("y=e^x")); // title above the plot
>xlabel(latex("x")); // "x" for x-axis
>ylabel(latex("y"),>vertical); // vertical "y" for y-axis
>label(latex("(0,1)"),0,1,color=blue): // label a point
\end{eulerprompt}
\eulerimg{17}{images/EMT4Plot2D-097.png}
\begin{eulercomment}
Sumbu dapat digambar secara terpisah dengan xaxis() dan yaxis().
\end{eulercomment}
\begin{eulerprompt}
>plot2d("x^3-x",<grid,<frame);
>xaxis(0,xx=-2:1,style="->"); yaxis(0,yy=-5:5,style="->"):
\end{eulerprompt}
\eulerimg{17}{images/EMT4Plot2D-098.png}
\begin{eulercomment}
Teks pada plot dapat diatur dengan label(). Dalam contoh berikut, "lc"
berarti bagian tengah bawah. Ini menetapkan posisi label relatif
terhadap koordinat plot.
\end{eulercomment}
\begin{eulerprompt}
>function f(x) &= x^3-x
\end{eulerprompt}
\begin{euleroutput}
  
                                   3
                                  x  - x
  
\end{euleroutput}
\begin{eulerprompt}
>plot2d(f,-1,1,>square);
>x0=fmin(f,0,1); // compute point of minimum
>label("Rel. Min.",x0,f(x0),pos="lc"): // add a label there
\end{eulerprompt}
\eulerimg{17}{images/EMT4Plot2D-099.png}
\begin{eulercomment}
Ada juga kotak teks.
\end{eulercomment}
\begin{eulerprompt}
>plot2d(&f(x),-1,1,-2,2); // function
>plot2d(&diff(f(x),x),>add,style="--",color=red); // derivative
>labelbox(["f","f'"],["-","--"],[black,red]): // label box
\end{eulerprompt}
\eulerimg{17}{images/EMT4Plot2D-100.png}
\begin{eulerprompt}
>plot2d(["exp(x)","1+x"],color=[black,blue],style=["-","-.-"]):
\end{eulerprompt}
\eulerimg{17}{images/EMT4Plot2D-101.png}
\begin{eulerprompt}
>gridstyle("->",color=gray,textcolor=gray,framecolor=gray);  ...
> plot2d("x^3-x",grid=1);   ...
> settitle("y=x^3-x",color=black); ...
> label("x",2,0,pos="bc",color=gray);  ...
> label("y",0,6,pos="cl",color=gray); ...
> reset():
\end{eulerprompt}
\eulerimg{27}{images/EMT4Plot2D-102.png}
\begin{eulercomment}
Untuk kontrol lebih lanjut, sumbu x dan sumbu y dapat dilakukan secara
manual.

Perintah fullwindow() memperluas jendela plot karena kita tidak lagi
memerlukan tempat untuk label di luar jendela plot. Gunakan
shrinkwindow() atau reset() untuk menyetel ulang ke default.
\end{eulercomment}
\begin{eulerprompt}
>fullwindow; ...
> gridstyle(color=darkgray,textcolor=darkgray); ...
> plot2d(["2^x","1","2^(-x)"],a=-2,b=2,c=0,d=4,<grid,color=4:6,<frame); ...
> xaxis(0,-2:1,style="->"); xaxis(0,2,"x",<axis); ...
> yaxis(0,4,"y",style="->"); ...
> yaxis(-2,1:4,>left); ...
> yaxis(2,2^(-2:2),style=".",<left); ...
> labelbox(["2^x","1","2^-x"],colors=4:6,x=0.8,y=0.2); ...
> reset:
\end{eulerprompt}
\eulerimg{27}{images/EMT4Plot2D-103.png}
\begin{eulercomment}
Berikut adalah contoh lain, di mana string Unicode digunakan dan
sumbunya berada di luar area plot.
\end{eulercomment}
\begin{eulerprompt}
>aspect(1.5); 
>plot2d(["sin(x)","cos(x)"],0,2pi,color=[red,green],<grid,<frame); ...
> xaxis(-1.1,(0:2)*pi,xt=["0",u"&pi;",u"2&pi;"],style="-",>ticks,>zero);  ...
> xgrid((0:0.5:2)*pi,<ticks); ...
> yaxis(-0.1*pi,-1:0.2:1,style="-",>zero,>grid); ...
> labelbox(["sin","cos"],colors=[red,green],x=0.5,y=0.2,>left); ...
> xlabel(u"&phi;"); ylabel(u"f(&phi;)"):
\end{eulerprompt}
\eulerimg{17}{images/EMT4Plot2D-104.png}
\begin{eulercomment}
Contoh soal:\\
1. Tentukan plot dari fungsi berikut:\\
\end{eulercomment}
\begin{eulerformula}
\[
5x^3-2x
\]
\end{eulerformula}
\begin{eulercomment}
Penyelesaian:
\end{eulercomment}
\begin{eulerprompt}
>aspect(1.5);
>plot2d("5x^3-2x",grid=0); // plot
>frame; xgrid([-2,0,2]); ygrid(0): // add frame and grid
\end{eulerprompt}
\eulerimg{17}{images/EMT4Plot2D-106.png}
\eulerheading{Plotting 2D Data}
\begin{eulercomment}
Jika x dan y adalah vektor data, maka data tersebut akan digunakan
sebagai koordinat x dan y pada suatu kurva. Dalam hal ini, a, b, c,
dan d, atau radius r dapat ditentukan, atau jendela plot akan
menyesuaikan secara otomatis dengan data. Alternatifnya, \textgreater{}persegi
dapat diatur untuk mempertahankan rasio aspek persegi.

Merencanakan ekspresi hanyalah singkatan dari plot data. Untuk plot
data, Anda memerlukan satu atau beberapa baris nilai x, dan satu atau
beberapa baris nilai y. Dari rentang dan nilai x, fungsi plot2d akan
menghitung data yang akan diplot, secara default dengan evaluasi
fungsi yang adaptif. Untuk plot titik gunakan "\textgreater{}titik", untuk garis
dan titik campuran gunakan "\textgreater{}addpoints".

Tapi Anda bisa memasukkan data secara langsung.

- Gunakan vektor baris untuk x dan y untuk satu fungsi.\\
- Matriks untuk x dan y diplot baris demi baris.

Berikut adalah contoh dengan satu baris untuk x dan y.
\end{eulercomment}
\begin{eulerprompt}
>x=-10:0.1:10; y=exp(-x^2)*x; plot2d(x,y):
\end{eulerprompt}
\eulerimg{17}{images/EMT4Plot2D-107.png}
\begin{eulercomment}
Data juga dapat diplot sebagai poin. Gunakan points=true untuk ini.
Plotnya berfungsi seperti poligon, tetapi hanya menggambar sudutnya
saja.

- style="...": Pilih dari "[]", "\textless{}\textgreater{}", "o", ".", "..", "+", "*", "[]#",
"\textless{} \textgreater{}#", "o#", "..#", "#", "\textbar{}".

Untuk memplot kumpulan titik, gunakan \textgreater{}titik. Jika warna merupakan
vektor warna, masing-masing titik\\
mendapat warna berbeda. Untuk matriks koordinat dan vektor kolom,
warna diterapkan pada baris matriks.\\
Parameter \textgreater{}addpoints menambahkan titik ke segmen garis untuk plot
data.
\end{eulercomment}
\begin{eulerprompt}
>xdata=[1,1.5,2.5,3,4]; ydata=[3,3.1,2.8,2.9,2.7]; // data
>plot2d(xdata,ydata,a=0.5,b=4.5,c=2.5,d=3.5,style="."); // lines
>plot2d(xdata,ydata,>points,>add,style="o"): // add points
\end{eulerprompt}
\eulerimg{17}{images/EMT4Plot2D-108.png}
\begin{eulerprompt}
>p=polyfit(xdata,ydata,1); // get regression line
>plot2d("polyval(p,x)",>add,color=red): // add plot of line
\end{eulerprompt}
\eulerimg{17}{images/EMT4Plot2D-109.png}
\eulerheading{Menggambar Daerah Yang Dibatasi Kurva}
\begin{eulercomment}
Plot data sebenarnya berbentuk poligon. Kita juga dapat memplot kurva
atau kurva terisi.

- filled=benar mengisi plot.\\
- style="...": Pilih dari "#", "/", "\textbackslash{}", "\textbackslash{}/".\\
- fillcolor: Lihat di atas untuk mengetahui warna yang tersedia.

Warna isian ditentukan oleh argumen "fillcolor", dan pada \textless{}outline
opsional, mencegah menggambar batas untuk semua gaya kecuali gaya
default.
\end{eulercomment}
\begin{eulerprompt}
>t=linspace(0,2pi,1000); // parameter for curve
>x=sin(t)*exp(t/pi); y=cos(t)*exp(t/pi); // x(t) and y(t)
>figure(1,2); aspect(16/9)
>figure(1); plot2d(x,y,r=10); // plot curve
>figure(2); plot2d(x,y,r=10,>filled,style="/",fillcolor=red); // fill curve
>figure(0):
\end{eulerprompt}
\eulerimg{14}{images/EMT4Plot2D-110.png}
\begin{eulercomment}
Dalam contoh berikut kita memplot elips terisi dan dua segi enam
terisi menggunakan kurva tertutup dengan 6 titik dengan gaya isian
berbeda.
\end{eulercomment}
\begin{eulerprompt}
>x=linspace(0,2pi,1000); plot2d(sin(x),cos(x)*0.5,r=1,>filled,style="/"):
\end{eulerprompt}
\eulerimg{14}{images/EMT4Plot2D-111.png}
\begin{eulerprompt}
>t=linspace(0,2pi,6); ...
>plot2d(cos(t),sin(t),>filled,style="/",fillcolor=red,r=1.2):
\end{eulerprompt}
\eulerimg{14}{images/EMT4Plot2D-112.png}
\begin{eulerprompt}
>t=linspace(0,2pi,6); plot2d(cos(t),sin(t),>filled,style="#"):
\end{eulerprompt}
\eulerimg{14}{images/EMT4Plot2D-113.png}
\begin{eulercomment}
Contoh lainnya adalah septagon yang kita buat dengan 7 titik pada
lingkaran satuan
\end{eulercomment}
\begin{eulerprompt}
>t=linspace(0,2pi,7);  ...
> plot2d(cos(t),sin(t),r=1,>filled,style="/",fillcolor=red):
\end{eulerprompt}
\eulerimg{14}{images/EMT4Plot2D-114.png}
\begin{eulercomment}
Berikut adalah himpunan nilai maksimal dari empat kondisi linier yang
kurang dari atau sama dengan 3. Ini adalah A[k].v\textless{}=3 untuk semua baris
A. Untuk mendapatkan sudut yang bagus, kita menggunakan n yang relatif
besar.
\end{eulercomment}
\begin{eulerprompt}
>A=[2,1;1,2;-1,0;0,-1];
>function f(x,y) := max([x,y].A');
>plot2d("f",r=4,level=[0;3],color=green,n=111):
\end{eulerprompt}
\eulerimg{14}{images/EMT4Plot2D-115.png}
\begin{eulercomment}
Poin utama dari bahasa matriks adalah memungkinkan pembuatan tabel
fungsi dengan mudah.
\end{eulercomment}
\begin{eulerprompt}
>t=linspace(0,2pi,1000); x=cos(3*t); y=sin(4*t);
\end{eulerprompt}
\begin{eulercomment}
Kami sekarang memiliki nilai vektor x dan y. plot2d() dapat memplot
nilai-nilai ini sebagai kurva yang menghubungkan titik-titik tersebut.
Plotnya bisa diisi. Pada kasus ini memberikan hasil yang bagus karena
aturan belitan, yang digunakan untuk isi.
\end{eulercomment}
\begin{eulerprompt}
>plot2d(x,y,<grid,<frame,>filled):
\end{eulerprompt}
\eulerimg{14}{images/EMT4Plot2D-116.png}
\begin{eulercomment}
Vektor interval diplot terhadap nilai x sebagai wilayah terisi\\
antara nilai interval yang lebih rendah dan lebih tinggi.

Hal ini dapat berguna untuk memplot kesalahan perhitungan. Tapi itu \\
bisa juga dapat digunakan untuk memplot kesalahan statistik.
\end{eulercomment}
\begin{eulerprompt}
>t=0:0.1:1; ...
> plot2d(t,interval(t-random(size(t)),t+random(size(t))),style="|");  ...
> plot2d(t,t,add=true):
\end{eulerprompt}
\eulerimg{14}{images/EMT4Plot2D-117.png}
\begin{eulercomment}
Jika x adalah vektor yang diurutkan, dan y adalah vektor interval,
maka plot2d akan memplot rentang interval yang terisi pada bidang.
Gaya isiannya sama dengan gaya poligon.
\end{eulercomment}
\begin{eulerprompt}
>t=-1:0.01:1; x=~t-0.01,t+0.01~; y=x^3-x;
>plot2d(t,y):
\end{eulerprompt}
\eulerimg{14}{images/EMT4Plot2D-118.png}
\begin{eulercomment}
Dimungkinkan untuk mengisi wilayah nilai untuk fungsi tertentu. Untuk\\
ini, level harus berupa matriks 2xn. Baris pertama adalah batas bawah\\
dan baris kedua berisi batas atas.
\end{eulercomment}
\begin{eulerprompt}
>expr := "2*x^2+x*y+3*y^4+y"; // define an expression f(x,y)
>plot2d(expr,level=[0;1],style="-",color=blue): // 0 <= f(x,y) <= 1
\end{eulerprompt}
\eulerimg{14}{images/EMT4Plot2D-119.png}
\begin{eulercomment}
Kita juga dapat mengisi rentang nilai seperti

\end{eulercomment}
\begin{eulerformula}
\[
-1 \le (x^2+y^2)^2-x^2+y^2 \le 0.
\]
\end{eulerformula}
\begin{eulercomment}
\end{eulercomment}
\begin{eulerprompt}
>plot2d("(x^2+y^2)^2-x^2+y^2",r=1.2,level=[-1;0],style="/"):
\end{eulerprompt}
\eulerimg{14}{images/EMT4Plot2D-121.png}
\begin{eulerprompt}
>plot2d("cos(x)","sin(x)^3",xmin=0,xmax=2pi,>filled,style="/"):
\end{eulerprompt}
\eulerimg{14}{images/EMT4Plot2D-122.png}
\begin{eulercomment}
Contoh soal:\\
1. Tentukan plot dari fungsi berikut:\\
\end{eulercomment}
\begin{eulerformula}
\[
f(x)= cos(t)
\]
\end{eulerformula}
\begin{eulerformula}
\[
f(x)= sin(2t)
\]
\end{eulerformula}
\begin{eulercomment}
Penyelesaian:
\end{eulercomment}
\begin{eulerprompt}
>t=linspace(0,4pi,8); plot2d(cos(t),sin(2t),>filled,style="#",fillcolor=red):
\end{eulerprompt}
\eulerimg{14}{images/EMT4Plot2D-125.png}
\eulerheading{Grafik Fungsi Parametrik}
\begin{eulercomment}
Nilai x tidak perlu diurutkan. (x,y) hanya menggambarkan sebuah kurva.
Jika x diurutkan, kurva tersebut merupakan grafik suatu fungsi.

Dalam contoh berikut, kita memplot spiral

\end{eulercomment}
\begin{eulerformula}
\[
\gamma(t) = t \cdot (\cos(2\pi t),\sin(2\pi t))
\]
\end{eulerformula}
\begin{eulercomment}
Kita perlu menggunakan banyak titik untuk tampilan yang halus atau
fungsi adaptif() untuk mengevaluasi ekspresi (lihat fungsi adaptif()
untuk lebih jelasnya).
\end{eulercomment}
\begin{eulerprompt}
>t=linspace(0,1,1000); ...
>plot2d(t*cos(2*pi*t),t*sin(2*pi*t),r=1):
\end{eulerprompt}
\eulerimg{14}{images/EMT4Plot2D-127.png}
\begin{eulercomment}
Sebagai alternatif, dimungkinkan untuk menggunakan dua ekspresi untuk
kurva. Berikut ini plot kurva yang sama seperti di atas.
\end{eulercomment}
\begin{eulerprompt}
>plot2d("x*cos(2*pi*x)","x*sin(2*pi*x)",xmin=0,xmax=1,r=1):
\end{eulerprompt}
\eulerimg{14}{images/EMT4Plot2D-128.png}
\begin{eulerprompt}
>t=linspace(0,1,1000); r=exp(-t); x=r*cos(2pi*t); y=r*sin(2pi*t);
>plot2d(x,y,r=1):
\end{eulerprompt}
\eulerimg{14}{images/EMT4Plot2D-129.png}
\begin{eulercomment}
Pada contoh berikutnya, kita memplot kurvanya

\end{eulercomment}
\begin{eulerformula}
\[
\gamma(t) = (r(t) \cos(t), r(t) \sin(t))
\]
\end{eulerformula}
\begin{eulercomment}
dengan

\end{eulercomment}
\begin{eulerformula}
\[
r(t) = 1 + \dfrac{\sin(3t)}{2}.
\]
\end{eulerformula}
\begin{eulerprompt}
>t=linspace(0,2pi,1000); r=1+sin(3*t)/2; x=r*cos(t); y=r*sin(t); ...
>plot2d(x,y,>filled,fillcolor=red,style="/",r=1.5):
\end{eulerprompt}
\eulerimg{14}{images/EMT4Plot2D-132.png}
\eulerheading{Menggambar Grafik Bilangan Kompleks}
\begin{eulercomment}
Serangkaian bilangan kompleks juga dapat diplot. Kemudian titik-titik
grid akan dihubungkan. Jika sejumlah garis kisi ditentukan (atau
vektor garis kisi 1x2) dalam argumen cgrid, hanya garis kisi tersebut
yang terlihat.

Matriks bilangan kompleks secara otomatis akan diplot sebagai
kisi-kisi pada bidang kompleks.

Pada contoh berikut, kita memplot gambar lingkaran satuan di bawah
fungsi eksponensial. Parameter cgrid menyembunyikan beberapa kurva
grid.
\end{eulercomment}
\begin{eulerprompt}
>aspect(); r=linspace(0,1,50); a=linspace(0,2pi,80)'; z=r*exp(I*a);...
>plot2d(z,a=-1.25,b=1.25,c=-1.25,d=1.25,cgrid=10):
\end{eulerprompt}
\eulerimg{27}{images/EMT4Plot2D-133.png}
\begin{eulerprompt}
>aspect(1.25); r=linspace(0,1,50); a=linspace(0,2pi,200)'; z=r*exp(I*a);
>plot2d(exp(z),cgrid=[40,10]):
\end{eulerprompt}
\eulerimg{21}{images/EMT4Plot2D-134.png}
\begin{eulerprompt}
>r=linspace(0,1,10); a=linspace(0,2pi,40)'; z=r*exp(I*a);
>plot2d(exp(z),>points,>add):
\end{eulerprompt}
\eulerimg{21}{images/EMT4Plot2D-135.png}
\begin{eulercomment}
Vektor bilangan kompleks secara otomatis diplot sebagai kurva pada
bidang kompleks dengan bagian nyata dan bagian imajiner.

Dalam contoh, kita memplot lingkaran satuan dengan

\end{eulercomment}
\begin{eulerformula}
\[
\gamma(t) = e^{it}
\]
\end{eulerformula}
\begin{eulerprompt}
>t=linspace(0,2pi,1000); ...
>plot2d(exp(I*t)+exp(4*I*t),r=2):
\end{eulerprompt}
\eulerimg{21}{images/EMT4Plot2D-137.png}
\eulerheading{Plot Statistik}
\begin{eulercomment}
Ada banyak fungsi yang dikhususkan pada plot statistik. Salah satu
plot yang sering digunakan adalah plot kolom.

Jumlah kumulatif dari nilai terdistribusi normal 0-1 menghasilkan
jalan acak.
\end{eulercomment}
\begin{eulerprompt}
>plot2d(cumsum(randnormal(1,1000))):
\end{eulerprompt}
\eulerimg{21}{images/EMT4Plot2D-138.png}
\begin{eulercomment}
Penggunaan dua baris menunjukkan jalan dalam dua dimensi.
\end{eulercomment}
\begin{eulerprompt}
>X=cumsum(randnormal(2,1000)); plot2d(X[1],X[2]):
\end{eulerprompt}
\eulerimg{21}{images/EMT4Plot2D-139.png}
\begin{eulerprompt}
>columnsplot(cumsum(random(10)),style="/",color=blue):
\end{eulerprompt}
\eulerimg{21}{images/EMT4Plot2D-140.png}
\begin{eulercomment}
Itu juga dapat menampilkan string sebagai label.
\end{eulercomment}
\begin{eulerprompt}
>months=["Jan","Feb","Mar","Apr","May","Jun", ...
>  "Jul","Aug","Sep","Oct","Nov","Dec"];
>values=[10,12,12,18,22,28,30,26,22,18,12,8];
>columnsplot(values,lab=months,color=red,style="-");
>title("Temperature"):
\end{eulerprompt}
\eulerimg{21}{images/EMT4Plot2D-141.png}
\begin{eulerprompt}
>k=0:10;
>plot2d(k,bin(10,k),>bar):
\end{eulerprompt}
\eulerimg{21}{images/EMT4Plot2D-142.png}
\begin{eulerprompt}
>plot2d(k,bin(10,k)); plot2d(k,bin(10,k),>points,>add):
\end{eulerprompt}
\eulerimg{21}{images/EMT4Plot2D-143.png}
\begin{eulerprompt}
>plot2d(normal(1000),normal(1000),>points,grid=6,style=".."):
\end{eulerprompt}
\eulerimg{21}{images/EMT4Plot2D-144.png}
\begin{eulerprompt}
>plot2d(normal(1,1000),>distribution,style="O"):
\end{eulerprompt}
\eulerimg{21}{images/EMT4Plot2D-145.png}
\begin{eulerprompt}
>plot2d("qnormal",0,5;2.5,0.5,>filled):
\end{eulerprompt}
\eulerimg{21}{images/EMT4Plot2D-146.png}
\begin{eulercomment}
Untuk memplot distribusi statistik eksperimental, Anda dapat
menggunakan distribution=n dengan plot2d.
\end{eulercomment}
\begin{eulerprompt}
>w=randexponential(1,1000); // exponential distribution
>plot2d(w,>distribution): // or distribution=n with n intervals
\end{eulerprompt}
\eulerimg{21}{images/EMT4Plot2D-147.png}
\begin{eulercomment}
Atau Anda dapat menghitung distribusi dari data dan memplot hasilnya
dengan \textgreater{}bar di plot3d, atau dengan plot kolom.
\end{eulercomment}
\begin{eulerprompt}
>w=normal(1000); // 0-1-normal distribution
>\{x,y\}=histo(w,10,v=[-6,-4,-2,-1,0,1,2,4,6]); // interval bounds v
>plot2d(x,y,>bar):
\end{eulerprompt}
\eulerimg{21}{images/EMT4Plot2D-148.png}
\begin{eulercomment}
Fungsi statplot() mengatur gaya dengan string sederhana.
\end{eulercomment}
\begin{eulerprompt}
>statplot(1:10,cumsum(random(10)),"b"):
\end{eulerprompt}
\eulerimg{21}{images/EMT4Plot2D-149.png}
\begin{eulerprompt}
>n=10; i=0:n; ...
>plot2d(i,bin(n,i)/2^n,a=0,b=10,c=0,d=0.3); ...
>plot2d(i,bin(n,i)/2^n,points=true,style="ow",add=true,color=blue):
\end{eulerprompt}
\eulerimg{21}{images/EMT4Plot2D-150.png}
\begin{eulercomment}
Selain itu, data dapat diplot sebagai batang. Dalam hal ini, x harus
diurutkan dan satu elemen lebih panjang dari y. Batangnya akan
memanjang dari x[i] hingga x[i+1] dengan nilai y[i]. Jika x berukuran
sama dengan y, maka x akan diperpanjang satu elemen dengan spasi
terakhir.

Gaya isian dapat digunakan seperti di atas.
\end{eulercomment}
\begin{eulerprompt}
>n=10; k=bin(n,0:n); ...
>plot2d(-0.5:n+0.5,k,bar=true,fillcolor=lightgray):
\end{eulerprompt}
\eulerimg{21}{images/EMT4Plot2D-151.png}
\begin{eulercomment}
Data untuk plot batang (batang=1) dan histogram (histogram=1) dapat
diberikan secara eksplisit dalam xv dan yv, atau dapat dihitung dari
distribusi empiris dalam xv dengan \textgreater{}distribusi (atau distribusi=n).
Histogram nilai xv akan dihitung secara otomatis dengan \textgreater{}histogram.
Jika \textgreater{}even ditentukan, nilai xv akan dihitung dalam interval bilangan
bulat.
\end{eulercomment}
\begin{eulerprompt}
>plot2d(normal(10000),distribution=50):
\end{eulerprompt}
\eulerimg{21}{images/EMT4Plot2D-152.png}
\begin{eulerprompt}
>k=0:10; m=bin(10,k); x=(0:11)-0.5; plot2d(x,m,>bar):
\end{eulerprompt}
\eulerimg{21}{images/EMT4Plot2D-153.png}
\begin{eulerprompt}
>columnsplot(m,k):
\end{eulerprompt}
\eulerimg{21}{images/EMT4Plot2D-154.png}
\begin{eulerprompt}
>plot2d(random(600)*6,histogram=6):
\end{eulerprompt}
\eulerimg{21}{images/EMT4Plot2D-155.png}
\begin{eulercomment}
Untuk distribusi, terdapat parameter distribution=n, yang menghitung
nilai secara otomatis dan mencetak distribusi relatif dengan n
sub-interval.
\end{eulercomment}
\begin{eulerprompt}
>plot2d(normal(1,1000),distribution=10,style="\(\backslash\)/"):
\end{eulerprompt}
\eulerimg{21}{images/EMT4Plot2D-156.png}
\begin{eulercomment}
Dengan parameter even=true, ini akan menggunakan interval bilangan
bulat.
\end{eulercomment}
\begin{eulerprompt}
>plot2d(intrandom(1,1000,10),distribution=10,even=true):
\end{eulerprompt}
\eulerimg{21}{images/EMT4Plot2D-157.png}
\begin{eulercomment}
Perhatikan bahwa ada banyak plot statistik yang mungkin berguna.
Silahkan lihat tutorial tentang statistik.
\end{eulercomment}
\begin{eulerprompt}
>columnsplot(getmultiplicities(1:6,intrandom(1,6000,6))):
\end{eulerprompt}
\eulerimg{21}{images/EMT4Plot2D-158.png}
\begin{eulerprompt}
>plot2d(normal(1,1000),>distribution); ...
>  plot2d("qnormal(x)",color=red,thickness=2,>add):
\end{eulerprompt}
\eulerimg{21}{images/EMT4Plot2D-159.png}
\begin{eulercomment}
Ada juga banyak plot khusus untuk statistik. Plot kotak menunjukkan
kuartil distribusi ini dan banyak outlier. Menurut definisinya,
outlier dalam plot kotak adalah data yang melebihi 1,5 kali rentang
50\% tengah plot.
\end{eulercomment}
\begin{eulerprompt}
>M=normal(5,1000); boxplot(quartiles(M)):
\end{eulerprompt}
\eulerimg{21}{images/EMT4Plot2D-160.png}
\begin{eulercomment}
Contoh Soal:\\
1. Tentukan plot dalam bentuk histogram sebanyak 6!\\
Penyelesaian:
\end{eulercomment}
\begin{eulerprompt}
>columnsplot(cumsum(random(6)),style="/",color=yellow):
\end{eulerprompt}
\eulerimg{21}{images/EMT4Plot2D-161.png}
\eulerheading{Fungsi Implisit}
\begin{eulercomment}
Plot implisit menunjukkan penyelesaian garis level f(x,y)=level,
dimana "level" dapat berupa nilai tunggal atau vektor nilai. Jika
level = "auto", akan ada garis level nc, yang akan tersebar antara
fungsi minimum dan maksimum secara merata. Warna yang lebih gelap atau
lebih terang dapat ditambahkan dengan \textgreater{}hue untuk menunjukkan nilai
fungsi. Untuk fungsi implisit, xv harus berupa fungsi atau ekspresi
parameter x dan y, atau alternatifnya, xv dapat berupa matriks nilai.

Euler dapat menandai garis level

\end{eulercomment}
\begin{eulerformula}
\[
f(x,y) = c
\]
\end{eulerformula}
\begin{eulercomment}
dari fungsi apa pun.

Untuk menggambar himpunan f(x,y)=c untuk satu atau lebih konstanta c,
Anda dapat menggunakan plot2d() dengan plot implisitnya pada bidang.
Parameter c adalah level=c, dimana c dapat berupa vektor garis level.
Selain itu, skema warna dapat digambar di latar belakang untuk
menunjukkan nilai fungsi setiap titik dalam plot. Parameter "n"
menentukan kehalusan plot.
\end{eulercomment}
\begin{eulerprompt}
>aspect(1.5); 
>plot2d("x^2+y^2-x*y-x",r=1.5,level=0,contourcolor=red):
\end{eulerprompt}
\eulerimg{17}{images/EMT4Plot2D-163.png}
\begin{eulerprompt}
>expr := "2*x^2+x*y+3*y^4+y"; // define an expression f(x,y)
>plot2d(expr,level=0): // Solutions of f(x,y)=0
\end{eulerprompt}
\eulerimg{17}{images/EMT4Plot2D-164.png}
\begin{eulerprompt}
>plot2d(expr,level=0:0.5:20,>hue,contourcolor=white,n=200): // nice
\end{eulerprompt}
\eulerimg{17}{images/EMT4Plot2D-165.png}
\begin{eulerprompt}
>plot2d(expr,level=0:0.5:20,>hue,>spectral,n=200,grid=4): // nicer
\end{eulerprompt}
\eulerimg{17}{images/EMT4Plot2D-166.png}
\begin{eulercomment}
Ini juga berfungsi untuk plot data. Namun Anda harus menentukan
rentangnya untuk label sumbu.
\end{eulercomment}
\begin{eulerprompt}
>x=-2:0.05:1; y=x'; z=expr(x,y);
>plot2d(z,level=0,a=-1,b=2,c=-2,d=1,>hue):
\end{eulerprompt}
\eulerimg{17}{images/EMT4Plot2D-167.png}
\begin{eulerprompt}
>plot2d("x^3-y^2",>contour,>hue,>spectral):
\end{eulerprompt}
\eulerimg{17}{images/EMT4Plot2D-168.png}
\begin{eulerprompt}
>plot2d("x^3-y^2",level=0,contourwidth=3,>add,contourcolor=red):
\end{eulerprompt}
\eulerimg{17}{images/EMT4Plot2D-169.png}
\begin{eulerprompt}
>z=z+normal(size(z))*0.2;
>plot2d(z,level=0.5,a=-1,b=2,c=-2,d=1):
\end{eulerprompt}
\eulerimg{17}{images/EMT4Plot2D-170.png}
\begin{eulerprompt}
>plot2d(expr,level=[0:0.2:5;0.05:0.2:5.05],color=lightgray):
\end{eulerprompt}
\eulerimg{17}{images/EMT4Plot2D-171.png}
\begin{eulerprompt}
>plot2d("x^2+y^3+x*y",level=1,r=4,n=100):
\end{eulerprompt}
\eulerimg{17}{images/EMT4Plot2D-172.png}
\begin{eulerprompt}
>plot2d("x^2+2*y^2-x*y",level=0:0.1:10,n=100,contourcolor=white,>hue):
\end{eulerprompt}
\eulerimg{17}{images/EMT4Plot2D-173.png}
\begin{eulercomment}
Dimungkinkan juga untuk mengisi set

\end{eulercomment}
\begin{eulerformula}
\[
a \le f(x,y) \le b
\]
\end{eulerformula}
\begin{eulercomment}
dengan rentang level.

Dimungkinkan untuk mengisi wilayah nilai untuk fungsi tertentu. Untuk
ini, level harus berupa matriks 2xn. Baris pertama adalah batas bawah
dan baris kedua berisi batas atas.
\end{eulercomment}
\begin{eulerprompt}
>plot2d(expr,level=[0;1],style="-",color=blue): // 0 <= f(x,y) <= 1
\end{eulerprompt}
\eulerimg{17}{images/EMT4Plot2D-175.png}
\begin{eulercomment}
Plot implisit juga dapat menunjukkan rentang level. Maka level harus
berupa matriks interval level 2xn, di mana baris pertama berisi awal
dan baris kedua berisi akhir setiap interval. Alternatifnya, vektor
baris sederhana dapat digunakan untuk level, dan parameter dl
memperluas nilai level ke interval.
\end{eulercomment}
\begin{eulerprompt}
>plot2d("x^4+y^4",r=1.5,level=[0;1],color=blue,style="/"):
\end{eulerprompt}
\eulerimg{17}{images/EMT4Plot2D-176.png}
\begin{eulerprompt}
>plot2d("x^2+y^3+x*y",level=[0,2,4;1,3,5],style="/",r=2,n=100):
\end{eulerprompt}
\eulerimg{17}{images/EMT4Plot2D-177.png}
\begin{eulerprompt}
>plot2d("x^2+y^3+x*y",level=-10:20,r=2,style="-",dl=0.1,n=100):
\end{eulerprompt}
\eulerimg{17}{images/EMT4Plot2D-178.png}
\begin{eulerprompt}
>plot2d("sin(x)*cos(y)",r=pi,>hue,>levels,n=100):
\end{eulerprompt}
\eulerimg{17}{images/EMT4Plot2D-179.png}
\begin{eulercomment}
Dimungkinkan juga untuk menandai suatu wilayah

\end{eulercomment}
\begin{eulerformula}
\[
a \le f(x,y) \le b.
\]
\end{eulerformula}
\begin{eulercomment}
Hal ini dilakukan dengan menambahkan level dengan dua baris.
\end{eulercomment}
\begin{eulerprompt}
>plot2d("(x^2+y^2-1)^3-x^2*y^3",r=1.3, ...
>  style="#",color=red,<outline, ...
>  level=[-2;0],n=100):
\end{eulerprompt}
\eulerimg{17}{images/EMT4Plot2D-181.png}
\begin{eulercomment}
Dimungkinkan untuk menentukan level tertentu. Misalnya, kita dapat
memplot solusi persamaan seperti

\end{eulercomment}
\begin{eulerformula}
\[
x^3-xy+x^2y^2=6
\]
\end{eulerformula}
\begin{eulerprompt}
>plot2d("x^3-x*y+x^2*y^2",r=6,level=1,n=100):
\end{eulerprompt}
\eulerimg{17}{images/EMT4Plot2D-183.png}
\begin{eulerprompt}
>function starplot1 (v, style="/", color=green, lab=none) ...
\end{eulerprompt}
\begin{eulerudf}
    if !holding() then clg; endif;
    w=window(); window(0,0,1024,1024);
    h=holding(1);
    r=max(abs(v))*1.2;
    setplot(-r,r,-r,r);
    n=cols(v); t=linspace(0,2pi,n);
    v=v|v[1]; c=v*cos(t); s=v*sin(t);
    cl=barcolor(color); st=barstyle(style);
    loop 1 to n
      polygon([0,c[#],c[#+1]],[0,s[#],s[#+1]],1);
      if lab!=none then
        rlab=v[#]+r*0.1;
        \{col,row\}=toscreen(cos(t[#])*rlab,sin(t[#])*rlab);
        ctext(""+lab[#],col,row-textheight()/2);
      endif;
    end;
    barcolor(cl); barstyle(st);
    holding(h);
    window(w);
  endfunction
\end{eulerudf}
\begin{eulercomment}
Tidak ada tanda centang kotak atau sumbu di sini. Selain itu, kami
menggunakan jendela penuh untuk plotnya.

Kami memanggil reset sebelum kami menguji plot ini untuk mengembalikan
default grafis. Ini tidak perlu dilakukan jika Anda yakin plot Anda
berhasil.
\end{eulercomment}
\begin{eulerprompt}
>reset; starplot1(normal(1,10)+5,color=red,lab=1:10):
\end{eulerprompt}
\eulerimg{27}{images/EMT4Plot2D-184.png}
\begin{eulercomment}
Terkadang, Anda mungkin ingin merencanakan sesuatu yang plot2d tidak
bisa lakukan, tapi hampir.

Dalam fungsi berikut, kita membuat plot impuls logaritmik. plot2d
dapat melakukan plot logaritmik, tetapi tidak untuk batang impuls.
\end{eulercomment}
\begin{eulerprompt}
>function logimpulseplot1 (x,y) ...
\end{eulerprompt}
\begin{eulerudf}
    \{x0,y0\}=makeimpulse(x,log(y)/log(10));
    plot2d(x0,y0,>bar,grid=0);
    h=holding(1);
    frame();
    xgrid(ticks(x));
    p=plot();
    for i=-10 to 10;
      if i<=p[4] and i>=p[3] then
         ygrid(i,yt="10^"+i);
      endif;
    end;
    holding(h);
  endfunction
\end{eulerudf}
\begin{eulercomment}
Mari kita uji dengan nilai yang terdistribusi secara eksponensial.
\end{eulercomment}
\begin{eulerprompt}
>aspect(1.5); x=1:10; y=-log(random(size(x)))*200; ...
>logimpulseplot1(x,y):
\end{eulerprompt}
\eulerimg{17}{images/EMT4Plot2D-185.png}
\begin{eulercomment}
Mari kita menganimasikan kurva 2D menggunakan plot langsung. Perintah
plot(x,y) hanya memplot kurva ke dalam jendela plot. setplot(a,b,c,d)
menyetel jendela ini.

Fungsi wait(0) memaksa plot muncul di jendela grafis. Jika tidak,
pengundian ulang akan dilakukan dalam interval waktu yang jarang.
\end{eulercomment}
\begin{eulerprompt}
>function animliss (n,m) ...
\end{eulerprompt}
\begin{eulerudf}
  t=linspace(0,2pi,500);
  f=0;
  c=framecolor(0);
  l=linewidth(2);
  setplot(-1,1,-1,1);
  repeat
    clg;
    plot(sin(n*t),cos(m*t+f));
    wait(0);
    if testkey() then break; endif;
    f=f+0.02;
  end;
  framecolor(c);
  linewidth(l);
  endfunction
\end{eulerudf}
\begin{eulercomment}
Tekan tombol apa saja untuk menghentikan animasi ini.
\end{eulercomment}
\begin{eulerprompt}
>animliss(2,3); // lihat hasilnya, jika sudah puas, tekan ENTER
\end{eulerprompt}
\begin{eulercomment}
Contoh Soal\\
1. Tentukan plot dari fungsi implisit berikut!\\
\end{eulercomment}
\begin{eulerformula}
\[
2x^5+y^4+xy
\]
\end{eulerformula}
\begin{eulercomment}
Penyelesaian:
\end{eulercomment}
\begin{eulerprompt}
>plot2d("2x^5+y^4+x*y",level=[0,3,6;1,3,5],style="/",r=2,n=100):
\end{eulerprompt}
\eulerimg{17}{images/EMT4Plot2D-187.png}
\eulerheading{Plot Logaritma}
\begin{eulercomment}
EMT menggunakan parameter "logplot" untuk skala logaritmik.\\
Plot logaritma dapat diplot menggunakan skala logaritma di y dengan
logplot=1, atau menggunakan skala logaritma di x dan y dengan
logplot=2, atau di x dengan logplot=3.

\end{eulercomment}
\begin{eulerttcomment}
 - logplot=1: y-logaritma
 - logplot=2: x-y-logaritma
 - logplot=3: x-logaritma
\end{eulerttcomment}
\begin{eulerprompt}
>plot2d("exp(x^3-x)*x^2",1,5,logplot=1):
\end{eulerprompt}
\eulerimg{17}{images/EMT4Plot2D-188.png}
\begin{eulerprompt}
>plot2d("exp(x+sin(x))",0,100,logplot=1):
\end{eulerprompt}
\eulerimg{17}{images/EMT4Plot2D-189.png}
\begin{eulerprompt}
>plot2d("exp(x+sin(x))",10,100,logplot=2):
\end{eulerprompt}
\eulerimg{17}{images/EMT4Plot2D-190.png}
\begin{eulerprompt}
>plot2d("gamma(x)",1,10,logplot=1):
\end{eulerprompt}
\eulerimg{17}{images/EMT4Plot2D-191.png}
\begin{eulerprompt}
>plot2d("log(x*(2+sin(x/100)))",10,1000,logplot=3):
\end{eulerprompt}
\eulerimg{17}{images/EMT4Plot2D-192.png}
\begin{eulercomment}
Ini juga berfungsi dengan plot data.
\end{eulercomment}
\begin{eulerprompt}
>x=10^(1:20); y=x^2-x;
>plot2d(x,y,logplot=2):
\end{eulerprompt}
\eulerimg{17}{images/EMT4Plot2D-193.png}
\begin{eulercomment}
Contoh Soal:\\
1. Tentukan plot dari fungsi berikut:\\
\end{eulercomment}
\begin{eulerformula}
\[
log(4x*(2+cos(x/75)))
\]
\end{eulerformula}
\begin{eulercomment}
Penyelesaian:
\end{eulercomment}
\begin{eulerprompt}
>plot2d("log(4x*(2+cos(x/75)))",10,900,logplot=3):
\end{eulerprompt}
\eulerimg{17}{images/EMT4Plot2D-195.png}
\eulerheading{Rujukan Lengkap Fungsi plot2d()}
\begin{eulercomment}
\end{eulercomment}
\begin{eulerttcomment}
  function plot2d (xv, yv, btest, a, b, c, d, xmin, xmax, r, n,  ..
  logplot, grid, frame, framecolor, square, color, thickness, style, ..
  auto, add, user, delta, points, addpoints, pointstyle, bar, histogram,  ..
  distribution, even, steps, own, adaptive, hue, level, contour,  ..
  nc, filled, fillcolor, outline, title, xl, yl, maps, contourcolor, ..
  contourwidth, ticks, margin, clipping, cx, cy, insimg, spectral,  ..
  cgrid, vertical, smaller, dl, niveau, levels)
\end{eulerttcomment}
\begin{eulercomment}
Multipurpose plot function for plots in the plane (2D plots). This function can do
plots of functions of one variables, data plots, curves in the plane, bar plots, grids
of complex numbers, and implicit plots of functions of two variables.

Parameters
\\
x,y       : equations, functions or data vectors\\
a,b,c,d   : Plot area (default a=-2,b=2)\\
r         : if r is set, then a=cx-r, b=cx+r, c=cy-r, d=cy+r\\
\end{eulercomment}
\begin{eulerttcomment}
            r can be a vector [rx,ry] or a vector [rx1,rx2,ry1,ry2].
\end{eulerttcomment}
\begin{eulercomment}
xmin,xmax : range of the parameter for curves\\
auto      : Determine y-range automatically (default)\\
square    : if true, try to keep square x-y-ranges\\
n         : number of intervals (default is adaptive)\\
grid      : 0 = no grid and labels,\\
\end{eulercomment}
\begin{eulerttcomment}
            1 = axis only,
            2 = normal grid (see below for the number of grid lines)
            3 = inside axis
            4 = no grid
            5 = full grid including margin
            6 = ticks at the frame
            7 = axis only
            8 = axis only, sub-ticks
\end{eulerttcomment}
\begin{eulercomment}
frame     : 0 = no frame\\
framecolor: color of the frame and the grid\\
margin    : number between 0 and 0.4 for the margin around the plot\\
color     : Color of curves. If this is a vector of colors,\\
\end{eulercomment}
\begin{eulerttcomment}
            it will be used for each row of a matrix of plots. In the case of
            point plots, it should be a column vector. If a row vector or a
            full matrix of colors is used for point plots, it will be used for
            each data point.
\end{eulerttcomment}
\begin{eulercomment}
thickness : line thickness for curves\\
\end{eulercomment}
\begin{eulerttcomment}
            This value can be smaller than 1 for very thin lines.
\end{eulerttcomment}
\begin{eulercomment}
style     : Plot style for lines, markers, and fills.\\
\end{eulercomment}
\begin{eulerttcomment}
            For points use
            "[]", "<>", ".", "..", "...",
            "*", "+", "|", "-", "o"
            "[]#", "<>#", "o#" (filled shapes)
            "[]w", "<>w", "ow" (non-transparent)
            For lines use
            "-", "--", "-.", ".", ".-.", "-.-", "->"
            For filled polygons or bar plots use
            "#", "#O", "O", "/", "\(\backslash\)", "\(\backslash\)/",
            "+", "|", "-", "t"
\end{eulerttcomment}
\begin{eulercomment}
points    : plot single points instead of line segments\\
addpoints : if true, plots line segments and points\\
add       : add the plot to the existing plot\\
user      : enable user interaction for functions\\
delta     : step size for user interaction\\
bar       : bar plot (x are the interval bounds, y the interval values)\\
histogram : plots the frequencies of x in n subintervals\\
distribution=n : plots the distribution of x with n subintervals\\
even      : use inter values for automatic histograms.\\
steps     : plots the function as a step function (steps=1,2)\\
adaptive  : use adaptive plots (n is the minimal number of steps)\\
level     : plot level lines of an implicit function of two variables\\
outline   : draws boundary of level ranges.
\\
If the level value is a 2xn matrix, ranges of levels will be drawn\\
in the color using the given fill style. If outline is true, it\\
will be drawn in the contour color. Using this feature, regions of\\
f(x,y) between limits can be marked.
\\
hue       : add hue color to the level plot to indicate the function\\
\end{eulercomment}
\begin{eulerttcomment}
            value
\end{eulerttcomment}
\begin{eulercomment}
contour   : Use level plot with automatic levels\\
nc        : number of automatic level lines\\
title     : plot title (default "")\\
xl, yl    : labels for the x- and y-axis\\
smaller   : if \textgreater{}0, there will be more space to the left for labels.\\
vertical  :\\
\end{eulercomment}
\begin{eulerttcomment}
  Turns vertical labels on or off. This changes the global variable
  verticallabels locally for one plot. The value 1 sets only vertical
  text, the value 2 uses vertical numerical labels on the y axis.
\end{eulerttcomment}
\begin{eulercomment}
filled    : fill the plot of a curve\\
fillcolor : fill color for bar and filled curves\\
outline   : boundary for filled polygons\\
logplot   : set logarithmic plots\\
\end{eulercomment}
\begin{eulerttcomment}
            1 = logplot in y,
            2 = logplot in xy,
            3 = logplot in x
\end{eulerttcomment}
\begin{eulercomment}
own       :\\
\end{eulercomment}
\begin{eulerttcomment}
  A string, which points to an own plot routine. With >user, you get
  the same user interaction as in plot2d. The range will be set
  before each call to your function.
\end{eulerttcomment}
\begin{eulercomment}
maps      : map expressions (0 is faster), functions are always mapped.\\
contourcolor : color of contour lines\\
contourwidth : width of contour lines\\
clipping  : toggles the clipping (default is true)\\
title     :\\
\end{eulercomment}
\begin{eulerttcomment}
  This can be used to describe the plot. The title will appear above
  the plot. Moreover, a label for the x and y axis can be added with
  xl="string" or yl="string". Other labels can be added with the
  functions label() or labelbox(). The title can be a unicode
  string or an image of a Latex formula.
\end{eulerttcomment}
\begin{eulercomment}
cgrid     :\\
\end{eulercomment}
\begin{eulerttcomment}
  Determines the number of grid lines for plots of complex grids.
  Should be a divisor of the the matrix size minus 1 (number of
  subintervals). cgrid can be a vector [cx,cy].
\end{eulerttcomment}
\begin{eulercomment}

Overview

The function can plot

- expressions, call collections or functions of one variable,\\
- parametric curves,\\
- x data against y data,\\
- implicit functions,\\
- bar plots,\\
- complex grids,\\
- polygons.

If a function or expression for xv is given, plot2d() will compute\\
values in the given range using the function or expression. The\\
expression must be an expression in the variable x. The range must\\
be defined in the parameters a and b unless the default range\\
[-2,2] should be used. The y-range will be computed automatically,\\
unless c and d are specified, or a radius r, which yields the range\\
[-r,r] for x and y. For plots of functions, plot2d will use an\\
adaptive evaluation of the function by default. To speed up the\\
plot for complicated functions, switch this off with \textless{}adaptive, and\\
optionally decrease the number of intervals n. Moreover, plot2d()\\
will by default use mapping. I.e., it will compute the plot element\\
for element. If your expression or your functions can handle a\\
vector x, you can switch that off with \textless{}maps for faster evaluation.

Note that adaptive plots are always computed element for element. \\
If functions or expressions for both xv and for yv are specified,\\
plot2d() will compute a curve with the xv values as x-coordinates\\
and the yv values as y-coordinates. In this case, a range should be\\
defined for the parameter using xmin, xmax. Expressions contained\\
in strings must always be expressions in the parameter variable x.
\end{eulercomment}
\end{eulernotebook}

\chapter{EMT plot 3D}
\begin{eulernotebook}
\eulerheading{Menggambar Plot 3D dengan EMT}
\begin{eulercomment}
Ini adalah pengenalan plot 3D di Euler. Kita memerlukan plot 3D untuk
memvisualisasikan fungsi dari dua variabel.

Euler menggambar fungsi-fungsi tersebut dengan menggunakan algoritme
pengurutan untuk menyembunyikan bagian-bagian di latar belakang.
Secara umum, Euler menggunakan proyeksi pusat. Standarnya adalah dari
kuadran x-y positif ke arah asal x=y=z=0, tetapi sudut=0° terlihat
dari arah sumbu-y. Sudut pandang dan ketinggian dapat diubah.

Euler can plot :

- memplot - permukaan dengan garis bayangan dan garis datar,\\
- awan titik,\\
- kurva parametrik,\\
- permukaan implisit.


Plot 3D suatu fungsi menggunakan plot3d. Cara termudah adalah dengan
memplot ekspresi dalam x dan y. Parameter r mengatur rentang plot
sekitar (0,0).
\end{eulercomment}
\begin{eulerprompt}
>aspect(1.5); plot3d("x^2+sin(y)",-5,5,0,6*pi):
\end{eulerprompt}
\eulerimg{17}{images/EMT4Plot3D-001.png}
\begin{eulerprompt}
>plot3d("x^2+x*sin(y)",-5,5,0,6*pi):
\end{eulerprompt}
\eulerimg{17}{images/EMT4Plot3D-002.png}
\begin{eulercomment}
Silakan lakukan modifikasi agar gambar "talang bergelombang" tersebut
tidak lurus melainkan melengkung/melingkar, baik melingkar secara
mendatar maupun melingkar turun/naik (seperti papan peluncur pada
kolam renang. Temukan rumusnya.

Berikut rumus agar gambar "talang bergelombang" tidak lurus melainkan
melengkung/melingkar, baik melingkar secara mendatar maupun melingkar
turun/naik:
\end{eulercomment}
\begin{eulerprompt}
>aspect(1.5); plot3d("x^2+sin(y)",r=pi):
\end{eulerprompt}
\eulerimg{17}{images/EMT4Plot3D-003.png}
\eulerheading{Fungsi dua Variabel}
\begin{eulercomment}
Untuk grafik suatu fungsi, gunakan -

- ekspresi sederhana dalam x dan y,\\
- nama fungsi dari dua variabel,\\
- atau matriks data.

Standarnya adalah kisi-kisi kawat berisi dengan warna berbeda di kedua
sisi. Perhatikan bahwa jumlah interval kisi default adalah 10, tetapi
plot menggunakan jumlah default persegi panjang 40x40 untuk membuat
permukaannya. Ini bisa diubah.

- n=40, n=[40,40]: jumlah garis grid di setiap arah.\\
- grid=10, grid=[10,10]: : jumlah garis grid di setiap arah.

Kami menggunakan default n=40 dan grid=10.
\end{eulercomment}
\begin{eulerprompt}
>plot3d("x^2+y^2"):
\end{eulerprompt}
\eulerimg{17}{images/EMT4Plot3D-004.png}
\begin{eulercomment}
Interaksi pengguna dimungkinkan dengan parameter \textgreater{}pengguna. Pengguna
dapat menekan tombol berikut.

- left,right,up,down: : putar sudut pandang,\\
- +,-: memperbesar atau memperkecil\\
- a: menghasilkan anaglyph (lihat di bawah)\\
- l:  beralih memutar sumber cahaya(lihat di bawah)\\
- space: atur ulang ke default\\
- return: interaksi akhir
\end{eulercomment}
\begin{eulerprompt}
>plot3d("exp(-x^2+y^2)",>user, ...
>  title="Turn with the vector keys (press return to finish)"):
\end{eulerprompt}
\eulerimg{17}{images/EMT4Plot3D-005.png}
\begin{eulercomment}
Rentang plot untuk fungsi dapat ditentukan dengan :

- a,b: rentang x\\
- c,d: rentang y\\
- r: persegi simetris di sekelilingnya(0,0).\\
- n: jumlah subinterval untuk plot.

Ada beberapa parameter untuk menskalakan fungsi atau mengubah tampilan
grafik.

fscale: menskalakan ke nilai fungsi (defaultnya adalah \textless{}fscale).\\
scale: angka atau vektor 1x2 untuk menskalakan ke arah x dan y\\
frame:  jenis bingkai (default 1).
\end{eulercomment}
\begin{eulerprompt}
>plot3d("exp(-(x^2+y^2)/5)",r=10,n=80,fscale=4,scale=1.2,frame=3,>user):
\end{eulerprompt}
\eulerimg{17}{images/EMT4Plot3D-006.png}
\begin{eulercomment}
Tampilan dapat diubah dengan berbagai cara.

- distance: jarak pandang ke plot.\\
- zoom: the zoom value.\\
- sudut: the angle to the negative y-axis in radians.\\
- height: the height of the view in radians.

Nilai default dapat diperiksa atau diubah dengan fungsi view(). Ini
mengembalikan parameter dalam urutan di atas.
\end{eulercomment}
\begin{eulerprompt}
>view
\end{eulerprompt}
\begin{euleroutput}
  [5,  2.6,  2,  0.4]
\end{euleroutput}
\begin{eulercomment}
Jarak yang lebih dekat membutuhkan lebih sedikit zoom. Efeknya lebih
seperti lensa sudut lebar.

ada contoh berikut, sudut=0 dan tinggi=0 dilihat dari sumbu y negatif.
Label sumbu untuk y disembunyikan dalam kasus ini.
\end{eulercomment}
\begin{eulerprompt}
>plot3d("x^2+y",distance=3,zoom=1,angle=pi/2,height=0):
\end{eulerprompt}
\eulerimg{17}{images/EMT4Plot3D-007.png}
\begin{eulercomment}
Plot selalu terlihat berada di tengah kubus plot. Anda dapat
memindahkan bagian tengah dengan parameter tengah.
\end{eulercomment}
\begin{eulerprompt}
>plot3d("x^4+y^2",a=0,b=1,c=-1,d=1,angle=-20°,height=20°, ...
>  center=[0.4,0,0],zoom=5):
\end{eulerprompt}
\eulerimg{17}{images/EMT4Plot3D-008.png}
\begin{eulercomment}
Plotnya diskalakan agar sesuai dengan unit kubus untuk dilihat. Jadi
tidak perlu mengubah jarak atau zoom tergantung ukuran plot. Namun
labelnya mengacu pada ukuran sebenarnya.

Jika Anda mematikannya dengan scale=false, Anda harus berhati-hati
agar plot tetap masuk ke dalam jendela plotting, dengan mengubah jarak
pandang atau zoom, dan memindahkan bagian tengah.
\end{eulercomment}
\begin{eulerprompt}
>plot3d("5*exp(-x^2-y^2)",r=2,<fscale,<scale,distance=13,height=50°, ...
>  center=[0,0,-2],frame=3):
\end{eulerprompt}
\eulerimg{17}{images/EMT4Plot3D-009.png}
\begin{eulercomment}
Plot kutub juga tersedia. Parameter polar=true menggambar plot kutub.
Fungsi tersebut harus tetap merupakan fungsi dari x dan y. Parameter
"fscale" menskalakan fungsi dengan skalanya sendiri. Kalau tidak,
fungsinya akan diskalakan agar sesuai dengan kubus.
\end{eulercomment}
\begin{eulerprompt}
>plot3d("1/(x^2+y^2+1)",r=5,>polar, ...
>fscale=2,>hue,n=100,zoom=4,>contour,color=blue):
\end{eulerprompt}
\eulerimg{17}{images/EMT4Plot3D-010.png}
\begin{eulerprompt}
>function f(r) := exp(-r/2)*cos(r); ...
>plot3d("f(x^2+y^2)",>polar,scale=[1,1,0.4],r=pi,frame=3,zoom=4):
\end{eulerprompt}
\eulerimg{17}{images/EMT4Plot3D-011.png}
\begin{eulercomment}
Parameter memutar memutar fungsi di x di sekitar sumbu x.

- rotate=1: Uses the x-axis\\
- rotate=2: Uses the z-axis
\end{eulercomment}
\begin{eulerprompt}
>plot3d("x^2+1",a=-1,b=1,rotate=true,grid=5):
\end{eulerprompt}
\eulerimg{17}{images/EMT4Plot3D-012.png}
\begin{eulerprompt}
>plot3d("x^2+1",a=-1,b=1,rotate=2,grid=5):
\end{eulerprompt}
\eulerimg{17}{images/EMT4Plot3D-013.png}
\begin{eulerprompt}
>plot3d("sqrt(25-x^2)",a=0,b=5,rotate=1):
\end{eulerprompt}
\eulerimg{17}{images/EMT4Plot3D-014.png}
\begin{eulerprompt}
>plot3d("x*sin(x)",a=0,b=6pi,rotate=2):
\end{eulerprompt}
\eulerimg{17}{images/EMT4Plot3D-015.png}
\begin{eulercomment}
Berikut adalah plot dengan tiga fungsi.
\end{eulercomment}
\begin{eulerprompt}
>plot3d("x","x^2+y^2","y",r=2,zoom=3.5,frame=3):
\end{eulerprompt}
\eulerimg{17}{images/EMT4Plot3D-016.png}
\begin{eulercomment}
Contoh Soal dan Penyelesaian:\\
1. Buatlah grafik dari fungsi berikut ini\\
\end{eulercomment}
\begin{eulerformula}
\[
f(x,y)=3-x^2-y^2
\]
\end{eulerformula}
\begin{eulercomment}
Penyelesaian:
\end{eulercomment}
\begin{eulerprompt}
>plot3d("x","3-x^2-y^2","y",r=5,zoom=8,frame=3):
\end{eulerprompt}
\eulerimg{17}{images/EMT4Plot3D-018.png}
\begin{eulercomment}
2. Buatlah grafik dari fungsi berikut ini\\
\end{eulercomment}
\begin{eulerformula}
\[
f(x,y)=\sqrt{16-x^2-y^2}
\]
\end{eulerformula}
\begin{eulercomment}
Penyelesaian:
\end{eulercomment}
\begin{eulerprompt}
>plot3d("(16-x^2-y^2)^(1/2)",>user, ...
>title= "Turn with the vector keys(press return to finish)"):
\end{eulerprompt}
\eulerimg{17}{images/EMT4Plot3D-020.png}
\eulerheading{Plot Kontur}
\begin{eulercomment}
Untuk plotnya, Euler menambahkan garis grid. Sebaliknya dimungkinkan
untuk menggunakan garis datar dan rona satu warna atau rona warna
spektral. Euler dapat menggambar ketinggian fungsi pada plot dengan
arsiran. Di semua plot 3D, Euler dapat menghasilkan anaglyph.

- \textgreater{}hue:  Mengaktifkan bayangan cahaya, bukan kabel.

\end{eulercomment}
\begin{eulerttcomment}
 >contour: : Membuat plot garis kontur otomatis pada plot.
\end{eulerttcomment}
\begin{eulercomment}
- level=... (or levels): A Vektor nilai garis kontur.

Standarnya adalah level="auto", yang menghitung beberapa garis level
secara otomatis. Seperti yang Anda lihat di plot, level sebenarnya
adalah rentang level.

Gaya default dapat diubah. Untuk plot kontur berikut, kami menggunakan
grid yang lebih halus berukuran 100x100 poin, menskalakan fungsi dan
plot, dan menggunakan sudut pandang yang berbeda.
\end{eulercomment}
\begin{eulerprompt}
>plot3d("exp(-x^2-y^2)",r=2,n=100,level="thin", ...
> >contour,>spectral,fscale=1,scale=1.1,angle=45°,height=20°):
\end{eulerprompt}
\eulerimg{17}{images/EMT4Plot3D-021.png}
\begin{eulerprompt}
>plot3d("exp(x*y)",angle=100°,>contour,color=green):
\end{eulerprompt}
\eulerimg{17}{images/EMT4Plot3D-022.png}
\begin{eulercomment}
Bayangan defaultnya menggunakan warna abu-abu. Namun rentang warna
spektral juga tersedia. \\
- \textgreater{}spectral: Menggunakan skema spektral default\\
- color=...: Menggunakan warna khusus atau skema

spektral Untuk plot berikut, kami menggunakan skema spektral default
dan menambah jumlah titik untuk mendapatkan tampilan yang sangat
halus.
\end{eulercomment}
\begin{eulerprompt}
>plot3d("x^2+y^2",>spectral,>contour,n=100):
\end{eulerprompt}
\eulerimg{17}{images/EMT4Plot3D-023.png}
\begin{eulercomment}
Selain garis level otomatis, kita juga dapat menetapkan nilai garis
level. Ini akan menghasilkan garis level yang tipis, bukan rentang
level.
\end{eulercomment}
\begin{eulerprompt}
>plot3d("x^2-y^2",0,5,0,5,level=-1:0.1:1,color=redgreen):
\end{eulerprompt}
\eulerimg{17}{images/EMT4Plot3D-024.png}
\begin{eulercomment}
Dalam plot berikut, kita menggunakan dua pita tingkat yang sangat luas
dari -0,1 hingga 1, dan dari 0,9 hingga 1. Ini dimasukkan sebagai
matriks dengan batas tingkat sebagai kolom.

Selain itu, kami melapisi grid dengan 10 interval di setiap arah.
\end{eulercomment}
\begin{eulerprompt}
>plot3d("x^2+y^3",level=[-0.1,0.9;0,1], ...
>  >spectral,angle=30°,grid=10,contourcolor=gray):
\end{eulerprompt}
\eulerimg{17}{images/EMT4Plot3D-025.png}
\begin{eulercomment}
Pada contoh berikut, kita memplot himpunan, di mana :

\end{eulercomment}
\begin{eulerformula}
\[
f(x,y) = x^y-y^x = 0
\]
\end{eulerformula}
\begin{eulercomment}
Kami menggunakan satu garis tipis untuk garis level.
\end{eulercomment}
\begin{eulerprompt}
>plot3d("x^y-y^x",level=0,a=0,b=6,c=0,d=6,contourcolor=red,n=100):
\end{eulerprompt}
\eulerimg{17}{images/EMT4Plot3D-026.png}
\begin{eulercomment}
Dimungkinkan untuk menampilkan bidang kontur di bawah plot. Warna dan
jarak ke plot dapat ditentukan.
\end{eulercomment}
\begin{eulerprompt}
>plot3d("x^2+y^4",>cp,cpcolor=green,cpdelta=0.2):
\end{eulerprompt}
\eulerimg{17}{images/EMT4Plot3D-027.png}
\begin{eulercomment}
Berikut beberapa gaya lainnya. Kami selalu mematikan bingkai, dan
menggunakan berbagai skema warna untuk plot dan kisi.
\end{eulercomment}
\begin{eulerprompt}
>figure(2,2); ...
>expr="y^3-x^2"; ...
>figure(1);  ...
>  plot3d(expr,<frame,>cp,cpcolor=spectral); ...
>figure(2);  ...
>  plot3d(expr,<frame,>spectral,grid=10,cp=2); ...
>figure(3);  ...
>  plot3d(expr,<frame,>contour,color=gray,nc=5,cp=3,cpcolor=greenred); ...
>figure(4);  ...
>  plot3d(expr,<frame,>hue,grid=10,>transparent,>cp,cpcolor=gray); ...
>figure(0):
\end{eulerprompt}
\eulerimg{17}{images/EMT4Plot3D-028.png}
\begin{eulercomment}
Ada beberapa skema spektral lainnya, yang diberi nomor dari 1 hingga
9. Namun Anda juga dapat menggunakan warna=nilai, di mana nilai :

- spectral: untuk rentang dari biru ke merah\\
- white: untuk rentang yang lebih redup \\
- yellowblue,purplegreen,blueyellow,greenred\\
- blueyellow, greenpurple,yellowblue,redgreen
\end{eulercomment}
\begin{eulerprompt}
>figure(3,3); ...
>for i=1:9;  ...
>  figure(i); plot3d("x^2+y^2",spectral=i,>contour,>cp,<frame,zoom=4);  ...
>end; ...
>figure(0):
\end{eulerprompt}
\eulerimg{17}{images/EMT4Plot3D-029.png}
\begin{eulercomment}
Sumber cahaya dapat diubah dengan l dan tombol kursor selama interaksi
pengguna. Itu juga dapat diatur dengan parameter.

- light: arah\\
- amb: ambient light between 0 and 1

Catatan : program tidak membuat perbedaan antara sisi plot. Tidak ada
bayangan. Untuk ini, Anda memerlukan Povray.
\end{eulercomment}
\begin{eulerprompt}
>plot3d("-x^2-y^2", ...
>  hue=true,light=[0,1,1],amb=0,user=true, ...
>  title="Press l and cursor keys (return to exit)"):
\end{eulerprompt}
\eulerimg{17}{images/EMT4Plot3D-030.png}
\begin{eulercomment}
Parameter warna mengubah warna permukaan. Warna garis level juga bisa
diubah.
\end{eulercomment}
\begin{eulerprompt}
>plot3d("-x^2-y^2",color=rgb(0.2,0.2,0),hue=true,frame=false, ...
>  zoom=3,contourcolor=red,level=-2:0.1:1,dl=0.01):
\end{eulerprompt}
\eulerimg{17}{images/EMT4Plot3D-031.png}
\begin{eulercomment}
Warna 0 memberikan efek pelangi yang istimewa.
\end{eulercomment}
\begin{eulerprompt}
>plot3d("x^2/(x^2+y^2+1)",color=0,hue=true,grid=10):
\end{eulerprompt}
\eulerimg{17}{images/EMT4Plot3D-032.png}
\begin{eulercomment}
Permukaannya juga bisa transparan.
\end{eulercomment}
\begin{eulerprompt}
>plot3d("x^2+y^2",>transparent,grid=10,wirecolor=red):
\end{eulerprompt}
\eulerimg{17}{images/EMT4Plot3D-033.png}
\begin{eulercomment}
Contoh Soal dan Penyelesaian:\\
1. Buatlah grafik dari fungsi berikut ini\\
\end{eulercomment}
\begin{eulerformula}
\[
z=y^2-x^2
\]
\end{eulerformula}
\begin{eulercomment}
Penyelesaian:
\end{eulercomment}
\begin{eulerprompt}
>plot3d("y^2-x^2",>cp,cpcolor=green,cpdelta=0.1):
\end{eulerprompt}
\eulerimg{17}{images/EMT4Plot3D-035.png}
\eulerheading{Plot Implisit}
\begin{eulercomment}
Ada juga plot implisit dalam tiga dimensi. Euler menghasilkan
pemotongan melalui objek. Fitur plot3d mencakup plot implisit. Plot
ini menunjukkan himpunan nol suatu fungsi dalam tiga variabel.
Permukaannya juga bisa transparan.\\
Solusi dari

\end{eulercomment}
\begin{eulerformula}
\[
f(x,y,z) = 0
\]
\end{eulerformula}
\begin{eulercomment}
dapat divisualisasikan dalam potongan yang sejajar dengan bidang xy-,
xz- dan yz.

- implicit=1: cut parallel to the y-z-plane\\
- implicit=2: cut parallel to the x-z-plane\\
- implicit=4: cut parallel to the x-y-plane

Tambahkan nilai berikut, jika Anda mau. Dalam contoh kita memplot :

\end{eulercomment}
\begin{eulerformula}
\[
M = \{ (x,y,z) : x^2+y^3+zy=1 \}
\]
\end{eulerformula}
\begin{eulerprompt}
>plot3d("x^2+y^3+z*y-1",r=5,implicit=3):
\end{eulerprompt}
\eulerimg{17}{images/EMT4Plot3D-038.png}
\begin{eulerprompt}
>c=1; d=1;
>plot3d("((x^2+y^2-c^2)^2+(z^2-1)^2)*((y^2+z^2-c^2)^2+(x^2-1)^2)*((z^2+x^2-c^2)^2+(y^2-1)^2)-d",r=2,<frame,>implicit,>user): 
\end{eulerprompt}
\eulerimg{17}{images/EMT4Plot3D-039.png}
\begin{eulerprompt}
>plot3d("x^2+y^2+4*x*z+z^3",>implicit,r=2,zoom=2.5):
\end{eulerprompt}
\eulerimg{17}{images/EMT4Plot3D-040.png}
\begin{eulercomment}
Contoh Soal dan Penyelesaian:\\
1. Buatlah grafik dari fungsi berikut ini\\
\end{eulercomment}
\begin{eulerformula}
\[
f(x,y,z)=16x^2+16y^2-9z^2
\]
\end{eulerformula}
\begin{eulercomment}
Penyelesaian:
\end{eulercomment}
\begin{eulerprompt}
>plot3d("16*x^2+16*y^2-9*z^2",>implicit,r=4,zoom=1.5):
\end{eulerprompt}
\eulerimg{17}{images/EMT4Plot3D-042.png}
\eulerheading{Memplot Data 3D}
\begin{eulercomment}
Sama seperti plot2d, plot3d menerima data. Untuk objek 3D, Anda perlu
menyediakan matriks nilai x-, y- dan z, atau tiga fungsi atau ekspresi
fx(x,y), fy(x,y), fz(x,y).

\end{eulercomment}
\begin{eulerformula}
\[
\gamma(t,s) = (x(t,s),y(t,s),z(t,s))
\]
\end{eulerformula}
\begin{eulercomment}
Karena x,y,z adalah matriks, kita asumsikan bahwa (t,s) melewati grid
persegi. Hasilnya, Anda dapat memplot gambar persegi panjang di ruang
angkasa.\\
Anda dapat menggunakan bahasa matriks Euler untuk menghasilkan
koordinat secara efektif.

Dalam contoh berikut, kita menggunakan vektor nilai t dan vektor kolom
nilai s untuk membuat parameter permukaan bola. Dalam gambar kita
dapat menandai wilayah, dalam kasus kita wilayah kutub.
\end{eulercomment}
\begin{eulerprompt}
>t=linspace(0,2pi,180); s=linspace(-pi/2,pi/2,90)'; ...
>x=cos(s)*cos(t); y=cos(s)*sin(t); z=sin(s); ...
>plot3d(x,y,z,>hue, ...
>color=blue,<frame,grid=[10,20], ...
>values=s,contourcolor=red,level=[90°-24°;90°-22°], ...
>scale=1.4,height=50°):
\end{eulerprompt}
\eulerimg{17}{images/EMT4Plot3D-044.png}
\begin{eulercomment}
Berikut ini contohnya yaitu grafik suatu fungsi.
\end{eulercomment}
\begin{eulerprompt}
>t=-1:0.1:1; s=(-1:0.1:1)'; plot3d(t,s,t*s,grid=10):
\end{eulerprompt}
\eulerimg{17}{images/EMT4Plot3D-045.png}
\begin{eulercomment}
Namun, kita bisa membuat berbagai macam permukaan. Berikut adalah
permukaan yang sama sebagai suatu fungsi :

\end{eulercomment}
\begin{eulerformula}
\[
x = y \ z
\]
\end{eulerformula}
\begin{eulerprompt}
>plot3d(t*s,t,s,angle=180°,grid=10):
\end{eulerprompt}
\eulerimg{17}{images/EMT4Plot3D-047.png}
\begin{eulercomment}
Dengan lebih banyak usaha, kita dapat menghasilkan banyak permukaan.

Dalam contoh berikut kita membuat tampilan bayangan dari bola yang
terdistorsi. Koordinat bola yang biasa adalah

\end{eulercomment}
\begin{eulerformula}
\[
\gamma(t,s) = (\cos(t)\cos(s),\sin(t)\sin(s),\cos(s))
\]
\end{eulerformula}
\begin{eulercomment}
dengan

\end{eulercomment}
\begin{eulerformula}
\[
0 \le t \le 2\pi, \quad \frac{-\pi}{2} \le s \le \frac{\pi}{2}.
\]
\end{eulerformula}
\begin{eulercomment}
Kami mengurangi hal ini dengan sebuah faktor

\end{eulercomment}
\begin{eulerformula}
\[
d(t,s) = \frac{\cos(4t)+\cos(8s)}{4}.
\]
\end{eulerformula}
\begin{eulerprompt}
>t=linspace(0,2pi,320); s=linspace(-pi/2,pi/2,160)'; ...
>d=1+0.2*(cos(4*t)+cos(8*s)); ...
>plot3d(cos(t)*cos(s)*d,sin(t)*cos(s)*d,sin(s)*d,hue=1, ...
>  light=[1,0,1],frame=0,zoom=5):
\end{eulerprompt}
\eulerimg{17}{images/EMT4Plot3D-051.png}
\begin{eulercomment}
Tentu saja, point cloud juga dimungkinkan. Untuk memplot data titik
dalam ruang, kita memerlukan tiga vektor untuk koordinat titik-titik
tersebut.

Gayanya sama seperti di plot2d dengan points=true;
\end{eulercomment}
\begin{eulerprompt}
>n=500;  ...
>  plot3d(normal(1,n),normal(1,n),normal(1,n),points=true,style="."):
\end{eulerprompt}
\eulerimg{17}{images/EMT4Plot3D-052.png}
\begin{eulercomment}
Dimungkinkan juga untuk memplot kurva dalam 3D. Dalam hal ini, lebih
mudah untuk menghitung terlebih dahulu titik-titik kurva. Untuk kurva
pada bidang kita menggunakan barisan koordinat dan parameter
wire=true.
\end{eulercomment}
\begin{eulerprompt}
>t=linspace(0,8pi,500); ...
>plot3d(sin(t),cos(t),t/10,>wire,zoom=3):
\end{eulerprompt}
\eulerimg{17}{images/EMT4Plot3D-053.png}
\begin{eulerprompt}
>t=linspace(0,4pi,1000); plot3d(cos(t),sin(t),t/2pi,>wire, ...
>linewidth=3,wirecolor=blue):
\end{eulerprompt}
\eulerimg{17}{images/EMT4Plot3D-054.png}
\begin{eulerprompt}
>X=cumsum(normal(3,100)); ...
> plot3d(X[1],X[2],X[3],>anaglyph,>wire):
\end{eulerprompt}
\eulerimg{17}{images/EMT4Plot3D-055.png}
\begin{eulercomment}
EMT juga dapat membuat plot dalam mode anaglyph. Untuk melihat plot
seperti itu, Anda memerlukan kacamata berwarna merah/cyan.
\end{eulercomment}
\begin{eulerprompt}
> plot3d("x^2+y^3",>anaglyph,>contour,angle=30°):
\end{eulerprompt}
\eulerimg{17}{images/EMT4Plot3D-056.png}
\begin{eulercomment}
Seringkali skema warna spektral digunakan untuk plot. Ini menekankan
ketinggian fungsinya.
\end{eulercomment}
\begin{eulerprompt}
>plot3d("x^2*y^3-y",>spectral,>contour,zoom=3.2):
\end{eulerprompt}
\eulerimg{17}{images/EMT4Plot3D-057.png}
\begin{eulercomment}
Euler juga dapat memplot permukaan yang diparameterisasi, jika
parameternya adalah nilai x, y, dan z dari gambar kotak persegi
panjang di ruang tersebut.

Untuk demo berikut, kami menyiapkan parameter u- dan v-, dan
menghasilkan koordinat ruang dari parameter tersebut.
\end{eulercomment}
\begin{eulerprompt}
>u=linspace(-1,1,10); v=linspace(0,2*pi,50)'; ...
>X=(3+u*cos(v/2))*cos(v); Y=(3+u*cos(v/2))*sin(v); Z=u*sin(v/2); ...
>plot3d(X,Y,Z,>anaglyph,<frame,>wire,scale=2.3):
\end{eulerprompt}
\eulerimg{17}{images/EMT4Plot3D-058.png}
\begin{eulercomment}
Berikut adalah contoh yang lebih rumit, yang megah dengan kacamata
merah/cyan.
\end{eulercomment}
\begin{eulerprompt}
>u:=linspace(-pi,pi,160); v:=linspace(-pi,pi,400)';  ...
>x:=(4*(1+.25*sin(3*v))+cos(u))*cos(2*v); ...
>y:=(4*(1+.25*sin(3*v))+cos(u))*sin(2*v); ...
> z=sin(u)+2*cos(3*v); ...
>plot3d(x,y,z,frame=0,scale=1.5,hue=1,light=[1,0,-1],zoom=2.8,>anaglyph):
\end{eulerprompt}
\eulerimg{17}{images/EMT4Plot3D-059.png}
\begin{eulercomment}
Contoh Soal dan Penyelesaian:\\
1. Buatlah grafik dari fungsi berikut ini\\
\end{eulercomment}
\begin{eulerformula}
\[
f(x,y)=2x^4-y^3+y
\]
\end{eulerformula}
\begin{eulercomment}
Penyelesaian:
\end{eulercomment}
\begin{eulerprompt}
>plot3d("2*x^4-y^3+y",>spectral,>contour,zoom=2):
\end{eulerprompt}
\eulerimg{17}{images/EMT4Plot3D-061.png}
\eulerheading{Plot Statistik}
\begin{eulercomment}
Plot batang juga dimungkinkan. Untuk ini, kita harus menyediakan

- x: vektor baris dengan n+1 elemen\\
- y: vektor kolom dengan n+1 elemen\\
- z: matriks nilai nxn.\\
z bisa lebih besar, tetapi hanya nilai nxn yang akan digunakan.

Dalam contoh ini, pertama-tama kita menghitung nilainya. Kemudian kita
sesuaikan x dan y, sehingga vektor-vektornya berpusat pada nilai yang
digunakan.
\end{eulercomment}
\begin{eulerprompt}
>x=-1:0.1:1; y=x'; z=x^2+y^2; ...
>xa=(x|1.1)-0.05; ya=(y_1.1)-0.05; ...
>plot3d(xa,ya,z,bar=true):
\end{eulerprompt}
\eulerimg{17}{images/EMT4Plot3D-062.png}
\begin{eulercomment}
Dimungkinkan untuk membagi plot suatu permukaan menjadi dua bagian
atau lebih.
\end{eulercomment}
\begin{eulerprompt}
>x=-1:0.1:1; y=x'; z=x+y; d=zeros(size(x)); ...
>plot3d(x,y,z,disconnect=2:2:20):
\end{eulerprompt}
\eulerimg{17}{images/EMT4Plot3D-063.png}
\begin{eulercomment}
Jika memuat atau menghasilkan matriks data M dari file dan perlu
memplotnya dalam 3D, Anda dapat menskalakan matriks ke [-1,1] dengan
skala(M), atau menskalakan matriks dengan \textgreater{}zscale. Hal ini dapat
dikombinasikan dengan faktor penskalaan individual yang diterapkan
sebagai tambahan.
\end{eulercomment}
\begin{eulerprompt}
>i=1:20; j=i'; ...
>plot3d(i*j^2+100*normal(20,20),>zscale,scale=[1,1,1.5],angle=-40°,zoom=1.8):
\end{eulerprompt}
\eulerimg{17}{images/EMT4Plot3D-064.png}
\begin{eulerprompt}
>Z=intrandom(5,100,6); v=zeros(5,6); ...
>loop 1 to 5; v[#]=getmultiplicities(1:6,Z[#]); end; ...
>columnsplot3d(v',scols=1:5,ccols=[1:5]):
\end{eulerprompt}
\eulerimg{17}{images/EMT4Plot3D-065.png}
\begin{eulercomment}
Contoh Soal dan Penyelesaian:\\
1. Buatlah grafik dari fungsi berikut ini\\
\end{eulercomment}
\begin{eulerformula}
\[
z=4x^2-9y^2
\]
\end{eulerformula}
\begin{eulercomment}
Penyelesaian:
\end{eulercomment}
\begin{eulerprompt}
>x=-1:0.1:1; y=x'; z=4*x^2-9*y^2; ...
>xa=(x|1.1)-0.05; ya=(y_1.1)-0.05; ...
>plot3d(xa,ya,z,bar=true):
\end{eulerprompt}
\eulerimg{17}{images/EMT4Plot3D-067.png}
\eulerheading{Permukaan Benda Putar}
\begin{eulerprompt}
>plot2d("(x^2+y^2-1)^3-x^2*y^3",r=1.3, ...
>style="#",color=red,<outline, ...
>level=[-2;0],n=100):
\end{eulerprompt}
\eulerimg{17}{images/EMT4Plot3D-068.png}
\begin{eulerprompt}
>ekspresi &= (x^2+y^2-1)^3-x^2*y^3; $ekspresi
\end{eulerprompt}
\begin{eulerformula}
\[
\left(y^2+x^2-1\right)^3-x^2\,y^3
\]
\end{eulerformula}
\begin{eulercomment}
Kami ingin memutar kurva hati di sekitar sumbu y. Inilah ungkapan yang
mendefinisikan hati:

\end{eulercomment}
\begin{eulerformula}
\[
f(x,y)=(x^2+y^2-1)^3-x^2.y^3.
\]
\end{eulerformula}
\begin{eulercomment}
Selanjutnya kita atur

\end{eulercomment}
\begin{eulerformula}
\[
x=r.cos(a),\quad y=r.sin(a).
\]
\end{eulerformula}
\begin{eulerprompt}
>function fr(r,a) &= ekspresi with [x=r*cos(a),y=r*sin(a)] | trigreduce; $fr(r,a)
\end{eulerprompt}
\begin{eulerformula}
\[
\left(r^2-1\right)^3+\frac{\left(\sin \left(5\,a\right)-\sin \left(  3\,a\right)-2\,\sin a\right)\,r^5}{16}
\]
\end{eulerformula}
\begin{eulercomment}
Hal ini memungkinkan untuk mendefinisikan fungsi numerik, yang
menyelesaikan r, jika a diberikan. Dengan fungsi tersebut kita dapat
memplot jantung yang diputar sebagai permukaan parametrik.
\end{eulercomment}
\begin{eulerprompt}
>function map f(a) := bisect("fr",0,2;a); ...
>t=linspace(-pi/2,pi/2,100); r=f(t);  ...
>s=linspace(pi,2pi,100)'; ...
>plot3d(r*cos(t)*sin(s),r*cos(t)*cos(s),r*sin(t), ...
>>hue,<frame,color=red,zoom=4,amb=0,max=0.7,grid=12,height=50°):
\end{eulerprompt}
\eulerimg{17}{images/EMT4Plot3D-073.png}
\begin{eulercomment}
Berikut ini adalah plot 3D dari gambar di atas yang diputar
mengelilingi sumbu z. Kami mendefinisikan fungsi yang mendeskripsikan
objek.
\end{eulercomment}
\begin{eulerprompt}
>function f(x,y,z) ...
\end{eulerprompt}
\begin{eulerudf}
  r=x^2+y^2;
  return (r+z^2-1)^3-r*z^3;
   endfunction
\end{eulerudf}
\begin{eulerprompt}
>plot3d("f(x,y,z)", ...
>xmin=0,xmax=1.2,ymin=-1.2,ymax=1.2,zmin=-1.2,zmax=1.4, ...
>implicit=1,angle=-30°,zoom=2.5,n=[10,100,60],>anaglyph):
\end{eulerprompt}
\eulerimg{17}{images/EMT4Plot3D-074.png}
\begin{eulercomment}
Contoh Soal dan Penyelesaian:\\
1. Buatlah grafik dari fungsi berikut ini\\
\end{eulercomment}
\begin{eulerformula}
\[
f(x,y)=(2x^2+y^2-2)^4-x^2y^3
\]
\end{eulerformula}
\begin{eulercomment}
Penyelesaian:
\end{eulercomment}
\begin{eulerprompt}
>plot2d("(2x^2+y^2-2)^4-x^2*y^3",r=3.3, ...
>style="#",color=orange,<outline, ...
>level=[-6;0],n=68):
\end{eulerprompt}
\eulerimg{17}{images/EMT4Plot3D-076.png}
\eulerheading{Plot 3D Khusus}
\begin{eulercomment}
Fungsi plot3d bagus untuk dimiliki, tetapi tidak memenuhi semua
kebutuhan. Selain rutinitas yang lebih mendasar, dimungkinkan untuk
mendapatkan plot berbingkai dari objek apa pun yang Anda suka.

Meskipun Euler bukan program 3D, ia dapat menggabungkan beberapa objek
dasar. Kami mencoba memvisualisasikan paraboloid dan garis
singgungnya.
\end{eulercomment}
\begin{eulerprompt}
>function myplot ...
\end{eulerprompt}
\begin{eulerudf}
    y=-1:0.01:1; x=(-1:0.01:1)';
    plot3d(x,y,0.2*(x-0.1)/2,<scale,<frame,>hue, ..
      hues=0.5,>contour,color=orange);
    h=holding(1);
    plot3d(x,y,(x^2+y^2)/2,<scale,<frame,>contour,>hue);
    holding(h);
  endfunction
\end{eulerudf}
\begin{eulercomment}
Sekarang framedplot() menyediakan bingkai, dan mengatur tampilan.
\end{eulercomment}
\begin{eulerprompt}
>framedplot("myplot",[-1,1,-1,1,0,1],height=0,angle=-30°, ...
>  center=[0,0,-0.7],zoom=3):
\end{eulerprompt}
\eulerimg{17}{images/EMT4Plot3D-077.png}
\begin{eulercomment}
Dengan cara yang sama, Anda dapat memplot bidang kontur secara manual.
Perhatikan bahwa plot3d() menyetel jendela ke fullwindow(), secara
default, tetapi plotcontourplane() berasumsi demikian.
\end{eulercomment}
\begin{eulerprompt}
>x=-1:0.02:1.1; y=x'; z=x^2-y^4;
>function myplot (x,y,z) ...
\end{eulerprompt}
\begin{eulerudf}
    zoom(2);
    wi=fullwindow();
    plotcontourplane(x,y,z,level="auto",<scale);
    plot3d(x,y,z,>hue,<scale,>add,color=white,level="thin");
    window(wi);
    reset();
  endfunction
\end{eulerudf}
\begin{eulerprompt}
>myplot(x,y,z):
\end{eulerprompt}
\eulerimg{27}{images/EMT4Plot3D-078.png}
\eulerheading{Animasi}
\begin{eulercomment}
Euler dapat menggunakan frame untuk melakukan pra-komputasi animasi.

Salah satu fungsi yang memanfaatkan teknik ini adalah memutar. Itu
dapat mengubah sudut pandang dan menggambar ulang plot 3D. Fungsi ini
memanggil addpage() untuk setiap plot baru. Akhirnya ia menganimasikan
plotnya.

Silakan pelajari sumber rotasi untuk melihat lebih detail.
\end{eulercomment}
\begin{eulerprompt}
>function testplot () := plot3d("x^2+y^3"); ...
>rotate("testplot"); testplot():
\end{eulerprompt}
\eulerimg{27}{images/EMT4Plot3D-079.png}
\begin{eulercomment}
Contoh Soal dan Penyelesaian:\\
1. Buatlah grafik dari fungsi berikut\\
\end{eulercomment}
\begin{eulerformula}
\[
f(x,y)=2x^4+y^6
\]
\end{eulerformula}
\begin{eulercomment}
Penyelesaian:
\end{eulercomment}
\begin{eulerprompt}
>function testplot () := plot3d("2x^4+y^6"); ...
>rotate("testplot"); testplot():
\end{eulerprompt}
\eulerimg{27}{images/EMT4Plot3D-081.png}
\eulerheading{Menggambar Povray}
\begin{eulercomment}
Dengan bantuan file Euler povray.e, Euler dapat menghasilkan file
Povray. Hasilnya sangat bagus untuk dilihat.

Anda perlu menginstal Povray (32bit atau 64bit) dari
http://www.povray.org/, dan meletakkan sub-direktori "bin" Povray ke jalur lingkungan, atau mengatur variabel "defaultpovray" dengan jalur lengkap yang mengarah ke "pvengine.exe".


Antarmuka Povray Euler menghasilkan file Povray di direktori home
pengguna, dan memanggil Povray untuk menguraikan file-file ini. Nama
file default adalah current.pov, dan direktori default adalah
eulerhome(), biasanya c:\textbackslash{}Users\textbackslash{}Username\textbackslash{}Euler. Povray menghasilkan
file PNG, yang dapat dimuat oleh Euler ke dalam notebook. Untuk
membersihkan file-file ini, gunakan povclear().


Fungsi pov3d memiliki semangat yang sama dengan plot3d. Ini dapat
menghasilkan grafik fungsi f(x,y), atau permukaan dengan koordinat
X,Y,Z dalam matriks, termasuk garis level opsional. Fungsi ini memulai
raytracer secara otomatis, dan memuat adegan ke dalam notebook Euler.


Selain pov3d(), ada banyak fungsi yang menghasilkan objek Povray.
Fungsi-fungsi ini mengembalikan string, yang berisi kode Povray untuk
objek. Untuk menggunakan fungsi ini, mulai file Povray dengan
povstart(). Kemudian gunakan writeln(...) untuk menulis objek ke file
adegan. Terakhir, akhiri file dengan povend(). Secara default,
raytracer akan dimulai, dan PNG akan dimasukkan ke dalam notebook
Euler.

Fungsi objek memiliki parameter yang disebut "tampilan", yang
memerlukan string dengan kode Povray untuk tekstur dan penyelesaian
objek. Fungsi povlook() dapat digunakan untuk menghasilkan string ini.
Ini memiliki parameter untuk warna, transparansi, Phong Shading dll.


Perhatikan bahwa alam semesta Povray memiliki sistem koordinat lain.
Antarmuka ini menerjemahkan semua koordinat ke sistem Povray. Jadi
Anda dapat terus berpikir dalam sistem koordinat Euler dengan z
menunjuk vertikal ke atas, dan sumbu x,y,z di tangan kanan. Fungsi
pov3d memiliki semangat yang sama dengan plot3d. Ini dapat
menghasilkan grafik fungsi f(x,y), atau permukaan dengan koordinat
X,Y,Z dalam matriks, termasuk garis level opsional. Fungsi ini memulai
raytracer secara otomatis, dan memuat adegan ke dalam notebook Euler.\\
Anda perlu memuat file povray
\end{eulercomment}
\begin{eulerprompt}
>load povray;
\end{eulerprompt}
\begin{eulercomment}
Pastikan, direktori Povray bin ada di jalurnya. Jika tidak, edit
variabel berikut sehingga berisi jalur ke povray yang dapat
dieksekusi.
\end{eulercomment}
\begin{eulerprompt}
>defaultpovray="C:\(\backslash\)Program Files\(\backslash\)POV-Ray\(\backslash\)v3.7\(\backslash\)bin\(\backslash\)pvengine.exe"
\end{eulerprompt}
\begin{euleroutput}
  C:\(\backslash\)Program Files\(\backslash\)POV-Ray\(\backslash\)v3.7\(\backslash\)bin\(\backslash\)pvengine.exe
\end{euleroutput}
\begin{eulercomment}
Untuk kesan pertama, kami memplot fungsi sederhana. Perintah berikut
menghasilkan file povray di direktori pengguna Anda, dan menjalankan
Povray untuk penelusuran sinar file ini.


Jika Anda memulai perintah berikut, GUI Povray akan terbuka,
menjalankan file, dan menutup secara otomatis. Karena alasan keamanan,
Anda akan ditanya apakah Anda ingin mengizinkan file exe dijalankan.
Anda dapat menekan batal untuk menghentikan pertanyaan lebih lanjut.
Anda mungkin harus menekan OK di jendela Povray untuk mengonfirmasi
dialog pengaktifan Povray.
\end{eulercomment}
\begin{eulerprompt}
>plot3d("x^2+y^2",zoom=2):
\end{eulerprompt}
\eulerimg{27}{images/EMT4Plot3D-082.png}
\begin{eulerprompt}
>pov3d("x^2+y^2",zoom=3);
\end{eulerprompt}
\eulerimg{28}{images/EMT4Plot3D-083.png}
\begin{eulercomment}
Kita dapat membuat fungsinya transparan dan menambahkan penyelesaian
lainnya. Kita juga dapat menambahkan garis level ke plot fungsi.
\end{eulercomment}
\begin{eulerprompt}
>pov3d("x^2+y^3",axiscolor=red,angle=-45°,>anaglyph, ...
>  look=povlook(cyan,0.2),level=-1:0.5:1,zoom=3.8);
\end{eulerprompt}
\eulerimg{27}{images/EMT4Plot3D-084.png}
\begin{eulercomment}
Terkadang perlu untuk mencegah penskalaan fungsi, dan menskalakan
fungsi secara manual.

Kita memplot himpunan titik pada bidang kompleks, dimana hasil kali
jarak ke 1 dan -1 sama dengan 1.
\end{eulercomment}
\begin{eulerprompt}
>pov3d("((x-1)^2+y^2)*((x+1)^2+y^2)/40",r=2, ...
>  angle=-120°,level=1/40,dlevel=0.005,light=[-1,1,1],height=10°,n=50, ...
>  <fscale,zoom=3.8);
\end{eulerprompt}
\eulerimg{28}{images/EMT4Plot3D-085.png}
\begin{eulercomment}
Contoh Soal dan Penyelesaian:\\
1. Buatlah grafik pov3d dari fungsi berikut ini\\
\end{eulercomment}
\begin{eulerformula}
\[
f(x,y)= 5x^3+4y^2
\]
\end{eulerformula}
\begin{eulercomment}
Penyelesaian:
\end{eulercomment}
\begin{eulerprompt}
>pov3d("5x^3+4y^2",axiscolor=red,angle=-60°,>anaglyph, ...
>  look=povlook(cyan,0.2),level=-1:0.5:1,zoom=4.6);
\end{eulerprompt}
\eulerimg{27}{images/EMT4Plot3D-087.png}
\eulerheading{Merencanakan dengan Koordinat}
\begin{eulercomment}
Daripada menggunakan fungsi, kita bisa memplotnya dengan koordinat.
Seperti di plot3d, kita memerlukan tiga matriks untuk mendefinisikan
objek.

Dalam contoh ini kita memutar suatu fungsi di sekitar sumbu z.
\end{eulercomment}
\begin{eulerprompt}
>function f(x) := x^3-x+1; ...
>x=-1:0.01:1; t=linspace(0,2pi,50)'; ...
>Z=x; X=cos(t)*f(x); Y=sin(t)*f(x); ...
>pov3d(X,Y,Z,angle=40°,look=povlook(red,0.1),height=50°,axis=0,zoom=4,light=[10,5,15]);
\end{eulerprompt}
\eulerimg{28}{images/EMT4Plot3D-088.png}
\begin{eulercomment}
Pada contoh berikut, kita memplot gelombang teredam. Kami menghasilkan
gelombang dengan bahasa matriks Euler.

Kami juga menunjukkan, bagaimana objek tambahan dapat ditambahkan ke
adegan pov3d. Untuk pembuatan objek, lihat contoh berikut. Perhatikan
bahwa plot3d menskalakan plot, sehingga cocok dengan kubus satuan.
\end{eulercomment}
\begin{eulerprompt}
>r=linspace(0,1,80); phi=linspace(0,2pi,80)'; ...
>x=r*cos(phi); y=r*sin(phi); z=exp(-5*r)*cos(8*pi*r)/3;  ...
>pov3d(x,y,z,zoom=6,axis=0,height=30°,add=povsphere([0.5,0,0.25],0.15,povlook(red)), ...
>  w=500,h=300);
\end{eulerprompt}
\eulerimg{16}{images/EMT4Plot3D-089.png}
\begin{eulercomment}
Dengan metode peneduh canggih Povray, sangat sedikit titik yang dapat
menghasilkan permukaan yang sangat halus. Hanya pada batas-batas dan
dalam bayangan, triknya mungkin terlihat jelas.

Untuk ini, kita perlu menjumlahkan vektor normal di setiap titik
matriks.
\end{eulercomment}
\begin{eulerprompt}
>Z &= x^2*y^3
\end{eulerprompt}
\begin{euleroutput}
  
                                   2  3
                                  x  y
  
\end{euleroutput}
\begin{eulercomment}
Persamaan permukaannya adalah [x,y,Z]. Kami menghitung dua turunan
dari x dan y dan mengambil perkalian silangnya sebagai normal.
\end{eulercomment}
\begin{eulerprompt}
>dx &= diff([x,y,Z],x); dy &= diff([x,y,Z],y);
\end{eulerprompt}
\begin{eulercomment}
Kami mendefinisikan normal sebagai produk silang dari turunan ini, dan
mendefinisikan fungsi koordinat
\end{eulercomment}
\begin{eulerprompt}
>N &= crossproduct(dx,dy); NX &= N[1]; NY &= N[2]; NZ &= N[3]; N,
\end{eulerprompt}
\begin{euleroutput}
  
                                 3       2  2
                         [- 2 x y , - 3 x  y , 1]
  
\end{euleroutput}
\begin{eulercomment}
Kami hanya menggunakan 25 poin.
\end{eulercomment}
\begin{eulerprompt}
>x=-1:0.5:1; y=x';
>pov3d(x,y,Z(x,y),angle=10°, ...
>  xv=NX(x,y),yv=NY(x,y),zv=NZ(x,y),<shadow);
\end{eulerprompt}
\eulerimg{28}{images/EMT4Plot3D-090.png}
\begin{eulercomment}
Berikut ini adalah simpul Trefoil yang dilakukan oleh A. Busser di
Povray. Ada versi yang lebih baik dalam contoh ini.

See: Examples\textbackslash{}Trefoil Knot \textbar{} Trefoil Knot

Untuk tampilan yang bagus dengan poin yang tidak terlalu banyak, kami
menambahkan vektor normal di sini. Kami menggunakan Maxima untuk
menghitung normalnya bagi kami. Pertama, tiga fungsi koordinat sebagai
ekspresi simbolik.
\end{eulercomment}
\begin{eulerprompt}
>X &= ((4+sin(3*y))+cos(x))*cos(2*y); ...
>Y &= ((4+sin(3*y))+cos(x))*sin(2*y); ...
>Z &= sin(x)+2*cos(3*y);
\end{eulerprompt}
\begin{eulercomment}
Kemudian kedua vektor turunan ke x dan y.
\end{eulercomment}
\begin{eulerprompt}
>dx &= diff([X,Y,Z],x); dy &= diff([X,Y,Z],y);
\end{eulerprompt}
\begin{eulercomment}
Sekarang normalnya, yaitu perkalian silang kedua turunannya.
\end{eulercomment}
\begin{eulerprompt}
>dn &= crossproduct(dx,dy);
\end{eulerprompt}
\begin{eulercomment}
Kami sekarang mengevaluasi semua ini secara numerik.
\end{eulercomment}
\begin{eulerprompt}
>x:=linspace(-%pi,%pi,40); y:=linspace(-%pi,%pi,100)';
\end{eulerprompt}
\begin{eulercomment}
Vektor normal adalah evaluasi ekspresi simbolik dn[i] untuk i=1,2,3.
Sintaksnya adalah \&"ekspresi"(parameter). Ini adalah alternatif dari
metode pada contoh sebelumnya, di mana kita mendefinisikan ekspresi
simbolik NX, NY, NZ terlebih dahulu.
\end{eulercomment}
\begin{eulerprompt}
>pov3d(X(x,y),Y(x,y),Z(x,y),>anaglyph,axis=0,zoom=5,w=450,h=350, ...
>  <shadow,look=povlook(blue), ...
>  xv=&"dn[1]"(x,y), yv=&"dn[2]"(x,y), zv=&"dn[3]"(x,y));
\end{eulerprompt}
\eulerimg{21}{images/EMT4Plot3D-091.png}
\begin{eulercomment}
Kami juga dapat menghasilkan grid dalam 3D.
\end{eulercomment}
\begin{eulerprompt}
>povstart(zoom=4); ...
>x=-1:0.5:1; r=1-(x+1)^2/6; ...
>t=(0°:30°:360°)'; y=r*cos(t); z=r*sin(t); ...
>writeln(povgrid(x,y,z,d=0.02,dballs=0.05)); ...
>povend();
\end{eulerprompt}
\eulerimg{28}{images/EMT4Plot3D-092.png}
\begin{eulercomment}
Dengan povgrid(), kurva dimungkinkan.
\end{eulercomment}
\begin{eulerprompt}
>povstart(center=[0,0,1],zoom=3.6); ...
>t=linspace(0,2,1000); r=exp(-t); ...
>x=cos(2*pi*10*t)*r; y=sin(2*pi*10*t)*r; z=t; ...
>writeln(povgrid(x,y,z,povlook(red))); ...
>writeAxis(0,2,axis=3); ...
>povend();
\end{eulerprompt}
\eulerimg{28}{images/EMT4Plot3D-093.png}
\eulerheading{Objek Povray}
\begin{eulercomment}
Di atas, kami menggunakan pov3d untuk memplot permukaan. Antarmuka
povray di Euler juga dapat menghasilkan objek Povray. Objek ini
disimpan sebagai string di Euler, dan perlu ditulis ke file Povray.

Kami memulai output dengan povstart().
\end{eulercomment}
\begin{eulerprompt}
>povstart(zoom=4);
\end{eulerprompt}
\begin{eulercomment}
Pertama kita mendefinisikan tiga silinder, dan menyimpannya dalam
string di Euler.

Fungsi povx() dll. hanya mengembalikan vektor [1,0,0], yang dapat
digunakan sebagai gantinya.
\end{eulercomment}
\begin{eulerprompt}
>c1=povcylinder(-povx,povx,1,povlook(red)); ...
>c2=povcylinder(-povy,povy,1,povlook(yellow)); ...
>c3=povcylinder(-povz,povz,1,povlook(blue)); ...
\end{eulerprompt}
\begin{eulercomment}
String tersebut berisi kode Povray, yang tidak perlu kita pahami pada
saat itu.

Fungsi povx() dll. hanya mengembalikan vektor [1,0,0], yang dapat
digunakan sebagai gantinya.
\end{eulercomment}
\begin{eulerprompt}
>c2
\end{eulerprompt}
\begin{euleroutput}
  cylinder \{ <0,0,-1>, <0,0,1>, 1
   texture \{ pigment \{ color rgb <0.941176,0.941176,0.392157> \}  \} 
   finish \{ ambient 0.2 \} 
   \}
\end{euleroutput}
\begin{eulercomment}
As you see, we added texture to the objects in three different colors.

Hal ini dilakukan oleh povlook(), yang mengembalikan string dengan
kode Povray yang relevan. Kita dapat menggunakan warna default Euler,
atau menentukan warna kita sendiri. Kita juga dapat menambahkan
transparansi, atau mengubah cahaya sekitar.
\end{eulercomment}
\begin{eulerprompt}
>povlook(rgb(0.1,0.2,0.3),0.1,0.5)
\end{eulerprompt}
\begin{euleroutput}
   texture \{ pigment \{ color rgbf <0.101961,0.2,0.301961,0.1> \}  \} 
   finish \{ ambient 0.5 \} 
  
\end{euleroutput}
\begin{eulercomment}
Sekarang kita mendefinisikan objek persimpangan, dan menulis hasilnya
ke file.\\
i dilakukan oleh povlook(), yang mengembalikan string dengan kode
Povray yang relevan. Kita dapat menggunakan warna default Euler, atau
menentukan warna kita sendiri. Kita juga dapat menambahkan
transparansi, atau mengubah cahaya sekitar.
\end{eulercomment}
\begin{eulerprompt}
>writeln(povintersection([c1,c2,c3]));
\end{eulerprompt}
\begin{eulercomment}
Persimpangan tiga silinder sulit untuk divisualisasikan jika Anda
belum pernah melihatnya sebelumnya.
\end{eulercomment}
\begin{eulerprompt}
>povend;
\end{eulerprompt}
\eulerimg{28}{images/EMT4Plot3D-094.png}
\begin{eulercomment}
Fungsi berikut menghasilkan fraktal secara rekursif.

Fungsi pertama menunjukkan bagaimana Euler menangani objek Povray
sederhana. Fungsi povbox() mengembalikan string, yang berisi koordinat
kotak, tekstur, dan hasil akhir.
\end{eulercomment}
\begin{eulerprompt}
>function onebox(x,y,z,d) := povbox([x,y,z],[x+d,y+d,z+d],povlook());
>function fractal (x,y,z,h,n) ...
\end{eulerprompt}
\begin{eulerudf}
   if n==1 then writeln(onebox(x,y,z,h));
   else
     h=h/3;
     fractal(x,y,z,h,n-1);
     fractal(x+2*h,y,z,h,n-1);
     fractal(x,y+2*h,z,h,n-1);
     fractal(x,y,z+2*h,h,n-1);
     fractal(x+2*h,y+2*h,z,h,n-1);
     fractal(x+2*h,y,z+2*h,h,n-1);
     fractal(x,y+2*h,z+2*h,h,n-1);
     fractal(x+2*h,y+2*h,z+2*h,h,n-1);
     fractal(x+h,y+h,z+h,h,n-1);
   endif;
  endfunction
\end{eulerudf}
\begin{eulerprompt}
>povstart(fade=10,<shadow);
>fractal(-1,-1,-1,2,4);
>povend();
\end{eulerprompt}
\eulerimg{28}{images/EMT4Plot3D-095.png}
\begin{eulercomment}
Perbedaan memungkinkan pemisahan satu objek dari objek lainnya.
Seperti persimpangan, ada bagian dari objek CSG di Povray.
\end{eulercomment}
\begin{eulerprompt}
>povstart(light=[5,-5,5],fade=10);
\end{eulerprompt}
\begin{eulercomment}
Untuk demonstrasi ini, kami mendefinisikan objek di Povray, alih-alih
menggunakan string di Euler. Definisi segera ditulis ke file.

Koordinat kotak -1 berarti [-1,-1,-1].
\end{eulercomment}
\begin{eulerprompt}
>povdefine("mycube",povbox(-1,1));
\end{eulerprompt}
\begin{eulercomment}
Kita bisa menggunakan objek ini di povobject(), yang mengembalikan
string seperti biasa.
\end{eulercomment}
\begin{eulerprompt}
>c1=povobject("mycube",povlook(red));
\end{eulerprompt}
\begin{eulercomment}
Kami membuat kubus kedua, dan memutar serta menskalakannya sedikit.
\end{eulercomment}
\begin{eulerprompt}
>c2=povobject("mycube",povlook(yellow),translate=[1,1,1], ...
>  rotate=xrotate(10°)+yrotate(10°), scale=1.2);
\end{eulerprompt}
\begin{eulercomment}
Lalu kita ambil selisih kedua benda tersebut.
\end{eulercomment}
\begin{eulerprompt}
>writeln(povdifference(c1,c2));
\end{eulerprompt}
\begin{eulercomment}
Sekarang tambahkan tiga sumbu.
\end{eulercomment}
\begin{eulerprompt}
>writeAxis(-1.2,1.2,axis=1); ...
>writeAxis(-1.2,1.2,axis=2); ...
>writeAxis(-1.2,1.2,axis=4); ...
>povend();
\end{eulerprompt}
\eulerimg{28}{images/EMT4Plot3D-096.png}
\eulerheading{Fungsi Implisit}
\begin{eulercomment}
Povray dapat memplot himpunan di mana f(x,y,z)=0, seperti parameter
implisit di plot3d. Namun hasilnya terlihat jauh lebih baik.

Sintaks untuk fungsinya sedikit berbeda. Anda tidak dapat menggunakan
keluaran ekspresi Maxima atau Euler.
\end{eulercomment}
\begin{eulerprompt}
>povstart(angle=70°,height=50°,zoom=4);
\end{eulerprompt}
\begin{eulercomment}
Buat permukaan implisit. Perhatikan sintaksis yang berbeda dalam
ekspresi.
\end{eulercomment}
\begin{eulerprompt}
>writeln(povsurface("pow(x,2)*y-pow(y,3)-pow(z,2)",povlook(green))); ...
>writeAxes(); ...
>povend();
\end{eulerprompt}
\eulerimg{28}{images/EMT4Plot3D-097.png}
\eulerheading{Objek Jaring}
\begin{eulercomment}
Dalam contoh ini, kami menunjukkan cara membuat objek mesh, dan
menggambarnya dengan informasi tambahan.

Kita ingin memaksimalkan xy pada kondisi x+y=1 dan mendemonstrasikan
sentuhan tangensial garis datar.
\end{eulercomment}
\begin{eulerprompt}
>povstart(angle=-10°,center=[0.5,0.5,0.5],zoom=7);
\end{eulerprompt}
\begin{eulercomment}
Kita tidak dapat menyimpan objek dalam string seperti sebelumnya,
karena terlalu besar. Jadi kita mendefinisikan objek dalam file Povray
menggunakan #declare. Fungsi povtriangle() melakukan ini secara
otomatis. Ia dapat menerima vektor normal seperti pov3d().

Berikut ini definisi objek mesh, dan segera menuliskannya ke dalam
file.
\end{eulercomment}
\begin{eulerprompt}
>X=0:0.02:1; y=x'; z=x*y; vx=-y; vy=-x; vz=1;
>mesh=povtriangles(x,y,z,"",vx,vy,vz);
\end{eulerprompt}
\begin{eulercomment}
Sekarang kita mendefinisikan dua cakram, yang akan berpotongan dengan
permukaan.
\end{eulercomment}
\begin{eulerprompt}
>cl=povdisc([0.5,0.5,0],[1,1,0],2); ...
>ll=povdisc([0,0,1/4],[0,0,1],2);
\end{eulerprompt}
\begin{eulercomment}
Tulis permukaannya dikurangi kedua cakram.
\end{eulercomment}
\begin{eulerprompt}
>writeln(povdifference(mesh,povunion([cl,ll]),povlook(green)));
\end{eulerprompt}
\begin{eulercomment}
Tuliskan kedua perpotongan tersebut.
\end{eulercomment}
\begin{eulerprompt}
>writeln(povintersection([mesh,cl],povlook(red))); ...
>writeln(povintersection([mesh,ll],povlook(gray)));
\end{eulerprompt}
\begin{eulercomment}
Tulis poin maksimal.
\end{eulercomment}
\begin{eulerprompt}
>writeln(povpoint([1/2,1/2,1/4],povlook(gray),size=2*defaultpointsize));
\end{eulerprompt}
\begin{eulercomment}
Tambahkan sumbu dan selesai.
\end{eulercomment}
\begin{eulerprompt}
>writeAxes(0,1,0,1,0,1,d=0.015); ...
>povend();
\end{eulerprompt}
\eulerimg{28}{images/EMT4Plot3D-098.png}
\eulerheading{Anaglyphs di Povray}
\begin{eulercomment}
Untuk menghasilkan anaglyph untuk kacamata merah/cyan, Povray harus
dijalankan dua kali dari posisi kamera berbeda. Ini menghasilkan dua
file Povray dan dua file PNG, yang dimuat dengan fungsi
loadanaglyph().

Tentu saja, Anda memerlukan kacamata berwarna merah/cyan untuk melihat
contoh berikut dengan benar.

Fungsi pov3d() memiliki saklar sederhana untuk menghasilkan anaglyph.
\end{eulercomment}
\begin{eulerprompt}
>pov3d("-exp(-x^2-y^2)/2",r=2,height=45°,>anaglyph, ...
>  center=[0,0,0.5],zoom=3.5);
\end{eulerprompt}
\eulerimg{27}{images/EMT4Plot3D-099.png}
\begin{eulercomment}
Jika Anda membuat adegan dengan objek, Anda perlu memasukkan pembuatan
adegan ke dalam fungsi, dan menjalankannya dua kali dengan nilai
berbeda untuk parameter anaglyph.
\end{eulercomment}
\begin{eulerprompt}
>function myscene ...
\end{eulerprompt}
\begin{eulerudf}
    s=povsphere(povc,1);
    cl=povcylinder(-povz,povz,0.5);
    clx=povobject(cl,rotate=xrotate(90°));
    cly=povobject(cl,rotate=yrotate(90°));
    c=povbox([-1,-1,0],1);
    un=povunion([cl,clx,cly,c]);
    obj=povdifference(s,un,povlook(red));
    writeln(obj);
    writeAxes();
  endfunction
\end{eulerudf}
\begin{eulercomment}
Fungsi povanaglyph() melakukan semua ini. Parameternya seperti
gabungan povstart() dan povend().
\end{eulercomment}
\begin{eulerprompt}
>povanaglyph("myscene",zoom=4.5);
\end{eulerprompt}
\eulerimg{27}{images/EMT4Plot3D-100.png}
\begin{eulercomment}
Contoh Soal dan Penyelesaian\\
1. Buatlah grafik dari fungsi berikut ini\\
\end{eulercomment}
\begin{eulerformula}
\[
f(x,y)=e^{2x^2+3y^2}
\]
\end{eulerformula}
\begin{eulercomment}
Penyelesaian:
\end{eulercomment}
\begin{eulerprompt}
>pov3d("-exp(2x^2+3y^2)/2",r=2,height=45°,>anaglyph, ...
>  center=[0,0,0.5],zoom=3.5);
\end{eulerprompt}
\eulerimg{27}{images/EMT4Plot3D-102.png}
\eulerheading{Mendefinisikan Objek sendiri}
\begin{eulercomment}
Antarmuka povray Euler berisi banyak objek. Namun Anda tidak dibatasi
pada hal ini. Anda dapat membuat objek sendiri, yang menggabungkan
objek lain, atau merupakan objek yang benar-benar baru.

Kami mendemonstrasikan torus. Perintah Povray untuk ini adalah
"torus". Jadi kami mengembalikan string dengan perintah ini dan
parameternya. Perhatikan bahwa torus selalu berpusat pada titik asal.
\end{eulercomment}
\begin{eulerprompt}
>function povdonat (r1,r2,look="") ...
\end{eulerprompt}
\begin{eulerudf}
    return "torus \{"+r1+","+r2+look+"\}";
  endfunction
\end{eulerudf}
\begin{eulercomment}
Ini torus pertama kami.
\end{eulercomment}
\begin{eulerprompt}
>t1=povdonat(0.8,0.2)
\end{eulerprompt}
\begin{euleroutput}
  torus \{0.8,0.2\}
\end{euleroutput}
\begin{eulercomment}
Mari kita gunakan objek ini untuk membuat torus kedua, diterjemahkan
dan diputar.
\end{eulercomment}
\begin{eulerprompt}
>t2=povobject(t1,rotate=xrotate(90°),translate=[0.8,0,0])
\end{eulerprompt}
\begin{euleroutput}
  object \{ torus \{0.8,0.2\}
   rotate 90 *x 
   translate <0.8,0,0>
   \}
\end{euleroutput}
\begin{eulercomment}
Sekarang kita tempatkan objek-objek tersebut ke dalam sebuah adegan.
Untuk tampilannya kami menggunakan Phong Shading.
\end{eulercomment}
\begin{eulerprompt}
>povstart(center=[0.4,0,0],angle=0°,zoom=3.8,aspect=1.5); ...
>writeln(povobject(t1,povlook(green,phong=1))); ...
>writeln(povobject(t2,povlook(green,phong=1))); ...
\end{eulerprompt}
\begin{eulerttcomment}
 >povend();
\end{eulerttcomment}
\begin{eulercomment}
memanggil program Povray. Namun, jika terjadi kesalahan, kesalahan
tersebut tidak ditampilkan. Oleh karena itu Anda harus menggunakan\\
\end{eulercomment}
\begin{eulerttcomment}
 
 >povend(<exit);
\end{eulerttcomment}
\begin{eulercomment}

jika ada yang tidak berhasil. Ini akan membiarkan jendela Povray
terbuka.
\end{eulercomment}
\begin{eulerprompt}
>povend(h=320,w=480);
\end{eulerprompt}
\eulerimg{18}{images/EMT4Plot3D-103.png}
\begin{eulercomment}
Berikut adalah contoh yang lebih rumit. Kami memecahkannya

\end{eulercomment}
\begin{eulerformula}
\[
Ax \le b, \quad x \ge 0, \quad c.x \to \text{Max.}
\]
\end{eulerformula}
\begin{eulercomment}
dan menunjukkan titik-titik yang layak dan optimal dalam plot 3D.
\end{eulercomment}
\begin{eulerprompt}
>A=[10,8,4;5,6,8;6,3,2;9,5,6];
>b=[10,10,10,10]';
>c=[1,1,1];
\end{eulerprompt}
\begin{eulercomment}
Pertama, mari kita periksa, apakah contoh ini punya solusinya.
\end{eulercomment}
\begin{eulerprompt}
>x=simplex(A,b,c,>max,>check)'
\end{eulerprompt}
\begin{euleroutput}
  [0,  1,  0.5]
\end{euleroutput}
\begin{eulercomment}
Yes, it has.

Selanjutnya kita mendefinisikan dua objek. Yang pertama adalah the
plane

\end{eulercomment}
\begin{eulerformula}
\[
a \cdot x \le b
\]
\end{eulerformula}
\begin{eulerprompt}
>function oneplane (a,b,look="") ...
\end{eulerprompt}
\begin{eulerudf}
    return povplane(a,b,look)
  endfunction
\end{eulerudf}
\begin{eulercomment}
Kemudian kita mendefinisikan perpotongan semua setengah ruang dan
sebuah kubus.
\end{eulercomment}
\begin{eulerprompt}
>function adm (A, b, r, look="") ...
\end{eulerprompt}
\begin{eulerudf}
    ol=[];
    loop 1 to rows(A); ol=ol|oneplane(A[#],b[#]); end;
    ol=ol|povbox([0,0,0],[r,r,r]);
    return povintersection(ol,look);
  endfunction
\end{eulerudf}
\begin{eulercomment}
Sekarang kita dapat merencanakan adegannya.
\end{eulercomment}
\begin{eulerprompt}
>povstart(angle=120°,center=[0.5,0.5,0.5],zoom=3.5); ...
>writeln(adm(A,b,2,povlook(green,0.4))); ...
>writeAxes(0,1.3,0,1.6,0,1.5); ...
\end{eulerprompt}
\begin{eulercomment}
Berikut ini adalah lingkaran di sekitar optimal.
\end{eulercomment}
\begin{eulerprompt}
>writeln(povintersection([povsphere(x,0.5),povplane(c,c.x')], ...
>  povlook(red,0.9)));
\end{eulerprompt}
\begin{eulercomment}
Dan kesalahan ke arah optimal.
\end{eulercomment}
\begin{eulerprompt}
>writeln(povarrow(x,c*0.5,povlook(red)));
\end{eulerprompt}
\begin{eulercomment}
Kami menambahkan teks ke layar. Teks hanyalah objek 3D. Kita perlu
menempatkan dan memutarnya sesuai dengan pandangan kita.
\end{eulercomment}
\begin{eulerprompt}
>writeln(povtext("Linear Problem",[0,0.2,1.3],size=0.05,rotate=5°)); ...
>povend();
\end{eulerprompt}
\eulerimg{28}{images/EMT4Plot3D-106.png}
\eulerheading{Contoh Lainnya}
\begin{eulercomment}
Anda dapat menemukan beberapa contoh Povray di Euler di file berikut.


ee: Examples/Dandelin Spheres\\
See: Examples/Donat Math\\
See: Examples/Trefoil Knot\\
See: Examples/Optimization by Affine Scaling
\end{eulercomment}
\end{eulernotebook}

\chapter{EMT Untuk Kalkulus}
\begin{eulernotebook}
\eulerheading{Kalkulus dengan EMT}
\begin{eulercomment}
Materi Kalkulus mencakup di antaranya:

- Fungsi (fungsi aljabar, trigonometri, eksponensial, logaritma,
komposisi fungsi)\\
- Limit Fungsi,\\
- Turunan Fungsi,\\
- Integral Tak Tentu,\\
- Integral Tentu dan Aplikasinya,\\
- Barisan dan Deret (kekonvergenan barisan dan deret).

EMT (bersama Maxima) dapat digunakan untuk melakukan semua perhitungan
di dalam kalkulus, baik secara numerik maupun analitik (eksak).

\end{eulercomment}
\eulersubheading{Mendefinisikan Fungsi}
\begin{eulercomment}
Terdapat beberapa cara mendefinisikan fungsi pada EMT, yakni:

- Menggunakan format nama\_fungsi := rumus fungsi (untuk fungsi
numerik),\\
- Menggunakan format nama\_fungsi \&= rumus fungsi (untuk fungsi
simbolik, namun dapat dihitung secara numerik),\\
- Menggunakan format nama\_fungsi \&\&= rumus fungsi (untuk fungsi
simbolik murni, tidak dapat dihitung langsung),\\
- Fungsi sebagai program EMT.

Setiap format harus diawali dengan perintah function (bukan sebagai
ekspresi).

Berikut adalah adalah beberapa contoh cara mendefinisikan fungsi:

\end{eulercomment}
\begin{eulerformula}
\[
f(x)=2x^2+e^{\sin(x)}.
\]
\end{eulerformula}
\begin{eulerprompt}
>function f(x) := 2*x^2+exp(sin(x)) // fungsi numerik
>f(0), f(1), f(pi)
\end{eulerprompt}
\begin{euleroutput}
  1
  4.31977682472
  20.7392088022
\end{euleroutput}
\begin{eulerprompt}
>f(a) // tidak dapat dihitung nilainya
\end{eulerprompt}
\begin{euleroutput}
  Variable or function a not found.
  Error in:
  f(a) // tidak dapat dihitung nilainya ...
     ^
\end{euleroutput}
\begin{eulercomment}
Silakan Anda plot kurva fungsi di atas!

Berikutnya kita definisikan fungsi:

\end{eulercomment}
\begin{eulerformula}
\[
g(x)=\frac{\sqrt{x^2-3x}}{x+1}.
\]
\end{eulerformula}
\begin{eulerprompt}
>function g(x) := sqrt(x^2-3*x)/(x+1)
>g(3)
\end{eulerprompt}
\begin{euleroutput}
  10
\end{euleroutput}
\begin{eulerprompt}
>g(0)
\end{eulerprompt}
\begin{euleroutput}
  1
\end{euleroutput}
\begin{eulerprompt}
>g(1) // kompleks, tidak dapat dihitung oleh fungsi numerik
\end{eulerprompt}
\begin{euleroutput}
  4
\end{euleroutput}
\begin{eulercomment}
Silakan Anda plot kurva fungsi di atas!
\end{eulercomment}
\begin{eulerprompt}
>f(g(5)) // komposisi fungsi
\end{eulerprompt}
\begin{euleroutput}
  17772221.041
\end{euleroutput}
\begin{eulerprompt}
>g(f(5))
\end{eulerprompt}
\begin{euleroutput}
  891.478954615
\end{euleroutput}
\begin{eulerprompt}
>function h(x) := f(g(x)) // definisi komposisi fungsi 
>h(5) // sama dengan f(g(5))
\end{eulerprompt}
\begin{euleroutput}
  17772221.041
\end{euleroutput}
\begin{eulercomment}
Silakan Anda plot kurva fungsi komposisi fungsi f dan g:

\end{eulercomment}
\begin{eulerformula}
\[
h(x)=f(g(x))
\]
\end{eulerformula}
\begin{eulercomment}
dan

\end{eulercomment}
\begin{eulerformula}
\[
u(x)=g(f(x))
\]
\end{eulerformula}
\begin{eulercomment}
bersama-sama kurva fungsi f dan g dalam satu bidang koordinat.
\end{eulercomment}
\begin{eulerprompt}
>f(0:10) // nilai-nilai f(0), f(1), f(2), ..., f(10)
\end{eulerprompt}
\begin{euleroutput}
  [2,  5.43656,  14.7781,  40.1711,  109.196,  296.826,  806.858,
  2193.27,  5961.92,  16206.2,  44052.9]
\end{euleroutput}
\begin{eulerprompt}
>fmap(0:10) // sama dengan f(0:10), berlaku untuk semua fungsi
\end{eulerprompt}
\begin{euleroutput}
  [2,  5.43656,  14.7781,  40.1711,  109.196,  296.826,  806.858,
  2193.27,  5961.92,  16206.2,  44052.9]
\end{euleroutput}
\begin{eulerprompt}
>gmap(200:210)
\end{eulerprompt}
\begin{euleroutput}
  [601,  604,  607,  610,  613,  616,  619,  622,  625,  628,  631]
\end{euleroutput}
\begin{eulercomment}
Misalkan kita akan mendefinisikan fungsi

\end{eulercomment}
\begin{eulerformula}
\[
f(x) = \begin{cases} x^3 & x>0 \\ x^2 & x\le 0. \end{cases}
\]
\end{eulerformula}
\begin{eulercomment}
Fungsi tersebut tidak dapat didefinisikan sebagai fungsi numerik
secara "inline" menggunakan format :=, melainkan didefinisikan sebagai
program. Perhatikan, kata "map" digunakan agar fungsi dapat menerima
vektor sebagai input, dan hasilnya berupa vektor. Jika tanpa kata
"map" fungsinya hanya dapat menerima input satu nilai.
\end{eulercomment}
\begin{eulerprompt}
>function map f(x) ...
\end{eulerprompt}
\begin{eulerudf}
    if x>0 then return x^3
    else return x^2
    endif;
  endfunction
\end{eulerudf}
\begin{eulerprompt}
>f(1)
\end{eulerprompt}
\begin{euleroutput}
  1
\end{euleroutput}
\begin{eulerprompt}
>f(-2)
\end{eulerprompt}
\begin{euleroutput}
  4
\end{euleroutput}
\begin{eulerprompt}
>f(-5:5)
\end{eulerprompt}
\begin{euleroutput}
  [25,  16,  9,  4,  1,  0,  1,  8,  27,  64,  125]
\end{euleroutput}
\begin{eulerprompt}
>aspect(1.5); plot2d("f(x)",-5,5):
\end{eulerprompt}
\eulerimg{12}{images/EMT4Kalkulus_Amalia Intan Arvitasari_22305144026-004.png}
\begin{eulerprompt}
>function f(x) &= 2*E^x // fungsi simbolik
\end{eulerprompt}
\begin{euleroutput}
  
                                      x
                                   2 E
  
\end{euleroutput}
\begin{eulerprompt}
>$f(a) // nilai fungsi secara simbolik
\end{eulerprompt}
\begin{eulerformula}
\[
2\,e^{a}
\]
\end{eulerformula}
\begin{eulerprompt}
>f(E) // nilai fungsi berupa bilangan desimal
\end{eulerprompt}
\begin{euleroutput}
  30.308524483
\end{euleroutput}
\begin{eulerprompt}
>$f(E), $float(%)
\end{eulerprompt}
\begin{eulerformula}
\[
30.30852448295852
\]
\end{eulerformula}
\eulerimg{0}{images/EMT4Kalkulus_Amalia Intan Arvitasari_22305144026-007-large.png}
\begin{eulerprompt}
>function g(x) &= 3*x+1
\end{eulerprompt}
\begin{euleroutput}
  
                                 3 x + 1
  
\end{euleroutput}
\begin{eulerprompt}
>function h(x) &= f(g(x)) // komposisi fungsi
\end{eulerprompt}
\begin{euleroutput}
  
                                   3 x + 1
                                2 E
  
\end{euleroutput}
\begin{eulerprompt}
>plot2d("h(x)",-1,1):
\end{eulerprompt}
\eulerimg{12}{images/EMT4Kalkulus_Amalia Intan Arvitasari_22305144026-008.png}
\eulerheading{Latihan}
\begin{eulercomment}
Bukalah buku Kalkulus. Cari dan pilih beberapa (paling sedikit 5
fungsi berbeda tipe/bentuk/jenis) fungsi dari buku tersebut, kemudian
definisikan fungsi-fungsi tersebut dan komposisinya di EMT pada
baris-baris perintah berikut (jika perlu tambahkan lagi). Untuk setiap
fungsi, hitung beberapa nilainya, baik untuk satu nilai maupun vektor.
Gambar grafik fungsi-fungsi tersebut dan komposisi-komposisi 2 fungsi.

Juga, carilah fungsi beberapa (dua) variabel. Lakukan hal sama seperti
di atas.

Soal 1:
\end{eulercomment}
\begin{eulerprompt}
>function r(x,y):=sqrt(25-(x^2+2y^2))
>plot3d("r",>user):
\end{eulerprompt}
\eulerimg{12}{images/EMT4Kalkulus_Amalia Intan Arvitasari_22305144026-009.png}
\begin{eulercomment}
Soal 2:

Misalkan kita mendefinisikan fungsi

\end{eulercomment}
\begin{eulerformula}
\[
f(x)= \begin{cases} 2x+3 & x>0 \\ x^2-2 & x\le 0. \end{cases}
\]
\end{eulerformula}
\begin{eulerprompt}
>function map f(x) ...
\end{eulerprompt}
\begin{eulerudf}
    if x>0 then return 2*x+3
    else return x^2-2
    endif;
  endfunction
\end{eulerudf}
\begin{eulerprompt}
>f(1)
\end{eulerprompt}
\begin{euleroutput}
  5
\end{euleroutput}
\begin{eulerprompt}
>f(4)
\end{eulerprompt}
\begin{euleroutput}
  11
\end{euleroutput}
\begin{eulercomment}
Soal 3:
\end{eulercomment}
\begin{eulerprompt}
>function k(x,y):=sqrt(1-(x^2+y^2))
>k(1,0)
\end{eulerprompt}
\begin{euleroutput}
  0
\end{euleroutput}
\begin{eulerprompt}
>k(0,1)
\end{eulerprompt}
\begin{euleroutput}
  0
\end{euleroutput}
\begin{eulercomment}
Soal 4:
\end{eulercomment}
\begin{eulerprompt}
>function m(x,y):= x^3-2*x*y+3*y
>m(-2,3)
\end{eulerprompt}
\begin{euleroutput}
  13
\end{euleroutput}
\begin{eulerprompt}
>m(1/9:2,2:3)
\end{eulerprompt}
\begin{euleroutput}
  [5.55693,  3.70508]
\end{euleroutput}
\begin{eulercomment}
Soal 5:
\end{eulercomment}
\begin{eulerprompt}
>function n(x):= x^2+5
>plot2d("n(x)",-5,5):
\end{eulerprompt}
\eulerimg{12}{images/EMT4Kalkulus_Amalia Intan Arvitasari_22305144026-011.png}
\begin{eulercomment}
\begin{eulercomment}
\eulerheading{Menghitung Limit}
\begin{eulercomment}
Perhitungan limit pada EMT dapat dilakukan dengan menggunakan fungsi
Maxima, yakni "limit". Fungsi "limit" dapat digunakan untuk menghitung
limit fungsi dalam bentuk ekspresi maupun fungsi yang sudah
didefinisikan sebelumnya. Nilai limit dapat dihitung pada sebarang
nilai atau pada tak hingga (-inf, minf, dan inf). Limit kiri dan limit
kanan juga dapat dihitung, dengan cara memberi opsi "plus" atau
"minus". Hasil limit dapat berupa nilai, "und" (tak definisi), "ind"
(tak tentu namun terbatas), "infinity" (kompleks tak hingga).

Perhatikan beberapa contoh berikut. Perhatikan cara menampilkan
perhitungan secara lengkap, tidak hanya menampilkan hasilnya saja.
\end{eulercomment}
\begin{eulerprompt}
>$showev('limit(sqrt(x^2-3*x)/(x+1),x,inf))
\end{eulerprompt}
\begin{eulerformula}
\[
\lim_{x\rightarrow \infty }{\frac{\sqrt{x^2-3\,x}}{x+1}}=1
\]
\end{eulerformula}
\begin{eulerprompt}
>$limit((x^3-13*x^2+51*x-63)/(x^3-4*x^2-3*x+18),x,3)
\end{eulerprompt}
\begin{eulerformula}
\[
-\frac{4}{5}
\]
\end{eulerformula}
\begin{eulerformula}
\[
\lim_{x\rightarrow 3}{\frac{x^3-13\,x^2+51\,x-63}{x^3-4\,x^2-3\,x+  18}}=-\frac{4}{5}
\]
\end{eulerformula}
\begin{eulercomment}
Fungsi tersebut diskontinu di titik x=3. Berikut adalah grafik
fungsinya.
\end{eulercomment}
\begin{eulerprompt}
>aspect(1.5); plot2d("(x^3-13*x^2+51*x-63)/(x^3-4*x^2-3*x+18)",0,4); plot2d(3,-4/5,>points,style="ow",>add):
\end{eulerprompt}
\eulerimg{12}{images/EMT4Kalkulus_Amalia Intan Arvitasari_22305144026-015.png}
\begin{eulerprompt}
>$limit(2*x*sin(x)/(1-cos(x)),x,0)
\end{eulerprompt}
\begin{eulerformula}
\[
4
\]
\end{eulerformula}
\begin{eulerformula}
\[
2\,\left(\lim_{x\rightarrow 0}{\frac{x\,\sin x}{1-\cos x}}\right)=4
\]
\end{eulerformula}
\begin{eulercomment}
Fungsi tersebut diskontinu di titik x=0. Berikut adalah grafik
fungsinya.
\end{eulercomment}
\begin{eulerprompt}
>plot2d("2*x*sin(x)/(1-cos(x))",-pi,pi); plot2d(0,4,>points,style="ow",>add):
\end{eulerprompt}
\eulerimg{12}{images/EMT4Kalkulus_Amalia Intan Arvitasari_22305144026-018.png}
\begin{eulerprompt}
>$limit(cot(7*h)/cot(5*h),h,0)
\end{eulerprompt}
\begin{eulerformula}
\[
\frac{5}{7}
\]
\end{eulerformula}
\begin{eulerformula}
\[
\lim_{h\rightarrow 0}{\frac{\cot \left(7\,h\right)}{\cot \left(5\,h  \right)}}=\frac{5}{7}
\]
\end{eulerformula}
\begin{eulercomment}
Fungsi tersebut juga diskontinu (karena tidak terdefinisi) di x=0.
Berikut adalah grafiknya.
\end{eulercomment}
\begin{eulerprompt}
>plot2d("cot(7*x)/cot(5*x)",-0.001,0.001); plot2d(0,5/7,>points,style="ow",>add):
\end{eulerprompt}
\eulerimg{12}{images/EMT4Kalkulus_Amalia Intan Arvitasari_22305144026-021.png}
\begin{eulerprompt}
>$showev('limit(((x/8)^(1/3)-1)/(x-8),x,8))
\end{eulerprompt}
\begin{eulerformula}
\[
\lim_{x\rightarrow 8}{\frac{\frac{x^{\frac{1}{3}}}{2}-1}{x-8}}=  \frac{1}{24}
\]
\end{eulerformula}
\begin{eulercomment}
Tunjukkan limit tersebut dengan grafik, seperti contoh-contoh
sebelumnya.\\
Penyelesaian:
\end{eulercomment}
\begin{eulerprompt}
>plot2d("((x/8)^(1/3)-1)/(x-8)",0,12); plot2d(8, 1/24,>points,style="ow",>add):
\end{eulerprompt}
\eulerimg{12}{images/EMT4Kalkulus_Amalia Intan Arvitasari_22305144026-023.png}
\begin{eulerprompt}
>$showev('limit(1/(2*x-1),x,0))
\end{eulerprompt}
\begin{eulerformula}
\[
\lim_{x\rightarrow 0}{\frac{1}{2\,x-1}}=-1
\]
\end{eulerformula}
\begin{eulercomment}
Tunjukkan limit tersebut dengan grafik, seperti contoh-contoh
sebelumnya.\\
Penyelesaian:
\end{eulercomment}
\begin{eulerprompt}
>plot2d("(1/(2*x-1))",-5,5); plot2d(0,-1,>points,style="ow",>add):
\end{eulerprompt}
\eulerimg{12}{images/EMT4Kalkulus_Amalia Intan Arvitasari_22305144026-025.png}
\begin{eulerprompt}
>$showev('limit((x^2-3*x-10)/(x-5),x,5))
\end{eulerprompt}
\begin{eulerformula}
\[
\lim_{x\rightarrow 5}{\frac{x^2-3\,x-10}{x-5}}=7
\]
\end{eulerformula}
\begin{eulercomment}
Tunjukkan limit tersebut dengan grafik, seperti contoh-contoh
sebelumnya.\\
Penyelesaian:
\end{eulercomment}
\begin{eulerprompt}
>plot2d("((x^2-3*x-10)/(x-5))",-2,6); plot2d(5,7,>points,style="ow",>add):
\end{eulerprompt}
\eulerimg{12}{images/EMT4Kalkulus_Amalia Intan Arvitasari_22305144026-027.png}
\begin{eulerprompt}
> 
>$showev('limit(sqrt(x^2+x)-x,x,inf))
\end{eulerprompt}
\begin{eulerformula}
\[
\lim_{x\rightarrow \infty }{\sqrt{x^2+x}-x}=\frac{1}{2}
\]
\end{eulerformula}
\begin{eulercomment}
Tunjukkan limit tersebut dengan grafik, seperti contoh-contoh
sebelumnya.\\
Penyelesaian:
\end{eulercomment}
\begin{eulerprompt}
>plot2d("(sqrt(x^2+x)-x)",0,10):
\end{eulerprompt}
\eulerimg{12}{images/EMT4Kalkulus_Amalia Intan Arvitasari_22305144026-029.png}
\begin{eulerprompt}
> 
>$showev('limit(abs(x-1)/(x-1),x,1,minus))
\end{eulerprompt}
\begin{eulerformula}
\[
\lim_{x\uparrow 1}{\frac{\left| x-1\right| }{x-1}}=-1
\]
\end{eulerformula}
\begin{eulercomment}
Hitung limit di atas untuk x menuju 1 dari kanan.\\
Tunjukkan limit tersebut dengan grafik, seperti contoh-contoh
sebelumnya.\\
Penyelesaian:
\end{eulercomment}
\begin{eulerprompt}
>plot2d("abs(x-1)/(x-1)",-1,3,-3,3); plot2d(1,-1,>points,style="ow",>add):
\end{eulerprompt}
\eulerimg{12}{images/EMT4Kalkulus_Amalia Intan Arvitasari_22305144026-031.png}
\begin{eulerprompt}
>$showev('limit(abs(x-1)/(x-1),x,1,plus))
\end{eulerprompt}
\begin{eulerformula}
\[
\lim_{x\downarrow 1}{\frac{\left| x-1\right| }{x-1}}=1
\]
\end{eulerformula}
\begin{eulerprompt}
>plot2d("abs(x-1)/(x-1)",-1,3,-3,3); plot2d(1,1,>points,style="ow",>add):
\end{eulerprompt}
\eulerimg{12}{images/EMT4Kalkulus_Amalia Intan Arvitasari_22305144026-033.png}
\begin{eulerprompt}
>$showev('limit(sin(x)/x,x,0))
\end{eulerprompt}
\begin{eulerformula}
\[
\lim_{x\rightarrow 0}{\frac{\sin x}{x}}=1
\]
\end{eulerformula}
\begin{eulerprompt}
>plot2d("sin(x)/x",-pi,pi); plot2d(0,1,>points,style="ow",>add):
\end{eulerprompt}
\eulerimg{12}{images/EMT4Kalkulus_Amalia Intan Arvitasari_22305144026-035.png}
\begin{eulerprompt}
>$showev('limit(sin(x^3)/x,x,0))
\end{eulerprompt}
\begin{eulerformula}
\[
\lim_{x\rightarrow 0}{\frac{\sin x^3}{x}}=0
\]
\end{eulerformula}
\begin{eulercomment}
Tunjukkan limit tersebut dengan grafik, seperti contoh-contoh
sebelumnya.\\
Penyelesaian:
\end{eulercomment}
\begin{eulerprompt}
>plot2d("sin(x^3)/x",-pi,pi); plot2d(0,0,>points,style="cow",>add):
\end{eulerprompt}
\eulerimg{12}{images/EMT4Kalkulus_Amalia Intan Arvitasari_22305144026-037.png}
\begin{eulerprompt}
>$showev('limit(log(x), x, minf))
\end{eulerprompt}
\begin{eulerformula}
\[
\lim_{x\rightarrow  -\infty }{\log x}={\it infinity}
\]
\end{eulerformula}
\begin{eulerprompt}
>$showev('limit((-2)^x,x, inf))
\end{eulerprompt}
\begin{eulerformula}
\[
\lim_{x\rightarrow \infty }{\left(-2\right)^{x}}={\it infinity}
\]
\end{eulerformula}
\begin{eulerprompt}
>$showev('limit(t-sqrt(2-t),t,2,minus))
\end{eulerprompt}
\begin{eulerformula}
\[
\lim_{t\uparrow 2}{t-\sqrt{2-t}}=2
\]
\end{eulerformula}
\begin{eulerprompt}
>$showev('limit(t-sqrt(2-t),t,2,plus))
\end{eulerprompt}
\begin{eulerformula}
\[
\lim_{t\downarrow 2}{t-\sqrt{2-t}}=2
\]
\end{eulerformula}
\begin{eulerprompt}
>$showev('limit(t-sqrt(2-t),t,5,plus)) // Perhatikan hasilnya
\end{eulerprompt}
\begin{eulerformula}
\[
\lim_{t\downarrow 5}{t-\sqrt{2-t}}=5-\sqrt{3}\,i
\]
\end{eulerformula}
\begin{eulerprompt}
>plot2d("x-sqrt(2-x)",0,2):
\end{eulerprompt}
\eulerimg{12}{images/EMT4Kalkulus_Amalia Intan Arvitasari_22305144026-043.png}
\begin{eulerprompt}
>$showev('limit((x^2-9)/(2*x^2-5*x-3),x,3))
\end{eulerprompt}
\begin{eulerformula}
\[
\lim_{x\rightarrow 3}{\frac{x^2-9}{2\,x^2-5\,x-3}}=\frac{6}{7}
\]
\end{eulerformula}
\begin{eulercomment}
Tunjukkan limit tersebut dengan grafik, seperti contoh-contoh
sebelumnya.\\
Penyelesaian:
\end{eulercomment}
\begin{eulerprompt}
>plot2d("(x^2-9)/(2*x^2-5*x-3)",-pi,2pi); plot2d(3,6/7,>points,style="ow",>add):
\end{eulerprompt}
\eulerimg{12}{images/EMT4Kalkulus_Amalia Intan Arvitasari_22305144026-045.png}
\begin{eulerprompt}
>$showev('limit((1-cos(x))/x,x,0))
\end{eulerprompt}
\begin{eulerformula}
\[
\lim_{x\rightarrow 0}{\frac{1-\cos x}{x}}=0
\]
\end{eulerformula}
\begin{eulercomment}
Tunjukkan limit tersebut dengan grafik, seperti contoh-contoh
sebelumnya.\\
Penyelesaian:
\end{eulercomment}
\begin{eulerprompt}
>plot2d("(1-cos(x))/x",-pi,pi); plot2d(0,0,>points,style="ow",>add):
\end{eulerprompt}
\eulerimg{12}{images/EMT4Kalkulus_Amalia Intan Arvitasari_22305144026-047.png}
\begin{eulerprompt}
>$showev('limit((x^2+abs(x))/(x^2-abs(x)),x,0))
\end{eulerprompt}
\begin{eulerformula}
\[
\lim_{x\rightarrow 0}{\frac{\left| x\right| +x^2}{x^2-\left| x  \right| }}=-1
\]
\end{eulerformula}
\begin{eulercomment}
Tunjukkan limit tersebut dengan grafik, seperti contoh-contoh
sebelumnya.\\
penyelesaian:
\end{eulercomment}
\begin{eulerprompt}
>plot2d("(x^2+abs(x))/(x^2-abs(x))",-1,1,-5,5); plot2d(0,-1,>points,style="ow",>add):
\end{eulerprompt}
\eulerimg{12}{images/EMT4Kalkulus_Amalia Intan Arvitasari_22305144026-049.png}
\begin{eulerprompt}
>$showev('limit((1+1/x)^x,x,inf))
\end{eulerprompt}
\begin{eulerformula}
\[
\lim_{x\rightarrow \infty }{\left(\frac{1}{x}+1\right)^{x}}=e
\]
\end{eulerformula}
\begin{eulerprompt}
>plot2d("(1+1/x)^x",0,1000):
\end{eulerprompt}
\eulerimg{12}{images/EMT4Kalkulus_Amalia Intan Arvitasari_22305144026-051.png}
\begin{eulerprompt}
>$showev('limit((1+k/x)^x,x,inf))
\end{eulerprompt}
\begin{eulerformula}
\[
\lim_{x\rightarrow \infty }{\left(\frac{k}{x}+1\right)^{x}}=e^{k}
\]
\end{eulerformula}
\begin{eulerprompt}
>$showev('limit((1+x)^(1/x),x,0))
\end{eulerprompt}
\begin{eulerformula}
\[
\lim_{x\rightarrow 0}{\left(x+1\right)^{\frac{1}{x}}}=e
\]
\end{eulerformula}
\begin{eulercomment}
Tunjukkan limit tersebut dengan grafik, seperti contoh-contoh
sebelumnya.\\
Penyelesaian:
\end{eulercomment}
\begin{eulerprompt}
>plot2d("(1+x)^(1/x)",-pi,2pi):
\end{eulerprompt}
\eulerimg{12}{images/EMT4Kalkulus_Amalia Intan Arvitasari_22305144026-054.png}
\begin{eulerprompt}
>$showev('limit((x/(x+k))^x,x,inf))
\end{eulerprompt}
\begin{eulerformula}
\[
\lim_{x\rightarrow \infty }{\left(\frac{x}{x+k}\right)^{x}}=e^ {- k   }
\]
\end{eulerformula}
\begin{eulerprompt}
>$showev('limit((E^x-E^2)/(x-2),x,2))
\end{eulerprompt}
\begin{eulerformula}
\[
\lim_{x\rightarrow 2}{\frac{e^{x}-e^2}{x-2}}=e^2
\]
\end{eulerformula}
\begin{eulercomment}
Tunjukkan limit tersebut dengan grafik, seperti contoh-contoh
sebelumnya.\\
Penyelesaian:
\end{eulercomment}
\begin{eulerprompt}
>plot2d("(E*x-E^2)/(x-2)",-10,10,-1,50):
\end{eulerprompt}
\eulerimg{12}{images/EMT4Kalkulus_Amalia Intan Arvitasari_22305144026-057.png}
\begin{eulerprompt}
>$showev('limit(sin(1/x),x,0))
\end{eulerprompt}
\begin{eulerformula}
\[
\lim_{x\rightarrow 0}{\sin \left(\frac{1}{x}\right)}={\it ind}
\]
\end{eulerformula}
\begin{eulerprompt}
>$showev('limit(sin(1/x),x,inf))
\end{eulerprompt}
\begin{eulerformula}
\[
\lim_{x\rightarrow \infty }{\sin \left(\frac{1}{x}\right)}=0
\]
\end{eulerformula}
\begin{eulerprompt}
>plot2d("sin(1/x)",-0.001,0.001):
\end{eulerprompt}
\eulerimg{12}{images/EMT4Kalkulus_Amalia Intan Arvitasari_22305144026-060.png}
\eulerheading{Latihan}
\begin{eulercomment}
Bukalah buku Kalkulus. Cari dan pilih beberapa (paling sedikit 5
fungsi berbeda tipe/bentuk/jenis) fungsi dari buku tersebut, kemudian
definisikan di EMT pada baris-baris perintah berikut (jika perlu
tambahkan lagi). Untuk setiap fungsi, hitung nilai limit fungsi
tersebut di beberapa nilai dan di tak hingga. Gambar grafik fungsi
tersebut untuk mengkonfirmasi nilai-nilai limit tersebut.

Soal 1:
\end{eulercomment}
\begin{eulerprompt}
>$showev('limit((x^2-x-6)/(x-3),x,3))
\end{eulerprompt}
\begin{eulerformula}
\[
\lim_{x\rightarrow 3}{\frac{x^2-x-6}{x-3}}=5
\]
\end{eulerformula}
\begin{eulerprompt}
>plot2d("(x^2-x-6)/(x-3)",2,4); plot2d(3,5,>points,style="ow",>add):
\end{eulerprompt}
\eulerimg{12}{images/EMT4Kalkulus_Amalia Intan Arvitasari_22305144026-062.png}
\begin{eulercomment}
Soal 2:
\end{eulercomment}
\begin{eulerprompt}
>$showev('limit((2*x^3)/(1+x^3),x, -inf))
\end{eulerprompt}
\begin{eulerformula}
\[
2\,\left(\lim_{x\rightarrow -\infty }{\frac{x^3}{x^3+1}}\right)=2
\]
\end{eulerformula}
\begin{eulerprompt}
>plot2d("(2*x^3)/(1+x^3)",-4,0,1,3); ...
>plot2d("2",style="--",color=blue,>add):
\end{eulerprompt}
\eulerimg{12}{images/EMT4Kalkulus_Amalia Intan Arvitasari_22305144026-064.png}
\begin{eulercomment}
Soal 3:
\end{eulercomment}
\begin{eulerprompt}
>$showev('limit((sin(x))^2/(2*x^2),x,0))
\end{eulerprompt}
\begin{eulerformula}
\[
\frac{\lim_{x\rightarrow 0}{\frac{\sin ^2x}{x^2}}}{2}=\frac{1}{2}
\]
\end{eulerformula}
\begin{eulerprompt}
>plot2d("(sin(x))^2/(2*x^2)",-1,1); plot2d(0,1/2,>points,style="ow",>add):
\end{eulerprompt}
\eulerimg{12}{images/EMT4Kalkulus_Amalia Intan Arvitasari_22305144026-066.png}
\begin{eulercomment}
Soal 4:
\end{eulercomment}
\begin{eulerprompt}
>$showev('limit(log(x/10),x,10))
\end{eulerprompt}
\begin{eulerformula}
\[
\lim_{x\rightarrow 10}{\log \left(\frac{x}{10}\right)}=0
\]
\end{eulerformula}
\begin{eulerprompt}
>plot2d("log(x/10)",-5,5):
\end{eulerprompt}
\eulerimg{12}{images/EMT4Kalkulus_Amalia Intan Arvitasari_22305144026-068.png}
\begin{eulercomment}
Soal 5:
\end{eulercomment}
\begin{eulerprompt}
>$showev('limit((1+2/(3*x))^(5*x),x,inf)) 
\end{eulerprompt}
\begin{eulerformula}
\[
\lim_{x\rightarrow \infty }{\left(\frac{2}{3\,x}+1\right)^{5\,x}}=e  ^{\frac{10}{3}}
\]
\end{eulerformula}
\begin{eulerprompt}
>plot2d("(1+2/(3*x))^(5*x)",-50,0,20,100):
\end{eulerprompt}
\eulerimg{12}{images/EMT4Kalkulus_Amalia Intan Arvitasari_22305144026-070.png}
\begin{eulercomment}
\begin{eulercomment}
\eulerheading{Turunan Fungsi}
\begin{eulercomment}
Definisi turunan:

\end{eulercomment}
\begin{eulerformula}
\[
f'(x) = \lim_{h\to 0} \frac{f(x+h)-f(x)}{h}
\]
\end{eulerformula}
\begin{eulercomment}
Berikut adalah contoh-contoh menentukan turunan fungsi dengan
menggunakan definisi turunan (limit).
\end{eulercomment}
\begin{eulerprompt}
>$showev('limit(((x+h)^2-x^2)/h,h,0)) // turunan x^2
\end{eulerprompt}
\begin{eulerformula}
\[
\lim_{h\rightarrow 0}{\frac{\left(x+h\right)^2-x^2}{h}}=2\,x
\]
\end{eulerformula}
\begin{eulerprompt}
>p &= expand((x+h)^2-x^2)|simplify; $p //pembilang dijabarkan dan disederhanakan
\end{eulerprompt}
\begin{eulerformula}
\[
2\,h\,x+h^2
\]
\end{eulerformula}
\begin{eulerprompt}
>q &=ratsimp(p/h); $q // ekspresi yang akan dihitung limitnya disederhanakan
\end{eulerprompt}
\begin{eulerformula}
\[
2\,x+h
\]
\end{eulerformula}
\begin{eulerprompt}
>$limit(q,h,0) // nilai limit sebagai turunan
\end{eulerprompt}
\begin{eulerformula}
\[
2\,x
\]
\end{eulerformula}
\begin{eulerprompt}
>$showev('limit(((x+h)^n-x^n)/h,h,0)) // turunan x^n
\end{eulerprompt}
\begin{eulerformula}
\[
\lim_{h\rightarrow 0}{\frac{\left(x+h\right)^{n}-x^{n}}{h}}=n\,x^{n  -1}
\]
\end{eulerformula}
\begin{eulercomment}
Mengapa hasilnya seperti itu? Tuliskan atau tunjukkan bahwa hasil
limit tersebut benar, sehingga benar turunan fungsinya benar.  Tulis
penjelasan Anda di komentar ini.

Sebagai petunjuk, ekspansikan (x+h)\textasciicircum{}n dengan menggunakan teorema
binomial.

Bukti:

\end{eulercomment}
\begin{eulerformula}
\[
f'(x) = \lim_{h\to 0} \frac{f(x+h)-f(x)}{h}
\]
\end{eulerformula}
\begin{eulercomment}
Untuk\\
\end{eulercomment}
\begin{eulerformula}
\[
f(x)=x^{n}
\]
\end{eulerformula}
\begin{eulerformula}
\[
\frac{d}{dx}sin(x) = \lim_{h\to 0} \frac{(x+h)^{n}-x^{n}}{h}
\]
\end{eulerformula}
\begin{eulercomment}
Dengan\\
\end{eulercomment}
\begin{eulerformula}
\[
(a+b)^{n}=\sum_{k=0}^n a^{k}b^{n-k}
\]
\end{eulerformula}
\begin{eulercomment}
maka\\
\end{eulercomment}
\begin{eulerformula}
\[
= \lim_{h\to 0} \frac{(x^{n}+\frac{n}{1!}x^{n-1}h+\frac{n(n-1)}{2!}x^{n-2}h^2+\frac{n(n-1)(n-2)}{3!}x^{n-3}h^{3}+...)-x^{n}}{h}
\]
\end{eulerformula}
\begin{eulerformula}
\[
= \lim_{h\to 0} \frac{n.x^{n-1}h+\frac{n(n-1)}{2!}x^{n-2}h^2+\frac{n(n-1)(n-2)}{3!}x^{n-3}h^{3}+...}{h}
\]
\end{eulerformula}
\begin{eulerformula}
\[
= \lim_{h\to 0} n.x^{n-1}+\frac{n(n-1)}{2!}.x^{n-2}h+\frac{n(n-1)(n-2)}{3!}.x^{n-3}h^{2}+...
\]
\end{eulerformula}
\begin{eulerformula}
\[
= n.x^{n-1}+0+0+...+0
\]
\end{eulerformula}
\begin{eulerformula}
\[
= n.x^{n-1}
\]
\end{eulerformula}
\begin{eulercomment}
Jadi, terbukti benar bahwa\\
\end{eulercomment}
\begin{eulerformula}
\[
f'(x^n) = n.x^{n-1}
\]
\end{eulerformula}
\begin{eulerprompt}
>$showev('limit((sin(x+h)-sin(x))/h,h,0)) // turunan sin(x)
\end{eulerprompt}
\begin{eulerformula}
\[
\lim_{h\rightarrow 0}{\frac{\sin \left(x+h\right)-\sin x}{h}}=\cos   x
\]
\end{eulerformula}
\begin{eulercomment}
Mengapa hasilnya seperti itu? Tuliskan atau tunjukkan bahwa hasil
limit tersebut\\
benar, sehingga benar turunan fungsinya benar.  Tulis penjelasan Anda
di komentar ini.

Sebagai petunjuk, ekspansikan sin(x+h) dengan menggunakan rumus jumlah
dua sudut.

Bukti:

\end{eulercomment}
\begin{eulerformula}
\[
f'(x) = \lim_{h\to 0} \frac{sin(x+h)-sin(x)}{h}
\]
\end{eulerformula}
\begin{eulercomment}
\end{eulercomment}
\begin{eulerformula}
\[
sin(a+b)=sin(a)cos(a)+cos(a)sin(b)
\]
\end{eulerformula}
\begin{eulercomment}
\end{eulercomment}
\begin{eulerformula}
\[
= \lim_{h\to 0} \frac{sin(x)cos(h)+cos(x)sin(h)-sin(x)}{h}
\]
\end{eulerformula}
\begin{eulercomment}
\end{eulercomment}
\begin{eulerformula}
\[
= \lim_{h\to 0} sinx.\frac{cos(h)-1}{h}+\lim_{h\to 0} cos(x).\frac{sin(h)}{h}
\]
\end{eulerformula}
\begin{eulerformula}
\[
= sin(x).0+cos(x).1
\]
\end{eulerformula}
\begin{eulercomment}
\end{eulercomment}
\begin{eulerformula}
\[
= cos(x)
\]
\end{eulerformula}
\begin{eulercomment}
Jadi, terbukti benar bahwa

\end{eulercomment}
\begin{eulerformula}
\[
f'(sin(x)) = cos(x)
\]
\end{eulerformula}
\begin{eulerprompt}
>$showev('limit((log(x+h)-log(x))/h,h,0)) // turunan log(x)
\end{eulerprompt}
\begin{eulerformula}
\[
\lim_{h\rightarrow 0}{\frac{\log \left(x+h\right)-\log x}{h}}=  \frac{1}{x}
\]
\end{eulerformula}
\begin{eulercomment}
Mengapa hasilnya seperti itu? Tuliskan atau tunjukkan bahwa hasil
limit tersebut\\
benar, sehingga benar turunan fungsinya benar.  Tulis penjelasan Anda
di komentar ini.

Sebagai petunjuk, gunakan sifat-sifat logaritma dan hasil limit pada
bagian sebelumnya di atas.

Bukti:

\end{eulercomment}
\begin{eulerformula}
\[
f'(x) = \lim_{h\to 0} \frac{log(x+h)-log x}{h}
\]
\end{eulerformula}
\begin{eulercomment}
\end{eulercomment}
\begin{eulerformula}
\[
=\lim_{h\to 0} \frac{\frac{d}{dh}(log(x+h)-log x)}{\frac{d}{dh}(h)}
\]
\end{eulerformula}
\begin{eulerformula}
\[
=\lim_{h\to 0} \frac{\frac{1}{x+h}}{1}
\]
\end{eulerformula}
\begin{eulerformula}
\[
=\lim_{h\to 0} \frac{1}{x+h}
\]
\end{eulerformula}
\begin{eulerformula}
\[
=\frac{1}{x}
\]
\end{eulerformula}
\begin{eulercomment}
Jadi, terbukti benar bahwa\\
\end{eulercomment}
\begin{eulerformula}
\[
f'(x) = \lim_{h\to 0} \frac{log(x+h)-log x}{h} = \frac{1}{x}
\]
\end{eulerformula}
\begin{eulerprompt}
>$showev('limit((1/(x+h)-1/x)/h,h,0)) // turunan 1/x
\end{eulerprompt}
\begin{eulerformula}
\[
\lim_{h\rightarrow 0}{\frac{\frac{1}{x+h}-\frac{1}{x}}{h}}=-\frac{1  }{x^2}
\]
\end{eulerformula}
\begin{eulerprompt}
>$showev('limit((E^(x+h)-E^x)/h,h,0)) // turunan f(x)=e^x
\end{eulerprompt}
\begin{euleroutput}
  Answering "Is x an integer?" with "integer"
  Answering "Is x an integer?" with "integer"
  Answering "Is x an integer?" with "integer"
  Answering "Is x an integer?" with "integer"
  Answering "Is x an integer?" with "integer"
  Maxima is asking
  Acceptable answers are: yes, y, Y, no, n, N, unknown, uk
  Is x an integer?
  
  Use assume!
  Error in:
   $showev('limit((E^(x+h)-E^x)/h,h,0)) // turunan f(x)=e^x ...
                                       ^
\end{euleroutput}
\begin{eulercomment}
Maxima bermasalah dengan limit:

\end{eulercomment}
\begin{eulerformula}
\[
\lim_{h\to 0}\frac{e^{x+h}-e^x}{h}.
\]
\end{eulerformula}
\begin{eulercomment}
Oleh karena itu diperlukan trik khusus agar hasilnya benar.
\end{eulercomment}
\begin{eulerprompt}
>$showev('limit((E^h-1)/h,h,0))
\end{eulerprompt}
\begin{eulerformula}
\[
\lim_{h\rightarrow 0}{\frac{e^{h}-1}{h}}=1
\]
\end{eulerformula}
\begin{eulerprompt}
>$showev('factor(E^(x+h)-E^x))
\end{eulerprompt}
\begin{eulerformula}
\[
{\it factor}\left(e^{x+h}-e^{x}\right)=\left(e^{h}-1\right)\,e^{x}
\]
\end{eulerformula}
\begin{eulerprompt}
>$showev('limit(factor((E^(x+h)-E^x)/h),h,0)) // turunan f(x)=e^x
\end{eulerprompt}
\begin{eulerformula}
\[
\left(\lim_{h\rightarrow 0}{\frac{e^{h}-1}{h}}\right)\,e^{x}=e^{x}
\]
\end{eulerformula}
\begin{eulerprompt}
>function f(x) &= x^x
\end{eulerprompt}
\begin{euleroutput}
  
                                     x
                                    x
  
\end{euleroutput}
\begin{eulerprompt}
>$showev('limit(f(x),x,0))
\end{eulerprompt}
\begin{eulerformula}
\[
\lim_{x\rightarrow 0}{x^{x}}=1
\]
\end{eulerformula}
\begin{eulercomment}
Silakan Anda gambar kurva

\end{eulercomment}
\begin{eulerformula}
\[
y=x^x.
\]
\end{eulerformula}
\begin{eulercomment}
Penyelesaian:
\end{eulercomment}
\begin{eulerprompt}
>plot2d("x^x"):
\end{eulerprompt}
\eulerimg{1}{images/EMT4Kalkulus_Amalia Intan Arvitasari_22305144026-109-large.png}
\begin{eulerprompt}
>$showev('limit((f(x+h)-f(x))/h,h,0)) // turunan f(x)=x^x
\end{eulerprompt}
\begin{eulerformula}
\[
\lim_{h\rightarrow 0}{\frac{\left(x+h\right)^{x+h}-x^{x}}{h}}=  {\it infinity}
\]
\end{eulerformula}
\begin{eulercomment}
Di sini Maxima juga bermasalah terkait limit:

\end{eulercomment}
\begin{eulerformula}
\[
\lim_{h\to 0} \frac{(x+h)^{x+h}-x^x}{h}.
\]
\end{eulerformula}
\begin{eulercomment}
Dalam hal ini diperlukan asumsi nilai x.
\end{eulercomment}
\begin{eulerprompt}
>&assume(x>0); $showev('limit((f(x+h)-f(x))/h,h,0)) // turunan f(x)=x^x
\end{eulerprompt}
\begin{eulerformula}
\[
\lim_{h\rightarrow 0}{\frac{\left(x+h\right)^{x+h}-x^{x}}{h}}=x^{x}  \,\left(\log x+1\right)
\]
\end{eulerformula}
\begin{eulercomment}
Mengapa hasilnya seperti itu? Tuliskan atau tunjukkan bahwa hasil limit tersebut benar, sehingga benar turunan fungsinya benar.
Tulis penjelasan Anda di komentar ini.
\end{eulercomment}
\begin{eulerprompt}
>&forget(x>0) // jangan lupa, lupakan asumsi untuk kembali ke semula
\end{eulerprompt}
\begin{euleroutput}
  
                                 [x > 0]
  
\end{euleroutput}
\begin{eulerprompt}
>&forget(x<0)
\end{eulerprompt}
\begin{euleroutput}
  
                                 [x < 0]
  
\end{euleroutput}
\begin{eulerprompt}
>&facts()
\end{eulerprompt}
\begin{euleroutput}
  
                                    []
  
\end{euleroutput}
\begin{eulerprompt}
>$showev('limit((asin(x+h)-asin(x))/h,h,0)) // turunan arcsin(x)
\end{eulerprompt}
\begin{eulerformula}
\[
\lim_{h\rightarrow 0}{\frac{\arcsin \left(x+h\right)-\arcsin x}{h}}=  \frac{1}{\sqrt{1-x^2}}
\]
\end{eulerformula}
\begin{eulercomment}
Mengapa hasilnya seperti itu? Tuliskan atau tunjukkan bahwa hasil limit tersebut benar, sehingga
benar turunan fungsinya benar. Tulis penjelasan Anda di komentar ini.
\end{eulercomment}
\begin{eulerprompt}
>$showev('limit((tan(x+h)-tan(x))/h,h,0)) // turunan tan(x)
\end{eulerprompt}
\begin{eulerformula}
\[
\lim_{h\rightarrow 0}{\frac{\tan \left(x+h\right)-\tan x}{h}}=  \frac{1}{\cos ^2x}
\]
\end{eulerformula}
\begin{eulercomment}
Mengapa hasilnya seperti itu? Tuliskan atau tunjukkan bahwa hasil limit tersebut benar, sehingga
benar turunan fungsinya benar. Tulis penjelasan Anda di komentar ini.
\end{eulercomment}
\begin{eulerprompt}
>function f(x) &= sinh(x) // definisikan f(x)=sinh(x)
\end{eulerprompt}
\begin{euleroutput}
  
                                 sinh(x)
  
\end{euleroutput}
\begin{eulerprompt}
>function df(x) &= limit((f(x+h)-f(x))/h,h,0); $df(x) // df(x) = f'(x)
\end{eulerprompt}
\begin{eulerformula}
\[
\frac{e^ {- x }\,\left(e^{2\,x}+1\right)}{2}
\]
\end{eulerformula}
\begin{eulercomment}
Hasilnya adalah cosh(x), karena

\end{eulercomment}
\begin{eulerformula}
\[
\frac{e^x+e^{-x}}{2}=\cosh(x).
\]
\end{eulerformula}
\begin{eulerprompt}
>plot2d(["f(x)","df(x)"],-pi,pi,color=[blue,red]):
\end{eulerprompt}
\eulerimg{0}{images/EMT4Kalkulus_Amalia Intan Arvitasari_22305144026-117-large.png}
\begin{eulerprompt}
>function f(x) &= sin(3*x^5+7)^2
\end{eulerprompt}
\begin{euleroutput}
  
                                 2    5
                              sin (3 x  + 7)
  
\end{euleroutput}
\begin{eulerprompt}
>diff(f,3), diffc(f,3)
\end{eulerprompt}
\begin{euleroutput}
  1198.32948904
  1198.72863721
\end{euleroutput}
\begin{eulercomment}
Apakah perbedaan diff dan diffc?
\end{eulercomment}
\begin{eulerprompt}
>$showev('diff(f(x),x))
\end{eulerprompt}
\begin{eulerformula}
\[
\frac{d}{d\,x}\,\sin ^2\left(3\,x^5+7\right)=30\,x^4\,\cos \left(3  \,x^5+7\right)\,\sin \left(3\,x^5+7\right)
\]
\end{eulerformula}
\begin{eulerprompt}
>$% with x=3
\end{eulerprompt}
\begin{eulerformula}
\[
{\it \%at}\left(\frac{d}{d\,x}\,\sin ^2\left(3\,x^5+7\right) , x=3  \right)=2430\,\cos 736\,\sin 736
\]
\end{eulerformula}
\begin{eulerprompt}
>$float(%)
\end{eulerprompt}
\begin{eulerformula}
\[
{\it \%at}\left(\frac{d^{1.0}}{d\,x^{1.0}}\,\sin ^2\left(3.0\,x^5+  7.0\right) , x=3.0\right)=1198.728637211748
\]
\end{eulerformula}
\begin{eulerprompt}
>plot2d(f,0,3.1):
\end{eulerprompt}
\eulerimg{0}{images/EMT4Kalkulus_Amalia Intan Arvitasari_22305144026-121-large.png}
\begin{eulerprompt}
>function f(x) &=5*cos(2*x)-2*x*sin(2*x) // mendifinisikan fungsi f
\end{eulerprompt}
\begin{euleroutput}
  
                        5 cos(2 x) - 2 x sin(2 x)
  
\end{euleroutput}
\begin{eulerprompt}
>function df(x) &=diff(f(x),x) // fd(x) = f'(x)
\end{eulerprompt}
\begin{euleroutput}
  
                       - 12 sin(2 x) - 4 x cos(2 x)
  
\end{euleroutput}
\begin{eulerprompt}
>$'f(1)=f(1), $float(f(1)), $'f(2)=f(2), $float(f(2)) // nilai f(1) dan f(2)
\end{eulerprompt}
\begin{eulerformula}
\[
-0.2410081230863468
\]
\end{eulerformula}
\eulerimg{0}{images/EMT4Kalkulus_Amalia Intan Arvitasari_22305144026-123-large.png}
\eulerimg{0}{images/EMT4Kalkulus_Amalia Intan Arvitasari_22305144026-124-large.png}
\eulerimg{12}{images/EMT4Kalkulus_Amalia Intan Arvitasari_22305144026-125.png}
\begin{eulerprompt}
>xp=solve("df(x)",1,2,0) // solusi f'(x)=0 pada interval [1, 2]
\end{eulerprompt}
\begin{euleroutput}
  1.35822987384
\end{euleroutput}
\begin{eulerprompt}
>df(xp), f(xp) // cek bahwa f'(xp)=0 dan nilai ekstrim di titik tersebut
\end{eulerprompt}
\begin{euleroutput}
  0
  -5.67530133759
\end{euleroutput}
\begin{eulerprompt}
>plot2d(["f(x)","df(x)"],0,2*pi,color=[blue,red]): //grafik fungsi dan turunannya
\end{eulerprompt}
\eulerimg{0}{images/EMT4Kalkulus_Amalia Intan Arvitasari_22305144026-126-large.png}
\begin{eulercomment}
Perhatikan titik-titik "puncak" grafik y=f(x) dan nilai turunan pada saat grafik fungsinya mencapai titik "puncak" tersebut.
\end{eulercomment}
\eulerheading{Latihan}
\begin{eulercomment}
Bukalah buku Kalkulus. Cari dan pilih beberapa (paling sedikit 5
fungsi berbeda tipe/bentuk/jenis) fungsi dari buku tersebut, kemudian
definisikan di EMT pada baris-baris perintah berikut (jika perlu
tambahkan lagi). Untuk setiap fungsi, tentukan turunannya dengan
menggunakan definisi turunan (limit), menggunakan perintah diff, dan
secara manual (langkah demi langkah yang dihitung dengan Maxima)
seperti contoh-contoh di atas. Gambar grafik fungsi asli dan fungsi
turunannya pada sumbu koordinat yang sama.

1. Tentukan nilai turunan berikut dan sketsakan grafiknya.\\
\end{eulercomment}
\begin{eulerformula}
\[
f(x)=2x^2+6
\]
\end{eulerformula}
\begin{eulercomment}
Penyelesaian:
\end{eulercomment}
\begin{eulerprompt}
>function f(x) &= 2*x^2+6; $f(x)
\end{eulerprompt}
\begin{eulerformula}
\[
2\,x^2+6
\]
\end{eulerformula}
\begin{eulerprompt}
>function df(x) &= limit((f(x+h)-f(x))/h,h,0); &df(x)//df(x)=f'(x)
\end{eulerprompt}
\begin{euleroutput}
  
                                   4 x
  
\end{euleroutput}
\begin{eulerprompt}
>plot2d(["f(x)","df(x)"],-pi,pi,color=[red,green]):
\end{eulerprompt}
\eulerimg{0}{images/EMT4Kalkulus_Amalia Intan Arvitasari_22305144026-129-large.png}
\begin{eulercomment}
2. Tentukan turunan dari fungsi berikut dan sketsakan grafiknya\\
\end{eulercomment}
\begin{eulerformula}
\[
f(x)=\frac{4x-1}{x-6}
\]
\end{eulerformula}
\begin{eulercomment}
Penyelesaian:
\end{eulercomment}
\begin{eulerprompt}
>function f(x) &= (4*x-1)/(x-6); $f(x)
\end{eulerprompt}
\begin{eulerformula}
\[
\frac{4\,x-1}{x-6}
\]
\end{eulerformula}
\begin{eulerprompt}
>function df(x) &= limit((f(x+h)-f(x))/h,h,0); $df(x) // df(x) = f'(x)
\end{eulerprompt}
\begin{eulerformula}
\[
-\frac{23}{x^2-12\,x+36}
\]
\end{eulerformula}
\begin{eulerprompt}
>plot2d(["f(x)","df(x)"],-10,10,color=[blue,red]):
\end{eulerprompt}
\eulerimg{1}{images/EMT4Kalkulus_Amalia Intan Arvitasari_22305144026-133-large.png}
\begin{eulercomment}
3. Tentukan turunan dari fungsi berikut dan sketsakan grafiknya\\
\end{eulercomment}
\begin{eulerformula}
\[
f(x)= \frac{3}{\sqrt{x-4}}
\]
\end{eulerformula}
\begin{eulercomment}
Penyelesaian:
\end{eulercomment}
\begin{eulerprompt}
>function f(x) &= 3/sqrt(x-4); $f(x)
\end{eulerprompt}
\begin{eulerformula}
\[
\frac{3}{\sqrt{x-4}}
\]
\end{eulerformula}
\begin{eulerprompt}
>function df(x) &= limit((f(x+h)-f(x))/h,h,0); $df(x) // df(x) = f'(x)function f(x) &= 3
\end{eulerprompt}
\begin{eulerformula}
\[
-\frac{3}{2\,\left(x-4\right)^{\frac{3}{2}}}
\]
\end{eulerformula}
\begin{eulerprompt}
>plot2d(["f(x)","df(x)"],-10,10,color=[yellow,red]):
\end{eulerprompt}
\eulerimg{0}{images/EMT4Kalkulus_Amalia Intan Arvitasari_22305144026-137-large.png}
\begin{eulercomment}
4. Tentukan turunan fungsi berikut dan sketsakan grafiknya\\
\end{eulercomment}
\begin{eulerformula}
\[
f(x) = 4sin(x)+2cos(x)
\]
\end{eulerformula}
\begin{eulercomment}
Penyelesaian:
\end{eulercomment}
\begin{eulerprompt}
>function f(x) &= (4*sin(x)+2*cos(x)); $f(x)
\end{eulerprompt}
\begin{eulerformula}
\[
4\,\sin x+2\,\cos x
\]
\end{eulerformula}
\begin{eulerprompt}
>function df(x) &= limit((f(x+h)-f(x))/h,h,0); &df(x)
\end{eulerprompt}
\begin{euleroutput}
  
                         - 2 (sin(x) - 2 cos(x))
  
\end{euleroutput}
\begin{eulerprompt}
>plot2d(["f(x)","df(x)"],-pi,pi,color=[blue,yellow]):
\end{eulerprompt}
\eulerimg{1}{images/EMT4Kalkulus_Amalia Intan Arvitasari_22305144026-140-large.png}
\begin{eulercomment}
5. Tentukan turunan dan grafik fungsi berikut dan sketsakan grafiknya.\\
\end{eulercomment}
\begin{eulerformula}
\[
f(x) = \frac{sin(x)+cos(x)}{cos(x)}
\]
\end{eulerformula}
\begin{eulercomment}
Penyelesaian:
\end{eulercomment}
\begin{eulerprompt}
>function f(x) &= (sin(x)+cos(x))/(cos(x)); $f(x)
\end{eulerprompt}
\begin{eulerformula}
\[
\frac{\sin x+\cos x}{\cos x}
\]
\end{eulerformula}
\begin{eulerprompt}
>function df(x) &= limit((f(x+h)-f(x))/h,h,0); $df(x) // df(x) = f'(x)
\end{eulerprompt}
\begin{eulerformula}
\[
\frac{\sin ^2x+\cos ^2x}{\cos ^2x}
\]
\end{eulerformula}
\begin{eulerprompt}
>plot2d(["f(x)","df(x)"],-pi,pi,color=[blue,yellow]):
\end{eulerprompt}
\eulerimg{1}{images/EMT4Kalkulus_Amalia Intan Arvitasari_22305144026-144-large.png}
\begin{euleroutput}
  
\end{euleroutput}
\eulerheading{Integral}
\begin{eulercomment}
EMT dapat digunakan untuk menghitung integral, baik integral tak tentu
maupun integral tentu. Untuk integral tak tentu (simbolik) sudah tentu
EMT menggunakan Maxima, sedangkan untuk perhitungan integral tentu EMT
sudah menyediakan beberapa fungsi yang mengimplementasikan algoritma
kuadratur (perhitungan integral tentu menggunakan metode numerik).

Pada notebook ini akan ditunjukkan perhitungan integral tentu dengan
menggunakan Teorema Dasar Kalkulus:

\end{eulercomment}
\begin{eulerformula}
\[
\int_a^b f(x)\ dx = F(b)-F(a), \quad \text{ dengan  } F'(x) = f(x).
\]
\end{eulerformula}
\begin{eulercomment}
Fungsi untuk menentukan integral adalah integrate. Fungsi ini dapat
digunakan untuk menentukan, baik integral tentu maupun tak tentu (jika
fungsinya memiliki antiderivatif). Untuk perhitungan integral tentu
fungsi integrate menggunakan metode numerik (kecuali fungsinya tidak
integrabel, kita tidak akan menggunakan metode ini).
\end{eulercomment}
\begin{eulerprompt}
>$showev('integrate(x^n,x))
\end{eulerprompt}
\begin{euleroutput}
  Answering "Is n equal to -1?" with "no"
\end{euleroutput}
\begin{eulerformula}
\[
\int {x^{n}}{\;dx}=\frac{x^{n+1}}{n+1}
\]
\end{eulerformula}
\begin{eulerprompt}
>$showev('integrate(1/(1+x),x))
\end{eulerprompt}
\begin{eulerformula}
\[
\int {\frac{1}{x+1}}{\;dx}=\log \left(x+1\right)
\]
\end{eulerformula}
\begin{eulerprompt}
>$showev('integrate(1/(1+x^2),x))
\end{eulerprompt}
\begin{eulerformula}
\[
\int {\frac{1}{x^2+1}}{\;dx}=\arctan x
\]
\end{eulerformula}
\begin{eulerprompt}
>$showev('integrate(1/sqrt(1-x^2),x))
\end{eulerprompt}
\begin{eulerformula}
\[
\int {\frac{1}{\sqrt{1-x^2}}}{\;dx}=\arcsin x
\]
\end{eulerformula}
\begin{eulerprompt}
>$showev('integrate(sin(x),x,0,pi))
\end{eulerprompt}
\begin{eulerformula}
\[
\int_{0}^{\pi}{\sin x\;dx}=2
\]
\end{eulerformula}
\begin{eulerprompt}
>plot2d("sin(x)",0,2*pi):
\end{eulerprompt}
\eulerimg{1}{images/EMT4Kalkulus_Amalia Intan Arvitasari_22305144026-151-large.png}
\begin{eulerprompt}
>$showev('integrate(sin(x),x,a,b))
\end{eulerprompt}
\begin{eulerformula}
\[
\int_{a}^{b}{\sin x\;dx}=\cos a-\cos b
\]
\end{eulerformula}
\begin{eulerprompt}
>$showev('integrate(x^n,x,a,b))
\end{eulerprompt}
\begin{euleroutput}
  Answering "Is n positive, negative or zero?" with "positive"
\end{euleroutput}
\begin{eulerformula}
\[
\int_{a}^{b}{x^{n}\;dx}=\frac{b^{n+1}}{n+1}-\frac{a^{n+1}}{n+1}
\]
\end{eulerformula}
\begin{eulerprompt}
>$showev('integrate(x^2*sqrt(2*x+1),x))
\end{eulerprompt}
\begin{eulerformula}
\[
\int {x^2\,\sqrt{2\,x+1}}{\;dx}=\frac{\left(2\,x+1\right)^{\frac{7  }{2}}}{28}-\frac{\left(2\,x+1\right)^{\frac{5}{2}}}{10}+\frac{\left(  2\,x+1\right)^{\frac{3}{2}}}{12}
\]
\end{eulerformula}
\begin{eulerprompt}
>$showev('integrate(x^2*sqrt(2*x+1),x,0,2))
\end{eulerprompt}
\begin{eulerformula}
\[
\int_{0}^{2}{x^2\,\sqrt{2\,x+1}\;dx}=\frac{2\,5^{\frac{5}{2}}}{21}-  \frac{2}{105}
\]
\end{eulerformula}
\begin{eulerprompt}
>$ratsimp(%)
\end{eulerprompt}
\begin{eulerformula}
\[
\int_{0}^{2}{x^2\,\sqrt{2\,x+1}\;dx}=\frac{2\,5^{\frac{7}{2}}-2}{  105}
\]
\end{eulerformula}
\begin{eulerprompt}
>$showev('integrate((sin(sqrt(x)+a)*E^sqrt(x))/sqrt(x),x,0,pi^2))
\end{eulerprompt}
\begin{eulerformula}
\[
\int_{0}^{\pi^2}{\frac{\sin \left(\sqrt{x}+a\right)\,e^{\sqrt{x}}}{  \sqrt{x}}\;dx}=\left(-e^{\pi}-1\right)\,\sin a+\left(e^{\pi}+1  \right)\,\cos a
\]
\end{eulerformula}
\begin{eulerprompt}
>$factor(%)
\end{eulerprompt}
\begin{eulerformula}
\[
\int_{0}^{\pi^2}{\frac{\sin \left(\sqrt{x}+a\right)\,e^{\sqrt{x}}}{  \sqrt{x}}\;dx}=\left(-e^{\pi}-1\right)\,\left(\sin a-\cos a\right)
\]
\end{eulerformula}
\begin{eulerprompt}
>function map f(x) &= E^(-x^2)
\end{eulerprompt}
\begin{euleroutput}
  
                                      2
                                   - x
                                  E
  
\end{euleroutput}
\begin{eulerprompt}
>$showev('integrate(f(x),x))
\end{eulerprompt}
\begin{eulerformula}
\[
\int {e^ {- x^2 }}{\;dx}=\frac{\sqrt{\pi}\,\mathrm{erf}\left(x  \right)}{2}
\]
\end{eulerformula}
\begin{eulercomment}
Fungsi f tidak memiliki antiturunan, integralnya masih memuat integral
lain.

\end{eulercomment}
\begin{eulerformula}
\[
erf(x) = \int \frac{e^{-x^2}}{\sqrt{\pi}} \ dx.
\]
\end{eulerformula}
\begin{eulercomment}
Kita tidak dapat menggunakan teorema Dasar kalkulus untuk menghitung
integral tentu fungsi tersebut jika semua batasnya berhingga. Dalam
hal ini dapat digunakan metode numerik (rumus kuadratur).

Misalkan kita akan menghitung:

\end{eulercomment}
\begin{eulerformula}
\[
\int_{0}^{\pi}{e^ {- x^2 }\;dx}
\]
\end{eulerformula}
\begin{eulerprompt}
>x=0:0.1:pi-0.1; plot2d(x,f(x+0.1),>bar); plot2d("f(x)",0,pi,>add):
\end{eulerprompt}
\eulerimg{1}{images/EMT4Kalkulus_Amalia Intan Arvitasari_22305144026-162-large.png}
\begin{eulercomment}
Integral tentu

\end{eulercomment}
\begin{eulerformula}
\[
\int_{0}^{\pi}{f\left(x\right)\;dx}
\]
\end{eulerformula}
\begin{eulercomment}
dapat dihampiri dengan jumlah luas persegi-persegi panjang di bawah
kurva y=f(x) tersebut. Langkah-langkahnya adalah sebagai berikut.
\end{eulercomment}
\begin{eulerprompt}
>t &= makelist(a,a,0,pi-0.1,0.1); // t sebagai list untuk menyimpan nilai-nilai x
>fx &= makelist(f(t[i]+0.1),i,1,length(t)); // simpan nilai-nilai f(x)
>// jangan menggunakan x sebagai list, kecuali Anda pakar Maxima!
\end{eulerprompt}
\begin{eulercomment}
Hasilnya adalah:

\end{eulercomment}
\begin{eulerformula}
\[
\int_{0}^{\pi}{x^2+50\;dx}=48.64225458159455
\]
\end{eulerformula}
\begin{eulercomment}
Jumlah tersebut diperoleh dari hasil kali lebar sub-subinterval (=0.1)
dan jumlah nilai-nilai f(x) untuk x = 0.1, 0.2, 0.3, ..., 3.2.
\end{eulercomment}
\begin{eulerprompt}
>0.1*sum(f(x+0.1)) // cek langsung dengan perhitungan numerik EMT
\end{eulerprompt}
\begin{euleroutput}
  0.836219610253
\end{euleroutput}
\begin{eulercomment}
Untuk mendapatkan nilai integral tentu yang mendekati nilai sebenarnya, lebar
sub-intervalnya dapat diperkecil lagi, sehingga daerah di bawah kurva tertutup
semuanya, misalnya dapat digunakan lebar subinterval 0.001. (Silakan dicoba!)

Meskipun Maxima tidak dapat menghitung integral tentu fungsi tersebut untuk
batas-batas yang berhingga, namun integral tersebut dapat dihitung secara eksak jika
batas-batasnya tak hingga. Ini adalah salah satu keajaiban di dalam matematika, yang
terbatas tidak dapat dihitung secara eksak, namun yang tak hingga malah dapat
dihitung secara eksak.
\end{eulercomment}
\begin{eulerprompt}
>$showev('integrate(f(x),x,0,inf))
\end{eulerprompt}
\begin{eulerformula}
\[
\int_{0}^{\infty }{e^ {- x^2 }\;dx}=\frac{\sqrt{\pi}}{2}
\]
\end{eulerformula}
\begin{eulercomment}
Tunjukkan kebenaran hasil di atas!

Berikut adalah contoh lain fungsi yang tidak memiliki antiderivatif, sehingga integral tentunya hanya
dapat dihitung dengan metode numerik.
\end{eulercomment}
\begin{eulerprompt}
>function f(x) &= x^x
\end{eulerprompt}
\begin{euleroutput}
  
                                     x
                                    x
  
\end{euleroutput}
\begin{eulerprompt}
>$showev('integrate(f(x),x,0,1))
\end{eulerprompt}
\begin{eulerformula}
\[
\int_{0}^{1}{x^{x}\;dx}=\int_{0}^{1}{x^{x}\;dx}
\]
\end{eulerformula}
\begin{eulerprompt}
>x=0:0.1:1-0.01; plot2d(x,f(x+0.01),>bar); plot2d("f(x)",0,1,>add):
\end{eulerprompt}
\eulerimg{1}{images/EMT4Kalkulus_Amalia Intan Arvitasari_22305144026-167-large.png}
\begin{eulercomment}
Maxima gagal menghitung integral tentu tersebut secara langsung menggunakan perintah
integrate. Berikut kita lakukan seperti contoh sebelumnya untuk mendapat hasil atau
pendekatan nilai integral tentu tersebut.
\end{eulercomment}
\begin{eulerprompt}
>t &= makelist(a,a,0,1-0.01,0.01);
>fx &= makelist(f(t[i]+0.01),i,1,length(t));
\end{eulerprompt}
\begin{eulercomment}
maxima: 'integrate(f(x),x,0,1) = 0.01*sum(fx[i],i,1,length(fx))

Apakah hasil tersebut cukup baik? perhatikan gambarnya.
\end{eulercomment}
\begin{eulerprompt}
>function f(x) &= sin(3*x^5+7)^2
\end{eulerprompt}
\begin{euleroutput}
  
                                 2    5
                              sin (3 x  + 7)
  
\end{euleroutput}
\begin{eulerprompt}
>integrate(f,0,1)
\end{eulerprompt}
\begin{euleroutput}
  0.542581176074
\end{euleroutput}
\begin{eulerprompt}
>&showev('integrate(f(x),x,0,1))
\end{eulerprompt}
\begin{euleroutput}
  
           1                           1              pi
          /                      gamma(-) sin(14) sin(--)
          [     2    5                 5              10
          I  sin (3 x  + 7) dx = ------------------------
          ]                                  1/5
          /                              10 6
           0
         4/5                  1          4/5                  1
   - (((6    gamma_incomplete(-, 6 I) + 6    gamma_incomplete(-, - 6 I))
                              5                               5
               4/5                    1
   sin(14) + (6    I gamma_incomplete(-, 6 I)
                                      5
      4/5                    1                       pi
   - 6    I gamma_incomplete(-, - 6 I)) cos(14)) sin(--) - 60)/120
                             5                       10
  
\end{euleroutput}
\begin{eulerprompt}
>&float(%)
\end{eulerprompt}
\begin{euleroutput}
  
           1.0
          /
          [       2      5
          I    sin (3.0 x  + 7.0) dx = 
          ]
          /
           0.0
  0.09820784258795788 - 0.008333333333333333
   (0.3090169943749474 (0.1367372182078336
   (4.192962712629476 I gamma__incomplete(0.2, 6.0 I)
   - 4.192962712629476 I gamma__incomplete(0.2, - 6.0 I))
   + 0.9906073556948704 (4.192962712629476 gamma__incomplete(0.2, 6.0 I)
   + 4.192962712629476 gamma__incomplete(0.2, - 6.0 I))) - 60.0)
  
\end{euleroutput}
\begin{eulerprompt}
>$showev('integrate(x*exp(-x),x,0,1)) // Integral tentu (eksak)
\end{eulerprompt}
\begin{eulerformula}
\[
\int_{0}^{1}{x\,e^ {- x }\;dx}=1-2\,e^ {- 1 }
\]
\end{eulerformula}
\eulerheading{Aplikasi Integral Tentu}
\begin{eulerprompt}
>plot2d("x^3-x",-0.1,1.1); plot2d("-x^2",>add);  ...
>b=solve("x^3-x+x^2",0.5); x=linspace(0,b,200); xi=flipx(x); ...
>plot2d(x|xi,x^3-x|-xi^2,>filled,style="|",fillcolor=1,>add): // Plot daerah antara 2 kurva
\end{eulerprompt}
\eulerimg{19}{images/EMT4Kalkulus_Amalia Intan Arvitasari_22305144026-169.png}
\begin{eulerprompt}
>a=solve("x^3-x+x^2",0), b=solve("x^3-x+x^2",1) // absis titik-titik potong kedua kurva
\end{eulerprompt}
\begin{euleroutput}
  0
  0.61803398875
\end{euleroutput}
\begin{eulerprompt}
>integrate("(-x^2)-(x^3-x)",a,b) // luas daerah yang diarsir
\end{eulerprompt}
\begin{euleroutput}
  0.0758191713542
\end{euleroutput}
\begin{eulercomment}
Hasil tersebut akan kita bandingkan dengan perhitungan secara analitik.
\end{eulercomment}
\begin{eulerprompt}
>a &= solve((-x^2)-(x^3-x),x); $a // menentukan absis titik potong kedua kurva secara eksak
\end{eulerprompt}
\begin{eulerformula}
\[
\left[ x=\frac{-\sqrt{5}-1}{2} , x=\frac{\sqrt{5}-1}{2} , x=0   \right] 
\]
\end{eulerformula}
\begin{eulerprompt}
>$showev('integrate(-x^2-x^3+x,x,0,(sqrt(5)-1)/2)) // Nilai integral secara eksak
\end{eulerprompt}
\begin{eulerformula}
\[
\int_{0}^{\frac{\sqrt{5}-1}{2}}{-x^3-x^2+x\;dx}=\frac{13-5^{\frac{3  }{2}}}{24}
\]
\end{eulerformula}
\begin{eulerprompt}
>$float(%)
\end{eulerprompt}
\begin{eulerformula}
\[
\int_{0.0}^{0.6180339887498949}{-1.0\,x^3-1.0\,x^2+x\;dx}=  0.07581917135421037
\]
\end{eulerformula}
\eulersubheading{Panjang Kurva}
\begin{eulercomment}
Hitunglah panjang kurva berikut ini dan luas daerah di dalam kurva
tersebut.

\end{eulercomment}
\begin{eulerformula}
\[
\gamma(t) = (r(t) \cos(t), r(t) \sin(t))
\]
\end{eulerformula}
\begin{eulercomment}
dengan

\end{eulercomment}
\begin{eulerformula}
\[
r(t) = 1 + \dfrac{\sin(3t)}{2},\quad 0\le t\le 2\pi.
\]
\end{eulerformula}
\begin{eulerprompt}
>t=linspace(0,2pi,1000); r=1+sin(3*t)/2; x=r*cos(t); y=r*sin(t); ...
>plot2d(x,y,>filled,fillcolor=red,style="/",r=1.5): // Kita gambar kurvanya terlebih dahulu
\end{eulerprompt}
\eulerimg{19}{images/EMT4Kalkulus_Amalia Intan Arvitasari_22305144026-175.png}
\begin{eulerprompt}
>function r(t) &= 1+sin(3*t)/2; $'r(t)=r(t)
\end{eulerprompt}
\begin{eulerformula}
\[
r\left(t\right)=\frac{\sin \left(3\,t\right)}{2}+1
\]
\end{eulerformula}
\begin{eulerprompt}
>function fx(t) &= r(t)*cos(t); $'fx(t)=fx(t)
\end{eulerprompt}
\begin{eulerformula}
\[
{\it fx}\left(t\right)=\cos t\,\left(\frac{\sin \left(3\,t\right)}{  2}+1\right)
\]
\end{eulerformula}
\begin{eulerprompt}
>function fy(t) &= r(t)*sin(t); $'fy(t)=fy(t)
\end{eulerprompt}
\begin{eulerformula}
\[
{\it fy}\left(t\right)=\sin t\,\left(\frac{\sin \left(3\,t\right)}{  2}+1\right)
\]
\end{eulerformula}
\begin{eulerprompt}
>function ds(t) &= trigreduce(radcan(sqrt(diff(fx(t),t)^2+diff(fy(t),t)^2))); $'ds(t)=ds(t)
\end{eulerprompt}
\begin{eulerformula}
\[
{\it ds}\left(t\right)=\frac{\sqrt{4\,\cos \left(6\,t\right)+4\,  \sin \left(3\,t\right)+9}}{2}
\]
\end{eulerformula}
\begin{eulerprompt}
>$integrate(ds(x),x,0,2*pi) //panjang (keliling) kurva
\end{eulerprompt}
\begin{eulerformula}
\[
\frac{\int_{0}^{2\,\pi}{\sqrt{4\,\cos \left(6\,x\right)+4\,\sin   \left(3\,x\right)+9}\;dx}}{2}
\]
\end{eulerformula}
\begin{eulercomment}
Maxima gagal melakukan perhitungan eksak integral tersebut.

Berikut kita hitung integralnya secara umerik dengan perintah EMT.
\end{eulercomment}
\begin{eulerprompt}
>integrate("ds(x)",0,2*pi)
\end{eulerprompt}
\begin{euleroutput}
  9.0749467823
\end{euleroutput}
\begin{eulercomment}
Spiral Logaritmik

\end{eulercomment}
\begin{eulerformula}
\[
x=e^{ax}\cos x,\ y=e^{ax}\sin x.
\]
\end{eulerformula}
\begin{eulerprompt}
>a=0.1; plot2d("exp(a*x)*cos(x)","exp(a*x)*sin(x)",r=2,xmin=0,xmax=2*pi):
\end{eulerprompt}
\eulerimg{19}{images/EMT4Kalkulus_Amalia Intan Arvitasari_22305144026-182.png}
\begin{eulerprompt}
>&kill(a) // hapus expresi a
\end{eulerprompt}
\begin{euleroutput}
  
                                   done
  
\end{euleroutput}
\begin{eulerprompt}
>function fx(t) &= exp(a*t)*cos(t); $'fx(t)=fx(t)
\end{eulerprompt}
\begin{eulerformula}
\[
{\it fx}\left(t\right)=e^{a\,t}\,\cos t
\]
\end{eulerformula}
\begin{eulerprompt}
>function fy(t) &= exp(a*t)*sin(t); $'fy(t)=fy(t)
\end{eulerprompt}
\begin{eulerformula}
\[
{\it fy}\left(t\right)=e^{a\,t}\,\sin t
\]
\end{eulerformula}
\begin{eulerprompt}
>function df(t) &= trigreduce(radcan(sqrt(diff(fx(t),t)^2+diff(fy(t),t)^2))); $'df(t)=df(t)
\end{eulerprompt}
\begin{eulerformula}
\[
{\it df}\left(t\right)=\sqrt{a^2+1}\,e^{a\,t}
\]
\end{eulerformula}
\begin{eulerprompt}
>S &=integrate(df(t),t,0,2*%pi); $S // panjang kurva (spiral)
\end{eulerprompt}
\begin{eulerformula}
\[
\sqrt{a^2+1}\,\left(\frac{e^{2\,\pi\,a}}{a}-\frac{1}{a}\right)
\]
\end{eulerformula}
\begin{eulerprompt}
>S(a=0.1) // Panjang kurva untuk a=0.1
\end{eulerprompt}
\begin{euleroutput}
  8.78817491636
\end{euleroutput}
\begin{eulercomment}
Soal:

Tunjukkan bahwa keliling lingkaran dengan jari-jari r adalah K=2.pi.r.

Berikut adalah contoh menghitung panjang parabola.
\end{eulercomment}
\begin{eulerprompt}
>plot2d("x^2",xmin=-1,xmax=1):
\end{eulerprompt}
\eulerimg{19}{images/EMT4Kalkulus_Amalia Intan Arvitasari_22305144026-187.png}
\begin{eulerprompt}
>$showev('integrate(sqrt(1+diff(x^2,x)^2),x,-1,1))
\end{eulerprompt}
\begin{eulerformula}
\[
\int_{-1}^{1}{\sqrt{4\,x^2+1}\;dx}=\frac{{\rm asinh}\; 2+2\,\sqrt{5  }}{2}
\]
\end{eulerformula}
\begin{eulerprompt}
>$float(%)
\end{eulerprompt}
\begin{eulerformula}
\[
\int_{-1.0}^{1.0}{\sqrt{4.0\,x^2+1.0}\;dx}=2.957885715089195
\]
\end{eulerformula}
\begin{eulerprompt}
>x=-1:0.2:1; y=x^2; plot2d(x,y);  ...
>  plot2d(x,y,points=1,style="o#",add=1):
\end{eulerprompt}
\eulerimg{19}{images/EMT4Kalkulus_Amalia Intan Arvitasari_22305144026-190.png}
\begin{eulercomment}
Panjang tersebut dapat dihampiri dengan menggunakan jumlah panjang ruas-ruas garis yang menghubungkan titik-titik pada parabola
tersebut.
\end{eulercomment}
\begin{eulerprompt}
>i=1:cols(x)-1; sum(sqrt((x[i+1]-x[i])^2+(y[i+1]-y[i])^2))
\end{eulerprompt}
\begin{euleroutput}
  2.95191957027
\end{euleroutput}
\begin{eulercomment}
Hasilnya mendekati panjang yang dihitung secara eksak. Untuk mendapatkan hampiran yang cukup akurat, jarak antar titik dapat
diperkecil, misalnya 0.1, 0.05, 0.01, dan seterusnya. Cobalah Anda ulangi perhitungannya dengan nilai-nilai tersebut.

\end{eulercomment}
\eulersubheading{Koordinat Kartesius}
\begin{eulercomment}
Berikut diberikan contoh perhitungan panjang kurva menggunakan koordinat Kartesius. Kita akan hitung panjang kurva dengan
persamaan implisit:

\end{eulercomment}
\begin{eulerformula}
\[
x^3+y^3-3xy=0.
\]
\end{eulerformula}
\begin{eulerprompt}
>z &= x^3+y^3-3*x*y; $z
\end{eulerprompt}
\begin{eulerformula}
\[
y^3-3\,x\,y+x^3
\]
\end{eulerformula}
\begin{eulerprompt}
>plot2d(z,r=2,level=0,n=100):
\end{eulerprompt}
\eulerimg{0}{images/EMT4Kalkulus_Amalia Intan Arvitasari_22305144026-192-large.png}
\begin{eulercomment}
Kita tertarik pada kurva di kuadran pertama.
\end{eulercomment}
\begin{eulerprompt}
>plot2d(z,a=0,b=2,c=0,d=2,level=[-10;0],n=100,contourwidth=3,style="/"):
\end{eulerprompt}
\eulerimg{19}{images/EMT4Kalkulus_Amalia Intan Arvitasari_22305144026-193.png}
\begin{eulercomment}
Kita selesaikan persamaannya untuk x.
\end{eulercomment}
\begin{eulerprompt}
>$z with y=l*x, sol &= solve(%,x); $sol
\end{eulerprompt}
\begin{eulerformula}
\[
\left[ x=\frac{3\,l}{l^3+1} , x=0 \right] 
\]
\end{eulerformula}
\eulerimg{0}{images/EMT4Kalkulus_Amalia Intan Arvitasari_22305144026-195-large.png}
\begin{eulercomment}
Kita gunakan solusi tersebut untuk mendefinisikan fungsi dengan Maxima.
\end{eulercomment}
\begin{eulerprompt}
>function f(l) &= rhs(sol[1]); $'f(l)=f(l)
\end{eulerprompt}
\begin{eulerformula}
\[
f\left(l\right)=\frac{3\,l}{l^3+1}
\]
\end{eulerformula}
\begin{eulercomment}
Fungsi tersebut juga dapat digunaka untuk menggambar kurvanya. Ingat,
bahwa fungsi tersebut adalah nilai x dan nilai y=l*x, yakni x=f(l) dan
y=l*f(l).
\end{eulercomment}
\begin{eulerprompt}
>plot2d(&f(x),&x*f(x),xmin=-0.5,xmax=2,a=0,b=2,c=0,d=2,r=1.5):
\end{eulerprompt}
\eulerimg{0}{images/EMT4Kalkulus_Amalia Intan Arvitasari_22305144026-197-large.png}
\begin{eulercomment}
Elemen panjang kurva adalah:

\end{eulercomment}
\begin{eulerformula}
\[
ds=\sqrt{f'(l)^2+(lf'(l)+f(l))^2}.
\]
\end{eulerformula}
\begin{eulerprompt}
>function ds(l) &= ratsimp(sqrt(diff(f(l),l)^2+diff(l*f(l),l)^2)); $'ds(l)=ds(l)
\end{eulerprompt}
\begin{eulerformula}
\[
{\it ds}\left(l\right)=\frac{\sqrt{9\,l^8+36\,l^6-36\,l^5-36\,l^3+  36\,l^2+9}}{\sqrt{l^{12}+4\,l^9+6\,l^6+4\,l^3+1}}
\]
\end{eulerformula}
\begin{eulerprompt}
>$integrate(ds(l),l,0,1)
\end{eulerprompt}
\begin{eulerformula}
\[
\int_{0}^{1}{\frac{\sqrt{9\,l^8+36\,l^6-36\,l^5-36\,l^3+36\,l^2+9}  }{\sqrt{l^{12}+4\,l^9+6\,l^6+4\,l^3+1}}\;dl}
\]
\end{eulerformula}
\begin{eulercomment}
Integral tersebut tidak dapat dihitung secara eksak menggunakan Maxima. Kita hitung integral etrsebut secara numerik dengan Euler.
Karena kurva simetris, kita hitung untuk nilai variabel integrasi dari 0 sampai 1, kemudian hasilnya dikalikan 2. 
\end{eulercomment}
\begin{eulerprompt}
>2*integrate("ds(x)",0,1)
\end{eulerprompt}
\begin{euleroutput}
  4.91748872168
\end{euleroutput}
\begin{eulerprompt}
>2*romberg(&ds(x),0,1)// perintah Euler lain untuk menghitung nilai hampiran integral yang sama
\end{eulerprompt}
\begin{euleroutput}
  4.91748872168
\end{euleroutput}
\begin{eulercomment}
Perhitungan di datas dapat dilakukan untuk sebarang fungsi x dan y dengan mendefinisikan fungsi EMT, misalnya kita beri nama
panjangkurva. Fungsi ini selalu memanggil Maxima untuk menurunkan fungsi yang diberikan.
\end{eulercomment}
\begin{eulerprompt}
>function panjangkurva(fx,fy,a,b) ...
\end{eulerprompt}
\begin{eulerudf}
  ds=mxm("sqrt(diff(@fx,x)^2+diff(@fy,x)^2)");
  return romberg(ds,a,b);
  endfunction
\end{eulerudf}
\begin{eulerprompt}
>panjangkurva("x","x^2",-1,1) // cek untuk menghitung panjang kurva parabola sebelumnya
\end{eulerprompt}
\begin{euleroutput}
  2.95788571509
\end{euleroutput}
\begin{eulercomment}
Bandingkan dengan nilai eksak di atas.
\end{eulercomment}
\begin{eulerprompt}
>2*panjangkurva(mxm("f(x)"),mxm("x*f(x)"),0,1) // cek contoh terakhir, bandingkan hasilnya!
\end{eulerprompt}
\begin{euleroutput}
  4.91748872168
\end{euleroutput}
\begin{eulercomment}
Kita hitung panjang spiral Archimides berikut ini dengan fungsi tersebut.
\end{eulercomment}
\begin{eulerprompt}
>plot2d("x*cos(x)","x*sin(x)",xmin=0,xmax=2*pi,square=1):
\end{eulerprompt}
\eulerimg{1}{images/EMT4Kalkulus_Amalia Intan Arvitasari_22305144026-201-large.png}
\begin{eulerprompt}
>panjangkurva("x*cos(x)","x*sin(x)",0,2*pi)
\end{eulerprompt}
\begin{euleroutput}
  21.2562941482
\end{euleroutput}
\begin{eulercomment}
Berikut kita definisikan fungsi yang sama namun dengan Maxima, untuk perhitungan eksak.
\end{eulercomment}
\begin{eulerprompt}
>&kill(ds,x,fx,fy)
\end{eulerprompt}
\begin{euleroutput}
  
                                   done
  
\end{euleroutput}
\begin{eulerprompt}
>function ds(fx,fy) &&= sqrt(diff(fx,x)^2+diff(fy,x)^2)
\end{eulerprompt}
\begin{euleroutput}
  
                             2              2
                    sqrt(diff (fy, x) + diff (fx, x))
  
\end{euleroutput}
\begin{eulerprompt}
>sol &= ds(x*cos(x),x*sin(x)); $sol // Kita gunakan untuk menghitung panjang kurva terakhir di atas
\end{eulerprompt}
\begin{eulerformula}
\[
\sqrt{\left(\cos x-x\,\sin x\right)^2+\left(\sin x+x\,\cos x\right)  ^2}
\]
\end{eulerformula}
\begin{eulerprompt}
>$sol | trigreduce | expand, $integrate(%,x,0,2*pi), %()
\end{eulerprompt}
\begin{eulerformula}
\[
\frac{{\rm asinh}\; \left(2\,\pi\right)+2\,\pi\,\sqrt{4\,\pi^2+1}}{  2}
\]
\end{eulerformula}
\eulerimg{0}{images/EMT4Kalkulus_Amalia Intan Arvitasari_22305144026-204-large.png}
\begin{euleroutput}
  21.2562941482
\end{euleroutput}
\begin{eulercomment}
Hasilnya sama dengan perhitungan menggunakan fungsi EMT.

Berikut adalah contoh lain penggunaan fungsi Maxima tersebut.
\end{eulercomment}
\begin{eulerprompt}
>plot2d("3*x^2-1","3*x^3-1",xmin=-1/sqrt(3),xmax=1/sqrt(3),square=1):
\end{eulerprompt}
\eulerimg{1}{images/EMT4Kalkulus_Amalia Intan Arvitasari_22305144026-205-large.png}
\begin{eulerprompt}
>sol &= radcan(ds(3*x^2-1,3*x^3-1)); $sol
\end{eulerprompt}
\begin{eulerformula}
\[
3\,x\,\sqrt{9\,x^2+4}
\]
\end{eulerformula}
\begin{eulerprompt}
>$showev('integrate(sol,x,0,1/sqrt(3))), $2*float(%) // panjang kurva di atas
\end{eulerprompt}
\begin{eulerformula}
\[
6.0\,\int_{0.0}^{0.5773502691896258}{x\,\sqrt{9.0\,x^2+4.0}\;dx}=  2.337835372767141
\]
\end{eulerformula}
\eulerimg{1}{images/EMT4Kalkulus_Amalia Intan Arvitasari_22305144026-208-large.png}
\eulersubheading{Sikloid}
\begin{eulercomment}
Berikut kita akan menghitung panjang kurva lintasan (sikloid) suatu
titik pada lingkaran yang berputar ke kanan pada permukaan datar.
Misalkan jari-jari lingkaran tersebut adalah r. Posisi titik pusat
lingkaran pada saat t adalah:

\end{eulercomment}
\begin{eulerformula}
\[
(rt,r).
\]
\end{eulerformula}
\begin{eulercomment}
Misalkan posisi titik pada lingkaran tersebut mula-mula (0,0) dan
posisinya pada saat t adalah:

\end{eulercomment}
\begin{eulerformula}
\[
(r(t-\sin(t)),r(1-\cos(t))).
\]
\end{eulerformula}
\begin{eulercomment}
Berikut kita plot lintasan tersebut dan beberapa posisi lingkaran
ketika t=0, t=pi/2, t=r*pi.
\end{eulercomment}
\begin{eulerprompt}
>x &= r*(t-sin(t))
\end{eulerprompt}
\begin{euleroutput}
  
                              r (t - sin(t))
  
\end{euleroutput}
\begin{eulerprompt}
>y &= r*(1-cos(t))
\end{eulerprompt}
\begin{euleroutput}
  
                              r (1 - cos(t))
  
\end{euleroutput}
\begin{eulercomment}
Berikut kita gambar sikloid untuk r=1.
\end{eulercomment}
\begin{eulerprompt}
>ex &= x-sin(x); ey &= 1-cos(x); aspect(1);
>plot2d(ex,ey,xmin=0,xmax=4pi,square=1); ...
>  plot2d("2+cos(x)","1+sin(x)",xmin=0,xmax=2pi,>add,color=blue); ...
>  plot2d([2,ex(2)],[1,ey(2)],color=red,>add); ...
>  plot2d(ex(2),ey(2),>points,>add,color=red); ...
>  plot2d("2pi+cos(x)","1+sin(x)",xmin=0,xmax=2pi,>add,color=blue); ...
>  plot2d([2pi,ex(2pi)],[1,ey(2pi)],color=red,>add);  ...
>  plot2d(ex(2pi),ey(2pi),>points,>add,color=red):
\end{eulerprompt}
\eulerimg{0}{images/EMT4Kalkulus_Amalia Intan Arvitasari_22305144026-211-large.png}
\begin{eulercomment}
Berikut dihitung panjang lintasan untuk 1 putaran penuh. (Jangan salah menduga bahwa panjang lintasan 1 putaran penuh sama dengan
keliling lingkaran!)
\end{eulercomment}
\begin{eulerprompt}
>ds &= radcan(sqrt(diff(ex,x)^2+diff(ey,x)^2)); $ds=trigsimp(ds) // elemen panjang kurva sikloid
\end{eulerprompt}
\begin{eulerformula}
\[
\sqrt{\sin ^2x+\cos ^2x-2\,\cos x+1}=\sqrt{2-2\,\cos x}
\]
\end{eulerformula}
\begin{eulerprompt}
>ds &= trigsimp(ds); $ds
\end{eulerprompt}
\begin{eulerformula}
\[
\sqrt{2-2\,\cos x}
\]
\end{eulerformula}
\begin{eulerprompt}
>$showev('integrate(ds,x,0,2*pi)) // hitung panjang sikloid satu putaran penuh
\end{eulerprompt}
\begin{eulerformula}
\[
\int_{0}^{2\,\pi}{\sqrt{2-2\,\cos x}\;dx}=8
\]
\end{eulerformula}
\begin{eulerprompt}
>integrate(mxm("ds"),0,2*pi) // hitung secara numerik
\end{eulerprompt}
\begin{euleroutput}
  8
\end{euleroutput}
\begin{eulerprompt}
>romberg(mxm("ds"),0,2*pi) // cara lain hitung secara numerik
\end{eulerprompt}
\begin{euleroutput}
  8
\end{euleroutput}
\begin{eulercomment}
Perhatikan, seperti terlihat pada gambar, panjang sikloid lebih besar
daripada keliling lingkarannya, yakni:

\end{eulercomment}
\begin{eulerformula}
\[
2\pi.
\]
\end{eulerformula}
\eulersubheading{Kurvatur (Kelengkungan) Kurva}
\begin{eulercomment}
image: Osculating.png

Aslinya, kelengkungan kurva diferensiabel (yakni, kurva mulus yang
tidak lancip) di titik P didefinisikan melalui lingkaran oskulasi
(yaitu, lingkaran yang melalui titik P dan terbaik memperkirakan,
paling banyak menyinggung kurva di sekitar P). Pusat dan radius
kelengkungan kurva di P adalah pusat dan radius lingkaran oskulasi.
Kelengkungan adalah kebalikan dari radius kelengkungan:

\end{eulercomment}
\begin{eulerformula}
\[
\kappa =\frac {1}{R}
\]
\end{eulerformula}
\begin{eulercomment}
dengan R adalah radius kelengkungan. (Setiap lingkaran memiliki
kelengkungan ini pada setiap titiknya, dapat diartikan, setiap
lingkaran berputar 2pi sejauh 2piR.)\\
Definisi ini sulit dimanipulasi dan dinyatakan ke dalam rumus untuk
kurva umum. Oleh karena itu digunakan definisi lain yang ekivalen.

\end{eulercomment}
\eulersubheading{Definisi Kurvatur dengan Fungsi Parametrik Panjang Kurva}
\begin{eulercomment}
Setiap kurva diferensiabel dapat dinyatakan dengan persamaan
parametrik terhadap panjang kurva s:

\end{eulercomment}
\begin{eulerformula}
\[
\gamma(s) = (x(s),\ y(s)),
\]
\end{eulerformula}
\begin{eulercomment}
dengan x dan y adalah fungsi riil yang diferensiabel, yang memenuhi:

\end{eulercomment}
\begin{eulerformula}
\[
\|\gamma'(s)\|=\sqrt{x'(s)^2+y'(s)^2}=1.
\]
\end{eulerformula}
\begin{eulercomment}
Ini berarti bahwa vektor singgung


\end{eulercomment}
\begin{eulerformula}
\[
\mathbf{T}(s)=(x'(s),\ y'(s))
\]
\end{eulerformula}
\begin{eulercomment}
memiliki norm 1 dan merupakan vektor singgung satuan.

Apabila kurvanya memiliki turunan kedua, artinya turunan kedua x dan y
ada, maka T'(s) ada. Vektor ini merupakan normal kurva yang arahnya
menuju pusat kurvatur, norm-nya merupakan nilai kurvatur
(kelengkungan):

\end{eulercomment}
\begin{eulerformula}
\[
 \begin{aligned}\mathbf{T}(s) &= \mathbf{\gamma}'(s),\\ \mathbf{T}^{2}(s) &=1\ \text{(konstanta)}\Rightarrow \mathbf{T}'(s)\cdot \mathbf{T}(s)=0\\ \kappa(s) &=\|\mathbf {T}'(s)\|= \|\mathbf{\gamma}''(s)\|=\sqrt{x''(s)^{2}+y''(s)^{2}}.\end{aligned}
\]
\end{eulerformula}
\begin{eulercomment}
Nilai

\end{eulercomment}
\begin{eulerformula}
\[
R(s)=\frac{1}{\kappa(s)}
\]
\end{eulerformula}
\begin{eulercomment}
disebut jari-jari (radius) kelengkungan kurva.

Bilangan riil

\end{eulercomment}
\begin{eulerformula}
\[
 k(s) = \pm\kappa(s)
\]
\end{eulerformula}
\begin{eulercomment}
disebut nilai kelengkungan bertanda.

Contoh:\\
Akan ditentukan kurvatur lingkaran

\end{eulercomment}
\begin{eulerformula}
\[
x=r\cos t,\ y= r\sin t.
\]
\end{eulerformula}
\begin{eulerprompt}
>fx &= r*cos(t); fy &=r*sin(t);
>&assume(t>0,r>0); s &=integrate(sqrt(diff(fx,t)^2+diff(fy,t)^2),t,0,t); s // elemen panjang kurva, panjang busur lingkaran (s)
\end{eulerprompt}
\begin{euleroutput}
  
                                   r t
  
\end{euleroutput}
\begin{eulerprompt}
>&kill(s); fx &= r*cos(s/r); fy &=r*sin(s/r); // definisi ulang persamaan parametrik terhadap s dengan substitusi t=s/r
>k &= trigsimp(sqrt(diff(fx,s,2)^2+diff(fy,s,2)^2)); $k // nilai kurvatur lingkaran dengan menggunakan definisi di atas
\end{eulerprompt}
\begin{eulerformula}
\[
\frac{1}{r}
\]
\end{eulerformula}
\begin{eulercomment}
Untuk representasi parametrik umum, misalkan

\end{eulercomment}
\begin{eulerformula}
\[
x = x(t),\ y= y(t)
\]
\end{eulerformula}
\begin{eulercomment}
merupakan persamaan parametrik untuk kurva bidang yang
terdiferensialkan dua kali. Kurvatur untuk kurva tersebut
didefinisikan sebagai

\end{eulercomment}
\begin{eulerformula}
\[
\begin{aligned}\kappa &= \frac{d\phi}{ds}=\frac{\frac{d\phi}{dt}}{\frac{ds}{dt}}\quad (\phi \text{ adalah sudut kemiringan garis singgung dan }s \text{ adalah panjang kurva})\\ &=\frac{\frac{d\phi}{dt}}{\sqrt{(\frac{dx}{dt})^2+(\frac{dy}{dt})^2}}= \frac{\frac{d\phi}{dt}}{\sqrt{x'(t)^2+y'(t)^2}}.\end{aligned}.
\]
\end{eulerformula}
\begin{eulercomment}
Selanjutnya, pembilang pada persamaan di atas dapat dicari sebagai
berikut.

\end{eulercomment}
\begin{eulerformula}
\[
\begin{aligned}\sec^2\phi\frac{d\phi}{dt} &= \frac{d}{dt}\left(\tan\phi\right)= \frac{d}{dt}\left(\frac{dy}{dx}\right)= \frac{d}{dt}\left(\frac{dy/dt}{dx/dt}\right)= \frac{d}{dt}\left(\frac{y'(t)}{x'(t)}\right)=\frac{x'(t)y''(t)-x''(t)y'(t)}{x'(t)^2}.\\ & \\ \frac{d\phi}{dt} &= \frac{1}{\sec^2\phi}\frac{x'(t)y''(t)-x''(t)y'(t)}{x'(t)^2}\\ &= \frac{1}{1+\tan^2\phi}\frac{x'(t)y''(t)-x''(t)y'(t)}{x'(t)^2}\\ &= \frac{1}{1+\left(\frac{y'(t)}{x'(t)}\right)^2}\frac{x'(t)y''(t)-x''(t)y'(t)}{x'(t)^2}\\ &= \frac{x'(t)y''(t)-x''(t)y'(t)}{x'(t)^2+y'(t)^2}.\end{aligned}
\]
\end{eulerformula}
\begin{eulercomment}
Jadi, rumus kurvatur untuk kurva parametrik

\end{eulercomment}
\begin{eulerformula}
\[
x=x(t),\ y=y(t)
\]
\end{eulerformula}
\begin{eulercomment}
adalah

\end{eulercomment}
\begin{eulerformula}
\[
\kappa(t) = \frac{x'(t)y''(t)-x''(t)y'(t)}{\left(x'(t)^2+y'(t)^2\right)^{3/2}}.
\]
\end{eulerformula}
\begin{eulercomment}
Jika kurvanya dinyatakan dengan persamaan parametrik pada koordinat
kutub

\end{eulercomment}
\begin{eulerformula}
\[
x=r(\theta)\cos\theta,\ y=r(\theta)\sin\theta,
\]
\end{eulerformula}
\begin{eulercomment}
maka rumus kurvaturnya adalah

\end{eulercomment}
\begin{eulerformula}
\[
\kappa(\theta) = \frac{r(\theta)^2+2r'(\theta)^2-r(\theta)r''(\theta)}{\left(r'(\theta)^2+r'(\theta)^2\right)^{3/2}}.
\]
\end{eulerformula}
\begin{eulercomment}
(Silakan Anda turunkan rumus tersebut!)

Contoh:\\
Lingkaran dengan pusat (0,0) dan jari-jari r dapat dinyatakan dengan
persamaan parametrik

\end{eulercomment}
\begin{eulerformula}
\[
x=r\cos t,\ y=r\sin t.
\]
\end{eulerformula}
\begin{eulercomment}
Nilai kelengkungan lingkaran tersebut adalah

\end{eulercomment}
\begin{eulerformula}
\[
\kappa(t)=\frac{x'(t)y''(t)-x''(t)y'(t)}{\left(x'(t)^2+y'(t)^2\right)^{3/2}}=\frac{r^2}{r^3}=\frac 1 r.
\]
\end{eulerformula}
\begin{eulercomment}
Hasil cocok dengan definisi kurvatur suatu kelengkungan.
\end{eulercomment}
\begin{eulercomment}
Kurva

\end{eulercomment}
\begin{eulerformula}
\[
y=f(x)
\]
\end{eulerformula}
\begin{eulercomment}
dapat dinyatakan ke dalam persamaan parametrik

\end{eulercomment}
\begin{eulerformula}
\[
x=t,\ y=f(t),\ \text{ dengan } x'(t)=1,\ x''(t)=0,
\]
\end{eulerformula}
\begin{eulercomment}
sehingga kurvaturnya adalah

\end{eulercomment}
\begin{eulerformula}
\[
\kappa(t) = \frac{y''(t)}{\left(1+y'(t)^2\right)^{3/2}}.
\]
\end{eulerformula}
\begin{eulercomment}
Contoh:\\
Akan ditentukan kurvatur parabola

\end{eulercomment}
\begin{eulerformula}
\[
y=ax^2+bx+c.
\]
\end{eulerformula}
\begin{eulerprompt}
>function f(x) &= a*x^2+b*x+c; $y=f(x)
\end{eulerprompt}
\begin{eulerformula}
\[
y=a\,x^2+b\,x+c
\]
\end{eulerformula}
\begin{eulerprompt}
>function k(x) &= (diff(f(x),x,2))/(1+diff(f(x),x)^2)^(3/2); $'k(x)=k(x) // kelengkungan parabola 
\end{eulerprompt}
\begin{eulerformula}
\[
k\left(x\right)=\frac{2\,a}{\left(\left(2\,a\,x+b\right)^2+1\right)  ^{\frac{3}{2}}}
\]
\end{eulerformula}
\begin{eulerprompt}
>function f(x) &= x^2+x+1; $y=f(x) // akan kita plot kelengkungan parabola untuk a=b=c=1
\end{eulerprompt}
\begin{eulerformula}
\[
y=x^2+x+1
\]
\end{eulerformula}
\begin{eulerprompt}
>function k(x) &= (diff(f(x),x,2))/(1+diff(f(x),x)^2)^(3/2); $'k(x)=k(x) // kelengkungan parabola 
\end{eulerprompt}
\begin{eulerformula}
\[
k\left(x\right)=\frac{2}{\left(\left(2\,x+1\right)^2+1\right)^{  \frac{3}{2}}}
\]
\end{eulerformula}
\begin{eulercomment}
Berikut kita gambar parabola tersebut beserta kurva kelengkungan,
kurva jari-jari kelengkungan dan salah satu lingkaran oskulasi di
titik puncak parabola. Perhatikan, puncak parabola dan jari-jari
lingkaran oskulasi di puncak parabola adalah

\end{eulercomment}
\begin{eulerformula}
\[
(-1/2,3/4),\ 1/k(2)=1/2,
\]
\end{eulerformula}
\begin{eulercomment}
sehingga pusat lingkaran oskulasi adalah (-1/2, 5/4).
\end{eulercomment}
\begin{eulerprompt}
>plot2d(["f(x)", "k(x)"],-2,1, color=[blue,red]); plot2d("1/k(x)",-1.5,1,color=green,>add); ...
>plot2d("-1/2+1/k(-1/2)*cos(x)","5/4+1/k(-1/2)*sin(x)",xmin=0,xmax=2pi,>add,color=blue):
\end{eulerprompt}
\eulerimg{0}{images/EMT4Kalkulus_Amalia Intan Arvitasari_22305144026-243-large.png}
\begin{eulercomment}
Untuk kurva yang dinyatakan dengan fungsi implisit

\end{eulercomment}
\begin{eulerformula}
\[
F(x,y)=0
\]
\end{eulerformula}
\begin{eulercomment}
dengan turunan-turunan parsial

\end{eulercomment}
\begin{eulerformula}
\[
F_x=\frac{\partial F}{\partial x},\ F_y=\frac{\partial F}{\partial y},\ F_{xy}=\frac{\partial}{\partial y}\left(\frac{\partial F}{\partial x}\right),\ F_{xx}=\frac{\partial}{\partial x}\left(\frac{\partial F}{\partial x}\right),\ F_{yy}=\frac{\partial}{\partial y}\left(\frac{\partial F}{\partial y}\right),
\]
\end{eulerformula}
\begin{eulercomment}
berlaku

\end{eulercomment}
\begin{eulerformula}
\[
F_x dx+ F_y dy = 0\text{ atau } \frac{dy}{dx}=-\frac{F_x}{F_y},
\]
\end{eulerformula}
\begin{eulercomment}
sehingga kurvaturnya adalah

\end{eulercomment}
\begin{eulerformula}
\[
\kappa =\frac {F_y^2F_{xx}-2F_xF_yF_{xy}+F_x^2F_{yy}}{\left(F_x^2+F_y^2\right)^{3/2}}.
\]
\end{eulerformula}
\begin{eulercomment}
(Silakan Anda turunkan sendiri!)

Contoh 1:\\
Parabola

\end{eulercomment}
\begin{eulerformula}
\[
y=ax^2+bx+c
\]
\end{eulerformula}
\begin{eulercomment}
dapat dinyatakan ke dalam persamaan implisit

\end{eulercomment}
\begin{eulerformula}
\[
ax^2+bx+c-y=0.
\]
\end{eulerformula}
\begin{eulerprompt}
>function F(x,y) &=a*x^2+b*x+c-y; $F(x,y)
\end{eulerprompt}
\begin{eulerformula}
\[
-y+a\,x^2+b\,x+c
\]
\end{eulerformula}
\begin{eulerprompt}
>Fx &= diff(F(x,y),x), Fxx &=diff(F(x,y),x,2), Fy &=diff(F(x,y),y), Fxy &=diff(diff(F(x,y),x),y), Fyy &=diff(F(x,y),y,2)  
\end{eulerprompt}
\begin{euleroutput}
  
                                2 a x + b
  
  
                                   2 a
  
  
                                   - 1
  
  
                                    0
  
  
                                    0
  
\end{euleroutput}
\begin{eulerprompt}
>function k(x) &= (Fy^2*Fxx-2*Fx*Fy*Fxy+Fx^2*Fyy)/(Fx^2+Fy^2)^(3/2); $'k(x)=k(x) // kurvatur parabola tersebut
\end{eulerprompt}
\begin{eulercomment}
Hasilnya sama dengan sebelumnya yang menggunakan persamaan parabola biasa.
\end{eulercomment}
\eulerheading{Latihan}
\begin{eulercomment}
- Bukalah buku Kalkulus.\\
- Cari dan pilih beberapa (paling sedikit 5 fungsi berbeda
tipe/bentuk/jenis) fungsi dari buku tersebut, kemudian definisikan di
EMT pada baris-baris perintah berikut (jika perlu tambahkan lagi).\\
- Untuk setiap fungsi, tentukan anti turunannya (jika ada), hitunglah
integral tentu dengan batas-batas yang menarik (Anda tentukan
sendiri), seperti contoh-contoh tersebut.\\
- Lakukan hal yang sama untuk fungsi-fungsi yang tidak dapat
diintegralkan (cari sedikitnya 3 fungsi).\\
- Gambar grafik fungsi dan daerah integrasinya pada sumbu koordinat
yang sama.\\
- Gunakan integral tentu untuk mencari luas daerah yang dibatasi oleh
dua kurva yang berpotongan di dua titik. (Cari dan gambar kedua kurva
dan arsir (warnai) daerah yang dibatasi oleh keduanya.)\\
- Gunakan integral tentu untuk menghitung volume benda putar kurva y=
f(x) yang diputar mengelilingi sumbu x dari x=a sampai x=b, yakni

\end{eulercomment}
\begin{eulerformula}
\[
V = \int_a^b \pi (f(x)^2\ dx.
\]
\end{eulerformula}
\begin{eulercomment}
(Pilih fungsinya dan gambar kurva dan benda putar yang dihasilkan.
Anda dapat mencari contoh-contoh bagaimana cara menggambar benda hasil
perputaran suatu kurva.)\\
- Gunakan integral tentu untuk menghitung panjang kurva y=f(x) dari
x=a sampai x=b dengan menggunakan rumus:

\end{eulercomment}
\begin{eulerformula}
\[
S = \int_a^b \sqrt{1+(f'(x))^2} \ dx.
\]
\end{eulerformula}
\begin{eulercomment}
(Pilih fungsi dan gambar kurvanya.)\\
- Apabila fungsi dinyatakan dalam koordinat kutub x=f(r,t), y=g(r,t),
r=h(t), x=a bersesuaian dengan t=t0 dan x=b bersesuian dengan t=t1,
maka rumus di atas akan menjadi:

\end{eulercomment}
\begin{eulerformula}
\[
S=\int_{t_0}^{t_1} \sqrt{x'(t)^2+y'(t)^2}\ dt.
\]
\end{eulerformula}
\begin{eulercomment}
- Pilih beberapa kurva menarik (selain lingkaran dan parabola) dari
buku  kalkulus. Nyatakan setiap kurva tersebut dalam bentuk:\\
\end{eulercomment}
\begin{eulerttcomment}
  a. koordinat Kartesius (persamaan y=f(x))
  b. koordinat kutub ( r=r(theta))
  c. persamaan parametrik x=x(t), y=y(t)
  d. persamaan implit F(x,y)=0
\end{eulerttcomment}
\begin{eulercomment}
- Tentukan kurvatur masing-masing kurva dengan menggunakan keempat
representasi tersebut (hasilnya harus sama).\\
- Gambarlah kurva asli, kurva kurvatur, kurva jari-jari lingkaran
oskulasi, dan salah satu lingkaran oskulasinya.

\end{eulercomment}
\eulersubheading{Jawab}
\begin{eulercomment}
Soal 1:
\end{eulercomment}
\begin{eulerprompt}
>function f(x) &= 5*x^2; $f(x)
\end{eulerprompt}
\begin{eulerformula}
\[
5\,x^2
\]
\end{eulerformula}
\begin{eulerprompt}
>$showev('integrate(f(x),x))
\end{eulerprompt}
\begin{eulerformula}
\[
5\,\int {x^2}{\;dx}=\frac{5\,x^3}{3}
\]
\end{eulerformula}
\begin{eulerprompt}
>$showev('integrate(f(x),x,2,3))
\end{eulerprompt}
\begin{eulerformula}
\[
5\,\int_{2}^{3}{x^2\;dx}=\frac{95}{3}
\]
\end{eulerformula}
\begin{eulerprompt}
>x=0.01:0.03:4; plot2d(x,f(x+0.01),>bar); plot2d("f(x)",2,3,>add):
\end{eulerprompt}
\eulerimg{29}{images/EMT4Kalkulus_Amalia Intan Arvitasari_22305144026-254.png}
\begin{eulercomment}
Soal 2:
\end{eulercomment}
\begin{eulerprompt}
>function f(x) &= cos(2*x+5); $f(x)
\end{eulerprompt}
\begin{eulerformula}
\[
\cos \left(2\,x+5\right)
\]
\end{eulerformula}
\begin{eulerprompt}
>$showev('integrate(f(x),x))
\end{eulerprompt}
\begin{eulerformula}
\[
\int {\cos \left(2\,x+5\right)}{\;dx}=\frac{\sin \left(2\,x+5  \right)}{2}
\]
\end{eulerformula}
\begin{eulerprompt}
>$showev('integrate(f(x),x,pi,2*pi))
\end{eulerprompt}
\begin{eulerformula}
\[
\int_{\pi}^{2\,\pi}{\cos \left(2\,x+5\right)\;dx}=0
\]
\end{eulerformula}
\begin{eulerprompt}
>x=0:0.05:pi-0.1; plot2d(x,f(x+0.03),>bar); plot2d("f(x)",pi,2*pi,>add):
\end{eulerprompt}
\eulerimg{29}{images/EMT4Kalkulus_Amalia Intan Arvitasari_22305144026-258.png}
\begin{eulercomment}
Soal 3:
\end{eulercomment}
\begin{eulerprompt}
>function f(x) &= (sin(x))*(cos(x))^2; $f(x)
\end{eulerprompt}
\begin{eulerformula}
\[
\cos ^2x\,\sin x
\]
\end{eulerformula}
\begin{eulerprompt}
>$showev('integrate(f(x),x))
\end{eulerprompt}
\begin{eulerformula}
\[
\int {\cos ^2x\,\sin x}{\;dx}=-\frac{\cos ^3x}{3}
\]
\end{eulerformula}
\begin{eulerprompt}
>$showev('integrate(f(x),x,0,pi))
\end{eulerprompt}
\begin{eulerformula}
\[
\int_{0}^{\pi}{\cos ^2x\,\sin x\;dx}=\frac{2}{3}
\]
\end{eulerformula}
\begin{eulerprompt}
>x=-pi:0.04:pi; plot2d(x,f(x+0.01),>bar); plot2d("f(x)",0,pi,>add):
\end{eulerprompt}
\eulerimg{29}{images/EMT4Kalkulus_Amalia Intan Arvitasari_22305144026-262.png}
\begin{eulercomment}
Soal 4:
\end{eulercomment}
\begin{eulerprompt}
>function f(x) &= (x^2*(2-x^3)^(1/2)); $f(x)
\end{eulerprompt}
\begin{eulerformula}
\[
x^2\,\sqrt{2-x^3}
\]
\end{eulerformula}
\begin{eulerprompt}
>$showev('integrate(f(x),x))
\end{eulerprompt}
\begin{eulerformula}
\[
\int {x^2\,\sqrt{2-x^3}}{\;dx}=-\frac{2\,\left(2-x^3\right)^{\frac{  3}{2}}}{9}
\]
\end{eulerformula}
\begin{eulerprompt}
>$showev('integrate(f(x),x,0,1))
\end{eulerprompt}
\begin{eulerformula}
\[
\int_{0}^{1}{x^2\,\sqrt{2-x^3}\;dx}=\frac{2^{\frac{5}{2}}}{9}-  \frac{2}{9}
\]
\end{eulerformula}
\begin{eulerprompt}
>x=-1:0.04:1; plot2d(x,f(x+0.01),>bar); plot2d("f(x)",0,1,>add):
\end{eulerprompt}
\eulerimg{29}{images/EMT4Kalkulus_Amalia Intan Arvitasari_22305144026-266.png}
\begin{eulercomment}
Soal 5:
\end{eulercomment}
\begin{eulerprompt}
>function f(x) &= sqrt(24-x^2); $f(x)
\end{eulerprompt}
\begin{eulerformula}
\[
\sqrt{24-x^2}
\]
\end{eulerformula}
\begin{eulerprompt}
>$showev('integrate(f(x),x))
\end{eulerprompt}
\begin{eulerformula}
\[
\int {\sqrt{24-x^2}}{\;dx}=12\,\arcsin \left(\frac{x}{2\,\sqrt{6}}  \right)+\frac{x\,\sqrt{24-x^2}}{2}
\]
\end{eulerformula}
\begin{eulerprompt}
>$showev('integrate(f(x),x,1,2))
\end{eulerprompt}
\begin{eulerformula}
\[
\int_{1}^{2}{\sqrt{24-x^2}\;dx}=12\,\arcsin \left(\frac{1}{\sqrt{6}  }\right)-\frac{24\,\arcsin \left(\frac{1}{2\,\sqrt{6}}\right)+\sqrt{  23}}{2}+2\,\sqrt{5}
\]
\end{eulerformula}
\begin{eulerprompt}
>x=-2:0.04:1; plot2d(x,f(x+0.01),>bar); plot2d("f(x)",1,2,>add):
\end{eulerprompt}
\eulerimg{29}{images/EMT4Kalkulus_Amalia Intan Arvitasari_22305144026-270.png}
\begin{eulercomment}
Soal 6:
\end{eulercomment}
\begin{eulerprompt}
>t &= makelist(a,a,0,1-0.01,0.01);
>fx &= makelist(f(t[i]+0.01),i,1,length(t));
>function f(x) &= x^2+50; $f(x) 
\end{eulerprompt}
\begin{eulerformula}
\[
x^2+50
\]
\end{eulerformula}
\begin{eulerprompt}
>x=0:0.1:pi-0.01; plot2d(x,f(x+0.01),>bar); plot2d("f(x)",0,pi,>add): 
\end{eulerprompt}
\eulerimg{29}{images/EMT4Kalkulus_Amalia Intan Arvitasari_22305144026-272.png}
\eulerheading{Barisan dan Deret}
\begin{eulercomment}
(Catatan: bagian ini belum lengkap. Anda dapat membaca contoh-contoh
pengguanaan EMT dan Maxima untuk menghitung limit barisan, rumus
jumlah parsial suatu deret, jumlah tak hingga suatu deret konvergen,
dan sebagainya. Anda dapat mengeksplor contoh-contoh di EMT atau
perbagai panduan penggunaan Maxima di software Maxima atau dari
Internet.)

Barisan dapat didefinisikan dengan beberapa cara di dalam EMT, di
antaranya:

- dengan cara yang sama seperti mendefinisikan vektor dengan
elemen-elemen beraturan (menggunakan titik dua ":");\\
- menggunakan perintah "sequence" dan rumus barisan (suku ke -n);\\
- menggunakan perintah "iterate" atau "niterate";\\
- menggunakan fungsi Maxima "create\_list" atau "makelist" untuk
menghasilkan barisan simbolik;\\
- menggunakan fungsi biasa yang inputnya vektor atau barisan;\\
- menggunakan fungsi rekursif.

EMT menyediakan beberapa perintah (fungsi) terkait barisan, yakni:

- sum: menghitung jumlah semua elemen suatu barisan\\
- cumsum: jumlah kumulatif suatu barisan\\
- differences: selisih antar elemen-elemen berturutan

EMT juga dapat digunakan untuk menghitung jumlah deret berhingga
maupun deret tak hingga, dengan menggunakan perintah (fungsi) "sum".
Perhitungan dapat dilakukan secara numerik maupun simbolik dan eksak.

Berikut adalah beberapa contoh perhitungan barisan dan deret
menggunakan EMT.
\end{eulercomment}
\begin{eulerprompt}
>1:10 // barisan sederhana
\end{eulerprompt}
\begin{euleroutput}
  [1,  2,  3,  4,  5,  6,  7,  8,  9,  10]
\end{euleroutput}
\begin{eulerprompt}
>1:2:30
\end{eulerprompt}
\begin{euleroutput}
  [1,  3,  5,  7,  9,  11,  13,  15,  17,  19,  21,  23,  25,  27,  29]
\end{euleroutput}
\eulerheading{Iterasi dan Barisan}
\begin{eulercomment}
EMT menyediakan fungsi iterate("g(x)", x0, n) untuk melakukan iterasi

\end{eulercomment}
\begin{eulerformula}
\[
x_{k+1}=g(x_k), \ x_0=x_0, k= 1, 2, 3, ..., n.
\]
\end{eulerformula}
\begin{eulercomment}
Berikut ini disajikan contoh-contoh penggunaan iterasi dan rekursi
dengan EMT. Contoh pertama menunjukkan pertumbuhan dari nilai awal
1000 dengan laju pertambahan 5\%, selama 10 periode.
\end{eulercomment}
\begin{eulerprompt}
>q=1.05; iterate("x*q",1000,n=10)'
\end{eulerprompt}
\begin{euleroutput}
           1000 
           1050 
         1102.5 
        1157.63 
        1215.51 
        1276.28 
         1340.1 
         1407.1 
        1477.46 
        1551.33 
        1628.89 
\end{euleroutput}
\begin{eulercomment}
Contoh berikutnya memperlihatkan bahaya menabung di bank pada masa sekarang! Dengan bunga
tabungan sebesar 6\% per tahun atau 0.5\% per bulan dipotong pajak 20\%, dan biaya administrasi
10000 per bulan, tabungan sebesar 1 juta tanpa diambil selama sekitar 10 tahunan akan habis
diambil oleh bank!
\end{eulercomment}
\begin{eulerprompt}
>r=0.005; plot2d(iterate("(1+0.8*r)*x-10000",1000000,n=130)):
\end{eulerprompt}
\eulerimg{29}{images/EMT4Kalkulus_Amalia Intan Arvitasari_22305144026-273.png}
\begin{eulercomment}
Silakan Anda coba-coba, dengan tabungan minimal berapa agar tidak akan habis diambil oleh
bank dengan ketentuan bunga dan biaya administrasi seperti di atas.

Berikut adalah perhitungan minimal tabungan agar aman di bank dengan bunga sebesar r dan
biaya administrasi a, pajak bunga 20\%.
\end{eulercomment}
\begin{eulerprompt}
>$solve(0.8*r*A-a,A), $% with [r=0.005, a=10] 
\end{eulerprompt}
\begin{eulerformula}
\[
\left[ A=2500.0 \right] 
\]
\end{eulerformula}
\eulerimg{0}{images/EMT4Kalkulus_Amalia Intan Arvitasari_22305144026-275-large.png}
\begin{eulercomment}
Berikut didefinisikan fungsi untuk menghitung saldo tabungan, kemudian dilakukan iterasi.
\end{eulercomment}
\begin{eulerprompt}
>function saldo(x,r,a) := round((1+0.8*r)*x-a,2);
>iterate(\{\{"saldo",0.005,10\}\},1000,n=6)
\end{eulerprompt}
\begin{euleroutput}
  [1000,  994,  987.98,  981.93,  975.86,  969.76,  963.64]
\end{euleroutput}
\begin{eulerprompt}
>iterate(\{\{"saldo",0.005,10\}\},2000,n=6)
\end{eulerprompt}
\begin{euleroutput}
  [2000,  1998,  1995.99,  1993.97,  1991.95,  1989.92,  1987.88]
\end{euleroutput}
\begin{eulerprompt}
>iterate(\{\{"saldo",0.005,10\}\},2500,n=6)
\end{eulerprompt}
\begin{euleroutput}
  [2500,  2500,  2500,  2500,  2500,  2500,  2500]
\end{euleroutput}
\begin{eulercomment}
Tabungan senilai 2,5 juta akan aman dan tidak akan berubah nilai (jika tidak ada penarikan),
sedangkan jika tabungan awal kurang dari 2,5 juta, lama kelamaan akan berkurang meskipun
tidak pernah dilakukan penarikan uang tabungan.
\end{eulercomment}
\begin{eulerprompt}
>iterate(\{\{"saldo",0.005,10\}\},3000,n=6)
\end{eulerprompt}
\begin{euleroutput}
  [3000,  3002,  3004.01,  3006.03,  3008.05,  3010.08,  3012.12]
\end{euleroutput}
\begin{eulercomment}
Tabungan yang lebih dari 2,5 juta baru akan bertambah jika tidak ada
penarikan.

Untuk barisan yang lebih kompleks dapat digunakan fungsi "sequence()".
Fungsi ini menghitung nilai-nilai x[n] dari semua nilai sebelumnya,
x[1],...,x[n-1] yang diketahui.\\
Berikut adalah contoh barisan Fibonacci.

\end{eulercomment}
\begin{eulerformula}
\[
x_n = x_{n-1}+x_{n-2}, \quad x_1=1, \quad x_2 =1
\]
\end{eulerformula}
\begin{eulerprompt}
>sequence("x[n-1]+x[n-2]",[1,1],15)
\end{eulerprompt}
\begin{euleroutput}
  [1,  1,  2,  3,  5,  8,  13,  21,  34,  55,  89,  144,  233,  377,  610]
\end{euleroutput}
\begin{eulercomment}
Barisan Fibonacci memiliki banyak sifat menarik, salah satunya adalah akar pangkat ke-n suku
ke-n akan konvergen ke pecahan emas:
\end{eulercomment}
\begin{eulerprompt}
>$'(1+sqrt(5))/2=float((1+sqrt(5))/2)
\end{eulerprompt}
\begin{eulerformula}
\[
\frac{\sqrt{5}+1}{2}=1.618033988749895
\]
\end{eulerformula}
\begin{eulerprompt}
>plot2d(sequence("x[n-1]+x[n-2]",[1,1],250)^(1/(1:250))):
\end{eulerprompt}
\eulerimg{29}{images/EMT4Kalkulus_Amalia Intan Arvitasari_22305144026-277.png}
\begin{eulercomment}
Barisan yang sama juga dapat dihasilkan dengan menggunakan loop.
\end{eulercomment}
\begin{eulerprompt}
>x=ones(500); for k=3 to 500; x[k]=x[k-1]+x[k-2]; end;
\end{eulerprompt}
\begin{eulercomment}
Rekursi dapat dilakukan dengan menggunakan rumus yang tergantung pada semua elemen
sebelumnya. Pada contoh berikut, elemen ke-n merupakan jumlah (n-1) elemen sebelumnya,
dimulai dengan 1 (elemen ke-1). Jelas, nilai elemen ke-n adalah 2\textasciicircum{}(n-2), untuk n=2, 4, 5,
....
\end{eulercomment}
\begin{eulerprompt}
>sequence("sum(x)",1,10)
\end{eulerprompt}
\begin{euleroutput}
  [1,  1,  2,  4,  8,  16,  32,  64,  128,  256]
\end{euleroutput}
\begin{eulercomment}
Selain menggunakan ekspresi dalam x dan n, kita juga dapat menggunakan
fungsi.

Pada contoh berikut, digunakan iterasi

\end{eulercomment}
\begin{eulerformula}
\[
x_n =A \cdot x_{n-1},
\]
\end{eulerformula}
\begin{eulercomment}
dengan A suatu matriks 2x2, dan setiap x[n] merupakan matriks/vektor
2x1.
\end{eulercomment}
\begin{eulerprompt}
>A=[1,1;1,2]; function suku(x,n) := A.x[,n-1]
>sequence("suku",[1;1],6)
\end{eulerprompt}
\begin{euleroutput}
  Real 2 x 6 matrix
  
              1             2             5            13     ...
              1             3             8            21     ...
\end{euleroutput}
\begin{eulercomment}
Hasil yang sama juga dapat diperoleh dengan menggunakan fungsi
perpangkatan matriks "matrixpower()". Cara ini lebih cepat, karena
hanya menggunakan perkalian matriks sebanyak log\_2(n).

\end{eulercomment}
\begin{eulerformula}
\[
x_n=A.x_{n-1}=A^2.x_{n-2}=A^3.x_{n-3}= ... = A^{n-1}.x_1.
\]
\end{eulerformula}
\begin{eulerprompt}
>sequence("matrixpower(A,n).[1;1]",1,6)
\end{eulerprompt}
\begin{euleroutput}
  Real 2 x 6 matrix
  
              1             5            13            34     ...
              1             8            21            55     ...
\end{euleroutput}
\eulerheading{Spiral Theodorus}
\begin{eulercomment}
image: Spiral\_of\_Theodorus.png\\
Spiral Theodorus (spiral segitiga siku-siku) dapat digambar secara
rekursif. Rumus rekursifnya adalah:

\end{eulercomment}
\begin{eulerformula}
\[
x_n = \left( 1 + \frac{i}{\sqrt{n-1}} \right) \, x_{n-1}, \quad x_1=1,
\]
\end{eulerformula}
\begin{eulercomment}
yang menghasilkan barisan bilangan kompleks.
\end{eulercomment}
\begin{eulerprompt}
>function g(n) := 1+I/sqrt(n)
\end{eulerprompt}
\begin{eulercomment}
Rekursinya dapat dijalankan sebanyak 17 untuk menghasilkan barisan 17 bilangan kompleks,
kemudian digambar bilangan-bilangan kompleksnya.
\end{eulercomment}
\begin{eulerprompt}
>x=sequence("g(n-1)*x[n-1]",1,17); plot2d(x,r=3.5); textbox(latex("Spiral\(\backslash\) Theodorus"),0.4):
\end{eulerprompt}
\eulerimg{29}{images/EMT4Kalkulus_Amalia Intan Arvitasari_22305144026-278.png}
\begin{eulercomment}
Selanjutnya dihubungan titik 0 dengan titik-titik kompleks tersebut menggunakan loop.
\end{eulercomment}
\begin{eulerprompt}
>for i=1:cols(x); plot2d([0,x[i]],>add); end:
\end{eulerprompt}
\eulerimg{29}{images/EMT4Kalkulus_Amalia Intan Arvitasari_22305144026-279.png}
\begin{eulerprompt}
> 
\end{eulerprompt}
\begin{eulercomment}
Spiral tersebut juga dapat didefinisikan menggunakan fungsi rekursif, yang tidak memmerlukan
indeks dan bilangan kompleks. Dalam hal ini diigunakan vektor kolom pada bidang.
\end{eulercomment}
\begin{eulerprompt}
>function gstep (v) ...
\end{eulerprompt}
\begin{eulerudf}
  w=[-v[2];v[1]];
  return v+w/norm(w);
  endfunction
\end{eulerudf}
\begin{eulercomment}
Jika dilakukan iterasi 16 kali dimulai dari [1;0] akan didapatkan matriks yang memuat
vektor-vektor dari setiap iterasi.
\end{eulercomment}
\begin{eulerprompt}
>x=iterate("gstep",[1;0],16); plot2d(x[1],x[2],r=3.5,>points):
\end{eulerprompt}
\eulerimg{29}{images/EMT4Kalkulus_Amalia Intan Arvitasari_22305144026-280.png}
\begin{eulercomment}
\begin{eulercomment}
\eulerheading{Kekonvergenan}
\begin{eulercomment}
Terkadang kita ingin melakukan iterasi sampai konvergen. Apabila iterasinya tidak konvergen
setelah ditunggu lama, Anda dapat menghentikannya dengan menekan tombol [ESC].
\end{eulercomment}
\begin{eulerprompt}
>iterate("cos(x)",1) // iterasi x(n+1)=cos(x(n)), dengan x(0)=1.
\end{eulerprompt}
\begin{euleroutput}
  0.739085133216
\end{euleroutput}
\begin{eulercomment}
Iterasi tersebut konvergen ke penyelesaian persamaan

\end{eulercomment}
\begin{eulerformula}
\[
x = \cos(x).
\]
\end{eulerformula}
\begin{eulercomment}
Iterasi ini juga dapat dilakukan pada interval, hasilnya adalah
barisan interval yang memuat akar tersebut.
\end{eulercomment}
\begin{eulerprompt}
>hasil := iterate("cos(x)",~1,2~) //iterasi x(n+1)=cos(x(n)), dengan interval awal (1, 2)
\end{eulerprompt}
\begin{euleroutput}
  ~0.739085133211,0.7390851332133~
\end{euleroutput}
\begin{eulercomment}
Jika interval hasil tersebut sedikit diperlebar, akan terlihat bahwa interval tersebut
memuat akar persamaan x=cos(x).
\end{eulercomment}
\begin{eulerprompt}
>h=expand(hasil,100), cos(h) << h
\end{eulerprompt}
\begin{euleroutput}
  ~0.73908513309,0.73908513333~
  1
\end{euleroutput}
\begin{eulercomment}
Iterasi juga dapat digunakan pada fungsi yang didefinisikan.
\end{eulercomment}
\begin{eulerprompt}
>function f(x) := (x+2/x)/2
\end{eulerprompt}
\begin{eulercomment}
Iterasi x(n+1)=f(x(n)) akan konvergen ke akar kuadrat 2.
\end{eulercomment}
\begin{eulerprompt}
>iterate("f",2), sqrt(2)
\end{eulerprompt}
\begin{euleroutput}
  1.41421356237
  1.41421356237
\end{euleroutput}
\begin{eulercomment}
Jika pada perintah iterate diberikan tambahan parameter n, maka hasil iterasinya akan
ditampilkan mulai dari iterasi pertama sampai ke-n.
\end{eulercomment}
\begin{eulerprompt}
>iterate("f",2,5)
\end{eulerprompt}
\begin{euleroutput}
  [2,  1.5,  1.41667,  1.41422,  1.41421,  1.41421]
\end{euleroutput}
\begin{eulercomment}
Untuk iterasi ini tidak dapat dilakukan terhadap interval.
\end{eulercomment}
\begin{eulerprompt}
>niterate("f",~1,2~,5)
\end{eulerprompt}
\begin{euleroutput}
  [ ~1,2~,  ~1,2~,  ~1,2~,  ~1,2~,  ~1,2~,  ~1,2~ ]
\end{euleroutput}
\begin{eulercomment}
Perhatikan, hasil iterasinya sama dengan interval awal. Alasannya adalah perhitungan dengan
interval bersifat terlalu longgar. Untuk meingkatkan perhitungan pada ekspresi dapat
digunakan pembagian intervalnya, menggunakan fungsi ieval().
\end{eulercomment}
\begin{eulerprompt}
>function s(x) := ieval("(x+2/x)/2",x,10)
\end{eulerprompt}
\begin{eulercomment}
Selanjutnya dapat dilakukan iterasi hingga diperoleh hasil optimal,
dan intervalnya tidak semakin mengecil. Hasilnya berupa interval yang
memuat akar persamaan:

\end{eulercomment}
\begin{eulerformula}
\[
x = \frac{1}{2} \left( x + \frac{2}{x} \right).
\]
\end{eulerformula}
\begin{eulercomment}
Satu-satunya solusi adalah\\
\end{eulercomment}
\begin{eulerformula}
\[
x = \sqrt2.
\]
\end{eulerformula}
\begin{eulerprompt}
>iterate("s",~1,2~)
\end{eulerprompt}
\begin{euleroutput}
  ~1.41421356236,1.41421356239~
\end{euleroutput}
\begin{eulercomment}
Fungsi "iterate()" juga dapat bekerja pada vektor. Berikut adalah
contoh fungsi vektor, yang menghasilkan rata-rata aritmetika dan
rata-rata geometri.

\end{eulercomment}
\begin{eulerformula}
\[
(a_{n+1},b_{n+1}) = \left( \frac{a_n+b_n}{2}, \sqrt{a_nb_n} \right)
\]
\end{eulerformula}
\begin{eulercomment}
Iterasi ke-n disimpan pada vektor kolom x[n].
\end{eulercomment}
\begin{eulerprompt}
>function g(x) := [(x[1]+x[2])/2;sqrt(x[1]*x[2])]
\end{eulerprompt}
\begin{eulercomment}
Iterasi dengan menggunakan fungsi tersebut akan konvergen ke rata-rata aritmetika dan
geometri dari nilai-nilai awal. 
\end{eulercomment}
\begin{eulerprompt}
>iterate("g",[1;5])
\end{eulerprompt}
\begin{euleroutput}
        2.60401 
        2.60401 
\end{euleroutput}
\begin{eulercomment}
Hasil tersebut konvergen agak cepat, seperti kita cek sebagai berikut.
\end{eulercomment}
\begin{eulerprompt}
>iterate("g",[1;5],4)
\end{eulerprompt}
\begin{euleroutput}
              1             3       2.61803       2.60403       2.60401 
              5       2.23607       2.59002       2.60399       2.60401 
\end{euleroutput}
\begin{eulercomment}
Iterasi pada interval dapat dilakukan dan stabil, namun tidak menunjukkan bahwa limitnya
pada batas-batas yang dihitung.
\end{eulercomment}
\begin{eulerprompt}
>iterate("g",[~1~;~5~],4)
\end{eulerprompt}
\begin{euleroutput}
  Interval 2 x 5 matrix
  
  ~0.999999999999999778,1.00000000000000022~     ...
  ~4.99999999999999911,5.00000000000000089~     ...
\end{euleroutput}
\begin{eulercomment}
Iterasi berikut konvergen sangat lambat.

\end{eulercomment}
\begin{eulerformula}
\[
x_{n+1} = \sqrt{x_n}.
\]
\end{eulerformula}
\begin{eulerprompt}
>iterate("sqrt(x)",2,10)
\end{eulerprompt}
\begin{euleroutput}
  [2,  1.41421,  1.18921,  1.09051,  1.04427,  1.0219,  1.01089,
  1.00543,  1.00271,  1.00135,  1.00068]
\end{euleroutput}
\begin{eulercomment}
Kekonvergenan iterasi tersebut dapat dipercepatdengan percepatan Steffenson:
\end{eulercomment}
\begin{eulerprompt}
>steffenson("sqrt(x)",2,10)
\end{eulerprompt}
\begin{euleroutput}
  [1.04888,  1.00028,  1,  1]
\end{euleroutput}
\eulerheading{Iterasi menggunakan Loop yang ditulis Langsung}
\begin{eulercomment}
Berikut adalah beberapa contoh penggunaan loop untuk melakukan iterasi yang ditulis langsung
pada baris perintah.
\end{eulercomment}
\begin{eulerprompt}
>x=2; repeat x=(x+2/x)/2; until x^2~=2; end; x,
\end{eulerprompt}
\begin{euleroutput}
  1.41421356237
\end{euleroutput}
\begin{eulercomment}
Penggabungan matriks menggunakan tanda "\textbar{}" dapat digunakan untuk menyimpan semua hasil
iterasi.
\end{eulercomment}
\begin{eulerprompt}
>v=[1]; for i=2 to 8; v=v|(v[i-1]*i); end; v,
\end{eulerprompt}
\begin{euleroutput}
  [1,  2,  6,  24,  120,  720,  5040,  40320]
\end{euleroutput}
\begin{eulercomment}
hasil iterasi juga dapat disimpan pada vektor yang sudah ada.
\end{eulercomment}
\begin{eulerprompt}
>v=ones(1,100); for i=2 to cols(v); v[i]=v[i-1]*i; end; ...
>plot2d(v,logplot=1); textbox(latex(&log(n)),x=0.5):
\end{eulerprompt}
\eulerimg{29}{images/EMT4Kalkulus_Amalia Intan Arvitasari_22305144026-281.png}
\begin{eulerprompt}
>A =[0.5,0.2;0.7,0.1]; b=[2;2]; ...
>x=[1;1]; repeat xnew=A.x-b; until all(xnew~=x); x=xnew; end; ...
>x,
\end{eulerprompt}
\begin{euleroutput}
       -7.09677 
       -7.74194 
\end{euleroutput}
\eulerheading{Iterasi di dalam Fungsi}
\begin{eulercomment}
Fungsi atau program juga dapat menggunakan iterasi dan dapat digunakan untuk melakukan iterasi. Berikut adalah beberapa contoh
iterasi di dalam fungsi.

Contoh berikut adalah suatu fungsi untuk menghitung berapa lama suatu iterasi konvergen. Nilai fungsi tersebut adalah hasil akhir
iterasi dan banyak iterasi sampai konvergen.
\end{eulercomment}
\begin{eulerprompt}
>function map hiter(f$,x0) ...
\end{eulerprompt}
\begin{eulerudf}
  x=x0;
  maxiter=0;
  repeat
    xnew=f$(x);
    maxiter=maxiter+1;
    until xnew~=x;
    x=xnew;
  end;
  return maxiter;
  endfunction
\end{eulerudf}
\begin{eulercomment}
Misalnya, berikut adalah iterasi untuk mendapatkan hampiran akar kuadrat 2, cukup cepat,
konvergen pada iterasi ke-5, jika dimulai dari hampiran awal 2.
\end{eulercomment}
\begin{eulerprompt}
>hiter("(x+2/x)/2",2)
\end{eulerprompt}
\begin{euleroutput}
  5
\end{euleroutput}
\begin{eulercomment}
Karena fungsinya didefinisikan menggunakan "map". maka nilai awalnya dapat berupa vektor.
\end{eulercomment}
\begin{eulerprompt}
>x=1.5:0.1:10; hasil=hiter("(x+2/x)/2",x); ...
>  plot2d(x,hasil):
\end{eulerprompt}
\eulerimg{29}{images/EMT4Kalkulus_Amalia Intan Arvitasari_22305144026-282.png}
\begin{eulercomment}
Dari gambar di atas terlihat bahwa kekonvergenan iterasinya semakin lambat, untuk nilai awal
semakin besar, namun penambahnnya tidak kontinu. Kita dapat menemukan kapan maksimum
iterasinya bertambah.
\end{eulercomment}
\begin{eulerprompt}
>hasil[1:10]
\end{eulerprompt}
\begin{euleroutput}
  [4,  5,  5,  5,  5,  5,  6,  6,  6,  6]
\end{euleroutput}
\begin{eulerprompt}
>x[nonzeros(differences(hasil))]
\end{eulerprompt}
\begin{euleroutput}
  [1.5,  2,  3.4,  6.6]
\end{euleroutput}
\begin{eulercomment}
maksimum iterasi sampai konvergen meningkat pada saat nilai awalnya 1.5, 2, 3.4, dan 6.6.

Contoh berikutnya adalah metode Newton pada polinomial kompleks berderajat 3.
\end{eulercomment}
\begin{eulerprompt}
>p &= x^3-1; newton &= x-p/diff(p,x); $newton
\end{eulerprompt}
\begin{eulerformula}
\[
x-\frac{x^3-1}{3\,x^2}
\]
\end{eulerformula}
\begin{eulercomment}
Selanjutnya didefinisikan fungsi untuk melakukan iterasi (aslinya 10 kali).
\end{eulercomment}
\begin{eulerprompt}
>function iterasi(f$,x,n=10) ...
\end{eulerprompt}
\begin{eulerudf}
  loop 1 to n; x=f$(x); end;
  return x;
  endfunction
\end{eulerudf}
\begin{eulercomment}
Kita mulai dengan menentukan titik-titik grid pada bidang kompleksnya.
\end{eulercomment}
\begin{eulerprompt}
>r=1.5; x=linspace(-r,r,501); Z=x+I*x'; W=iterasi(newton,Z);
\end{eulerprompt}
\begin{eulercomment}
Berikut adalah akar-akar polinomial di atas.
\end{eulercomment}
\begin{eulerprompt}
>z=&solve(p)()
\end{eulerprompt}
\begin{euleroutput}
  [ -0.5+0.866025i,  -0.5-0.866025i,  1+0i  ]
\end{euleroutput}
\begin{eulercomment}
Untuk menggambar hasil iterasinya, dihitung jarak dari hasil iterasi ke-10 ke masing-masing
akar, kemudian digunakan untuk menghitung warna yang akan digambar, yang menunjukkan limit
untuk masing-masing nilai awal. 

Fungsi plotrgb() menggunakan jendela gambar terkini untuk menggambar warna RGB sebagai
matriks.
\end{eulercomment}
\begin{eulerprompt}
>C=rgb(max(abs(W-z[1]),1),max(abs(W-z[2]),1),max(abs(W-z[3]),1)); ...
>  plot2d(none,-r,r,-r,r); plotrgb(C):
\end{eulerprompt}
\eulerimg{29}{images/EMT4Kalkulus_Amalia Intan Arvitasari_22305144026-284.png}
\eulerheading{Iterasi Simbolik}
\begin{eulercomment}
Seperti sudah dibahas sebelumnya, untuk menghasilkan barisan ekspresi simbolik dengan Maxima
dapat digunakan fungsi makelist().
\end{eulercomment}
\begin{eulerprompt}
>&powerdisp:true // untuk menampilkan deret pangkat mulai dari suku berpangkat terkecil
\end{eulerprompt}
\begin{euleroutput}
  
                                   true
  
\end{euleroutput}
\begin{eulerprompt}
>deret &= makelist(taylor(exp(x),x,0,k),k,1,3); $deret // barisan deret Taylor untuk e^x
\end{eulerprompt}
\begin{eulerformula}
\[
\left[ 1+x , 1+x+\frac{x^2}{2} , 1+x+\frac{x^2}{2}+\frac{x^3}{6}   \right] 
\]
\end{eulerformula}
\begin{eulercomment}
Untuk mengubah barisan deret tersebut menjadi vektor string di EMT digunakan fungsi
mxm2str(). Selanjutnya, vektor string/ekspresi hasilnya dapat digambar seperti menggambar
vektor eskpresi pada EMT.
\end{eulercomment}
\begin{eulerprompt}
>plot2d("exp(x)",0,3); // plot fungsi aslinya, e^x
>plot2d(mxm2str("deret"),>add,color=4:6): // plot ketiga deret taylor hampiran fungsi tersebut
\end{eulerprompt}
\eulerimg{29}{images/EMT4Kalkulus_Amalia Intan Arvitasari_22305144026-286.png}
\begin{eulercomment}
Selain cara di atas dapat juga dengan cara menggunakan indeks pada vektor/list yang
dihasilkan.
\end{eulercomment}
\begin{eulerprompt}
>$deret[3]
\end{eulerprompt}
\begin{eulerformula}
\[
1+x+\frac{x^2}{2}+\frac{x^3}{6}
\]
\end{eulerformula}
\begin{eulerprompt}
>plot2d(["exp(x)",&deret[1],&deret[2],&deret[3]],0,3,color=1:4):
\end{eulerprompt}
\eulerimg{29}{images/EMT4Kalkulus_Amalia Intan Arvitasari_22305144026-288.png}
\begin{eulerprompt}
>$sum(sin(k*x)/k,k,1,5)
\end{eulerprompt}
\begin{eulerformula}
\[
\sin x+\frac{\sin \left(2\,x\right)}{2}+\frac{\sin \left(3\,x  \right)}{3}+\frac{\sin \left(4\,x\right)}{4}+\frac{\sin \left(5\,x  \right)}{5}
\]
\end{eulerformula}
\begin{eulercomment}
Berikut adalah cara menggambar kurva

\end{eulercomment}
\begin{eulerformula}
\[
y=\sin(x) + \dfrac{\sin 3x}{3} + \dfrac{\sin 5x}{5} + \ldots.
\]
\end{eulerformula}
\begin{eulerprompt}
>plot2d(&sum(sin((2*k+1)*x)/(2*k+1),k,0,20),0,2pi):
\end{eulerprompt}
\eulerimg{29}{images/EMT4Kalkulus_Amalia Intan Arvitasari_22305144026-290.png}
\begin{eulercomment}
Hal serupa juga dapat dilakukan dengan menggunakan matriks, misalkan
kita akan menggambar kurva

\end{eulercomment}
\begin{eulerformula}
\[
y = \sum_{k=1}^{100} \dfrac{\sin(kx)}{k},\quad 0\le x\le 2\pi.
\]
\end{eulerformula}
\begin{eulercomment}
\end{eulercomment}
\begin{eulerprompt}
>x=linspace(0,2pi,1000); k=1:100; y=sum(sin(k*x')/k)'; plot2d(x,y):
\end{eulerprompt}
\eulerimg{29}{images/EMT4Kalkulus_Amalia Intan Arvitasari_22305144026-291.png}
\eulerheading{Tabel Fungsi}
\begin{eulercomment}
Terdapat cara menarik untuk menghasilkan barisan dengan ekspresi Maxima. Perintah
mxmtable() berguna untuk menampilkan dan menggambar barisan dan menghasilkan barisan sebagai
vektor kolom. 

Sebagai contoh berikut adalah barisan turunan ke-n x\textasciicircum{}x di x=1.
\end{eulercomment}
\begin{eulerprompt}
>mxmtable("diffat(x^x,x=1,n)","n",1,8,frac=1);
\end{eulerprompt}
\begin{euleroutput}
          1 
          2 
          3 
          8 
         10 
         54 
        -42 
        944 
\end{euleroutput}
\eulerimg{29}{images/EMT4Kalkulus_Amalia Intan Arvitasari_22305144026-292.png}
\begin{eulerprompt}
>$'sum(k, k, 1, n) = factor(ev(sum(k, k, 1, n),simpsum=true)) // simpsum:menghitung deret secara simbolik
\end{eulerprompt}
\begin{euleroutput}
  
\end{euleroutput}
\begin{eulerprompt}
>$'sum(1/(3^k+k), k, 0, inf) = factor(ev(sum(1/(3^k+k), k, 0, inf),simpsum=true))
\end{eulerprompt}
\begin{eulercomment}
Di sini masih gagal, hasilnya tidak dihitung.
\end{eulercomment}
\begin{eulerprompt}
>$'sum(1/x^2, x, 1, inf)= ev(sum(1/x^2, x, 1, inf),simpsum=true) // ev: menghitung nilai ekspresi
>$'sum((-1)^(k-1)/k, k, 1, inf) = factor(ev(sum((-1)^(x-1)/x, x, 1, inf),simpsum=true))
\end{eulerprompt}
\begin{eulercomment}
Di sini masih gagal, hasilnya tidak dihitung.
\end{eulercomment}
\begin{eulerprompt}
>$'sum((-1)^k/(2*k-1), k, 1, inf) = factor(ev(sum((-1)^k/(2*k-1), k, 1, inf),simpsum=true))
>$ev(sum(1/n!, n, 0, inf),simpsum=true)
\end{eulerprompt}
\begin{eulercomment}
Di sini masih gagal, hasilnya tidak dihitung, harusnya hasilnya e.
\end{eulercomment}
\begin{eulerprompt}
>&assume(abs(x)<1); $'sum(a*x^k, k, 0, inf)=ev(sum(a*x^k, k, 0, inf),simpsum=true), &forget(abs(x)<1);
\end{eulerprompt}
\begin{eulercomment}
Deret geometri tak hingga, dengan asumsi rasional antara -1 dan 1.
\end{eulercomment}
\begin{eulerprompt}
>$'sum(x^k/k!,k,0,inf)=ev(sum(x^k/k!,k,0,inf),simpsum=true)
>$limit(sum(x^k/k!,k,0,n),n,inf)
>function d(n) &= sum(1/(k^2-k),k,2,n); $'d(n)=d(n)
>$d(10)=ev(d(10),simpsum=true)
>$d(100)=ev(d(100),simpsum=true)
\end{eulerprompt}
\eulerheading{Deret Taylor}
\begin{eulercomment}
Deret Taylor suatu fungsi f yang diferensiabel sampai tak hingga di
sekitar x=a adalah:

\end{eulercomment}
\begin{eulerformula}
\[
f(x) = \sum_{k=0}^\infty \frac{(x-a)^k f^{(k)}(a)}{k!}.
\]
\end{eulerformula}
\begin{eulerprompt}
>$'e^x =taylor(exp(x),x,0,10) // deret Taylor e^x di sekitar x=0, sampai suku ke-11
>$'log(x)=taylor(log(x),x,1,10)// deret log(x) di sekitar x=1
\end{eulerprompt}
\end{eulernotebook}

\chapter{EMT Untuk Geometri}
\begin{eulercomment}
\eulerheading{Visualisasi dan Perhitungan Geometri dengan EMT}
\begin{eulercomment}
Euler menyediakan beberapa fungsi untuk melakukan visualisasi dan
perhitungan geometri, baik secara numerik maupun analitik (seperti
biasanya tentunya, menggunakan Maxima). Fungsi-fungsi untuk
visualisasi dan perhitungan geometeri tersebut disimpan di dalam file
program "geometry.e", sehingga file tersebut harus dipanggil sebelum
menggunakan fungsi-fungsi atau perintah-perintah untuk geometri.
\end{eulercomment}
\begin{eulerprompt}
>load geometry
\end{eulerprompt}
\begin{euleroutput}
  Numerical and symbolic geometry.
\end{euleroutput}
\eulersubheading{Fungsi-fungsi Geometri}
\begin{eulercomment}
Fungsi-fungsi untuk Menggambar Objek Geometri:

\end{eulercomment}
\begin{eulerttcomment}
  defaultd:=textheight()*1.5: nilai asli untuk parameter d
  setPlotrange(x1,x2,y1,y2): menentukan rentang x dan y pada bidang
\end{eulerttcomment}
\begin{eulercomment}
koordinat\\
\end{eulercomment}
\begin{eulerttcomment}
  setPlotRange(r): pusat bidang koordinat (0,0) dan batas-batas
\end{eulerttcomment}
\begin{eulercomment}
sumbu-x dan y adalah -r sd r\\
\end{eulercomment}
\begin{eulerttcomment}
  plotPoint (P, "P"): menggambar titik P dan diberi label "P"
  plotSegment (A,B, "AB", d): menggambar ruas garis AB, diberi label
\end{eulerttcomment}
\begin{eulercomment}
"AB" sejauh d\\
\end{eulercomment}
\begin{eulerttcomment}
  plotLine (g, "g", d): menggambar garis g diberi label "g" sejauh d
  plotCircle (c,"c",v,d): Menggambar lingkaran c dan diberi label "c"
  plotLabel (label, P, V, d): menuliskan label pada posisi P
\end{eulerttcomment}
\begin{eulercomment}

Fungsi-fungsi Geometri Analitik (numerik maupun simbolik):

\end{eulercomment}
\begin{eulerttcomment}
  turn(v, phi): memutar vektor v sejauh phi
  turnLeft(v):   memutar vektor v ke kiri
  turnRight(v):  memutar vektor v ke kanan
  normalize(v): normal vektor v
  crossProduct(v, w): hasil kali silang vektorv dan w.
  lineThrough(A, B): garis melalui A dan B, hasilnya [a,b,c] sdh.
\end{eulerttcomment}
\begin{eulercomment}
ax+by=c.\\
\end{eulercomment}
\begin{eulerttcomment}
  lineWithDirection(A,v): garis melalui A searah vektor v
  getLineDirection(g): vektor arah (gradien) garis g
  getNormal(g): vektor normal (tegak lurus) garis g
  getPointOnLine(g):  titik pada garis g
  perpendicular(A, g):  garis melalui A tegak lurus garis g
  parallel (A, g):  garis melalui A sejajar garis g
  lineIntersection(g, h):  titik potong garis g dan h
  projectToLine(A, g):   proyeksi titik A pada garis g
  distance(A, B):  jarak titik A dan B
  distanceSquared(A, B):  kuadrat jarak A dan B
  quadrance(A, B): kuadrat jarak A dan B
  areaTriangle(A, B, C):  luas segitiga ABC
  computeAngle(A, B, C):   besar sudut <ABC
  angleBisector(A, B, C): garis bagi sudut <ABC
  circleWithCenter (A, r): lingkaran dengan pusat A dan jari-jari r
  getCircleCenter(c):  pusat lingkaran c
  getCircleRadius(c):  jari-jari lingkaran c
  circleThrough(A,B,C):  lingkaran melalui A, B, C
  middlePerpendicular(A, B): titik tengah AB
  lineCircleIntersections(g, c): titik potong garis g dan lingkran c
  circleCircleIntersections (c1, c2):  titik potong lingkaran c1 dan
\end{eulerttcomment}
\begin{eulercomment}
c2\\
\end{eulercomment}
\begin{eulerttcomment}
  planeThrough(A, B, C):  bidang melalui titik A, B, C
\end{eulerttcomment}
\begin{eulercomment}

Fungsi-fungsi Khusus Untuk Geometri Simbolik:

\end{eulercomment}
\begin{eulerttcomment}
  getLineEquation (g,x,y): persamaan garis g dinyatakan dalam x dan y
  getHesseForm (g,x,y,A): bentuk Hesse garis g dinyatakan dalam x dan
\end{eulerttcomment}
\begin{eulercomment}
y dengan titik A pada\\
\end{eulercomment}
\begin{eulerttcomment}
  sisi positif (kanan/atas) garis
  quad(A,B): kuadrat jarak AB
  spread(a,b,c): Spread segitiga dengan panjang sisi-sisi a,b,c, yakni
\end{eulerttcomment}
\begin{eulercomment}
sin(alpha)\textasciicircum{}2 dengan\\
\end{eulercomment}
\begin{eulerttcomment}
  alpha sudut yang menghadap sisi a.
  crosslaw(a,b,c,sa): persamaan 3 quads dan 1 spread pada segitiga
\end{eulerttcomment}
\begin{eulercomment}
dengan panjang sisi a, b, c.\\
\end{eulercomment}
\begin{eulerttcomment}
  triplespread(sa,sb,sc): persamaan 3 spread sa,sb,sc yang memebntuk
\end{eulerttcomment}
\begin{eulercomment}
suatu segitiga\\
\end{eulercomment}
\begin{eulerttcomment}
  doublespread(sa): Spread sudut rangkap Spread 2*phi, dengan
\end{eulerttcomment}
\begin{eulercomment}
sa=sin(phi)\textasciicircum{}2 spread a.

\end{eulercomment}
\eulersubheading{Contoh 1: Luas, Lingkaran Luar, Lingkaran Dalam Segitiga}
\begin{eulercomment}
Untuk menggambar objek-objek geometri, langkah pertama adalah
menentukan rentang sumbu-sumbu koordinat. Semua objek geometri akan
digambar pada satu bidang koordinat, sampai didefinisikan bidang
koordinat yang baru.
\end{eulercomment}
\begin{eulerprompt}
>setPlotRange(-0.5,2.5,-0.5,2.5); // mendefinisikan bidang koordinat baru 
\end{eulerprompt}
\begin{eulercomment}
Sekarang, tetapkan 3 titik dan plotkan.
\end{eulercomment}
\begin{eulerprompt}
>A=[1,0]; plotPoint(A,"A"); // definisi dan gambar tiga titik
>B=[0,1]; plotPoint(B,"B");
>C=[2,2]; plotPoint(C,"C");
\end{eulerprompt}
\begin{eulercomment}
Kemudian tiga segmen.
\end{eulercomment}
\begin{eulerprompt}
>plotSegment(A,B,"c"); // c=AB
>plotSegment(B,C,"a"); // a=BC
>plotSegment(A,C,"b"); // b=AC
\end{eulerprompt}
\begin{eulercomment}
Fungsi geometri mencakup fungsi untuk membuat garis dan lingkaran.
Format untuk garis adalah [a,b,c], yang merepresentasikan garis dengan
persamaan ax+by=c.
\end{eulercomment}
\begin{eulerprompt}
>lineThrough(B,C) // garis yang melalui B dan C
\end{eulerprompt}
\begin{euleroutput}
  [-1,  2,  2]
\end{euleroutput}
\begin{eulercomment}
Hitung garis tegak lurus yang melalui A pada BC.
\end{eulercomment}
\begin{eulerprompt}
>h=perpendicular(A,lineThrough(B,C)); // garis h tegak lurus BC melalui A
\end{eulerprompt}
\begin{eulercomment}
Dan perpotongannya dengan BC.
\end{eulercomment}
\begin{eulerprompt}
>D=lineIntersection(h,lineThrough(B,C)); // D adalah titik potong h dan BC
\end{eulerprompt}
\begin{eulercomment}
Plotkan.
\end{eulercomment}
\begin{eulerprompt}
>plotPoint(D,value=1); // koordinat D ditampilkan
>aspect(1); plotSegment(A,D): // tampilkan semua gambar hasil plot...()
\end{eulerprompt}
\eulerimg{29}{images/EMT4Geometry_Amalia Intan Arvitasari_22305144026_Mat B-001.png}
\begin{eulercomment}
Hitung luas ABC:

\end{eulercomment}
\begin{eulerformula}
\[
L_{\triangle ABC}= \frac{1}{2}AD.BC.
\]
\end{eulerformula}
\begin{eulerprompt}
>norm(A-D)*norm(B-C)/2 // AD=norm(A-D), BC=norm(B-C)
\end{eulerprompt}
\begin{euleroutput}
  1.5
\end{euleroutput}
\begin{eulercomment}
Bandingkan dengan rumus determinan.
\end{eulercomment}
\begin{eulerprompt}
>areaTriangle(A,B,C) // hitung luas segitiga langsung dengan fungsi
\end{eulerprompt}
\begin{euleroutput}
  1.5
\end{euleroutput}
\begin{eulercomment}
Cara lain menghitung luas segitigas ABC:
\end{eulercomment}
\begin{eulerprompt}
>distance(A,D)*distance(B,C)/2
\end{eulerprompt}
\begin{euleroutput}
  1.5
\end{euleroutput}
\begin{eulercomment}
Sudut pada C.
\end{eulercomment}
\begin{eulerprompt}
>degprint(computeAngle(B,C,A))
\end{eulerprompt}
\begin{euleroutput}
  36°52'11.63''
\end{euleroutput}
\begin{eulercomment}
Sekarang, lingkarilah segitiga tersebut.
\end{eulercomment}
\begin{eulerprompt}
>c=circleThrough(A,B,C); // lingkaran luar segitiga ABC
>R=getCircleRadius(c); // jari2 lingkaran luar 
>O=getCircleCenter(c); // titik pusat lingkaran c 
>plotPoint(O,"O"); // gambar titik "O"
>plotCircle(c,"Lingkaran luar segitiga ABC"):
\end{eulerprompt}
\eulerimg{29}{images/EMT4Geometry_Amalia Intan Arvitasari_22305144026_Mat B-003.png}
\begin{eulercomment}
Tampilkan koordinat titik pusat dan jari-jari lingkaran luar.
\end{eulercomment}
\begin{eulerprompt}
>O, R
\end{eulerprompt}
\begin{euleroutput}
  [1.16667,  1.16667]
  1.17851130198
\end{euleroutput}
\begin{eulercomment}
Sekarang akan digambar lingkaran dalam segitiga ABC. Titik pusat lingkaran dalam adalah
titik potong garis-garis bagi sudut.
\end{eulercomment}
\begin{eulerprompt}
>l=angleBisector(A,C,B); // garis bagi <ACB
>g=angleBisector(C,A,B); // garis bagi <CAB
>P=lineIntersection(l,g) // titik potong kedua garis bagi sudut
\end{eulerprompt}
\begin{euleroutput}
  [0.86038,  0.86038]
\end{euleroutput}
\begin{eulercomment}
Tambahkan semuanya ke plot.
\end{eulercomment}
\begin{eulerprompt}
>color(5); plotLine(l); plotLine(g); color(1); // gambar kedua garis bagi sudut
>plotPoint(P,"P"); // gambar titik potongnya
>r=norm(P-projectToLine(P,lineThrough(A,B))) // jari-jari lingkaran dalam
\end{eulerprompt}
\begin{euleroutput}
  0.509653732104
\end{euleroutput}
\begin{eulerprompt}
>plotCircle(circleWithCenter(P,r),"Lingkaran dalam segitiga ABC"): // gambar lingkaran dalam
\end{eulerprompt}
\eulerimg{29}{images/EMT4Geometry_Amalia Intan Arvitasari_22305144026_Mat B-004.png}
\eulersubheading{Latihan}
\begin{eulercomment}
1. Tentukan ketiga titik singgung lingkaran dalam dengan sisi-sisi
segitiga ABC.\\
2. Gambar segitiga dengan titik-titik sudut ketiga titik singgung
tersebut. Merupakan segitiga apakah itu?\\
3. Hitung luas segitiga tersebut.\\
4. Tunjukkan bahwa garis bagi sudut yang ke tiga juga melalui titik
pusat lingkaran dalam.\\
5. Gambar jari-jari lingkaran dalam.\\
6. Hitung luas lingkaran luar dan luas lingkaran dalam segitiga ABC.
Adakah hubungan antara luas kedua lingkaran tersebut dengan luas
segitiga ABC?\\
\end{eulercomment}
\eulersubheading{}
\begin{eulercomment}
Jawab\\
1.
\end{eulercomment}
\begin{eulerprompt}
>setPlotRange(-0.5,7,-0.5,7);
>O=[6,1]; plotPoint(O,"O");
>P=[0,5]; plotPoint(P,"P");
>Q=[5,6]; plotPoint(Q,"Q");
\end{eulerprompt}
\begin{eulercomment}
2. 
\end{eulercomment}
\begin{eulerprompt}
>plotSegment(O,P,"q");
>plotSegment(P,Q,"o");
>plotSegment(Q,O,"p");
>lineThrough(O,P)
\end{eulerprompt}
\begin{euleroutput}
  [-4,  -6,  -30]
\end{euleroutput}
\begin{eulerprompt}
>h=perpendicular(Q,lineThrough(O,P));
\end{eulerprompt}
\begin{eulercomment}
merupakan segitiga sama kaki
\end{eulercomment}
\begin{eulerprompt}
>D=lineIntersection(h,lineThrough(O,P));
>plotPoint(D,value=1);
>aspect(1); plotSegment(Q,D):
\end{eulerprompt}
\eulerimg{29}{images/EMT4Geometry_Amalia Intan Arvitasari_22305144026_Mat B-005.png}
\begin{eulercomment}
3. Akan ditentukan luas segitiga di atas\\
\end{eulercomment}
\begin{eulerformula}
\[
L {\triangle ABC}= \frac{1}{2}OP.QD.
\]
\end{eulerformula}
\begin{eulerprompt}
>norm(O-P)*norm(Q-D)/2
\end{eulerprompt}
\begin{euleroutput}
  13
\end{euleroutput}
\begin{eulercomment}
Jadi luas segitiga diatas adalah 13


4.
\end{eulercomment}
\begin{eulerprompt}
>c=circleThrough(O,P,Q);
>L=getCircleRadius(c);
>K=getCircleCenter(c);
>plotPoint(K,"k");
>plotCircle(c,"lingkaran luar segitiga OPQ"):
\end{eulerprompt}
\eulerimg{29}{images/EMT4Geometry_Amalia Intan Arvitasari_22305144026_Mat B-007.png}
\begin{eulerprompt}
>K,L
\end{eulerprompt}
\begin{euleroutput}
  [3,  3]
  Space between commands expected!
  Found: L (character 76)
  You can disable this in the Options menu.
  Error in:
  K,L ...
    ^
\end{euleroutput}
\begin{eulercomment}
Jari-jari lingkaran luar segitiga adalah 3
\end{eulercomment}
\begin{eulerprompt}
>l=angleBisector(O,Q,P);
>g=angleBisector(Q,O,P);
>z=lineIntersection(l,g)
\end{eulerprompt}
\begin{euleroutput}
  [3.82843,  4.24264]
\end{euleroutput}
\begin{eulerprompt}
>color(5); plotLine(l); plotLine(g); color(1);
\end{eulerprompt}
\begin{eulercomment}
5.
\end{eulercomment}
\begin{eulerprompt}
>plotPoint(z,"z");
\end{eulerprompt}
\begin{eulercomment}
Jari-jari lingkaran dalam segitiga adalah 1.49
\end{eulercomment}
\begin{eulerprompt}
>r=norm(z-projectToLine(z,lineThrough(O,P)))
\end{eulerprompt}
\begin{euleroutput}
  1.49346823813
\end{euleroutput}
\begin{eulerprompt}
>plotCircle(circleWithCenter(z,r),"lingkaran dalam segitiga OPQ"):
\end{eulerprompt}
\eulerimg{29}{images/EMT4Geometry_Amalia Intan Arvitasari_22305144026_Mat B-008.png}
\begin{eulercomment}
6.

Rumus luas lingkaran\\
\end{eulercomment}
\begin{eulerformula}
\[
\pi r^2
\]
\end{eulerformula}
\begin{eulercomment}
Luas lingkaran luar\\
\end{eulercomment}
\begin{eulerformula}
\[
\pi r^2, r=3
\]
\end{eulerformula}
\begin{eulercomment}
sehingga
\end{eulercomment}
\begin{eulerprompt}
>pi
\end{eulerprompt}
\begin{euleroutput}
  3.14159265359
\end{euleroutput}
\begin{eulerprompt}
>r:=3
\end{eulerprompt}
\begin{euleroutput}
  3
\end{euleroutput}
\begin{eulerprompt}
>L=pi*r^2
\end{eulerprompt}
\begin{euleroutput}
  28.2743338823
\end{euleroutput}
\begin{eulercomment}
Luas lingkaran dalam\\
\end{eulercomment}
\begin{eulerformula}
\[
\pi r^2, r=1.49346823813
\]
\end{eulerformula}
\begin{eulercomment}
sehingga
\end{eulercomment}
\begin{eulerprompt}
>pi
\end{eulerprompt}
\begin{euleroutput}
  3.14159265359
\end{euleroutput}
\begin{eulerprompt}
>r=1.49346823813
\end{eulerprompt}
\begin{euleroutput}
  1.49346823813
\end{euleroutput}
\begin{eulerprompt}
>L2=pi*r^2
\end{eulerprompt}
\begin{euleroutput}
  7.0071570979
\end{euleroutput}
\begin{eulercomment}
hubungan:\\
lingkaran dalam dan luas segitiga\\
jari-jari lingkaran dalam = luas segitiga dibagi setengah keliling
segitiga\\
\end{eulercomment}
\begin{eulerformula}
\[
\frac{L\triangle OPQ}{\frac{1}{2}K\triangle OPQ} = r
\]
\end{eulerformula}
\begin{eulercomment}
\begin{eulercomment}
\eulerheading{Contoh 2: Geometri Smbolik}
\begin{eulercomment}
Kita dapat menghitung geometri eksak dan simbolik menggunakan Maxima.

File geometri.e menyediakan fungsi yang sama (dan lebih banyak lagi)
di Maxima. Namun, kita dapat menggunakan komputasi simbolik sekarang.
\end{eulercomment}
\begin{eulerprompt}
>A &= [1,0]; B &= [0,1]; C &= [2,2]; // menentukan tiga titik A, B, C
\end{eulerprompt}
\begin{eulercomment}
Fungsi untuk garis dan lingkaran bekerja seperti fungsi Euler, tetapi
menyediakan komputasi simbolis.
\end{eulercomment}
\begin{eulerprompt}
>c &= lineThrough(B,C) // c=BC
\end{eulerprompt}
\begin{euleroutput}
  
                               [- 1, 2, 2]
  
\end{euleroutput}
\begin{eulercomment}
Kita bisa mendapatkan persamaan garis dengan mudah.
\end{eulercomment}
\begin{eulerprompt}
>$getLineEquation(c,x,y), $solve(%,y) | expand // persamaan garis c
\end{eulerprompt}
\begin{eulerformula}
\[
2\,y-x=2
\]
\end{eulerformula}
\begin{eulerformula}
\[
\left[ y=\frac{x}{2}+1 \right] 
\]
\end{eulerformula}
\begin{eulerprompt}
>$getLineEquation(lineThrough([x1,y1],[x2,y2]),x,y), $solve(%,y) // persamaan garis melalui(x1, y1) dan (x2, y2)
\end{eulerprompt}
\begin{eulerformula}
\[
x\,\left({\it y_1}-{\it y_2}\right)+\left({\it x_2}-{\it x_1}
 \right)\,y={\it x_1}\,\left({\it y_1}-{\it y_2}\right)+\left(
 {\it x_2}-{\it x_1}\right)\,{\it y_1}
\]
\end{eulerformula}
\begin{eulerformula}
\[
\left[ y=\frac{-\left({\it x_1}-x\right)\,{\it y_2}-\left(x-
 {\it x_2}\right)\,{\it y_1}}{{\it x_2}-{\it x_1}} \right] 
\]
\end{eulerformula}
\begin{eulerprompt}
>$getLineEquation(lineThrough(A,[x1,y1]),x,y) // persamaan garis melalui A dan (x1, y1)
\end{eulerprompt}
\begin{eulerformula}
\[
\left({\it x_1}-1\right)\,y-x\,{\it y_1}=-{\it y_1}
\]
\end{eulerformula}
\begin{eulerprompt}
>h &= perpendicular(A,lineThrough(B,C)) // h melalui A tegak lurus BC
\end{eulerprompt}
\begin{euleroutput}
  
                                [2, 1, 2]
  
\end{euleroutput}
\begin{eulerprompt}
>Q &= lineIntersection(c,h) // Q titik potong garis c=BC dan h
\end{eulerprompt}
\begin{euleroutput}
  
                                   2  6
                                  [-, -]
                                   5  5
  
\end{euleroutput}
\begin{eulerprompt}
>$projectToLine(A,lineThrough(B,C)) // proyeksi A pada BC
\end{eulerprompt}
\begin{eulerformula}
\[
\left[ \frac{2}{5} , \frac{6}{5} \right] 
\]
\end{eulerformula}
\begin{eulerprompt}
>$distance(A,Q) // jarak AQ
\end{eulerprompt}
\begin{eulerformula}
\[
\frac{3}{\sqrt{5}}
\]
\end{eulerformula}
\begin{eulerprompt}
>cc &= circleThrough(A,B,C); $cc // (titik pusat dan jari-jari) lingkaran melalui A, B, C
\end{eulerprompt}
\begin{eulerformula}
\[
\left[ \frac{7}{6} , \frac{7}{6} , \frac{5}{3\,\sqrt{2}} \right] 
\]
\end{eulerformula}
\begin{eulerprompt}
>r&=getCircleRadius(cc); $r , $float(r) // tampilkan nilai jari-jari
\end{eulerprompt}
\begin{eulerformula}
\[
\frac{5}{3\,\sqrt{2}}
\]
\end{eulerformula}
\begin{eulerformula}
\[
1.178511301977579
\]
\end{eulerformula}
\begin{eulerprompt}
>$computeAngle(A,C,B) // nilai <ACB
\end{eulerprompt}
\begin{eulerformula}
\[
\arccos \left(\frac{4}{5}\right)
\]
\end{eulerformula}
\begin{eulerprompt}
>$solve(getLineEquation(angleBisector(A,C,B),x,y),y)[1] // persamaan garis bagi <ACB
\end{eulerprompt}
\begin{eulerformula}
\[
y=x
\]
\end{eulerformula}
\begin{eulerprompt}
>P &= lineIntersection(angleBisector(A,C,B),angleBisector(C,B,A)); $P // titik potong 2 garis bagi sudut
\end{eulerprompt}
\begin{eulerformula}
\[
\left[ \frac{\sqrt{2}\,\sqrt{5}+2}{6} , \frac{\sqrt{2}\,\sqrt{5}+2
 }{6} \right] 
\]
\end{eulerformula}
\begin{eulerprompt}
>P() // hasilnya sama dengan perhitungan sebelumnya
\end{eulerprompt}
\begin{euleroutput}
  [0.86038,  0.86038]
\end{euleroutput}
\eulersubheading{Perpotongan Garis dan Lingkaran}
\begin{eulercomment}
Tentu, kita juga dapat memotong garis dengan lingkaran dan lingkaran
dengan lingkaran.
\end{eulercomment}
\begin{eulerprompt}
>A &:= [1,0]; c=circleWithCenter(A,4);
>B &:= [1,2]; C &:= [2,1]; l=lineThrough(B,C);
>setPlotRange(5); plotCircle(c); plotLine(l);
\end{eulerprompt}
\begin{eulercomment}
Perpotongan garis dengan lingkaran menghasilkan dua titik dan nilai
titik perpotongannya.
\end{eulercomment}
\begin{eulerprompt}
>\{P1,P2,f\}=lineCircleIntersections(l,c);
>P1, P2, f
\end{eulerprompt}
\begin{euleroutput}
  [4.64575,  -1.64575]
  [-0.645751,  3.64575]
  2
\end{euleroutput}
\begin{eulerprompt}
>plotPoint(P1); plotPoint(P2):
\end{eulerprompt}
\eulerimg{29}{images/EMT4Geometry_Amalia Intan Arvitasari_22305144026_Mat B-026.png}
\begin{eulercomment}
Hal yang sama pada Maxima.
\end{eulercomment}
\begin{eulerprompt}
>c &= circleWithCenter(A,4) // lingkaran dengan pusat A jari-jari 4
\end{eulerprompt}
\begin{euleroutput}
  
                                [1, 0, 4]
  
\end{euleroutput}
\begin{eulerprompt}
>l &= lineThrough(B,C) // garis l melalui B dan C
\end{eulerprompt}
\begin{euleroutput}
  
                                [1, 1, 3]
  
\end{euleroutput}
\begin{eulerprompt}
>$lineCircleIntersections(l,c) | radcan, // titik potong lingkaran c dan garis l
\end{eulerprompt}
\begin{eulerformula}
\[
\left[ \left[ \sqrt{7}+2 , 1-\sqrt{7} \right]  , \left[ 2-\sqrt{7}
  , \sqrt{7}+1 \right]  \right] 
\]
\end{eulerformula}
\begin{eulercomment}
Akan ditunjukkan bahwa sudut-sudut yang menghadap busur yang sama
adalah sama besar.
\end{eulercomment}
\begin{eulerprompt}
>C=A+normalize([-2,-3])*4; plotPoint(C); plotSegment(P1,C); plotSegment(P2,C);
>degprint(computeAngle(P1,C,P2))
\end{eulerprompt}
\begin{euleroutput}
  69°17'42.68''
\end{euleroutput}
\begin{eulerprompt}
>C=A+normalize([-4,-3])*4; plotPoint(C); plotSegment(P1,C); plotSegment(P2,C);
>degprint(computeAngle(P1,C,P2))
\end{eulerprompt}
\begin{euleroutput}
  69°17'42.68''
\end{euleroutput}
\begin{eulerprompt}
>insimg;
\end{eulerprompt}
\eulerimg{29}{images/EMT4Geometry_Amalia Intan Arvitasari_22305144026_Mat B-028.png}
\eulersubheading{Garis Sumbu}
\begin{eulercomment}
Berikut adalah langkah-langkah menggambar garis sumbu ruas garis AB:

1. Gambar lingkaran dengan pusat A melalui B.\\
2. Gambar lingkaran dengan pusat B melalui A.\\
3. Tarik garis melallui kedua titik potong kedua lingkaran tersebut. Garis ini merupakan
garis sumbu (melalui titik tengah dan tegak lurus) AB.
\end{eulercomment}
\begin{eulerprompt}
>A=[2,2]; B=[-1,-2];
>c1=circleWithCenter(A,distance(A,B));
>c2=circleWithCenter(B,distance(A,B));
>\{P1,P2,f\}=circleCircleIntersections(c1,c2);
>l=lineThrough(P1,P2);
>setPlotRange(5); plotCircle(c1); plotCircle(c2);
>plotPoint(A); plotPoint(B); plotSegment(A,B); plotLine(l):
\end{eulerprompt}
\eulerimg{29}{images/EMT4Geometry_Amalia Intan Arvitasari_22305144026_Mat B-029.png}
\begin{eulercomment}
Selanjutnya, kita lakukan hal yang sama di Maxima dengan koordinat
umum.
\end{eulercomment}
\begin{eulerprompt}
>A &= [a1,a2]; B &= [b1,b2];
>c1 &= circleWithCenter(A,distance(A,B));
>c2 &= circleWithCenter(B,distance(A,B));
>P &= circleCircleIntersections(c1,c2); P1 &= P[1]; P2 &= P[2];
\end{eulerprompt}
\begin{eulercomment}
Persamaan untuk perpotongan cukup rumit. Tetapi kita dapat
menyederhanakannya, jika kita menyelesaikan untuk y.
\end{eulercomment}
\begin{eulerprompt}
>g &= getLineEquation(lineThrough(P1,P2),x,y);
>$solve(g,y)
\end{eulerprompt}
\begin{eulerformula}
\[
\left[ y=\frac{-\left(2\,{\it b_1}-2\,{\it a_1}\right)\,x+{\it b_2}
 ^2+{\it b_1}^2-{\it a_2}^2-{\it a_1}^2}{2\,{\it b_2}-2\,{\it a_2}}
  \right] 
\]
\end{eulerformula}
\begin{eulercomment}
Ini memang sama dengan tegak lurus tengah, yang dihitung dengan cara
yang sangat berbeda.
\end{eulercomment}
\begin{eulerprompt}
>$solve(getLineEquation(middlePerpendicular(A,B),x,y),y)
\end{eulerprompt}
\begin{eulerformula}
\[
\left[ y=\frac{-\left(2\,{\it b_1}-2\,{\it a_1}\right)\,x+{\it b_2}
 ^2+{\it b_1}^2-{\it a_2}^2-{\it a_1}^2}{2\,{\it b_2}-2\,{\it a_2}}
  \right] 
\]
\end{eulerformula}
\begin{eulerprompt}
>h &=getLineEquation(lineThrough(A,B),x,y);
>$solve(h,y)
\end{eulerprompt}
\begin{eulerformula}
\[
\left[ y=\frac{\left({\it b_2}-{\it a_2}\right)\,x-{\it a_1}\,
 {\it b_2}+{\it a_2}\,{\it b_1}}{{\it b_1}-{\it a_1}} \right] 
\]
\end{eulerformula}
\begin{eulercomment}
Perhatikan hasil kali gradien garis g dan h adalah:

\end{eulercomment}
\begin{eulerformula}
\[
\frac{-(b_1-a_1)}{(b_2-a_2)}\times \frac{(b_2-a_2)}{(b_1-a_1)} = -1.
\]
\end{eulerformula}
\begin{eulercomment}
Artinya kedua garis tegak lurus.
\end{eulercomment}
\eulerheading{Contoh 3: Rumus Heron}
\begin{eulercomment}
Rumus Heron menyatakan bahwa luas segitiga dengan panjang sisi-sisi a,
b dan c adalah:

\end{eulercomment}
\begin{eulerformula}
\[
L = \sqrt{s(s-a)(s-b)(s-c)}\quad \text{ dengan } s=(a+b+c)/2,
\]
\end{eulerformula}
\begin{eulercomment}
atau bisa ditulis dalam bentuk lain:

\end{eulercomment}
\begin{eulerformula}
\[
L = \frac{1}{4}\sqrt{(a+b+c)(b+c-a)(a+c-b)(a+b-c)}
\]
\end{eulerformula}
\begin{eulercomment}
Untuk membuktikan hal ini kita misalkan C(0,0), B(a,0) dan A(x,y),
b=AC, c=AB. Luas segitiga ABC adalah

\end{eulercomment}
\begin{eulerformula}
\[
L_{\triangle ABC}=\frac{1}{2}a\times y.
\]
\end{eulerformula}
\begin{eulercomment}
Nilai y didapat dengan menyelesaikan sistem persamaan:

\end{eulercomment}
\begin{eulerformula}
\[
x^2+y^2=b^2, \quad (x-a)^2+y^2=c^2.
\]
\end{eulerformula}
\begin{eulerprompt}
>setPlotRange(-1,10,-1,8); plotPoint([0,0], "C(0,0)"); plotPoint([5.5,0], "B(a,0)");  ...
> plotPoint([7.5,6], "A(x,y)");
>plotSegment([0,0],[5.5,0], "a",25); plotSegment([5.5,0],[7.5,6],"c",15);  ...
>plotSegment([0,0],[7.5,6],"b",25); 
>plotSegment([7.5,6],[7.5,0],"t=y",25):
\end{eulerprompt}
\eulerimg{29}{images/EMT4Geometry_Amalia Intan Arvitasari_22305144026_Mat B-038.png}
\begin{eulerprompt}
>remvalue x,y,a,b,c
>&assume(a>0); sol &= solve([x^2+y^2=b^2,(x-a)^2+y^2=c^2],[x,y])
\end{eulerprompt}
\begin{euleroutput}
  
                   2    2    2
                - c  + b  + a
          [[x = --------------, y = 
                     2 a
            4      2  2      2  2    4      2  2    4
    sqrt(- c  + 2 b  c  + 2 a  c  - b  + 2 a  b  - a )
  - --------------------------------------------------], 
                           2 a
          2    2    2
       - c  + b  + a
  [x = --------------, y = 
            2 a
          4      2  2      2  2    4      2  2    4
  sqrt(- c  + 2 b  c  + 2 a  c  - b  + 2 a  b  - a )
  --------------------------------------------------]]
                         2 a
  
\end{euleroutput}
\begin{eulercomment}
Ekstrak solusi y.
\end{eulercomment}
\begin{eulerprompt}
>ysol &= y with sol[2][2]; $'y=sqrt(factor(ysol^2))
\end{eulerprompt}
\begin{eulerformula}
\[
y=\frac{\sqrt{\left(-c+b+a\right)\,\left(c-b+a\right)\,\left(c+b-a
 \right)\,\left(c+b+a\right)}}{2\,a}
\]
\end{eulerformula}
\begin{eulercomment}
Kita dapatkan rumus Heron.
\end{eulercomment}
\begin{eulerprompt}
>function H(a,b,c) &= sqrt(factor((ysol*a/2)^2)); $'H(a,b,c)=H(a,b,c)
\end{eulerprompt}
\begin{eulerformula}
\[
H\left(a , b , c\right)=\frac{\sqrt{\left(-c+b+a\right)\,\left(c-b+
 a\right)\,\left(c+b-a\right)\,\left(c+b+a\right)}}{4}
\]
\end{eulerformula}
\begin{eulerprompt}
>$'Luas=H(2,5,6) // luas segitiga dengan panjang sisi-sisi 2, 5, 6
\end{eulerprompt}
\begin{eulerformula}
\[
{\it Luas}=\frac{3\,\sqrt{39}}{4}
\]
\end{eulerformula}
\begin{eulercomment}
Tentu, setiap segitiga siku-siku merupakan kasus yang sering dijumpai.
\end{eulercomment}
\begin{eulerprompt}
>H(3,4,5) //luas segitiga siku-siku dengan panjang sisi 3, 4, 5
\end{eulerprompt}
\begin{euleroutput}
  6
\end{euleroutput}
\begin{eulercomment}
Dan jelas bahwa ini adalah segitiga dengan luas maksimal dan kedua
sisinya 3 dan 4.
\end{eulercomment}
\begin{eulerprompt}
>aspect (1.5); plot2d(&H(3,4,x),1,7): // Kurva luas segitiga sengan panjang sisi 3, 4, x (1<= x <=7)
\end{eulerprompt}
\eulerimg{19}{images/EMT4Geometry_Amalia Intan Arvitasari_22305144026_Mat B-042.png}
\begin{eulercomment}
Kasus umum juga bisa digunakan.
\end{eulercomment}
\begin{eulerprompt}
>$solve(diff(H(a,b,c)^2,c)=0,c)
\end{eulerprompt}
\begin{eulerformula}
\[
\left[ c=-\sqrt{b^2+a^2} , c=\sqrt{b^2+a^2} , c=0 \right] 
\]
\end{eulerformula}
\begin{eulercomment}
Sekarang, mari kita cari himpunan semua titik di mana b+c=d untuk
suatu konstanta d. Sudah diketahui bahwa ini adalah sebuah elips.
\end{eulercomment}
\begin{eulerprompt}
>s1 &= subst(d-c,b,sol[2]); $s1
\end{eulerprompt}
\begin{eulerformula}
\[
\left[ x=\frac{\left(d-c\right)^2-c^2+a^2}{2\,a} , y=\frac{\sqrt{-
 \left(d-c\right)^4+2\,c^2\,\left(d-c\right)^2+2\,a^2\,\left(d-c
 \right)^2-c^4+2\,a^2\,c^2-a^4}}{2\,a} \right] 
\]
\end{eulerformula}
\begin{eulercomment}
Dan buatlah fungsi-fungsi dari hal ini.
\end{eulercomment}
\begin{eulerprompt}
>function fx(a,c,d) &= rhs(s1[1]); $fx(a,c,d), function fy(a,c,d) &= rhs(s1[2]); $fy(a,c,d)
\end{eulerprompt}
\begin{eulerformula}
\[
\frac{\left(d-c\right)^2-c^2+a^2}{2\,a}
\]
\end{eulerformula}
\begin{eulerformula}
\[
\frac{\sqrt{-\left(d-c\right)^4+2\,c^2\,\left(d-c\right)^2+2\,a^2\,
 \left(d-c\right)^2-c^4+2\,a^2\,c^2-a^4}}{2\,a}
\]
\end{eulerformula}
\begin{eulercomment}
Sekarang kita dapat menggambar himpunannya. Sisi b bervariasi dari 1
hingga 4. Sudah diketahui bahwa kita mendapatkan sebuah elips.
\end{eulercomment}
\begin{eulerprompt}
>aspect(1); plot2d(&fx(3,x,5),&fy(3,x,5),xmin=1,xmax=4,square=1):
\end{eulerprompt}
\eulerimg{29}{images/EMT4Geometry_Amalia Intan Arvitasari_22305144026_Mat B-047.png}
\begin{eulercomment}
Kita dapat memeriksa persamaan umum untuk elips ini, yaitu

\end{eulercomment}
\begin{eulerformula}
\[
\frac{(x-x_m)^2}{u^2}+\frac{(y-y_m)}{v^2}=1,
\]
\end{eulerformula}
\begin{eulercomment}
di mana (xm, ym) adalah titik pusat, serta u dan v adalah setengah
sumbu.
\end{eulercomment}
\begin{eulerprompt}
>$ratsimp((fx(a,c,d)-a/2)^2/u^2+fy(a,c,d)^2/v^2 with [u=d/2,v=sqrt(d^2-a^2)/2])
\end{eulerprompt}
\begin{eulerformula}
\[
1
\]
\end{eulerformula}
\begin{eulercomment}
Kita lihat bahwa tinggi dan luas segitiga adalah maksimal untuk x=0.
Dengan demikian, luas segitiga dengan a+b+c=d adalah maksimal, jika
segitiga tersebut sama sisi. Kita ingin membuktikannya secara
analitis.
\end{eulercomment}
\begin{eulerprompt}
>eqns &= [diff(H(a,b,d-(a+b))^2,a)=0,diff(H(a,b,d-(a+b))^2,b)=0]; $eqns
\end{eulerprompt}
\begin{eulerformula}
\[
\left[ \frac{d\,\left(d-2\,a\right)\,\left(d-2\,b\right)}{8}-\frac{
 \left(-d+2\,b+2\,a\right)\,d\,\left(d-2\,b\right)}{8}=0 , \frac{d\,
 \left(d-2\,a\right)\,\left(d-2\,b\right)}{8}-\frac{\left(-d+2\,b+2\,
 a\right)\,d\,\left(d-2\,a\right)}{8}=0 \right] 
\]
\end{eulerformula}
\begin{eulercomment}
Kita mendapatkan beberapa minima, yang termasuk dalam segitiga dengan
satu sisi 0, dan solusi a = b = c = d / 3.
\end{eulercomment}
\begin{eulerprompt}
>$solve(eqns,[a,b])
\end{eulerprompt}
\begin{eulerformula}
\[
\left[ \left[ a=\frac{d}{3} , b=\frac{d}{3} \right]  , \left[ a=0
  , b=\frac{d}{2} \right]  , \left[ a=\frac{d}{2} , b=0 \right]  , 
 \left[ a=\frac{d}{2} , b=\frac{d}{2} \right]  \right] 
\]
\end{eulerformula}
\begin{eulercomment}
Ada juga metode Lagrange, yang memaksimalkan H(a,b,c)\textasciicircum{}2 sehubungan
dengan a+b+d=d.
\end{eulercomment}
\begin{eulerprompt}
>&solve([diff(H(a,b,c)^2,a)=la,diff(H(a,b,c)^2,b)=la, ...
>   diff(H(a,b,c)^2,c)=la,a+b+c=d],[a,b,c,la])
\end{eulerprompt}
\begin{euleroutput}
  
                       d      d
          [[a = 0, b = -, c = -, la = 0], 
                       2      2
       d             d                d      d
  [a = -, b = 0, c = -, la = 0], [a = -, b = -, c = 0, la = 0], 
       2             2                2      2
                              3
       d      d      d       d
  [a = -, b = -, c = -, la = ---]]
       3      3      3       108
  
\end{euleroutput}
\begin{eulercomment}
Kita bisa membuat plot situassi.
\end{eulercomment}
\begin{eulercomment}
Pertama, tetapkan titik-titik di Maxima.
\end{eulercomment}
\begin{eulerprompt}
>A &= at([x,y],sol[2]); $A
\end{eulerprompt}
\begin{eulerformula}
\[
\left[ \frac{-c^2+b^2+a^2}{2\,a} , \frac{\sqrt{-c^4+2\,b^2\,c^2+2\,
 a^2\,c^2-b^4+2\,a^2\,b^2-a^4}}{2\,a} \right] 
\]
\end{eulerformula}
\begin{eulerprompt}
>B &= [0,0]; $B, C &= [a,0]; $C
\end{eulerprompt}
\begin{eulerformula}
\[
\left[ 0 , 0 \right] 
\]
\end{eulerformula}
\begin{eulerformula}
\[
\left[ a , 0 \right] 
\]
\end{eulerformula}
\begin{eulercomment}
Kemudian, tetapkan rentang plot, dan titik-titik plot.
\end{eulercomment}
\begin{eulerprompt}
>setPlotRange(0,5,-2,3); ...
>a=4; b=3; c=2; ...
>plotPoint(mxmeval("B"),"B"); plotPoint(mxmeval("C"),"C"); ...
>plotPoint(mxmeval("A"),"A"):
\end{eulerprompt}
\eulerimg{29}{images/EMT4Geometry_Amalia Intan Arvitasari_22305144026_Mat B-055.png}
\begin{eulercomment}
Plot segmen-segmen tersebut.
\end{eulercomment}
\begin{eulerprompt}
>plotSegment(mxmeval("A"),mxmeval("C")); ...
>plotSegment(mxmeval("B"),mxmeval("C")); ...
>plotSegment(mxmeval("B"),mxmeval("A")):
\end{eulerprompt}
\eulerimg{29}{images/EMT4Geometry_Amalia Intan Arvitasari_22305144026_Mat B-056.png}
\begin{eulercomment}
Hitung garis tegak lurus tengah dalam Maxima.
\end{eulercomment}
\begin{eulerprompt}
>h &= middlePerpendicular(A,B); g &= middlePerpendicular(B,C);
\end{eulerprompt}
\begin{eulercomment}
Dan titik pusat lingkaran.
\end{eulercomment}
\begin{eulerprompt}
>U &= lineIntersection(h,g);
\end{eulerprompt}
\begin{eulercomment}
Kita peroleh rumus untuk jari-jari lingkaran.\\
circle.
\end{eulercomment}
\begin{eulerprompt}
>&assume(a>0,b>0,c>0); $distance(U,B) | radcan
\end{eulerprompt}
\begin{eulerformula}
\[
\frac{i\,a\,b\,c}{\sqrt{c-b-a}\,\sqrt{c-b+a}\,\sqrt{c+b-a}\,\sqrt{c
 +b+a}}
\]
\end{eulerformula}
\begin{eulercomment}
Mari kita tambahkan ke dalam plot.
\end{eulercomment}
\begin{eulerprompt}
>plotPoint(U()); ...
>plotCircle(circleWithCenter(mxmeval("U"),mxmeval("distance(U,C)"))):
\end{eulerprompt}
\eulerimg{29}{images/EMT4Geometry_Amalia Intan Arvitasari_22305144026_Mat B-058.png}
\begin{eulercomment}
Dengan menggunakan geometri, kami memperoleh rumus sederhana

\end{eulercomment}
\begin{eulerformula}
\[
\frac{a}{\sin(\alpha)}=2r
\]
\end{eulerformula}
\begin{eulercomment}
untuk radius. Kita bisa mengecek, apakah bernilai benar dengan Maxima.
Maxima akan memperhitungkannya hanya jika kita mengkuadratkannya.
\end{eulercomment}
\begin{eulerprompt}
>$c^2/sin(computeAngle(A,B,C))^2  | factor
\end{eulerprompt}
\begin{eulerformula}
\[
-\frac{4\,a^2\,b^2\,c^2}{\left(c-b-a\right)\,\left(c-b+a\right)\,
 \left(c+b-a\right)\,\left(c+b+a\right)}
\]
\end{eulerformula}
\eulerheading{Contoh 4: Garis Euler dan Parabola}
\begin{eulercomment}
Garis Euler adalah garis yang ditentukan dari segitiga apa pun yang
tidak sama sisi. Garis ini merupakan garis tengah segitiga, dan
melewati beberapa titik penting yang ditentukan dari segitiga,
termasuk ortosentrum, circumcenter, centroid, titik Exeter, dan pusat
lingkaran sembilan titik segitiga.

Sebagai demonstrasi, kami menghitung dan memplot garis Euler dalam
sebuah segitiga.

Pertama, kita mendefinisikan sudut-sudut segitiga dalam Euler. Kita
menggunakan definisi, yang terlihat dalam ekspresi simbolis.
\end{eulercomment}
\begin{eulerprompt}
>A::=[-1,-1]; B::=[2,0]; C::=[1,2];
\end{eulerprompt}
\begin{eulercomment}
Untuk memplot objek geometris, kita menyiapkan area plot, dan
menambahkan titik-titiknya. Semua plot objek geometris ditambahkan ke
plot tersebut.
\end{eulercomment}
\begin{eulerprompt}
>setPlotRange(3); plotPoint(A,"A"); plotPoint(B,"B"); plotPoint(C,"C");
\end{eulerprompt}
\begin{eulercomment}
Kita juga bisa menambahkan sisi-sisi segitiga.
\end{eulercomment}
\begin{eulerprompt}
>plotSegment(A,B,""); plotSegment(B,C,""); plotSegment(C,A,""):
\end{eulerprompt}
\eulerimg{29}{images/EMT4Geometry_Amalia Intan Arvitasari_22305144026_Mat B-061.png}
\begin{eulercomment}
Berikut ini adalah luas area segitiga, dengan menggunakan rumus
determinan. Tentu, kita harus mengambil nilai absolut dari hasil ini.
\end{eulercomment}
\begin{eulerprompt}
>$areaTriangle(A,B,C)
\end{eulerprompt}
\begin{eulerformula}
\[
-\frac{7}{2}
\]
\end{eulerformula}
\begin{eulercomment}
Kita dapat menghitung koefisien sisi c.
\end{eulercomment}
\begin{eulerprompt}
>c &= lineThrough(A,B)
\end{eulerprompt}
\begin{euleroutput}
  
                              [- 1, 3, - 2]
  
\end{euleroutput}
\begin{eulercomment}
Dan juga mendapatkan formula untuk baris ini.\\
menggunakan rumus determinan. Tentu, kita harus mengambil nilai
absolut dari hasil ini.
\end{eulercomment}
\begin{eulerprompt}
>$getLineEquation(c,x,y)
\end{eulerprompt}
\begin{eulerformula}
\[
3\,y-x=-2
\]
\end{eulerformula}
\begin{eulercomment}
Untuk bentuk Hesse, kita perlu menentukan sebuah titik, sehingga titik
tersebut berada di sisi positif dari bentuk Hesse. Dengan memasukkan
titik tersebut akan dihasilkan jarak positif ke garis.
\end{eulercomment}
\begin{eulerprompt}
>$getHesseForm(c,x,y,C), $at(%,[x=C[1],y=C[2]])
\end{eulerprompt}
\begin{eulerformula}
\[
\frac{3\,y-x+2}{\sqrt{10}}
\]
\end{eulerformula}
\begin{eulerformula}
\[
\frac{7}{\sqrt{10}}
\]
\end{eulerformula}
\begin{eulercomment}
Sekarang kita menghitung keliling ABC.
\end{eulercomment}
\begin{eulerprompt}
>LL &= circleThrough(A,B,C); $getCircleEquation(LL,x,y)
\end{eulerprompt}
\begin{eulerformula}
\[
\left(y-\frac{5}{14}\right)^2+\left(x-\frac{3}{14}\right)^2=\frac{
 325}{98}
\]
\end{eulerformula}
\begin{eulerprompt}
>O &= getCircleCenter(LL); $O
\end{eulerprompt}
\begin{eulerformula}
\[
\left[ \frac{3}{14} , \frac{5}{14} \right] 
\]
\end{eulerformula}
\begin{eulercomment}
Plot lingkaran dan titikpusatnya. Cu dan U adalah simbolik. Kami
mengevaluasi ekspresi ini untuk Euler.
\end{eulercomment}
\begin{eulerprompt}
>plotCircle(LL()); plotPoint(O(),"O"):
\end{eulerprompt}
\eulerimg{29}{images/EMT4Geometry_Amalia Intan Arvitasari_22305144026_Mat B-068.png}
\begin{eulercomment}
Kita dapat menghitung perpotongan ketinggian di ABC (pusat
ortosentrum) secara numerik dengan perintah berikut ini.
\end{eulercomment}
\begin{eulerprompt}
>H &= lineIntersection(perpendicular(A,lineThrough(C,B)),...
>  perpendicular(B,lineThrough(A,C))); $H
\end{eulerprompt}
\begin{eulerformula}
\[
\left[ \frac{11}{7} , \frac{2}{7} \right] 
\]
\end{eulerformula}
\begin{eulercomment}
Sekarang kita dapat menghitung garis Euler dari segitiga tersebut.
\end{eulercomment}
\begin{eulerprompt}
>el &= lineThrough(H,O); $getLineEquation(el,x,y)
\end{eulerprompt}
\begin{eulerformula}
\[
-\frac{19\,y}{14}-\frac{x}{14}=-\frac{1}{2}
\]
\end{eulerformula}
\begin{eulercomment}
Tambahkan ke plot kita.
\end{eulercomment}
\begin{eulerprompt}
>plotPoint(H(),"H"); plotLine(el(),"Garis Euler"):
\end{eulerprompt}
\eulerimg{29}{images/EMT4Geometry_Amalia Intan Arvitasari_22305144026_Mat B-071.png}
\begin{eulercomment}
Pusat gravitasi harus berada pada garis ini.
\end{eulercomment}
\begin{eulerprompt}
>M &= (A+B+C)/3; $getLineEquation(el,x,y) with [x=M[1],y=M[2]]
\end{eulerprompt}
\begin{eulerformula}
\[
-\frac{1}{2}=-\frac{1}{2}
\]
\end{eulerformula}
\begin{eulerprompt}
>plotPoint(M(),"M"): // titik berat
\end{eulerprompt}
\eulerimg{29}{images/EMT4Geometry_Amalia Intan Arvitasari_22305144026_Mat B-073.png}
\begin{eulercomment}
Teori mengatakan bahwa MH = 2*MO. Kita perlu menyederhanakan dengan
radcan untuk mencapai hal ini.
\end{eulercomment}
\begin{eulerprompt}
>$distance(M,H)/distance(M,O)|radcan
\end{eulerprompt}
\begin{eulerformula}
\[
2
\]
\end{eulerformula}
\begin{eulercomment}
Fungsi-fungsi ini juga mencakup fungsi untuk sudut.
\end{eulercomment}
\begin{eulerprompt}
>$computeAngle(A,C,B), degprint(%())
\end{eulerprompt}
\begin{eulerformula}
\[
\arccos \left(\frac{4}{\sqrt{5}\,\sqrt{13}}\right)
\]
\end{eulerformula}
\begin{euleroutput}
  60°15'18.43''
\end{euleroutput}
\begin{eulercomment}
Persamaan untuk titik tengah lingkaran ini tidak begitu bagus.
\end{eulercomment}
\begin{eulerprompt}
>Q &= lineIntersection(angleBisector(A,C,B),angleBisector(C,B,A))|radcan; $Q
\end{eulerprompt}
\begin{eulerformula}
\[
\left[ \frac{\left(2^{\frac{3}{2}}+1\right)\,\sqrt{5}\,\sqrt{13}-15
 \,\sqrt{2}+3}{14} , \frac{\left(\sqrt{2}-3\right)\,\sqrt{5}\,\sqrt{
 13}+5\,2^{\frac{3}{2}}+5}{14} \right] 
\]
\end{eulerformula}
\begin{eulercomment}
Mari kita hitung juga ekspresi untuk jari-jari lingkaran yang
tertulis.
\end{eulercomment}
\begin{eulerprompt}
>r &= distance(Q,projectToLine(Q,lineThrough(A,B)))|ratsimp; $r
\end{eulerprompt}
\begin{eulerformula}
\[
\frac{\sqrt{\left(-41\,\sqrt{2}-31\right)\,\sqrt{5}\,\sqrt{13}+115
 \,\sqrt{2}+614}}{7\,\sqrt{2}}
\]
\end{eulerformula}
\begin{eulerprompt}
>LD &=  circleWithCenter(Q,r); // Lingkaran dalam
\end{eulerprompt}
\begin{eulercomment}
Mari kita tambahkan ke dalam plot.
\end{eulercomment}
\begin{eulerprompt}
>color(5); plotCircle(LD()):
\end{eulerprompt}
\eulerimg{29}{images/EMT4Geometry_Amalia Intan Arvitasari_22305144026_Mat B-078.png}
\eulersubheading{Parabola}
\begin{eulercomment}
Selanjutnya akan dicari persamaan tempat kedudukan titik-titik yang berjarak sama ke titik C
dan ke garis AB.
\end{eulercomment}
\begin{eulerprompt}
>p &= getHesseForm(lineThrough(A,B),x,y,C)-distance([x,y],C); $p='0
\end{eulerprompt}
\begin{eulerformula}
\[
\frac{3\,y-x+2}{\sqrt{10}}-\sqrt{\left(2-y\right)^2+\left(1-x
 \right)^2}=0
\]
\end{eulerformula}
\begin{eulercomment}
Persamaan tersebut dapat digambar menjadi satu dengan gambar sebelumnya.
\end{eulercomment}
\begin{eulerprompt}
>plot2d(p,level=0,add=1,contourcolor=6):
\end{eulerprompt}
\eulerimg{29}{images/EMT4Geometry_Amalia Intan Arvitasari_22305144026_Mat B-080.png}
\begin{eulercomment}
Ini seharusnya berupa suatu fungsi, tetapi solver default Maxima hanya
dapat menemukan solusinya, jika kita mengkuadratkan persamaannya.
Akibatnya, kita mendapatkan solusi palsu.
\end{eulercomment}
\begin{eulerprompt}
>akar &= solve(getHesseForm(lineThrough(A,B),x,y,C)^2-distance([x,y],C)^2,y)
\end{eulerprompt}
\begin{euleroutput}
  
          [y = - 3 x - sqrt(70) sqrt(9 - 2 x) + 26, 
                                y = - 3 x + sqrt(70) sqrt(9 - 2 x) + 26]
  
\end{euleroutput}
\begin{eulercomment}
Solusi pertama adalah

maxima: akar[1]

Dengan menambahkan solusi pertama ke dalam plot, menunjukkan bahwa itu
memang jalur yang kita cari. Teori mengatakan bahwa itu adalah
parabola yang diputar.
\end{eulercomment}
\begin{eulerprompt}
>plot2d(&rhs(akar[1]),add=1):
\end{eulerprompt}
\eulerimg{29}{images/EMT4Geometry_Amalia Intan Arvitasari_22305144026_Mat B-081.png}
\begin{eulerprompt}
>function g(x) &= rhs(akar[1]); $'g(x)= g(x)// fungsi yang mendefinisikan kurva di atas
\end{eulerprompt}
\begin{eulerformula}
\[
g\left(x\right)=-3\,x-\sqrt{70}\,\sqrt{9-2\,x}+26
\]
\end{eulerformula}
\begin{eulerprompt}
>T &=[-1, g(-1)]; // ambil sebarang titik pada kurva tersebut
>dTC &= distance(T,C); $fullratsimp(dTC), $float(%) // jarak T ke C
\end{eulerprompt}
\begin{eulerformula}
\[
\sqrt{1503-54\,\sqrt{11}\,\sqrt{70}}
\]
\end{eulerformula}
\begin{eulerformula}
\[
2.135605779339061
\]
\end{eulerformula}
\begin{eulerprompt}
>U &= projectToLine(T,lineThrough(A,B)); $U // proyeksi T pada garis AB 
\end{eulerprompt}
\begin{eulerformula}
\[
\left[ \frac{80-3\,\sqrt{11}\,\sqrt{70}}{10} , \frac{20-\sqrt{11}\,
 \sqrt{70}}{10} \right] 
\]
\end{eulerformula}
\begin{eulerprompt}
>dU2AB &= distance(T,U); $fullratsimp(dU2AB), $float(%) // jatak T ke AB
\end{eulerprompt}
\begin{eulerformula}
\[
\sqrt{1503-54\,\sqrt{11}\,\sqrt{70}}
\]
\end{eulerformula}
\begin{eulerformula}
\[
2.135605779339061
\]
\end{eulerformula}
\begin{eulercomment}
Ternyata jarak T ke C sama dengan jarak T ke AB. Coba Anda pilih titik T yang lain dan
ulangi perhitungan-perhitungan di atas untuk menunjukkan bahwa hasilnya juga sama.
\end{eulercomment}
\begin{eulercomment}

\begin{eulercomment}
\eulerheading{Contoh 5: Trigonometri Rasional}
\begin{eulercomment}
Hal ini terinspirasi dari sebuah perkataan N.J. Wildberger. Dalam
bukunya "Proporsi Ilahi", Wildberger mengusulkan untuk mengganti
gagasan klasik tentang jarak dan sudut dengan quadrance dan spread.
Dengan menggunakan ini, memang memungkinkan untuk menghindari fungsi
trigonometri dalam banyak contoh, dan tetap "rasional".

Berikut ini, saya akan memperkenalkan konsep-konsepnya, dan memecahkan
beberapa masalah. Saya menggunakan komputasi simbolis Maxima di sini,
yang menyembunyikan keuntungan utama dari trigonometri rasional yang
komputasinya dapat dilakukan dengan kertas dan pensil saja. Anda
dipersilakan untuk memeriksa hasilnya tanpa komputer.

Intinya adalah bahwa komputasi rasional simbolis sering kali
memberikan hasil yang sederhana. Sebaliknya, trigonometri klasik
menghasilkan hasil trigonometri yang rumit, yang dievaluasi dengan
perkiraan numerik saja.
\end{eulercomment}
\begin{eulerprompt}
>load geometry;
\end{eulerprompt}
\begin{eulercomment}
Untuk pengenalan pertama, kita menggunakan segitiga persegi panjang
dengan proporsi Mesir yang terkenal 3, 4, dan 5. Perintah berikut ini
adalah perintah Euler untuk memplot geometri bidang yang terdapat pada
file Euler "geometry.e".
\end{eulercomment}
\begin{eulerprompt}
>C&:=[0,0]; A&:=[4,0]; B&:=[0,3]; ...
>setPlotRange(-1,5,-1,5); ...
>plotPoint(A,"A"); plotPoint(B,"B"); plotPoint(C,"C"); ...
>plotSegment(B,A,"c"); plotSegment(A,C,"b"); plotSegment(C,B,"a"); ...
>insimg(30);
\end{eulerprompt}
\eulerimg{29}{images/EMT4Geometry_Amalia Intan Arvitasari_22305144026_Mat B-088.png}
\begin{eulercomment}
Tentu,

\end{eulercomment}
\begin{eulerformula}
\[
\sin(w_a)=\frac{a}{c},
\]
\end{eulerformula}
\begin{eulercomment}
di mana wa adalah sudut di A. Cara biasa untuk menghitung sudut ini,
adalah dengan mengambil kebalikan dari fungsi sinus. Hasilnya adalah
sudut yang tidak dapat dicerna, yang hanya dapat dicetak kira-kira.
\end{eulercomment}
\begin{eulerprompt}
>wa := arcsin(3/5); degprint(wa)
\end{eulerprompt}
\begin{euleroutput}
  36°52'11.63''
\end{euleroutput}
\begin{eulercomment}
Trigonometri rasional mencoba menghindari hal ini.

Gagasan pertama trigonometri rasional adalah kuadrat, yang
menggantikan jarak. Sebenarnya, ini hanyalah jarak yang dikuadratkan.
Berikut ini, a, b, dan c menunjukkan kuadran sisi-sisinya.

Teorema Pythogoras menjadi a+b=c. Maka, teorema Pythogoras menjadi
a+b=c.
\end{eulercomment}
\begin{eulerprompt}
>a &= 3^2; b &= 4^2; c &= 5^2; &a+b=c
\end{eulerprompt}
\begin{euleroutput}
  
                                 25 = 25
  
\end{euleroutput}
\begin{eulercomment}
Gagasan kedua dari trigonometri rasional adalah penyebaran. Penyebaran
mengukur bukaan di antara garis-garis. Bernilai 0, jika garis-garisnya
sejajar, dan bernilai 1, jika garis-garisnya persegi panjang. Ini
adalah kuadrat dari sinus sudut antara kedua garis tersebut.

Penyebaran garis AB dan AC pada gambar di atas didefinisikan sebagai

\end{eulercomment}
\begin{eulerformula}
\[
s_a = \sin(\alpha)^2 = \frac{a}{c},
\]
\end{eulerformula}
\begin{eulercomment}
di mana a dan c adalah kuadran dari segitiga persegi panjang dengan
satu sudut di A.
\end{eulercomment}
\begin{eulerprompt}
>sa &= a/c; $sa
\end{eulerprompt}
\begin{eulerformula}
\[
\frac{9}{25}
\]
\end{eulerformula}
\begin{eulercomment}
Tentu saja, hal ini lebih mudah dihitung daripada sudut. Tetapi Anda
kehilangan sifat bahwa sudut dapat ditambahkan dengan mudah.

Tentu saja kita bisa mengonversi nilai perkiraan kita untuk sudut wa
ke sprad, dan mencetaknya sebagai pecahan.
\end{eulercomment}
\begin{eulerprompt}
>fracprint(sin(wa)^2)
\end{eulerprompt}
\begin{euleroutput}
  9/25
\end{euleroutput}
\begin{eulercomment}
Hukum kosinus trgonometri klasik diterjemahkan ke dalam "crosslaw"
berikut ini.

\end{eulercomment}
\begin{eulerformula}
\[
(c+b-a)^2 = 4 b c \, (1-s_a)
\]
\end{eulerformula}
\begin{eulercomment}
Di sini, a, b, dan c adalah kuadran dari sisi-sisi segitiga, dan sa
adalah spread di sudut A. Sisi a, seperti biasa, berlawanan dengan
sudut A.

Hukum-hukum ini diimplementasikan dalam file geometry.e yang kita muat
ke dalam Euler.
\end{eulercomment}
\begin{eulerprompt}
>$crosslaw(aa,bb,cc,saa)
\end{eulerprompt}
\begin{eulerformula}
\[
\left[ \left({\it bb}-{\it aa}+\frac{7}{6}\right)^2 , \left(
 {\it bb}-{\it aa}+\frac{7}{6}\right)^2 , \left({\it bb}-{\it aa}+
 \frac{5}{3\,\sqrt{2}}\right)^2 \right] =\left[ \frac{14\,{\it bb}\,
 \left(1-{\it saa}\right)}{3} , \frac{14\,{\it bb}\,\left(1-{\it saa}
 \right)}{3} , \frac{5\,2^{\frac{3}{2}}\,{\it bb}\,\left(1-{\it saa}
 \right)}{3} \right] 
\]
\end{eulerformula}
\begin{eulercomment}
Dalam kasus kita, kita dapatkan
\end{eulercomment}
\begin{eulerprompt}
>$crosslaw(a,b,c,sa)
\end{eulerprompt}
\begin{eulerformula}
\[
1024=1024
\]
\end{eulerformula}
\begin{eulercomment}
Mari kita gunakan crosslaw ini untuk mencari sebaran di A. Untuk
melakukannya, kita buat crosslaw untuk kuadran a, b, dan c, dan
selesaikan untuk sebaran sa yang tidak diketahui.

Anda bisa melakukan ini dengan tangan dengan mudah, tetapi disini saya
menggunakan Maxima. Tentu saja, kami mendapatkan hasil yang sudah kami
dapatkan.
\end{eulercomment}
\begin{eulerprompt}
>$crosslaw(a,b,c,x), $solve(%,x)
\end{eulerprompt}
\begin{eulerformula}
\[
1024=1600\,\left(1-x\right)
\]
\end{eulerformula}
\begin{eulerformula}
\[
\left[ x=\frac{9}{25} \right] 
\]
\end{eulerformula}
\begin{eulercomment}
Kita sudah mengetahui hal ini. Definisi spread adalah kasus khusus
dari crosslaw.

Kita juga dapat menyelesaikannya untuk a, b, c secara umum. Hasilnya
adalah sebuah rumus yang menghitung penyebaran sudut segitiga dengan
kuadran ketiga sisinya.
\end{eulercomment}
\begin{eulerprompt}
>$solve(crosslaw(aa,bb,cc,x),x)
\end{eulerprompt}
\begin{eulerformula}
\[
\left[ \left[ \frac{168\,{\it bb}\,x+36\,{\it bb}^2+\left(-72\,
 {\it aa}-84\right)\,{\it bb}+36\,{\it aa}^2-84\,{\it aa}+49}{36} , 
 \frac{168\,{\it bb}\,x+36\,{\it bb}^2+\left(-72\,{\it aa}-84\right)
 \,{\it bb}+36\,{\it aa}^2-84\,{\it aa}+49}{36} , \frac{15\,2^{\frac{
 5}{2}}\,{\it bb}\,x+18\,{\it bb}^2+\left(-36\,{\it aa}-15\,2^{\frac{
 3}{2}}\right)\,{\it bb}+18\,{\it aa}^2-15\,2^{\frac{3}{2}}\,{\it aa}
 +25}{18} \right] =0 \right] 
\]
\end{eulerformula}
\begin{eulercomment}
Kita dapat membuat sebuah fungsi dari hasil tersebut. Fungsi seperti
itu sudah didefinisikan dalam file geometry.e dari Euler.
\end{eulercomment}
\begin{eulerprompt}
>$spread(a,b,c)
\end{eulerprompt}
\begin{eulerformula}
\[
\frac{9}{25}
\]
\end{eulerformula}
\begin{eulercomment}
Sebagai contoh, kita dapat menggunakannya untuk menghitung sudut
segitiga dengan sisi

\end{eulercomment}
\begin{eulerformula}
\[
a, \quad a, \quad \frac{4a}{7}
\]
\end{eulerformula}
\begin{eulercomment}
Hasilnya rasional, hal ini tidak mudah didapat jika kita menggunakan
trigonometri klasik.
\end{eulercomment}
\begin{eulerprompt}
>$spread(a,a,4*a/7)
\end{eulerprompt}
\begin{eulerformula}
\[
\frac{6}{7}
\]
\end{eulerformula}
\begin{eulercomment}
Ini adalah sudut dalam derajat.
\end{eulercomment}
\begin{eulerprompt}
>degprint(arcsin(sqrt(6/7)))
\end{eulerprompt}
\begin{euleroutput}
  67°47'32.44''
\end{euleroutput}
\eulersubheading{Contoh Lain}
\begin{eulercomment}
Sekarang, mari kita mencoba contoh lebih lanjut.

Kita tetapkan tiga sudut segitiga sebagai berikut.
\end{eulercomment}
\begin{eulerprompt}
>A&:=[1,2]; B&:=[4,3]; C&:=[0,4]; ...
>setPlotRange(-1,5,1,7); ...
>plotPoint(A,"A"); plotPoint(B,"B"); plotPoint(C,"C"); ...
>plotSegment(B,A,"c"); plotSegment(A,C,"b"); plotSegment(C,B,"a"); ...
>insimg;
\end{eulerprompt}
\eulerimg{29}{images/EMT4Geometry_Amalia Intan Arvitasari_22305144026_Mat B-100.png}
\begin{eulercomment}
Dengan menggunakan Pythogoras, dapat dengan mudah untuk menghitung
jarak antara dua titik. Pertama-tama saya menggunakan jarak fungsi
dari file Euler untuk geometri. Jarak fungsi menggunakan geometri
klasik.
\end{eulercomment}
\begin{eulerprompt}
>$distance(A,B)
\end{eulerprompt}
\begin{eulerformula}
\[
\sqrt{10}
\]
\end{eulerformula}
\begin{eulercomment}
Euler juga memiliki fungsi untuk kuadransi antara dua titik.

Pada contoh berikut, karena c+b bukan a, maka segitiga tersebut tidak
berbentuk siku-siku.
\end{eulercomment}
\begin{eulerprompt}
>c &= quad(A,B); $c, b &= quad(A,C); $b, a &= quad(B,C); $a,
\end{eulerprompt}
\begin{eulerformula}
\[
10
\]
\end{eulerformula}
\begin{eulerformula}
\[
5
\]
\end{eulerformula}
\begin{eulerformula}
\[
17
\]
\end{eulerformula}
\begin{eulercomment}
Pertama, mari kita menghitung sudut tradisional. Fungsi computeAngle
menggunakan metode yang biasa berdasarkan hasil kali titik dari dua
vektor. Hasilnya adalah beberapa perkiraan titik mengambang.

\end{eulercomment}
\begin{eulerformula}
\[
A=<1,2>\quad B=<4,3>,\quad C=<0,4>
\]
\end{eulerformula}
\begin{eulerformula}
\[
\mathbf{a}=C-B=<-4,1>,\quad \mathbf{c}=A-B=<-3,-1>,\quad \beta=\angle ABC
\]
\end{eulerformula}
\begin{eulerformula}
\[
\mathbf{a}.\mathbf{c}=|\mathbf{a}|.|\mathbf{c}|\cos \beta
\]
\end{eulerformula}
\begin{eulerformula}
\[
\cos \angle ABC =\cos\beta=\frac{\mathbf{a}.\mathbf{c}}{|\mathbf{a}|.|\mathbf{c}|}=\frac{12-1}{\sqrt{17}\sqrt{10}}=\frac{11}{\sqrt{17}\sqrt{10}}
\]
\end{eulerformula}
\begin{eulerprompt}
>wb &= computeAngle(A,B,C); $wb, $(wb/pi*180)()
\end{eulerprompt}
\begin{eulerformula}
\[
\arccos \left(\frac{11}{\sqrt{10}\,\sqrt{17}}\right)
\]
\end{eulerformula}
\begin{euleroutput}
  32.4711922908
\end{euleroutput}
\begin{eulercomment}
Dengan menggunakan pensil dan kertas, kita dapat melakukan hal yang
sama dengan cross law. Kita masukkan kuadran a, b, dan c ke dalam
hukum silang dan selesaikan untuk x.

\end{eulercomment}
\begin{eulerformula}
\[
A=<1,2>\quad B=<4,3>,\quad C=<0,4>
\]
\end{eulerformula}
\begin{eulerformula}
\[
\mathbf{a}=C-B=<-4,1>,\quad \mathbf{c}=A-B=<-3,-1>,\quad \beta=\angle ABC
\]
\end{eulerformula}
\begin{eulerformula}
\[
\mathbf{a}.\mathbf{c}=|\mathbf{a}|.|\mathbf{c}|\cos \beta
\]
\end{eulerformula}
\begin{eulerformula}
\[
\cos \angle ABC =\cos\beta=\frac{\mathbf{a}.\mathbf{c}}{|\mathbf{a}|.|\mathbf{c}|}=\frac{12-1}{\sqrt{17}\sqrt{10}}=\frac{11}{\sqrt{17}\sqrt{10}}
\]
\end{eulerformula}
\begin{eulerprompt}
>$crosslaw(a,b,c,x), $solve(%,x), //(b+c-a)^=4b.c(1-x)
\end{eulerprompt}
\begin{eulerformula}
\[
4=200\,\left(1-x\right)
\]
\end{eulerformula}
\begin{eulerformula}
\[
\left[ x=\frac{49}{50} \right] 
\]
\end{eulerformula}
\begin{eulercomment}
Itulah yang dilakukan oleh fungsi spread yang didefinisikan dalam
"geometry.e".
\end{eulercomment}
\begin{eulerprompt}
>sb &= spread(b,a,c); $sb
\end{eulerprompt}
\begin{eulerformula}
\[
\frac{49}{170}
\]
\end{eulerformula}
\begin{eulercomment}
Maxima mendapatkan hasil yang sama dengan menggunakan trigonometri
biasa, jika kita memaksakannya. Maxima menyelesaikan suku
sin(arccos(...)) menjadi hasil pecahan. Sebagian besar siswa tidak
dapat melakukan ini.
\end{eulercomment}
\begin{eulerprompt}
>$sin(computeAngle(A,B,C))^2
\end{eulerprompt}
\begin{eulerformula}
\[
\frac{49}{170}
\]
\end{eulerformula}
\begin{eulercomment}
Setelah kita memiliki penyebaran di B, kita dapat menghitung tinggi ha
di sisi a. Ingatlah bahwa menurut definisi

\end{eulercomment}
\begin{eulerformula}
\[
s_b=\frac{h_a}{c}
\]
\end{eulerformula}
\begin{eulercomment}
\end{eulercomment}
\begin{eulerprompt}
>ha &= c*sb; $ha
\end{eulerprompt}
\begin{eulerformula}
\[
\frac{49}{17}
\]
\end{eulerformula}
\begin{eulercomment}
Gambar berikut ini telah dibuat dengan program geometri C.a.R., yang
dapat menggambar kuadran dan penyebaran.

image: (20) Rational\_Geometry\_CaR.png

Menurut definisi, panjang ha adalah akar kuadrat dari kuadrannya.
\end{eulercomment}
\begin{eulerprompt}
>$sqrt(ha)
\end{eulerprompt}
\begin{eulerformula}
\[
\frac{7}{\sqrt{17}}
\]
\end{eulerformula}
\begin{eulercomment}
Sekarang kita dapat menghitung luas segitiga. Jangan lupa, bahwa kita
berurusan dengan kuadran!
\end{eulercomment}
\begin{eulerprompt}
>$sqrt(ha)*sqrt(a)/2
\end{eulerprompt}
\begin{eulerformula}
\[
\frac{7}{2}
\]
\end{eulerformula}
\begin{eulercomment}
Rumus determinan yang biasa menghasilkan hasil yang sama.
\end{eulercomment}
\begin{eulerprompt}
>$areaTriangle(B,A,C)
\end{eulerprompt}
\begin{eulerformula}
\[
\frac{7}{2}
\]
\end{eulerformula}
\eulersubheading{Rumus Heron}
\begin{eulercomment}
Sekarang, mari kita selesaikan masalah ini secara umum!
\end{eulercomment}
\begin{eulerprompt}
>&remvalue(a,b,c,sb,ha);
\end{eulerprompt}
\begin{eulercomment}
Pertama-tama kita menghitung luas di B untuk segitiga dengan sisi a,
b, dan c. Kemudian kita menghitung luas kuadrat ("quadrea"?),
faktorkan dengan Maxima, dan kita dapatkan rumus Heron yang terkenal.

Memang, hal ini sulit dilakukan dengan menggunakan pensil dan kertas.
\end{eulercomment}
\begin{eulerprompt}
>$spread(b^2,c^2,a^2), $factor(%*c^2*a^2/4)
\end{eulerprompt}
\begin{eulerformula}
\[
\frac{-c^4-\left(-2\,b^2-2\,a^2\right)\,c^2-b^4+2\,a^2\,b^2-a^4}{4
 \,a^2\,c^2}
\]
\end{eulerformula}
\begin{eulerformula}
\[
\frac{\left(-c+b+a\right)\,\left(c-b+a\right)\,\left(c+b-a\right)\,
 \left(c+b+a\right)}{16}
\]
\end{eulerformula}
\eulersubheading{Aturan Triple Spread}
\begin{eulercomment}
Kekurangan dari spread adalah mereka tidak lagi hanya menambahkan
seperti sudut.

Namun, three spread dari sebuah segitiga memenuhi aturan "triple
spread" berikut ini.
\end{eulercomment}
\begin{eulerprompt}
>&remvalue(sa,sb,sc); $triplespread(sa,sb,sc)
\end{eulerprompt}
\begin{eulerformula}
\[
\left({\it sc}+{\it sb}+{\it sa}\right)^2=2\,\left({\it sc}^2+
 {\it sb}^2+{\it sa}^2\right)+4\,{\it sa}\,{\it sb}\,{\it sc}
\]
\end{eulerformula}
\begin{eulercomment}
Aturan ini berlaku untuk tiga sudut yang berjumlah 180°.

\end{eulercomment}
\begin{eulerformula}
\[
\alpha+\beta+\gamma=\pi
\]
\end{eulerformula}
\begin{eulercomment}
Karena spread dari

\end{eulercomment}
\begin{eulerformula}
\[
\alpha, \pi-\alpha
\]
\end{eulerformula}
\begin{eulercomment}
sama, aturan triple spread juga benar, jika

\end{eulercomment}
\begin{eulerformula}
\[
\alpha+\beta=\gamma
\]
\end{eulerformula}
\begin{eulercomment}
Karena spread dari sudut negatif sama, aturan triple spread juga
berlaku, jika

\end{eulercomment}
\begin{eulerformula}
\[
\alpha+\beta+\gamma=0
\]
\end{eulerformula}
\begin{eulercomment}
Contohnya, kita dapat menghitung spread dari sudut 60°. Hasilnya
adalah 3/4. Namun, persamaan ini memiliki solusi kedua, di mana semua
spread adalah 0
\end{eulercomment}
\begin{eulerprompt}
>$solve(triplespread(x,x,x),x)
\end{eulerprompt}
\begin{eulerformula}
\[
\left[ x=\frac{3}{4} , x=0 \right] 
\]
\end{eulerformula}
\begin{eulercomment}
Penyebaran 90° jelas adalah 1. Jika dua sudut ditambahkan ke 90°,
penyebarannya akan menyelesaikan persamaan penyebaran tiga dengan a,
b, 1. Dengan perhitungan berikut, kita mendapatkan a + b = 1.
\end{eulercomment}
\begin{eulerprompt}
>$triplespread(x,y,1), $solve(%,x)
\end{eulerprompt}
\begin{eulerformula}
\[
\left(y+x+1\right)^2=2\,\left(y^2+x^2+1\right)+4\,x\,y
\]
\end{eulerformula}
\begin{eulerformula}
\[
\left[ x=1-y \right] 
\]
\end{eulerformula}
\begin{eulercomment}
Karena penyebaran 180°-t sama dengan penyebaran t, rumus triple spread
juga berlaku, jika salah satu sudut adalah jumlah atau selisih dari
dua sudut lainnya.

Jadi, kita dapat menemukan penyebaran sudut dua kali lipat. Perhatikan
bahwa ada dua solusi lagi. Kita jadikan ini sebuah fungsi.
\end{eulercomment}
\begin{eulerprompt}
>$solve(triplespread(a,a,x),x), function doublespread(a) &= factor(rhs(%[1]))
\end{eulerprompt}
\begin{eulerformula}
\[
\left[ x=4\,a-4\,a^2 , x=0 \right] 
\]
\end{eulerformula}
\begin{euleroutput}
  
                              - 4 (a - 1) a
  
\end{euleroutput}
\eulersubheading{Garis Bagi Sudut}
\begin{eulercomment}
Ini adalah situasi yang sudah kita ketahui.
\end{eulercomment}
\begin{eulerprompt}
>C&:=[0,0]; A&:=[4,0]; B&:=[0,3]; ...
>setPlotRange(-1,5,-1,5); ...
>plotPoint(A,"A"); plotPoint(B,"B"); plotPoint(C,"C"); ...
>plotSegment(B,A,"c"); plotSegment(A,C,"b"); plotSegment(C,B,"a"); ...
>insimg;
\end{eulerprompt}
\eulerimg{29}{images/EMT4Geometry_Amalia Intan Arvitasari_22305144026_Mat B-134.png}
\begin{eulercomment}
Mari kita hitung panjang garis bagi sudut di A. Tetapi kita ingin
menyelesaikannya untuk a, b, c secara umum.
\end{eulercomment}
\begin{eulerprompt}
>&remvalue(a,b,c);
\end{eulerprompt}
\begin{eulercomment}
Jadi, pertama-tama kita hitung penyebaran sudut yang dibelah dua di A,
dengan menggunakan rumus triple spread.

Masalah dengan rumus ini muncul lagi. Rumus ini memiliki dua solusi.
Kita harus memilih salah satu solusi yang benar. Solusi lainnya
mengacu pada sudut yang dibagi dua 180°-wa.
\end{eulercomment}
\begin{eulerprompt}
>$triplespread(x,x,a/(a+b)), $solve(%,x), sa2 &= rhs(%[1]); $sa2
\end{eulerprompt}
\begin{eulerformula}
\[
\left(2\,x+\frac{a}{b+a}\right)^2=2\,\left(2\,x^2+\frac{a^2}{\left(
 b+a\right)^2}\right)+\frac{4\,a\,x^2}{b+a}
\]
\end{eulerformula}
\begin{eulerformula}
\[
\left[ x=\frac{-\sqrt{b}\,\sqrt{b+a}+b+a}{2\,b+2\,a} , x=\frac{
 \sqrt{b}\,\sqrt{b+a}+b+a}{2\,b+2\,a} \right] 
\]
\end{eulerformula}
\begin{eulerformula}
\[
\frac{-\sqrt{b}\,\sqrt{b+a}+b+a}{2\,b+2\,a}
\]
\end{eulerformula}
\begin{eulercomment}
Mari kita cek rectangle Mesir.
\end{eulercomment}
\begin{eulerprompt}
>$sa2 with [a=3^2,b=4^2]
\end{eulerprompt}
\begin{eulerformula}
\[
\frac{1}{10}
\]
\end{eulerformula}
\begin{eulercomment}
Kita bisa mencetak sudut dalam Euler, setelah mentransfer spread ke
radian.
\end{eulercomment}
\begin{eulerprompt}
>wa2 := arcsin(sqrt(1/10)); degprint(wa2)
\end{eulerprompt}
\begin{euleroutput}
  18°26'5.82''
\end{euleroutput}
\begin{eulercomment}
Titik P adalah perpotongan garis bagi sudut dengan sumbu y.
\end{eulercomment}
\begin{eulerprompt}
>P := [0,tan(wa2)*4]
\end{eulerprompt}
\begin{euleroutput}
  [0,  1.33333]
\end{euleroutput}
\begin{eulerprompt}
>plotPoint(P,"P"); plotSegment(A,P):
\end{eulerprompt}
\eulerimg{29}{images/EMT4Geometry_Amalia Intan Arvitasari_22305144026_Mat B-139.png}
\begin{eulercomment}
Mari kita cek sudut-sudutnya dalam contoh spesifik.
\end{eulercomment}
\begin{eulerprompt}
>computeAngle(C,A,P), computeAngle(P,A,B)
\end{eulerprompt}
\begin{euleroutput}
  0.321750554397
  0.321750554397
\end{euleroutput}
\begin{eulercomment}
Sekarang kita hitung panjang garis bagi AP

Kami menggunakan teorema sinus dalam segitiga APC. Teorema ini
menyatakan bahwa

\end{eulercomment}
\begin{eulerformula}
\[
\frac{BC}{\sin(w_a)} = \frac{AC}{\sin(w_b)} = \frac{AB}{\sin(w_c)}
\]
\end{eulerformula}
\begin{eulercomment}
berlaku dalam segitiga apa pun. Kuadratkan, ini diterjemahkan ke dalam
apa yang disebut "spread law"

\end{eulercomment}
\begin{eulerformula}
\[
\frac{a}{s_a} = \frac{b}{s_b} = \frac{c}{s_b}
\]
\end{eulerformula}
\begin{eulercomment}
di mana a, b, c menunjukkan kuadrat jarak.

Karena spread CPA adalah 1-sa2, kita bisa mendapatkan bisa/1 =
b/(1-sa2) dan dapat menghitung bisa (kuadrat jarak dari garis bagi
sudut).
\end{eulercomment}
\begin{eulerprompt}
>&factor(ratsimp(b/(1-sa2))); bisa &= %; $bisa
\end{eulerprompt}
\begin{eulerformula}
\[
\frac{2\,b\,\left(b+a\right)}{\sqrt{b}\,\sqrt{b+a}+b+a}
\]
\end{eulerformula}
\begin{eulercomment}
Mari kita periksa rumus ini untuk nilai Mesir.
\end{eulercomment}
\begin{eulerprompt}
>sqrt(mxmeval("at(bisa,[a=3^2,b=4^2])")), distance(A,P)
\end{eulerprompt}
\begin{euleroutput}
  4.21637021356
  4.21637021356
\end{euleroutput}
\begin{eulercomment}
Kita juga dapat menghitungg P menggunakan rumus spread.
\end{eulercomment}
\begin{eulerprompt}
>py&=factor(ratsimp(sa2*bisa)); $py
\end{eulerprompt}
\begin{eulerformula}
\[
-\frac{b\,\left(\sqrt{b}\,\sqrt{b+a}-b-a\right)}{\sqrt{b}\,\sqrt{b+
 a}+b+a}
\]
\end{eulerformula}
\begin{eulercomment}
Nilainya sama dengan yang kita dapatkan dengan rumus trigonometri.
\end{eulercomment}
\begin{eulerprompt}
>sqrt(mxmeval("at(py,[a=3^2,b=4^2])"))
\end{eulerprompt}
\begin{euleroutput}
  1.33333333333
\end{euleroutput}
\eulersubheading{Chord Angle}
\begin{eulercomment}
Lihatlah situasi berikut ini.
\end{eulercomment}
\begin{eulerprompt}
>setPlotRange(1.2); ...
>color(1); plotCircle(circleWithCenter([0,0],1)); ...
>A:=[cos(1),sin(1)]; B:=[cos(2),sin(2)]; C:=[cos(6),sin(6)]; ...
>plotPoint(A,"A"); plotPoint(B,"B"); plotPoint(C,"C"); ...
>color(3); plotSegment(A,B,"c"); plotSegment(A,C,"b"); plotSegment(C,B,"a"); ...
>color(1); O:=[0,0];  plotPoint(O,"0"); ...
>plotSegment(A,O); plotSegment(B,O); plotSegment(C,O,"r"); ...
>insimg;
\end{eulerprompt}
\eulerimg{29}{images/EMT4Geometry_Amalia Intan Arvitasari_22305144026_Mat B-144.png}
\begin{eulercomment}
Kita dapat menggunakan Maxima untuk menyelesaikan rumus triple spread
untuk sudut-sudut di pusat O untuk r. Dengan demikian kita mendapatkan
rumus untuk jari-jari kuadrat dari pericircle dalam hal kuadran
sisisisinya.

Kali ini, Maxima menghasilkan beberapa angka nol yang rumit, yang kita
abaikan.
\end{eulercomment}
\begin{eulerprompt}
>&remvalue(a,b,c,r); // hapus nilai-nilai sebelumnya untuk perhitungan baru
>rabc &= rhs(solve(triplespread(spread(b,r,r),spread(a,r,r),spread(c,r,r)),r)[4]); $rabc
\end{eulerprompt}
\begin{eulerformula}
\[
-\frac{a\,b\,c}{c^2-2\,b\,c+a\,\left(-2\,c-2\,b\right)+b^2+a^2}
\]
\end{eulerformula}
\begin{eulercomment}
Kita dapat menjadikannya sebuah fungsi Euler.
\end{eulercomment}
\begin{eulerprompt}
>function periradius(a,b,c) &= rabc;
\end{eulerprompt}
\begin{eulercomment}
Mari kita periksa hasilnya untuk titik A, B, C.
\end{eulercomment}
\begin{eulerprompt}
>a:=quadrance(B,C); b:=quadrance(A,C); c:=quadrance(A,B);
\end{eulerprompt}
\begin{eulercomment}
Untuk radiusnya 1.
\end{eulercomment}
\begin{eulerprompt}
>periradius(a,b,c)
\end{eulerprompt}
\begin{euleroutput}
  1
\end{euleroutput}
\begin{eulercomment}
Faktanya adalah, bahwa spread CBA hanya bergantung pada b dan c. Ini
adalah teorema chord angle.
\end{eulercomment}
\begin{eulerprompt}
>$spread(b,a,c)*rabc | ratsimp
\end{eulerprompt}
\begin{eulerformula}
\[
\frac{b}{4}
\]
\end{eulerformula}
\begin{eulercomment}
Kenyataannya, spread adalah b/(4r), dan kita lihat bahwa sudut chord b
adalah setengah dari sudut tengah.
\end{eulercomment}
\begin{eulerprompt}
>$doublespread(b/(4*r))-spread(b,r,r) | ratsimp
\end{eulerprompt}
\begin{eulerformula}
\[
0
\]
\end{eulerformula}
\begin{eulercomment}
\begin{eulercomment}
\eulerheading{Contoh 6: Jarak Minimal pada Bidang}
\begin{eulercomment}
\end{eulercomment}
\eulersubheading{Preliminary remark}
\begin{eulercomment}
Fungsi pada sebuah titik M pada bidang, menetapkan jarak AM antara
titik tetap A dan M, memiliki garis-garis tingkat yang cukup
sederhana: lingkaran yang berpusat di A.
\end{eulercomment}
\begin{eulerprompt}
>&remvalue();
>A=[-1,-1];
>function d1(x,y):=sqrt((x-A[1])^2+(y-A[2])^2)
>fcontour("d1",xmin=-2,xmax=0,ymin=-2,ymax=0,hue=1, ...
>title="If you see ellipses, please set your window square"):
\end{eulerprompt}
\eulerimg{29}{images/EMT4Geometry_Amalia Intan Arvitasari_22305144026_Mat B-148.png}
\begin{eulercomment}
dan grafiknya juga cukup sederhana: bagian atas kerucut:
\end{eulercomment}
\begin{eulerprompt}
>plot3d("d1",xmin=-2,xmax=0,ymin=-2,ymax=0):
\end{eulerprompt}
\eulerimg{29}{images/EMT4Geometry_Amalia Intan Arvitasari_22305144026_Mat B-149.png}
\begin{eulercomment}
Tentu saja, nilai minimum 0 diperoleh di A.

\end{eulercomment}
\eulersubheading{Dua Titik}
\begin{eulercomment}
Sekarang kita lihat fungsi MA+MB di mana A dan B adalah dua titik
(tetap). Ini adalah "well-known fact" bahwa kurva level adalah elips,
titik fokusnya adalah A dan B; kecuali AB minimum yang konstan pada
segmen [AB]:
\end{eulercomment}
\begin{eulerprompt}
>B=[1,-1];
>function d2(x,y):=d1(x,y)+sqrt((x-B[1])^2+(y-B[2])^2)
>fcontour("d2",xmin=-2,xmax=2,ymin=-3,ymax=1,hue=1):
\end{eulerprompt}
\eulerimg{29}{images/EMT4Geometry_Amalia Intan Arvitasari_22305144026_Mat B-150.png}
\begin{eulercomment}
Grafiknya lebih menarik:
\end{eulercomment}
\begin{eulerprompt}
>plot3d("d2",xmin=-2,xmax=2,ymin=-3,ymax=1):
\end{eulerprompt}
\eulerimg{29}{images/EMT4Geometry_Amalia Intan Arvitasari_22305144026_Mat B-151.png}
\begin{eulercomment}
Pembatasan pada garis (AB) lebih terkenal:
\end{eulercomment}
\begin{eulerprompt}
>plot2d("abs(x+1)+abs(x-1)",xmin=-3,xmax=3):
\end{eulerprompt}
\eulerimg{29}{images/EMT4Geometry_Amalia Intan Arvitasari_22305144026_Mat B-152.png}
\begin{eulercomment}
\end{eulercomment}
\eulersubheading{Tiga Titik}
\begin{eulercomment}
Sekarang, hal-hal menjadi kurang sederhana: Hal ini sedikit kurang
dikenal bahwa MA+MB+MC mencapai titik minimum pada satu titik bidang,
tetapi untuk menentukannya tidak sesederhana itu:

1) Jika salah satu sudut segitiga ABC lebih dari 120° (katakanlah di
A), maka sudut minimum dicapai pada titik ini (katakanlah AB+AC).\\
Contoh:
\end{eulercomment}
\begin{eulerprompt}
>C=[-4,1];
>function d3(x,y):=d2(x,y)+sqrt((x-C[1])^2+(y-C[2])^2)
>plot3d("d3",xmin=-5,xmax=3,ymin=-4,ymax=4);
>insimg;
\end{eulerprompt}
\eulerimg{29}{images/EMT4Geometry_Amalia Intan Arvitasari_22305144026_Mat B-153.png}
\begin{eulerprompt}
>fcontour("d3",xmin=-4,xmax=1,ymin=-2,ymax=2,hue=1,title="The minimum is on A");
>P=(A_B_C_A)'; plot2d(P[1],P[2],add=1,color=12);
>insimg;
\end{eulerprompt}
\eulerimg{29}{images/EMT4Geometry_Amalia Intan Arvitasari_22305144026_Mat B-154.png}
\begin{eulercomment}
2) Tetapi jika semua sudut segitiga ABC kurang dari 120°, minimumnya
adalah pada titik F di bagian dalam segitiga, yang merupakan
satu-satunya titik yang melihat sisi-sisi ABC dengan sudut yang sama
(masing-masing 120°):
\end{eulercomment}
\begin{eulerprompt}
>C=[-0.5,1];
>plot3d("d3",xmin=-2,xmax=2,ymin=-2,ymax=2):
\end{eulerprompt}
\eulerimg{29}{images/EMT4Geometry_Amalia Intan Arvitasari_22305144026_Mat B-155.png}
\begin{eulerprompt}
>fcontour("d3",xmin=-2,xmax=2,ymin=-2,ymax=2,hue=1,title="The Fermat point");
>P=(A_B_C_A)'; plot2d(P[1],P[2],add=1,color=12);
>insimg;
\end{eulerprompt}
\eulerimg{29}{images/EMT4Geometry_Amalia Intan Arvitasari_22305144026_Mat B-156.png}
\begin{eulercomment}
Ini menjadi kegiatan yang menarik untuk merealisasikan gambar di atas
dengan perangkat lunak geometri; sebagai contoh, saya tahu sebuah
perangkat lunak yang ditulis dalam bahasa Java yang memiliki instruksi
"contour lines"...

Semua hal di atas telah ditemukan oleh seorang hakim Perancis bernama
Pierre de Fermat; ia menulis surat kepada para ahli dilet lainnya
seperti pendeta Marin Mersenne dan Blaise Pascal yang bekerja di
bagian pajak pendapatan. Jadi titik unik F sedemikian rupa sehingga
FA+FB+FC minimal, disebut titik Fermat dari segitiga. Namun tampaknya
beberapa tahun sebelumnya, Torriccelli dari Italia telah menemukan
titik ini sebelum Fermat menemukannya! Bagaimanapun juga, tradisi yang
berlaku adalah mencatat titik F ini...

\end{eulercomment}
\eulersubheading{Empat Titik}
\begin{eulercomment}
Langkah selanjutnya adalah menambahkan titik ke-4 D dan mencoba
meminimumkan MA+MB+MC+MD; misalkan Anda adalah operator TV kabel dan
ingin menemukan di bidang mana Anda harus meletakkan antena sehingga
Anda dapat mencakup empat desa dan menggunakan panjang kabel sesedikit
mungkin!
\end{eulercomment}
\begin{eulerprompt}
>D=[1,1];
>function d4(x,y):=d3(x,y)+sqrt((x-D[1])^2+(y-D[2])^2)
>plot3d("d4",xmin=-1.5,xmax=1.5,ymin=-1.5,ymax=1.5):
\end{eulerprompt}
\eulerimg{29}{images/EMT4Geometry_Amalia Intan Arvitasari_22305144026_Mat B-157.png}
\begin{eulerprompt}
>fcontour("d4",xmin=-1.5,xmax=1.5,ymin=-1.5,ymax=1.5,hue=1);
>P=(A_B_C_D)'; plot2d(P[1],P[2],points=1,add=1,color=12);
>insimg;
\end{eulerprompt}
\eulerimg{29}{images/EMT4Geometry_Amalia Intan Arvitasari_22305144026_Mat B-158.png}
\begin{eulercomment}
Masih ada nilai minimum dan tidak ada vertice A, B, C, maupun D.
\end{eulercomment}
\begin{eulerprompt}
>function f(x):=d4(x[1],x[2])
>neldermin("f",[0.2,0.2])
\end{eulerprompt}
\begin{euleroutput}
  [0.142858,  0.142857]
\end{euleroutput}
\begin{eulercomment}
Tampaknya dalam kasus ini, koordinat titik optimal adalah rasional
atau mendekati rasional...

Karena ABCD adalah sebuah bujur sangkar, kita berharap bahwa titik
optimalnya adalah pusat dari ABCD:
\end{eulercomment}
\begin{eulerprompt}
>C=[-1,1];
>plot3d("d4",xmin=-1,xmax=1,ymin=-1,ymax=1):
\end{eulerprompt}
\eulerimg{29}{images/EMT4Geometry_Amalia Intan Arvitasari_22305144026_Mat B-159.png}
\begin{eulerprompt}
>fcontour("d4",xmin=-1.5,xmax=1.5,ymin=-1.5,ymax=1.5,hue=1);
>P=(A_B_C_D)'; plot2d(P[1],P[2],add=1,color=12,points=1);
>insimg;
\end{eulerprompt}
\eulerimg{29}{images/EMT4Geometry_Amalia Intan Arvitasari_22305144026_Mat B-160.png}
\eulerheading{Contoh 7: Bola Dandelin dengan Povray}
\begin{eulercomment}
Anda dapat menjalankan demonstrasi ini, jika Anda telah menginstal
Povray, dan pvengine.exe pada jalur program.

Pertama, kita menghitung jari-jari bola.

Jika Anda melihat gambar di bawah ini, Anda dapat melihat bahwa kita
membutuhkan dua lingkaran yang menyentuh dua garis yang membentuk
kerucut, dan satu garis yang membentuk bidang yang memotong kerucut.

Kita gunakan file geometry.e dari Euler untuk ini.
\end{eulercomment}
\begin{eulerprompt}
>load geometry;
\end{eulerprompt}
\begin{eulercomment}
Pertama, dua garis yang membentuk kerucut.
\end{eulercomment}
\begin{eulerprompt}
>g1 &= lineThrough([0,0],[1,a])
\end{eulerprompt}
\begin{euleroutput}
  
                               [- a, 1, 0]
  
\end{euleroutput}
\begin{eulerprompt}
>g2 &= lineThrough([0,0],[-1,a])
\end{eulerprompt}
\begin{euleroutput}
  
                              [- a, - 1, 0]
  
\end{euleroutput}
\begin{eulercomment}
Kemudian baris ketiga.
\end{eulercomment}
\begin{eulerprompt}
>g &= lineThrough([-1,0],[1,1])
\end{eulerprompt}
\begin{euleroutput}
  
                               [- 1, 2, 1]
  
\end{euleroutput}
\begin{eulercomment}
Kita plotkan semuanya sejauh ini.
\end{eulercomment}
\begin{eulerprompt}
>setPlotRange(-1,1,0,2);
>color(black); plotLine(g(),"")
>a:=2; color(blue); plotLine(g1(),""), plotLine(g2(),""):
\end{eulerprompt}
\eulerimg{29}{images/EMT4Geometry_Amalia Intan Arvitasari_22305144026_Mat B-161.png}
\begin{eulercomment}
Sekarang, kita ambil titik umum pada sumbu y.
\end{eulercomment}
\begin{eulerprompt}
>P &= [0,u]
\end{eulerprompt}
\begin{euleroutput}
  
                                  [0, u]
  
\end{euleroutput}
\begin{eulercomment}
Hitung jarak ke g1. 
\end{eulercomment}
\begin{eulerprompt}
>d1 &= distance(P,projectToLine(P,g1)); $d1
\end{eulerprompt}
\begin{eulerformula}
\[
\sqrt{\left(\frac{a^2\,u}{a^2+1}-u\right)^2+\frac{a^2\,u^2}{\left(a
 ^2+1\right)^2}}
\]
\end{eulerformula}
\begin{eulercomment}
Compute the distance to g.
\end{eulercomment}
\begin{eulerprompt}
>d &= distance(P,projectToLine(P,g)); $d
\end{eulerprompt}
\begin{eulerformula}
\[
\sqrt{\left(\frac{u+2}{5}-u\right)^2+\frac{\left(2\,u-1\right)^2}{
 25}}
\]
\end{eulerformula}
\begin{eulercomment}
Dan temukan pusat kedua lingkaran yang jaraknya sama.
\end{eulercomment}
\begin{eulerprompt}
>sol &= solve(d1^2=d^2,u); $sol
\end{eulerprompt}
\begin{eulerformula}
\[
\left[ u=\frac{-\sqrt{5}\,\sqrt{a^2+1}+2\,a^2+2}{4\,a^2-1} , u=
 \frac{\sqrt{5}\,\sqrt{a^2+1}+2\,a^2+2}{4\,a^2-1} \right] 
\]
\end{eulerformula}
\begin{eulercomment}
Ada dua solusi.\\
Kami mengevaluasi solusi simbolis, dan menemukan kedua pusat, dan
kedua jarak.
\end{eulercomment}
\begin{eulerprompt}
>u := sol()
\end{eulerprompt}
\begin{euleroutput}
  [0.333333,  1]
\end{euleroutput}
\begin{eulerprompt}
>dd := d()
\end{eulerprompt}
\begin{euleroutput}
  [0.149071,  0.447214]
\end{euleroutput}
\begin{eulercomment}
Plot lingkaran ke dalam gambar.
\end{eulercomment}
\begin{eulerprompt}
>color(red);
>plotCircle(circleWithCenter([0,u[1]],dd[1]),"");
>plotCircle(circleWithCenter([0,u[2]],dd[2]),"");
>insimg;
\end{eulerprompt}
\eulerimg{29}{images/EMT4Geometry_Amalia Intan Arvitasari_22305144026_Mat B-165.png}
\eulersubheading{Plot dengan Povray}
\begin{eulercomment}
Selanjutnya kita plotkan semuanya dengan Povray. Perhatikan bahwa Anda
mengubah perintah apa pun dalam urutan perintah Povray berikut ini,
dan jalankan kembali semua perintah dengan Shift-Return.

Pertama, kita memuat fungsi povray.
\end{eulercomment}
\begin{eulerprompt}
>load povray;
>defaultpovray="C:\(\backslash\)Program Files\(\backslash\)POV-Ray\(\backslash\)v3.7\(\backslash\)bin\(\backslash\)pvengine.exe"
\end{eulerprompt}
\begin{euleroutput}
  C:\(\backslash\)Program Files\(\backslash\)POV-Ray\(\backslash\)v3.7\(\backslash\)bin\(\backslash\)pvengine.exe
\end{euleroutput}
\begin{eulercomment}
Kita siapkan scene dengan tepat.
\end{eulercomment}
\begin{eulerprompt}
>povstart(zoom=11,center=[0,0,0.5],height=10°,angle=140°);
\end{eulerprompt}
\begin{eulercomment}
Selanjutnya kita menulis dua bola ke file Povray.
\end{eulercomment}
\begin{eulerprompt}
>writeln(povsphere([0,0,u[1]],dd[1],povlook(red)));
>writeln(povsphere([0,0,u[2]],dd[2],povlook(red)));
\end{eulerprompt}
\begin{eulercomment}
Dan kecurut, tranparan.
\end{eulercomment}
\begin{eulerprompt}
>writeln(povcone([0,0,0],0,[0,0,a],1,povlook(lightgray,1)));
\end{eulerprompt}
\begin{eulercomment}
Kita menghasilkan bidang yang terbatas pada kerucut.
\end{eulercomment}
\begin{eulerprompt}
>gp=g();
>pc=povcone([0,0,0],0,[0,0,a],1,"");
>vp=[gp[1],0,gp[2]]; dp=gp[3];
>writeln(povplane(vp,dp,povlook(blue,0.5),pc));
\end{eulerprompt}
\begin{eulercomment}
Sekarang kita menghasilkan dua titik pada lingkaran, di mana bola
menyentuh kerucut.
\end{eulercomment}
\begin{eulerprompt}
>function turnz(v) := return [-v[2],v[1],v[3]]
>P1=projectToLine([0,u[1]],g1()); P1=turnz([P1[1],0,P1[2]]);
>writeln(povpoint(P1,povlook(yellow)));
>P2=projectToLine([0,u[2]],g1()); P2=turnz([P2[1],0,P2[2]]);
>writeln(povpoint(P2,povlook(yellow)));
\end{eulerprompt}
\begin{eulercomment}
Kemudian, kita menghasilkan dua titik di mana bola-bola tersebut
menyentuh bidang. Ini adalah fokus elips.
\end{eulercomment}
\begin{eulerprompt}
>P3=projectToLine([0,u[1]],g()); P3=[P3[1],0,P3[2]];
>writeln(povpoint(P3,povlook(yellow)));
>P4=projectToLine([0,u[2]],g()); P4=[P4[1],0,P4[2]];
>writeln(povpoint(P4,povlook(yellow)));
\end{eulerprompt}
\begin{eulercomment}
Selanjutnya kita menghitung perpotongan P1P2 dengan bidang.
\end{eulercomment}
\begin{eulerprompt}
>t1=scalp(vp,P1)-dp; t2=scalp(vp,P2)-dp; P5=P1+t1/(t1-t2)*(P2-P1);
>writeln(povpoint(P5,povlook(yellow)));
\end{eulerprompt}
\begin{eulercomment}
Kita menghubungkan titik-titik dengan segmen garis.
\end{eulercomment}
\begin{eulerprompt}
>writeln(povsegment(P1,P2,povlook(yellow)));
>writeln(povsegment(P5,P3,povlook(yellow)));
>writeln(povsegment(P5,P4,povlook(yellow)));
\end{eulerprompt}
\begin{eulercomment}
Sekarang, kita menghasilkan pita abu-abu, di mana bola-bola menyentuh
kerucut.
\end{eulercomment}
\begin{eulerprompt}
>pcw=povcone([0,0,0],0,[0,0,a],1.01);
>pc1=povcylinder([0,0,P1[3]-defaultpointsize/2],[0,0,P1[3]+defaultpointsize/2],1);
>writeln(povintersection([pcw,pc1],povlook(gray)));
>pc2=povcylinder([0,0,P2[3]-defaultpointsize/2],[0,0,P2[3]+defaultpointsize/2],1);
>writeln(povintersection([pcw,pc2],povlook(gray)));
\end{eulerprompt}
\begin{eulercomment}
Mulai program Povray.
\end{eulercomment}
\begin{eulerprompt}
>povend();
\end{eulerprompt}
\eulerimg{20}{images/EMT4Geometry_Amalia Intan Arvitasari_22305144026_Mat B-166.png}
\begin{eulercomment}
Untuk mendapatkan Anaglyph ini, kita perlu memasukkan semuanya ke
dalam fungsi scene. Fungsi ini akan digunakan dua kali nanti.
\end{eulercomment}
\begin{eulerprompt}
>function scene () ...
\end{eulerprompt}
\begin{eulerudf}
  global a,u,dd,g,g1,defaultpointsize;
  writeln(povsphere([0,0,u[1]],dd[1],povlook(red)));
  writeln(povsphere([0,0,u[2]],dd[2],povlook(red)));
  writeln(povcone([0,0,0],0,[0,0,a],1,povlook(lightgray,1)));
  gp=g();
  pc=povcone([0,0,0],0,[0,0,a],1,"");
  vp=[gp[1],0,gp[2]]; dp=gp[3];
  writeln(povplane(vp,dp,povlook(blue,0.5),pc));
  P1=projectToLine([0,u[1]],g1()); P1=turnz([P1[1],0,P1[2]]);
  writeln(povpoint(P1,povlook(yellow)));
  P2=projectToLine([0,u[2]],g1()); P2=turnz([P2[1],0,P2[2]]);
  writeln(povpoint(P2,povlook(yellow)));
  P3=projectToLine([0,u[1]],g()); P3=[P3[1],0,P3[2]];
  writeln(povpoint(P3,povlook(yellow)));
  P4=projectToLine([0,u[2]],g()); P4=[P4[1],0,P4[2]];
  writeln(povpoint(P4,povlook(yellow)));
  t1=scalp(vp,P1)-dp; t2=scalp(vp,P2)-dp; P5=P1+t1/(t1-t2)*(P2-P1);
  writeln(povpoint(P5,povlook(yellow)));
  writeln(povsegment(P1,P2,povlook(yellow)));
  writeln(povsegment(P5,P3,povlook(yellow)));
  writeln(povsegment(P5,P4,povlook(yellow)));
  pcw=povcone([0,0,0],0,[0,0,a],1.01);
  pc1=povcylinder([0,0,P1[3]-defaultpointsize/2],[0,0,P1[3]+defaultpointsize/2],1);
  writeln(povintersection([pcw,pc1],povlook(gray)));
  pc2=povcylinder([0,0,P2[3]-defaultpointsize/2],[0,0,P2[3]+defaultpointsize/2],1);
  writeln(povintersection([pcw,pc2],povlook(gray)));
  endfunction
\end{eulerudf}
\begin{eulercomment}
Anda memerlukan kacamata merah/cyan untuk mengapresiasi efek berikut
ini.
\end{eulercomment}
\begin{eulerprompt}
>povanaglyph("scene",zoom=11,center=[0,0,0.5],height=10°,angle=140°);
\end{eulerprompt}
\eulerimg{19}{images/EMT4Geometry_Amalia Intan Arvitasari_22305144026_Mat B-167.png}
\eulerheading{Contoh 8: Geometri Bumi}
\begin{eulercomment}
Dalam buku catatan ini, kita ingin melakukan beberapa komputasi bola.
Fungsi-fungsi tersebut terdapat pada file "spherical.e" di dalam
folder examples. Kita perlu memuat file tersebut terlebih dahulu.
\end{eulercomment}
\begin{eulerprompt}
>load "spherical.e";
\end{eulerprompt}
\begin{eulercomment}
Untuk memasukkan posisi geografis, kita menggunakan vektor dengan dua
koordinat dalam radian (utara dan timur,  nilai negatif untuk selatan
dan barat). Berikut ini adalah koordinat untuk Kampus FMIPA UNY.
\end{eulercomment}
\begin{eulerprompt}
>FMIPA=[rad(-7,-46.467),rad(110,23.05)]
\end{eulerprompt}
\begin{euleroutput}
  [-0.13569,  1.92657]
\end{euleroutput}
\begin{eulercomment}
Anda dapat mencetak posisi ini dengan sposprint (spherical position
print).
\end{eulercomment}
\begin{eulerprompt}
>sposprint(FMIPA) // posisi garis lintang dan garis bujur FMIPA UNY
\end{eulerprompt}
\begin{euleroutput}
  S 7°46.467' E 110°23.050'
\end{euleroutput}
\begin{eulercomment}
Mari kita tambahkan dua kota lagi, Solo dan Semarang.
\end{eulercomment}
\begin{eulerprompt}
>Solo=[rad(-7,-34.333),rad(110,49.683)]; Semarang=[rad(-6,-59.05),rad(110,24.533)];
>sposprint(Solo), sposprint(Semarang),
\end{eulerprompt}
\begin{euleroutput}
  S 7°34.333' E 110°49.683'
  S 6°59.050' E 110°24.533'
\end{euleroutput}
\begin{eulercomment}
Pertama, kita menghitung vektor dari satu titik ke titik lainnya pada
bola ideal. Vektor ini adalah [heading, distance] dalam radian. Untuk
menghitung jarak di bumi, kita kalikan dengan jari-jari bumi pada
garis lintang 7°.
\end{eulercomment}
\begin{eulerprompt}
>br=svector(FMIPA,Solo); degprint(br[1]), br[2]*rearth(7°)->km // perkiraan jarak FMIPA-Solo
\end{eulerprompt}
\begin{euleroutput}
  65°20'26.60''
  53.8945384608
\end{euleroutput}
\begin{eulercomment}
This is a good approximation. The following routines use even better
approximations. On such a short distance the result is almost the
same.

Ini adalah aproksimasi yang baik. Rutinitas berikut ini menggunakan
aproksimasi yang lebih baik lagi. Pada jarak yang pendek, hasilnya
hampir sama.
\end{eulercomment}
\begin{eulerprompt}
>esdist(FMIPA,Semarang)->" km" // perkiraan jarak FMIPA-Semarang
\end{eulerprompt}
\begin{euleroutput}
  Commands must be separated by semicolon or comma!
  Found:  // perkiraan jarak FMIPA-Semarang (character 32)
  You can disable this in the Options menu.
  Error in:
  esdist(FMIPA,Semarang)->" km" // perkiraan jarak FMIPA-Semaran ...
                               ^
\end{euleroutput}
\begin{eulercomment}
Terdapat fungsi untuk heading, dengan mempertimbangkan bentuk elips
bumi. Sekali lagi, kami mencetak dengan cara yang canggih.
\end{eulercomment}
\begin{eulerprompt}
>sdegprint(esdir(FMIPA,Solo))
\end{eulerprompt}
\begin{euleroutput}
       65.34°
\end{euleroutput}
\begin{eulercomment}
Sudut segitiga melebihi 180° pada bola.
\end{eulercomment}
\begin{eulerprompt}
>asum=sangle(Solo,FMIPA,Semarang)+sangle(FMIPA,Solo,Semarang)+sangle(FMIPA,Semarang,Solo); degprint(asum)
\end{eulerprompt}
\begin{euleroutput}
  180°0'10.77''
\end{euleroutput}
\begin{eulercomment}
Ini bisa digunakan untuk menghitung luas area segitiga. Catatan: Untuk
segitiga kecil, cara ini tidak akurat karena kesalahan pengurangan
dalam asum-pi.
\end{eulercomment}
\begin{eulerprompt}
>(asum-pi)*rearth(48°)^2->"km^2", // perkiraan luas segitiga FMIPA-Solo-Semarang
\end{eulerprompt}
\begin{euleroutput}
  2116.02948749km^2
\end{euleroutput}
\begin{eulercomment}
Ada sebuah fungsi untuk hal ini, yang menggunakan garis lintang
rata-rata segitiga untuk menghitung radius bumi, dan menangani
kesalahan pembulatan untuk segitiga yang sangat kecil.
\end{eulercomment}
\begin{eulerprompt}
>esarea(Solo,FMIPA,Semarang)->" km^2", //perkiraan yang sama dengan fungsi esarea()
\end{eulerprompt}
\begin{euleroutput}
  2123.64310526 km^2
\end{euleroutput}
\begin{eulercomment}
Kita juga dapat menambahkan vektor ke posisi. Sebuah vektor berisi
heading dan distance, keduanya dalam radian. Untuk mendapatkan sebuah
vektor, kita menggunakan svector. Untuk menambahkan sebuah vektor ke
sebuah posisi, kita menggunakan saddvector.
\end{eulercomment}
\begin{eulerprompt}
>v=svector(FMIPA,Solo); sposprint(saddvector(FMIPA,v)), sposprint(Solo),
\end{eulerprompt}
\begin{euleroutput}
  S 7°34.333' E 110°49.683'
  S 7°34.333' E 110°49.683'
\end{euleroutput}
\begin{eulercomment}
Fungsi-fungsi ini mengasumsikan bola yang ideal. Hal yang sama di
bumi.
\end{eulercomment}
\begin{eulerprompt}
>sposprint(esadd(FMIPA,esdir(FMIPA,Solo),esdist(FMIPA,Solo))), sposprint(Solo),
\end{eulerprompt}
\begin{euleroutput}
  S 7°34.333' E 110°49.683'
  S 7°34.333' E 110°49.683'
\end{euleroutput}
\begin{eulercomment}
Mari kita beralih ke contoh yang lebih besar, Tugu Jogja dan Monas
Jakarta (menggunakan Google Earth untuk mencari koordinatnya).
\end{eulercomment}
\begin{eulerprompt}
>Tugu=[-7.7833°,110.3661°]; Monas=[-6.175°,106.811944°];
>sposprint(Tugu), sposprint(Monas)
\end{eulerprompt}
\begin{euleroutput}
  S 7°46.998' E 110°21.966'
  S 6°10.500' E 106°48.717'
\end{euleroutput}
\begin{eulercomment}
Menurut Google Earth, jaraknya adalah 429,66 km. Kami mendapatkan
aproksimasi yang bagus.
\end{eulercomment}
\begin{eulerprompt}
>esdist(Tugu,Monas)->"km", // perkiraan jarak Tugu Jogja - Monas Jakarta
\end{eulerprompt}
\begin{euleroutput}
  431.565659488km
\end{euleroutput}
\begin{eulercomment}
Judulnya sama dengan yang dihitung di Google Earth.
\end{eulercomment}
\begin{eulerprompt}
>degprint(esdir(Tugu,Monas))
\end{eulerprompt}
\begin{euleroutput}
  294°17'2.85''
\end{euleroutput}
\begin{eulercomment}
Namun demikian, kita tidak lagi mendapatkan posisi target yang tepat,
jika kita menambahkan arah dan jarak ke posisi semula. Hal ini
terjadi, karena kita tidak menghitung fungsi inversi secara tepat,
tetapi mengambil perkiraan radius bumi di sepanjang jalur.
\end{eulercomment}
\begin{eulerprompt}
>sposprint(esadd(Tugu,esdir(Tugu,Monas),esdist(Tugu,Monas)))
\end{eulerprompt}
\begin{euleroutput}
  S 6°10.500' E 106°48.717'
\end{euleroutput}
\begin{eulercomment}
Namun demikian, kesalahannya tidak begitu besar.
\end{eulercomment}
\begin{eulerprompt}
>sposprint(Monas),
\end{eulerprompt}
\begin{euleroutput}
  S 6°10.500' E 106°48.717'
\end{euleroutput}
\begin{eulercomment}
Tentu saja, kita tidak bisa berlayar dengan arah yang sama dari satu
tujuan ke tujuan lainnya, jika kita ingin mengambil jalur terpendek.
Bayangkan, Anda terbang ke arah NE mulai dari titik mana pun di bumi.
Kemudian Anda akan berputar ke kutub utara. Lingkaran besar tidak
mengikuti arah yang konstan!

Perhitungan berikut ini menunjukkan bahwa kita melenceng jauh dari
tujuan yang benar, jika kita menggunakan arah yang sama selama
perjalanan.
\end{eulercomment}
\begin{eulerprompt}
>dist=esdist(Tugu,Monas); hd=esdir(Tugu,Monas);
\end{eulerprompt}
\begin{eulercomment}
Sekarang kita tambahkan 10 kali sepersepuluh dari jarak tersebut,
dengan menggunakan arah ke Monas, kita sampai di Tugu.
\end{eulercomment}
\begin{eulerprompt}
>p=Tugu; loop 1 to 10; p=esadd(p,hd,dist/10); end;
\end{eulerprompt}
\begin{eulercomment}
Hasilnya jauh berbeda.
\end{eulercomment}
\begin{eulerprompt}
>sposprint(p), skmprint(esdist(p,Monas))
\end{eulerprompt}
\begin{euleroutput}
  S 6°11.250' E 106°48.372'
       1.529km
\end{euleroutput}
\begin{eulercomment}
Sebagai contoh lain, mari kita ambil dua titik di bumi pada garis
lintang yang sama.
\end{eulercomment}
\begin{eulerprompt}
>P1=[30°,10°]; P2=[30°,50°];
\end{eulerprompt}
\begin{eulercomment}
Jalur terpendek dari P1 ke P2 bukanlah lingkaran lintang 30°, tetapi
jalur yang lebih pendek yang dimulai 10° lebih jauh ke utara di P1.
\end{eulercomment}
\begin{eulerprompt}
>sdegprint(esdir(P1,P2))
\end{eulerprompt}
\begin{euleroutput}
       79.69°
\end{euleroutput}
\begin{eulercomment}
Namun, jika kita mengikuti pembacaan kompas ini, kita akan berputar ke
kutub utara! Jadi, kita harus menyesuaikan arah kita di sepanjang
jalan. Untuk tujuan kasar, kita sesuaikan pada 1/10 dari jarak total.
\end{eulercomment}
\begin{eulerprompt}
>p=P1;  dist=esdist(P1,P2); ...
>  loop 1 to 10; dir=esdir(p,P2); sdegprint(dir), p=esadd(p,dir,dist/10); end;
\end{eulerprompt}
\begin{euleroutput}
       79.69°
       81.67°
       83.71°
       85.78°
       87.89°
       90.00°
       92.12°
       94.22°
       96.29°
       98.33°
\end{euleroutput}
\begin{eulercomment}
Jaraknya tidak tepat, karena kita akan menambahkan sedikit kesalahan,
jika kita mengikuti arah yang sama terlalu lama.
\end{eulercomment}
\begin{eulerprompt}
>skmprint(esdist(p,P2))
\end{eulerprompt}
\begin{euleroutput}
       0.203km
\end{euleroutput}
\begin{eulercomment}
Kita mendapatkan aproksimasi yang baik, jika kita menyesuaikan arah
setiap 1/100 dari total jarak dari Tugu ke Monas.
\end{eulercomment}
\begin{eulerprompt}
>p=Tugu; dist=esdist(Tugu,Monas); ...
>  loop 1 to 100; p=esadd(p,esdir(p,Monas),dist/100); end;
>skmprint(esdist(p,Monas))
\end{eulerprompt}
\begin{euleroutput}
       0.000km
\end{euleroutput}
\begin{eulercomment}
Untuk keperluan navigasi, kita bisa mendapatkan urutan posisi GPS di
sepanjang Bundaran HI menuju Monas dengan fungsi navigate.
\end{eulercomment}
\begin{eulerprompt}
>load spherical; v=navigate(Tugu,Monas,10); ...
>  loop 1 to rows(v); sposprint(v[#]), end;
\end{eulerprompt}
\begin{euleroutput}
  S 7°46.998' E 110°21.966'
  S 7°37.422' E 110°0.573'
  S 7°27.829' E 109°39.196'
  S 7°18.219' E 109°17.834'
  S 7°8.592' E 108°56.488'
  S 6°58.948' E 108°35.157'
  S 6°49.289' E 108°13.841'
  S 6°39.614' E 107°52.539'
  S 6°29.924' E 107°31.251'
  S 6°20.219' E 107°9.977'
  S 6°10.500' E 106°48.717'
\end{euleroutput}
\begin{eulercomment}
Kita menulis sebuah fungsi, yang memetakan bumi, dua posisi, dan
posisi di antaranya.
\end{eulercomment}
\begin{eulerprompt}
>function testplot ...
\end{eulerprompt}
\begin{eulerudf}
  useglobal;
  plotearth;
  plotpos(Tugu,"Tugu Jogja"); plotpos(Monas,"Tugu Monas");
  plotposline(v);
  endfunction
\end{eulerudf}
\begin{eulercomment}
Sekarang, plotkan semuanya.
\end{eulercomment}
\begin{eulerprompt}
>plot3d("testplot",angle=25, height=6,>own,>user,zoom=4):
\end{eulerprompt}
\eulerimg{29}{images/EMT4Geometry_Amalia Intan Arvitasari_22305144026_Mat B-168.png}
\begin{eulercomment}
Atau gunakan plot3d untuk mendapatkan tampilan anaglyph. Ini terlihat
sangat bagus dengan kacamata merah/cyan.
\end{eulercomment}
\begin{eulerprompt}
>plot3d("testplot",angle=25,height=6,distance=5,own=1,anaglyph=1,zoom=4):
\end{eulerprompt}
\eulerimg{29}{images/EMT4Geometry_Amalia Intan Arvitasari_22305144026_Mat B-169.png}
\eulerheading{Latihan}
\begin{eulercomment}
1. Gambarlah segi-n beraturan jika diketahui titik pusat O, n, dan
jarak titik pusat ke titik-titik sudut segi-n tersebut (jari-jari
lingkaran luar segi-n), r.

Petunjuk:

- Besar sudut pusat yang menghadap masing-masing sisi segi-n adalah
(360/n).\\
- Titik-titik sudut segi-n merupakan perpotongan lingkaran luar segi-n
dan garis-garis yang melalui pusat dan saling membentuk sudut sebesar
kelipatan (360/n).\\
- Untuk n ganjil, pilih salah satu titik sudut adalah di atas.\\
- Untuk n genap, pilih 2 titik di kanan dan kiri lurus dengan titik
pusat.\\
- Anda dapat menggambar segi-3, 4, 5, 6, 7, dst beraturan.

2. Gambarlah suatu parabola yang melalui 3 titik yang diketahui.

Petunjuk:\\
- Misalkan persamaan parabolanya y= ax\textasciicircum{}2+bx+c.\\
- Substitusikan koordinat titik-titik yang diketahui ke persamaan
tersebut.\\
- Selesaikan SPL yang terbentuk untuk mendapatkan nilai-nilai a, b, c.

3. Gambarlah suatu segi-4 yang diketahui keempat titik sudutnya,
misalnya A, B, C, D.\\
\end{eulercomment}
\begin{eulerttcomment}
   - Tentukan apakah segi-4 tersebut merupakan segi-4 garis singgung
\end{eulerttcomment}
\begin{eulercomment}
(sisinya-sisintya merupakan garis singgung lingkaran yang sama yakni
lingkaran dalam segi-4 tersebut).\\
\end{eulercomment}
\begin{eulerttcomment}
   - Suatu segi-4 merupakan segi-4 garis singgung apabila keempat
\end{eulerttcomment}
\begin{eulercomment}
garis bagi sudutnya bertemu di satu titik.\\
\end{eulercomment}
\begin{eulerttcomment}
   - Jika segi-4 tersebut merupakan segi-4 garis singgung, gambar
\end{eulerttcomment}
\begin{eulercomment}
lingkaran dalamnya.\\
\end{eulercomment}
\begin{eulerttcomment}
   - Tunjukkan bahwa syarat suatu segi-4 merupakan segi-4 garis
\end{eulerttcomment}
\begin{eulercomment}
singgung apabila hasil kali panjang sisi-sisi yang berhadapan sama.

4. Gambarlah suatu ellips jika diketahui kedua titik fokusnya,
misalnya P dan Q. Ingat ellips dengan fokus P dan Q adalah tempat
kedudukan titik-titik yang jumlah jarak ke P dan ke Q selalu sama
(konstan).

5. Gambarlah suatu hiperbola jika diketahui kedua titik fokusnya,
misalnya P dan Q. Ingat ellips dengan fokus P dan Q adalah tempat
kedudukan titik-titik yang selisih jarak ke P dan ke Q selalu sama
(konstan).\\
\end{eulercomment}
\eulersubheading{}
\begin{eulercomment}
Jawab:\\
1.
\end{eulercomment}
\begin{eulerprompt}
>load geometry
\end{eulerprompt}
\begin{euleroutput}
  Numerical and symbolic geometry.
\end{euleroutput}
\begin{eulerprompt}
>setPlotRange(-3.5,3.5,-3.5,3.5);
>O=[0,0]; plotPoint(O,"O");
>A=[-2,-1]; plotPoint(A,"A");
>B=[2,-1]; plotPoint(B,"B");
>C=[0,2*3^0.5-1]; plotPoint(A,"A");
>plotSegment(A,B,"c");
>plotSegment(B,C,"a");
>plotSegment(A,C,"b");
>aspect(1):
\end{eulerprompt}
\eulerimg{29}{images/EMT4Geometry_Amalia Intan Arvitasari_22305144026_Mat B-170.png}
\begin{eulerprompt}
>c=circleThrough(A,B,C):
\end{eulerprompt}
\eulerimg{29}{images/EMT4Geometry_Amalia Intan Arvitasari_22305144026_Mat B-171.png}
\begin{eulerprompt}
>R=getCircleRadius(c);
>O=getCircleCenter(c)
\end{eulerprompt}
\begin{euleroutput}
  [0,  0.154701]
\end{euleroutput}
\begin{eulerprompt}
>plotCircle(c,"Lingkaran luar segitiga ABC"):
\end{eulerprompt}
\eulerimg{29}{images/EMT4Geometry_Amalia Intan Arvitasari_22305144026_Mat B-172.png}
\begin{eulercomment}
2.
\end{eulercomment}
\begin{eulerprompt}
>setPlotRange(5); 
>K=[-4,0];  L=[4,0] ; M=[0,2];
>plotPoint(K,"K"); plotPoint(L,"L"); plotPoint(M,"M"):
\end{eulerprompt}
\eulerimg{29}{images/EMT4Geometry_Amalia Intan Arvitasari_22305144026_Mat B-173.png}
\begin{eulerprompt}
>sol &= solve([16*a+8*b=-c,16*a-8*b=-c,c=2],[a,b,c])
\end{eulerprompt}
\begin{euleroutput}
  
                                1
                        [[a = - -, b = 0, c = 2]]
                                8
  
\end{euleroutput}
\begin{eulerprompt}
>function y&=-1/8*(x^2)-0*x+2
\end{eulerprompt}
\begin{euleroutput}
  
                                       2
                                      x
                                  2 - --
                                      8
  
\end{euleroutput}
\begin{eulerprompt}
>plot2d("-1/8*(x^2)-0*x+2",-5,5,-5,5); ...
>plotPoint(K,"K"); plotPoint(L,"L"); plotPoint(M,"M"): 
\end{eulerprompt}
\eulerimg{29}{images/EMT4Geometry_Amalia Intan Arvitasari_22305144026_Mat B-174.png}
\begin{eulercomment}
3
\end{eulercomment}
\begin{eulerprompt}
>setPlotRange(5);
>A=[-3,-3];  B=[3,-3] ; C=[3,3]; D=[-3,3];
>plotPoint(A,"A"); plotPoint(B,"B"); plotPoint(C,"C"); plotPoint(D,"D");
>plotSegment(A,B,""); plotSegment(B,C,""); plotSegment(C,D,""); plotSegment(D,A,""):
\end{eulerprompt}
\eulerimg{29}{images/EMT4Geometry_Amalia Intan Arvitasari_22305144026_Mat B-175.png}
\begin{eulerprompt}
>l=angleBisector(A,B,C);
>g=angleBisector(B,C,D);
>P=lineIntersection(l,g)
\end{eulerprompt}
\begin{euleroutput}
  [0,  0]
\end{euleroutput}
\begin{eulerprompt}
>color(5); plotLine(l); plotLine(g); color(1);
>plotPoint(P,"P"):
\end{eulerprompt}
\eulerimg{29}{images/EMT4Geometry_Amalia Intan Arvitasari_22305144026_Mat B-176.png}
\begin{eulerprompt}
>r=norm(P-projectToLine(P,lineThrough(A,B)))
\end{eulerprompt}
\begin{euleroutput}
  3
\end{euleroutput}
\begin{eulerprompt}
>plotCircle(circleWithCenter(P,r),"Lingkaran dalam segiempat ABC"):
\end{eulerprompt}
\eulerimg{29}{images/EMT4Geometry_Amalia Intan Arvitasari_22305144026_Mat B-177.png}
\begin{eulercomment}
Dari gambar di atas, terlihat bahwa sisi-sisinya merupakan garis
singgung lingkaran yang sama yaitu lingkaran dalam segiempat.

Kemudian akan ditunjukkan bahwa hasil kali panjang sisi-sisi yang
berhadapan sama.
\end{eulercomment}
\begin{eulerprompt}
>AB=norm(A-B) // panjang sisi AB
\end{eulerprompt}
\begin{euleroutput}
  6
\end{euleroutput}
\begin{eulerprompt}
>BC=norm(B-C) // panjang sisi ABC
\end{eulerprompt}
\begin{euleroutput}
  6
\end{euleroutput}
\begin{eulerprompt}
>CD=norm(C-D) // panjang sisi CD
\end{eulerprompt}
\begin{euleroutput}
  6
\end{euleroutput}
\begin{eulerprompt}
>DA=norm(D-A) // panjang sisi DA
\end{eulerprompt}
\begin{euleroutput}
  6
\end{euleroutput}
\begin{eulerprompt}
>AB.CD
\end{eulerprompt}
\begin{euleroutput}
  36
\end{euleroutput}
\begin{eulerprompt}
>DA.BC
\end{eulerprompt}
\begin{euleroutput}
  36
\end{euleroutput}
\begin{eulercomment}
4. 
\end{eulercomment}
\begin{eulerprompt}
>P=[-1,2]; Q=[1,2];
>function d1(x,y):=sqrt((x-P[1])^2+(y-P[2])^2)
>function d2(x,y):=d1(x,y)+sqrt((x-Q[1])^2+(y-Q[2])^2)
>fcontour("d2",xmin=-2,xmax=2,ymin=0,ymax=4,hue=1):
\end{eulerprompt}
\eulerimg{29}{images/EMT4Geometry_Amalia Intan Arvitasari_22305144026_Mat B-178.png}
\begin{eulerprompt}
>plot3d("d2",xmin=-2,xmax=2,ymin=0,ymax=4,hue=1):
\end{eulerprompt}
\eulerimg{29}{images/EMT4Geometry_Amalia Intan Arvitasari_22305144026_Mat B-179.png}
\begin{eulercomment}
Batasan garis PQ
\end{eulercomment}
\begin{eulerprompt}
>plot2d("abs(x+1)+abs(x-1)",xmin=-3,xmax=3):
\end{eulerprompt}
\eulerimg{29}{images/EMT4Geometry_Amalia Intan Arvitasari_22305144026_Mat B-180.png}
\begin{eulercomment}
5.
\end{eulercomment}
\begin{eulerprompt}
>P=[-1,2]; Q=[1,2];
>function d3(x,y):=sqrt((x-P[1])^2+(y-P[2])^2)
>function d4(x,y):=d3(x,y)+sqrt((x-Q[1])^2+(y-Q[2])^2
>fcontour("d3",xmin=-4,xmax=2,ymin=0,ymax=4,hue=1):
\end{eulerprompt}
\eulerimg{29}{images/EMT4Geometry_Amalia Intan Arvitasari_22305144026_Mat B-181.png}
\begin{eulerprompt}
>plot3d("d3",xmin=-2,xmax=2,ymin=0,ymax=4,hue=1):
\end{eulerprompt}
\eulerimg{29}{images/EMT4Geometry_Amalia Intan Arvitasari_22305144026_Mat B-182.png}
\begin{eulerprompt}
>plot2d("abs(x+1)+abs(x-1)",xmin=-3,xmax=3):
\end{eulerprompt}
\eulerimg{29}{images/EMT4Geometry_Amalia Intan Arvitasari_22305144026_Mat B-183.png}
\end{eulernotebook}

\chapter{EMT Untuk Statistika}
\begin{eulernotebook}
\begin{eulercomment}
Nama : Amalia Intan Arvitasari\\
Kelas: Matematika B\\
NIM  : 22305144026

\end{eulercomment}
\eulersubheading{Sub Topik 5 : Perhitungan Analisis Data Statistika Deskriptif}
\begin{eulercomment}
Pada sub topik 5 part 1 ini akan dibahas mengenai:\\
1. Penjelasan umum mengenai statistika deskriptif\\
2. Ruang lingkup kajian pada analisis statistika deskriptif yang
meliputi distribusi frekuensi, mean, median, dan modus.

\end{eulercomment}
\eulersubheading{Penjelasan Umum Mengenai statistika Deskriptif}
\begin{eulercomment}
Statistika deskriptif yaitu statistik yang mempelajari tata cara
mengumpulkan, menyusun, menyajikan, dan menganalisis data yang
berwujud angka agar dapat memberikan gambaran yang teratur, ringkas,
dan jelas mengenai suatu gejala atau keadaan peristiwa. Analisis ini
hanya berupa akumulasi data dasar dalam bentuk deskripsi semata dalam
arti tidak mencari atau menerangkan saling hubungan, menguji
hipotesis, membuat ramalan atau melakukan penarikan kesimpulan.
Analisis data yang tergolong statistik deskriptif terdiri dari
distribusi frekuensi, ukuran pemusatan data(mean,median,modus), ukuran
letak(kuartil,desil,persentil), ukuran dispersi(jangkauan atau
rentang,varians,simpangan baku), dan teknis statistik lain yang
bertujuan hanya untuk mengetahui gambaran atau kecenderungan data
tanpa bermaksud melakukan generalisasi.

\end{eulercomment}
\eulersubheading{Distribusi Frekuensi}
\begin{eulercomment}
Distribusi frekuensi merupakan ruang lingkup kajian pada analisis
statistik deskriptif. Distribusi frekuensi adalah alat penyajian data
berbentuk kolom dan lajur (tabel)yang di dalamnya dibuat angka yang
menggambarkan pancaran frekuensi dari variabel yang sedang menjadi
objek. Unsur-unsur distribusi frekuensi yaitu kelas(kelompok nilai
data yang ditulis dalam bentuk interval),batas bawah(jangkauan
terendah dari kelas),batas atas(jangkauan tertinggi dari kelas),tepi
bawah kelas(batas bawah dikurangi ketelitian data),tepi atas
kelas(batas atas kelas ditambah ketelitian data),banyak kelas(1+3,3
log n),dan panjang kelas.\\
Dalam statistik terdapat berbagai macam distribusi frekuensi,
diantaranya:\\
1. Distribusi frekuensi biasa\\
\end{eulercomment}
\begin{eulerttcomment}
   Distribusi frekuensi biasa adalah distribusi
   frekuensi yang berisikan jumlah frekuensi
   dari setiap kelompok data atau kelas.
\end{eulerttcomment}
\begin{eulercomment}
2. Distribusi frekuensi relatif\\
\end{eulercomment}
\begin{eulerttcomment}
   Distribusi frekuensi relatif adalah
   distribusi frekuensi yang dinyatakan dalam
   bentuk persentase atau desimal. Besarnya
   frekuensi relatif yaitu frekuensi absolut
   setiap kelas dibagi total frekuensi dikali
   100%.
\end{eulerttcomment}
\begin{eulercomment}
3. Distribusi frekuensi kumulatif\\
\end{eulercomment}
\begin{eulerttcomment}
   Menunjukkan seberapa besar jumlah frekuensi
   pada tingkat kelas tertentu yang diperoleh
   dengan menjumlahkan atau mengurangkan
   frekuensi pada kelas tertentu dengan
   frekuensi kelas sebelumnya. Distribusi
   frekuensi kumulatif terdiri dari 2 macam
   yaitu distribusi kumulatif kurang dari dan
   distribusi kumulatif lebih dari.
\end{eulerttcomment}
\begin{eulercomment}

Contoh Soal 1:\\
1.Disajikan data urut yaitu
45,48,49,50,52,52,52,53,53,54,54,54,54,54,56,56,
56,56,57,57,58,58,58,58,58,58,58,59,59,60,60,60,
62,62,62,63,63,64,64,65,67,68,69,70,70,71,73,74.\\
Buatlah distribusi frekuensi berdasarkan data diatas!\\
Penyeleaian:\\
\end{eulercomment}
\begin{eulerttcomment}
         - Menentukan range
           range= nilai maks-nilai min
                = 74-45
                = 29
         - Menentukan banyak kelas dengan aturan
           struges.
           = 1+3,3 log n, n banyaknya data
           = 1+3,3 log 48
           = 6,64
           = 7
         - Menentukan panjang kelas
\end{eulerttcomment}
\begin{eulerformula}
\[
p=\frac {range}{banyak kelas}
\]
\end{eulerformula}
\begin{eulerformula}
\[
p=\frac {29}{7}
\]
\end{eulerformula}
\begin{eulerformula}
\[
p= 4.14=5
\]
\end{eulerformula}
\begin{eulercomment}
Berdasarkan pertimbangan beberapa unsur dalam data urut diatas yaitu
nilai minimum 45, nilai maksimum 74, banyak kelas yaitu 7, dan panjang
kelas yaitu 5 maka dapat dibuat tabel distribusi frekuensi dengan
batas bawah kelas pertama yaitu 43 dan batas atas kelas ketujuh yaitu
77. Sehingga dapat ditentukan tepi bawah kelas pertama yaitu
43-0.5=42.5 dan tepi atas kelas ketujuh yaitu 77+0.5=77.5.
\end{eulercomment}
\begin{eulerprompt}
>r=42.5:5:77.5; v=[1,6,13,15,6,5,2];
>T:=r[1:7]' | r[2:8]' | v'; writetable(T,labc=["TB","TA","Frek"])
\end{eulerprompt}
\begin{euleroutput}
          TB        TA      Frek
        42.5      47.5         1
        47.5      52.5         6
        52.5      57.5        13
        57.5      62.5        15
        62.5      67.5         6
        67.5      72.5         5
        72.5      77.5         2
\end{euleroutput}
\begin{eulercomment}
Mencari titik tengah 
\end{eulercomment}
\begin{eulerprompt}
>(T[,1]+T[,2])/2 // the midpoint of each interval 
\end{eulerprompt}
\begin{euleroutput}
             45 
             50 
             55 
             60 
             65 
             70 
             75 
\end{euleroutput}
\begin{eulercomment}
Sajian dalam bentuk histogram
\end{eulercomment}
\begin{eulerprompt}
>plot2d(r,v,a=40,b=80,c=0,d=20,bar=1,style="\(\backslash\)/"):
\end{eulerprompt}
\eulerimg{29}{images/SubTopik 5.1_Amalia Intan Arvitasari_22305144026_Mat B-004.png}
\eulersubheading{Rata-Rata Hitung(Mean)}
\begin{eulercomment}
Rata-Rata hitung biasa juga disebut sebagai rerata atau mean merupakan
ruang lingkup kajian pada analisis statistika deskriptif yang termasuk
dalam ukuran pemusatan data.\\
\end{eulercomment}
\begin{eulerformula}
\[
\mbox{ Rata-Rata hitung ini disimbolkan dengan } {\mu} \mbox{ untuk data populasi } dan
\]
\end{eulerformula}
\begin{eulerformula}
\[
\bar {X} \mbox{ untuk data sampel }.
\]
\end{eulerformula}
\begin{eulercomment}
Rata-rata hitung atau mean merupakan nilai yang menunjukkan pusat dari
nilai data dan dapat mewakili keterpusatan data. Mean dapat diperoleh
dengan membagi jumlah nilai-nilai data dengan jumlah individu(cacah
data). Perhitungan mean dibagi dua yaitu mean data tunggal dan mean
data kelompok.\\
\end{eulercomment}
\eulersubheading{1. Perhitungan mean pada data tunggal}
\begin{eulercomment}
Pada data tunggal, perhitungannya yaitu dengan cara menjumlahkan semua
nilai dan dibagi banyak data. Rumus yang digunakan adalah sebagai
berikut:\\
\end{eulercomment}
\begin{eulerformula}
\[
\bar{X} = \frac{\sum x_i}{n}
\]
\end{eulerformula}
\begin{eulerformula}
\[
\mu = \frac{\sum x_i}{N}
\]
\end{eulerformula}
\begin{eulercomment}
Keterangan:\\
\end{eulercomment}
\begin{eulerformula}
\[
\bar {X}=\mbox{ Rata-Rata hitung atau mean untuk data sampel}
\]
\end{eulerformula}
\begin{eulerformula}
\[
\mu=\mbox{ Rata-Rata hitung atau mean untuk data populasi }
\]
\end{eulerformula}
\begin{eulerformula}
\[
\sum x_i=\mbox{ jumlah dari nilai data ke-i }
\]
\end{eulerformula}
\begin{eulerttcomment}
           n = banyaknya data dalam sampel
           N - banyaknya data dalam populasi
\end{eulerttcomment}
\begin{eulercomment}
Data tunggal juga dapat disajikan dalam tabel distribusi.\\
Misalnya diberikan data\\
\end{eulercomment}
\begin{eulerformula}
\[
x_1,x_2,...,x_n
\]
\end{eulerformula}
\begin{eulercomment}
yang memiliki frekuensi berturut-turut\\
\end{eulercomment}
\begin{eulerformula}
\[
f_1,f_2,...,f_n
\]
\end{eulerformula}
\begin{eulercomment}
maka rataan hitung sampel atau rataan hitung populasi dari data yang
disajikan dalam daftar distribusi itu ditentukan dengan rumus:

Untuk rata-rata hitung sampel,

\end{eulercomment}
\begin{eulerformula}
\[
\bar{x}=\frac{\sum_{i=1}^{n} f_i x_i}{\sum_{i=1}^{n} f_i}
\]
\end{eulerformula}
\begin{eulercomment}
Untuk rata-rata hitung populasi,

\end{eulercomment}
\begin{eulerformula}
\[
\mu=\frac{\sum_{i=1}^{n} f_i x_i}{\sum_{i=1}^{n} f_i}
\]
\end{eulerformula}
\begin{eulercomment}
Kita dapat mengetahui nilai rata-rata hitung(mean)pada data tunggal
dengan menggunakan perintah EMT yaitu 'mean(x)' dan 'mean(x,f)'.

Contoh Soal 1:\\
1. Seorang pelatih tembak ingin mengevaluasi nilai ketangkasan delapan
anak buahnya jenis senapan yang dipakai M-16 dengan jarak 300 meter
dan masing-masing mendapat nilai 76,85,70,65,40,70,50,dan 80.
Berapakah rata-rata nilai ketangkasan delapan anak tersebut?\\
Penyelesain:
\end{eulercomment}
\begin{eulerprompt}
>x=[76,85,70,65,40,70,50,80]; mean(x),
\end{eulerprompt}
\begin{euleroutput}
  67
\end{euleroutput}
\begin{eulercomment}
Diketahui:\\
\end{eulercomment}
\begin{eulerformula}
\[
\sum x_i={76+85+70+65+40+70+80}=536
\]
\end{eulerformula}
\begin{eulerttcomment}
                 n = 8
\end{eulerttcomment}
\begin{eulerformula}
\[
\bar{X} = \frac{\sum x_i}{n}
\]
\end{eulerformula}
\begin{eulerformula}
\[
\bar{X} = \frac{536}{8}
\]
\end{eulerformula}
\begin{eulerformula}
\[
\bar{X} = 67
\]
\end{eulerformula}
\begin{eulercomment}
Sehingga rata-rata nilai ketangkasan delapan anak tersebut adalah 67

Contoh Soal 2:\\
Banyaknya pegawai di 5 apotik adalah 3,5,6,4,dan 6. Dengan memandang
data itu sebagai data populasi, hitunglah nilai rata-rata banyaknya
pegawai di 5 apotik tersebut!\\
Penyelesaian:
\end{eulercomment}
\begin{eulerprompt}
>x=[3,5,6,4,6]; mean(x),
\end{eulerprompt}
\begin{euleroutput}
  4.8
\end{euleroutput}
\begin{eulercomment}
Diketahui:\\
\end{eulercomment}
\begin{eulerformula}
\[
\sum x_i={3+5+6+4+6}=24
\]
\end{eulerformula}
\begin{eulerttcomment}
                        N = 5
\end{eulerttcomment}
\begin{eulerformula}
\[
\mu = \frac{\sum x_i}{N}
\]
\end{eulerformula}
\begin{eulerformula}
\[
\mu = \frac{24}{5}
\]
\end{eulerformula}
\begin{eulerformula}
\[
\mu = 4.8
\]
\end{eulerformula}
\begin{eulercomment}
Sehingga nilai rata-rata banyaknya pegawai di 5 apotik tersebut adalah
4.8

Contoh Soal 3:\\
Diberikan data berat kambing di suatu peternakan yang memelihara 50
kambing. kambing dengan berat 45kg terdapat 5 ekor, kambing dengan
berat 46kg terdapat 10 ekor, kambing dengan berat 47kg terdapat 6
ekor, kambing dengan berat 48kg terdapat 9 ekor, kambing dengan berat
49kg terdapat 7 ekor, dan kambing dengan berat 50kg terdapat 13 ekor.
Tentukan rata-rata berat kambing di peternakan tersebut.\\
Penyelesaian:
\end{eulercomment}
\begin{eulerprompt}
>x=[45,46,47,48,49,50], f=[5,10,6,9,7,13]      //Mendeskripsikan data dan frekuensi
\end{eulerprompt}
\begin{euleroutput}
  [45,  46,  47,  48,  49,  50]
  [5,  10,  6,  9,  7,  13]
\end{euleroutput}
\begin{eulerprompt}
>mean(x,f)    //Menghitung rata-rata  
\end{eulerprompt}
\begin{euleroutput}
  47.84
\end{euleroutput}
\begin{eulercomment}
Jadi, rata-rata berat kambing di peternakan tersebut adalah 47.84 kg
\end{eulercomment}
\begin{eulercomment}


\end{eulercomment}
\eulersubheading{2. Perhitungan mean pada data kelompok}
\begin{eulercomment}
Jika data yang sudah dikelompokkan dalam hitungan distribusi frekuensi
maka data tersebut akan berbaur sehingga keaslian data itu akan hilang
bercampur dengan data lain menurut kelasnya, hanya dalam perhitungan
mean pada data kelompok diambil titik tengahnya untuk mewakili setiap
kelas interval. Adapun rumus yang digunakan untuk menghitung mean pada
data kelompok yaitu:\\
\end{eulercomment}
\begin{eulerformula}
\[
\bar{X} = \frac{\sum t_i f_i}{\sum f_i}
\]
\end{eulerformula}
\begin{eulerttcomment}
   Keterangan:
\end{eulerttcomment}
\begin{eulerformula}
\[
  \sum t_i f_i = \mbox{jumlah dari perkalian antara titik tengah tiap kelas dan frekuensi tiap kelas}
\]
\end{eulerformula}
\begin{eulerformula}
\[
\sum f_i = \mbox{jumlah dari frekuensi tiap kelas}
\]
\end{eulerformula}
\begin{eulercomment}
Kita dapat mengetahui nilai rata-rata hitung(mean) pada data kelompok
dengan menggunakan perintah EMT yaitu 'mean(t,v)'dimana t menunjukkan
titik tengah dan v menunjukkan frekuensi.

Misalkan suatu data berkelompok terdiri dari n kelas dengan nilai
tengah masing-masing kelas secara berturut-turut adalah\\
\end{eulercomment}
\begin{eulerformula}
\[
t_1,t_2,...,t_n
\]
\end{eulerformula}
\begin{eulercomment}
dan masing-masing frekuensinya adalah\\
\end{eulercomment}
\begin{eulerformula}
\[
f_1,f_2,...,f_n
\]
\end{eulerformula}
\begin{eulercomment}
maka untuk menghitung rata-rata data tabel distribusi seperti ini di
EMT, dapat dilakukan dengan cara berikut:\\
1. Menentukan tepi bawah kelas (Tb), panjang kelas (P), dan tepi atas
kelas (Ta) dengan rumus :

\end{eulercomment}
\begin{eulerformula}
\[
T_b=a-0,5
\]
\end{eulerformula}
\begin{eulerformula}
\[
P=(b-a)+1
\]
\end{eulerformula}
\begin{eulerformula}
\[
P=\frac{range}{banyak kelas}
\]
\end{eulerformula}
\begin{eulerformula}
\[
T_a=b+0.5
\]
\end{eulerformula}
\begin{eulercomment}
dengan a = batas bawah kelas dan b = batas atas kelas

2. Mendeskripsikan data dalam bentuk tabel, dengan perintah

\textgreater{} r=tepi bawah terkecil:panjang kelas:tepi atas terbesar;
v=[frekuensi];\\
\textgreater{} T:=r[1:jumlah kelas]' \textbar{} r[2:jumlah kelas + 1]' \textbar{} v';
writetable(T,labc=["tepi bawah","tepi atas","frekuensi"])

3. Menghitung nilai tengah kelas, dengan perintah

\textgreater{} (T[,1]+T[,2])/2

4. Mengubah baris menjadi kolom

\textgreater{} t=fold(r,[0.5,0.5])

5. Menghitung rata-rata, dengan perintah

\textgreater{} mean(t,v)

Contoh Soal:\\
1. Data berikut menunjukkan nilai yang diperoleh 50 siswa SMP 1 Playen
pada Ujian Nasional mata pelajaran matematika.\\
Siswa yang mendapat nilai dalam rentang 61-65 sebanyak 2 orang, dalam
rentang 66-70 sebanyak 5 orang, dalam rentang 71-75 sebanyak 8 orang,
dalam rentang 76-80 sebanyak 10 orang, dalam rentang 81-85 sebanyak 12
orang, dalam rentang 86-90 sebanyak 9 orang, dan dalam rentang 91-95
sebanyak 4 orang.\\
Tentukan rata-rata nilai yang diperoleh 50 siswa tersebut!\\
Penyelesaian:\\
Menentukan tepi bawah kelas yang terkecil
\end{eulercomment}
\begin{eulerprompt}
>61-0.5
\end{eulerprompt}
\begin{euleroutput}
  60.5
\end{euleroutput}
\begin{eulercomment}
Menentukan panjang kelas
\end{eulercomment}
\begin{eulerprompt}
>(65-61)+1
\end{eulerprompt}
\begin{euleroutput}
  5
\end{euleroutput}
\begin{eulercomment}
Menentukan tepi atas kelas yang terbesar
\end{eulercomment}
\begin{eulerprompt}
>95+0.5
\end{eulerprompt}
\begin{euleroutput}
  95.5
\end{euleroutput}
\begin{eulerprompt}
>r=60.5:5:95.5; v=[2,5,8,10,12,9,4];
>T:=r[1:7]' | r[2:8]' | v'; writetable(T,labc=["TB","TA","Frek"])
\end{eulerprompt}
\begin{euleroutput}
          TB        TA      Frek
        60.5      65.5         2
        65.5      70.5         5
        70.5      75.5         8
        75.5      80.5        10
        80.5      85.5        12
        85.5      90.5         9
        90.5      95.5         4
\end{euleroutput}
\begin{eulercomment}
Menentukan titik tengah
\end{eulercomment}
\begin{eulerprompt}
>(T[,1]+T[,2])/2 // the midpoint of each interval
\end{eulerprompt}
\begin{euleroutput}
             63 
             68 
             73 
             78 
             83 
             88 
             93 
\end{euleroutput}
\begin{eulerprompt}
>t=fold(r,[0.5,0.5])
\end{eulerprompt}
\begin{euleroutput}
  [63,  68,  73,  78,  83,  88,  93]
\end{euleroutput}
\begin{eulercomment}
Menentukan mean(rata-rata)
\end{eulercomment}
\begin{eulerprompt}
>mean(t,v)
\end{eulerprompt}
\begin{euleroutput}
  79.8
\end{euleroutput}
\begin{eulercomment}
Diketahui:\\
\end{eulercomment}
\begin{eulerformula}
\[
\sum t_i f_i=(2)(63)+(5)(68)+(8)(73)+(10)(78)+(12)(83)+(9)(88)+(4)(93)=3.990
\]
\end{eulerformula}
\begin{eulerformula}
\[
\sum f_i=2+5+8+10+12+9+4=50
\]
\end{eulerformula}
\begin{eulerformula}
\[
\bar{X} = \frac{\sum t_i f_i}{\sum f_i}
\]
\end{eulerformula}
\begin{eulerformula}
\[
\bar{X} = \frac{3.990}{50}
\]
\end{eulerformula}
\begin{eulerformula}
\[
\bar{X} = 79,8
\]
\end{eulerformula}
\begin{eulercomment}
Jadi rata-rata nilai yang diperoleh 50 siswa tersebut adalah 79,8

\end{eulercomment}
\eulersubheading{3. Perhitungan Rata-Rata dari file yang tersimpan dalam direktori}
\begin{eulercomment}
Dalam perhitungan rata-rata hitung(mean), kita dapat menggunakan file
yang tersimpan dalam direktori.

Contoh 1:\\
Misalnya kita akan menghitung nilai rata-rata(mean) yang terdapat
dalam file "test.dat"
\end{eulercomment}
\begin{eulerprompt}
>filename="test.dat"; ...
>V=random(3,3); writematrix(V,filename);
>printfile(filename),
\end{eulerprompt}
\begin{euleroutput}
  0.6554163483485531,0.2009951854518572,0.8936223876466522
  0.2818865431288053,0.5250003829714993,0.3141267749950177
  0.4446156782993733,0.2994744556282315,0.2826898577756425
  
\end{euleroutput}
\begin{eulerprompt}
>readmatrix(filename)
\end{eulerprompt}
\begin{euleroutput}
       0.655416      0.200995      0.893622 
       0.281887         0.525      0.314127 
       0.444616      0.299474       0.28269 
\end{euleroutput}
\begin{eulerprompt}
>mean(V),
\end{eulerprompt}
\begin{euleroutput}
       0.583345 
       0.373671 
        0.34226 
\end{euleroutput}
\begin{eulercomment}
Contoh 2:\\
Misalnya akan dihitung nilai rata-rata dari file 'mtcars.csv'
\end{eulercomment}
\begin{eulerprompt}
>filename="mtcars.csv";  ...
>V=random(4,4); writematrix(V,filename);
>printfile(filename)
\end{eulerprompt}
\begin{euleroutput}
  0.8832274852666707,0.2709059757306277,0.7044190158488305,0.217693385437102
  0.4453627564273003,0.3084110353211584,0.9145409086421026,0.1935854779811416
  0.4633868194128757,0.0951529788263157,0.5950170162024192,0.4311837813121366
  0.7286804774486648,0.4651641815616542,0.3230318624794988,0.5251835885646776
  
\end{euleroutput}
\begin{eulerprompt}
>readmatrix(filename)
\end{eulerprompt}
\begin{euleroutput}
       0.883227      0.270906      0.704419      0.217693 
       0.445363      0.308411      0.914541      0.193585 
       0.463387      0.095153      0.595017      0.431184 
        0.72868      0.465164      0.323032      0.525184 
\end{euleroutput}
\begin{eulerprompt}
>mean(V[1])
\end{eulerprompt}
\begin{euleroutput}
  0.519061465571
\end{euleroutput}
\begin{eulerprompt}
>mean(V[2])
\end{eulerprompt}
\begin{euleroutput}
  0.465475044593
\end{euleroutput}
\begin{eulerprompt}
>mean(V[3])
\end{eulerprompt}
\begin{euleroutput}
  0.396185148938
\end{euleroutput}
\begin{eulerprompt}
>mean(V[4])
\end{eulerprompt}
\begin{euleroutput}
  0.510515027514
\end{euleroutput}
\begin{eulerprompt}
>mean(V)
\end{eulerprompt}
\begin{euleroutput}
       0.519061 
       0.465475 
       0.396185 
       0.510515 
\end{euleroutput}
\eulersubheading{Median}
\begin{eulercomment}
Median merupakah ruang lingkup kajian pada analisis statistika
deskriptif yang termasuk dalam ukuran pemusatan data. Median adalah
suatu nilai yang berada di tengah data setelah diurutkan dari data
yang terkecil sampai yang terbesar atau sebaliknya. Dalam perhitungan
statistik, median dibagi menjadi 2 bagian yaitu median untuk data
tunggal dan median untuk data kelompok.\\
\end{eulercomment}
\eulersubheading{1. Median data tunggal}
\begin{eulercomment}
Jika jumlah suatu data(n) berjumlah ganjil maka nilai mediannya adalah
sama dengan data yang memiliki nilai di urutan paling tengah yang
memiliki nomor urut k, dimana untuk menentukan nilai k dapat dihitung
menggunakan rumus:\\
\end{eulercomment}
\begin{eulerformula}
\[
k = \frac{n+1}{2}
\]
\end{eulerformula}
\begin{eulercomment}
Jika jumlah suatu data(n) berjumlah genap,maka untuk menghitung
mediannya dengan menggunakan rumus:\\
\end{eulercomment}
\begin{eulerformula}
\[
k = \frac{n}{2}
\]
\end{eulerformula}
\begin{eulerformula}
\[
Me = \frac{1}{2}({X_k+X_{k+1})}
\]
\end{eulerformula}
\begin{eulercomment}
Kita dapat mengetahui nilai median pada data tunggal dengan
menggunakan perintah EMT yaitu 'median(data)'

Contoh Soal 1:\\
Diketahui sebuah data hasil nilai Ujian Akhir Semester mata kuliah
Filsafat Ilmu 11 mahasiswa sebagai berikut:
85,90,80,95,50,75,30,60,65,40,70.\\
Tentukan nilai median dari data tersebut!\\
Penyelesaian:
\end{eulercomment}
\begin{eulerprompt}
>data=[85,90,80,95,50,75,30,60,65,40,70];
>urut=sort(data)
\end{eulerprompt}
\begin{euleroutput}
  [30,  40,  50,  60,  65,  70,  75,  80,  85,  90,  95]
\end{euleroutput}
\begin{eulercomment}
Dalam menentukan median, langkah pertama yang harus dilakukan adalah
dengan mengurutkan data tersebut dari yang terkecil sampai yang
terbesar dengan fungsi sort(data). Fungsi sort(data) dalam EMT
digunakan untuk mengurutkan elemen-elemen dalam suatu vektor atau
matriks(dari nilai terkecil ke nilai terbesar).
\end{eulercomment}
\begin{eulerprompt}
>median(data)
\end{eulerprompt}
\begin{euleroutput}
  70
\end{euleroutput}
\begin{eulercomment}
Diketahui bahwa kasus ini merupakan data yang berjumlah ganjil,
sehingga nilai median untuk kasus ini adalah sama dengan data yang
memiliki nilai di urutan paling tengah yang memiliki nomor urut k.\\
\end{eulercomment}
\begin{eulerformula}
\[
k = \frac{n+1}{2}
\]
\end{eulerformula}
\begin{eulerformula}
\[
k = \frac{11+1}{2}
\]
\end{eulerformula}
\begin{eulerformula}
\[
k = 6
\]
\end{eulerformula}
\begin{eulerformula}
\[
Me = X_6 = 70
\]
\end{eulerformula}
\begin{eulercomment}
Jadi nilai median dari data hasil Ujian Akhir Semester(UAS) mata
kuliah Filsafat Ilmu 11 mahasiswa adalah 70

Contoh Soal 2:\\
Data upah dari 8 karyawan yang dinyatakan dalam rupiah adalah sebagai
berikut:\\
20,80,75,60,50,85,45,90.\\
Tentukan nilai median dari data tersebut!\\
Penyelesaian:
\end{eulercomment}
\begin{eulerprompt}
>data=[20,80,75,60,50,85,45,90];
>urut=sort(data)
\end{eulerprompt}
\begin{euleroutput}
  [20,  45,  50,  60,  75,  80,  85,  90]
\end{euleroutput}
\begin{eulerprompt}
>median(data)
\end{eulerprompt}
\begin{euleroutput}
  67.5
\end{euleroutput}
\begin{eulercomment}
Diketahui bahwa kasus ini merupakan data yang berjumlah genap,
sehingga nilai median untuk kasus ini adalah terletak pada data ke-k
dan data ke-(k+1).\\
\end{eulercomment}
\begin{eulerformula}
\[
k = \frac{n}{2}
\]
\end{eulerformula}
\begin{eulerformula}
\[
k = \frac{8}{2}
\]
\end{eulerformula}
\begin{eulerformula}
\[
k = 4
\]
\end{eulerformula}
\begin{eulercomment}
\end{eulercomment}
\begin{eulerformula}
\[
Me = \frac{1}{2}({X_k+X_{k+1})}
\]
\end{eulerformula}
\begin{eulerformula}
\[
Me = \frac{1}{2}({X_4+X_{5})}
\]
\end{eulerformula}
\begin{eulerformula}
\[
Me = \frac{1}{2}({60+75})
\]
\end{eulerformula}
\begin{eulerformula}
\[
Me = 67.5
\]
\end{eulerformula}
\begin{eulercomment}
Jadi nilai median pada data upah 8 karyawan adalah 67.5
\end{eulercomment}
\begin{eulercomment}

\end{eulercomment}
\eulersubheading{2. Median data kelompok}
\begin{eulercomment}
Untuk menghitung median pada data kelompok, dapat menggunakan rumus di
bawah ini:\\
\end{eulercomment}
\begin{eulerformula}
\[
Me = Tb+\frac{\frac {1}{2}{n}-f_{ks}}{f_m}.{p}
\]
\end{eulerformula}
\begin{eulerttcomment}
   Keterangan:
      Tb = Tepi bawah kelas median
       n = Total frekuensi
     fks = Frekuensi kumulatif sebelum median
      fm = frekuensi median
       p = panjang kelas
\end{eulerttcomment}
\begin{eulercomment}
Untuk menghitung median data berkelompok di EMT, dapat dilakukan
dengan cara berikut:\\
1. Menentukan tepi bawah kelas (Tb), panjang kelas (P), dan tepi atas
kelas (Ta) dengan rumus :

\end{eulercomment}
\begin{eulerformula}
\[
T_b=a-0,5
\]
\end{eulerformula}
\begin{eulerformula}
\[
P=(b-a)+1
\]
\end{eulerformula}
\begin{eulerformula}
\[
T_a=b+0.5
\]
\end{eulerformula}
\begin{eulercomment}
dengan a = batas bawah kelas dan b = batas atas kelas

2. Mendeskripsikan data dalam bentuk tabel, dengan perintah

\textgreater{} r=tepi bawah terkecil:panjang kelas:tepi atas terbesar;
v=[frekuensi];\\
\textgreater{} T:=r[1:jumlah kelas]' \textbar{} r[2:jumlah kelas + 1]' \textbar{} v';
writetable(T,labc=["tepi bawah","tepi atas","frekuensi"]))

3. Mendeskripsikan tepi bawah kelas median, panjang kelas median,
banyak data,frekuensi kumulatif sebelum median, frekuensi kelas median

\textgreater{} Tb=(tepi bawah kelas median), p=(panjang kelas median), n=(Total
frekuensi), Fks=(frekuensi kumulatif sebelum median), fm=(frekuensi
kelas median)

4. Menghitung median data dengan perintah:

\textgreater{} Tb+p*(1/2*n-Fks)/fm

Contoh Soal:\\
1. Berikut adalah data hasil dari pengukuran berat badan 50 siswa SD
Negeri Tambakrejo. Dari ke 50 siswa, mayoritas siswa memiliki berat
badan yang ideal. Siswa yang mempunyai berat badan dalam rentang 21-26
kg sebanyak 5 orang, yang mempunyai berat badan dalam rentang 27-32 kg
sebanyak 10 orang, yang mempunyai berat badan dalam rentang 33-38 kg
sebanyak 12 orang, yang mempunyai berat badan dalam rentang 39-44 kg
sebanyak 14 orang, yang mempunyai berat badan dalam rentang 45-50 kg
sebanyak 7 orang, dan yang mempunyai berat badan 51-56 kg sebanyak 2
orang. Tentukan median dari data hasil pengukuran berat badan 50 siswa
di SD tersebut!\\
Penyelesaian:\\
Menentukan tepi bawah kelas yang terkecil
\end{eulercomment}
\begin{eulerprompt}
>21-0.5
\end{eulerprompt}
\begin{euleroutput}
  20.5
\end{euleroutput}
\begin{eulercomment}
Menentukan panjang kelas
\end{eulercomment}
\begin{eulerprompt}
>(26-21)+1
\end{eulerprompt}
\begin{euleroutput}
  6
\end{euleroutput}
\begin{eulercomment}
Menentukan tepi atas kelas yang terbesar
\end{eulercomment}
\begin{eulerprompt}
>56+0.5
\end{eulerprompt}
\begin{euleroutput}
  56.5
\end{euleroutput}
\begin{eulerprompt}
>r=20.5:6:56.5; v=[5,10,12,14,7,2];
>T:=r[1:6]' | r[2:7]' | v'; writetable(T,labc=["TB","TA","frek"])
\end{eulerprompt}
\begin{euleroutput}
          TB        TA      frek
        20.5      26.5         5
        26.5      32.5        10
        32.5      38.5        12
        38.5      44.5        14
        44.5      50.5         7
        50.5      56.5         2
\end{euleroutput}
\begin{eulercomment}
Berdasarkan data, median berada pada urutan ke 25, maka median berada
pada kelas 32.5-38.5.
\end{eulercomment}
\begin{eulerprompt}
>Tb=32.5, p=6, n=50, Fks=15, fm=12
\end{eulerprompt}
\begin{euleroutput}
  32.5
  6
  50
  15
  12
\end{euleroutput}
\begin{eulerprompt}
>Tb+p*(1/2*n-Fks)/fm
\end{eulerprompt}
\begin{euleroutput}
  37.5
\end{euleroutput}
\begin{eulercomment}
Diketahui bahwa median berada di data ke 25\\
\end{eulercomment}
\begin{eulerformula}
\[
Me = Tb+\frac{\frac {1}{2}{n}-f_{ks}}{f_m}.{p}
\]
\end{eulerformula}
\begin{eulerformula}
\[
Me = 32.5+\frac{\frac{1}{2}({50})-15}{12}.{6}
\]
\end{eulerformula}
\begin{eulerformula}
\[
Me = 32.5+5
\]
\end{eulerformula}
\begin{eulerformula}
\[
Me = 37.5
\]
\end{eulerformula}
\begin{eulercomment}
Jadi median dari data hasil pengukuran berat badan 50 siswa SD
Tambakrejo adalah 37.5
\end{eulercomment}
\eulersubheading{Modus}
\begin{eulercomment}
Modus merupakan ruang lingkup kajian pada analisis statistika
deskriptif yang termasuk dalam ukuran pemusatan data. Modus adalah
nilai yang sering muncul diantara sebaran data atau nilai yang
memiliki frekuensi tertinggi dalam distribusi data. Modus terdiri dari
2 jenis yaitu modus untuk data tunggal dan modus untuk data kelompok.\\
1. Modus data tunggal\\
\end{eulercomment}
\begin{eulerttcomment}
   Cara menentukan modus untuk data tunggal
   terbilang cukup mudah yaitu dengan
   mengurutkan data dari yang terkecil ke yang
   terbesar sehingga data-data yang memiliki
   nilai yang sama akan berdekatan satu sama
   lain, lalu mencari frekuensi dari masing-
   masing data dan pilih data dengan frekuensi
   tertinggi.
\end{eulerttcomment}
\begin{eulercomment}
2. Modus data kelompok\\
\end{eulercomment}
\begin{eulerttcomment}
   Jika data telah dikelompokkan, maka telah
   disajikan dalam bentuk distribusi frekuensi.
   Berikut rumus untuk mencari modus data
   kelompok:
\end{eulerttcomment}
\begin{eulerformula}
\[
Mo = Tb+\frac{d_1}{d_1+d_2}.{c}
\]
\end{eulerformula}
\begin{eulerttcomment}
   Keterangan:
     Tb = Tepi bawah
     d1 = selisih f modus dengan f sebelumnya
     d2 = selisih f modus dengan f sesudahnya
      c = panjang kelas
\end{eulerttcomment}
\begin{eulercomment}
Untuk menghitung modus data berkelompok di EMT, dapat dilakukan dengan
cara berikut:\\
1. Menentukan tepi bawah kelas (Tb), panjang kelas (P), dan tepi atas
kelas (Ta) dengan rumus :

\end{eulercomment}
\begin{eulerformula}
\[
T_b=a-0,5
\]
\end{eulerformula}
\begin{eulerformula}
\[
P=(b-a)+1
\]
\end{eulerformula}
\begin{eulerformula}
\[
T_a=b+0.5
\]
\end{eulerformula}
\begin{eulercomment}
dengan a = batas bawah kelas dan b = batas atas kelas

2. Mendeskripsikan data dalam bentuk tabel, dengan perintah

\textgreater{} r=tepi bawah terkecil:panjang kelas:tepi atas terbesar;
v=[frekuensi];\\
\textgreater{} T:=r[1:jumlah kelas]' \textbar{} r[2:jumlah kelas + 1]' \textbar{} v';
writetable(T,labc=["tepi bawah","tepi atas","frekuensi"]))

3. Mendeskripsikan tepi bawah kelas modus, panjang kelas modus,selisih
frekuensi modus dengan frekuensi sebelumnya, selisih frekuensi modus
dengan frekuensi sesudahnya

\textgreater{} Tb=(tepi bawah kelas modus), p=(panjang kelas modus), d1=(selisih
frekuensi modus dengan frekunsi sebelumnya), d2=(selisih frekuensi
modus dengan frekuensi sesudahnya)

4. Menghitung modus dengan perintah:

\textgreater{} Tb+p*d1/(d1+d2)

Contoh Soal:\\
1. Berikut adalah data hasil dari pengukuran berat badan 30 siswa SD
Negeri Tambakrejo. Dari ke 30 siswa, mayoritas siswa memiliki berat
badan yang ideal. Siswa yang mempunyai berat badan dalam rentang 21-25
kg sebanyak 2 orang, yang mempunyai berat badan dalam rentang 26-30 kg
sebanyak 8 orang, yang mempunyai berat badan dalam rentang 31-35 kg
sebanyak 9 orang, yang mempunyai berat badan dalam rentang 36-40 kg
sebanyak 6 orang, yang mempunyai berat badan dalam rentang 41-45 kg
sebanyak 3 orang, dan yang mempunyai berat badan 46-50 kg sebanyak 2
orang. Tentukan modus dari data hasil pengukuran berat badan 30 siswa
di SD tersebut!\\
Penyelesaian:\\
Menentukan tepi bawah kelas yang terkecil
\end{eulercomment}
\begin{eulerprompt}
>21-0.5
\end{eulerprompt}
\begin{euleroutput}
  20.5
\end{euleroutput}
\begin{eulercomment}
Menentukan panjang kelas
\end{eulercomment}
\begin{eulerprompt}
>(25-21)+1
\end{eulerprompt}
\begin{euleroutput}
  5
\end{euleroutput}
\begin{eulercomment}
Menentukan tepi atas yang terbesar
\end{eulercomment}
\begin{eulerprompt}
>50+0.5
\end{eulerprompt}
\begin{euleroutput}
  50.5
\end{euleroutput}
\begin{eulerprompt}
>r=20.5:5:50.5; v=[2,8,9,6,3,2];
>T:=r[1:6]' | r[2:7]' | v'; writetable(T,labc=["TB","TA","frek"])
\end{eulerprompt}
\begin{euleroutput}
          TB        TA      frek
        20.5      25.5         2
        25.5      30.5         8
        30.5      35.5         9
        35.5      40.5         6
        40.5      45.5         3
        45.5      50.5         2
\end{euleroutput}
\begin{eulercomment}
Berdasarkan data, modus berada pada kelas 30.5-35.5.
\end{eulercomment}
\begin{eulerprompt}
>Tb=30.5, p=5, d1=1, d2=3
\end{eulerprompt}
\begin{euleroutput}
  30.5
  5
  1
  3
\end{euleroutput}
\begin{eulerprompt}
>Tb+p*d1/(d1+d2)
\end{eulerprompt}
\begin{euleroutput}
  31.75
\end{euleroutput}
\begin{eulercomment}
Jadi modus dari data hasil pengukuran berat badan 30 siswa di SD
Tambakrejo adalah 31.75
\end{eulercomment}
\end{eulernotebook}
\end{document}









